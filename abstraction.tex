%\subsection{Upper bound}
To prove that the "query coveravility problem" is in NP in the special case where there is only one register\lu{trouver abbréviation pour BNRA+1 reg}, we will present an abstraction on "configurations" and "runs". Intuitively, our abstraction keeps track of a set of reachable states and what we shall name a "gang". A "gang" represents a "boss" state and a set of "followers" states, it should represent all configurations in which a process is on the "boss" state and kept its initial value $v$ and other processes are on the "followers" states with register value $\val$. We will define an abstract semantics, and prove it to be sound and complete. Finally, we will present a NP-algorithm on this abstraction to answer the "query coverability problem".

In order to make proofs easier, and simplify the problem, we first present some preliminaries lemmas which allow us to (i) reduce our analysis to protocols that does not have receptions with actions $\quotemarks{\ne}$, or local tests, (ii) only consider "non-contradictory" queries.

\subsection{Preliminaries}



%\begin{definition}
%	Given two runs $\run, \run'$ over sets of agents $\agents, \agents'$ respectively, we define their ""parallel composition"" as the run over $\agents\times \set{0} \cup \agents'\times \set{1}$ defined as follows: 
%\end{definition}

\cortoin{Copycats are not used in the proofs. Remove?}
\luin{sometimes "copycat principle"}

In this model, given a run, we can clone it. Take a disjoint set of agents, with a disjoint set of initial values, each agent can mimick the behaviour of one agent of the initial run and reach the same state than its original agent. This idea is formalized in the following lemma.

\begin{lemma}[Weak copycat]\label{lem:weak_copycat}
Let $q \in Q$ and  $\run : \config_0 \step{*} \config_f$ a "run" over a set of agents $\agents$ with some distinguished agent $a_m \in \agents$ such that $\st{\config_f}(a_m) = q$. There exist $\agents' \subseteq \nats$ finite s.t. $\agents \cap \agents' = \emptyset$, $a_c \in \agents'$ and a "run" $\run': \config_0' \step{*} \config_f'$ over $\agents \cup \agents'$ such that:
\begin{itemize}
\item for all $a \in \agents$, $\config_f'(a) = \config_f(a)$ (agents of $\agents$ end the same in $\run$ and $\run'$),
\item for all $a \in \agents$, $a' \in \agents'$, $\data{\config_f}(a) \ne \data{\config_f'}(a')$,
\item $\st{\config_f}(a_m) = \st{\config_f'}(a_c)$.
\end{itemize}
\end{lemma}


\ifproofs
\begin{proof}
Let $\agents' \subseteq \nats$ such that $|\agents'|= |\agents|$ and $\agents \cap \agents' = \emptyset$.
There exists a bijection $\psi: \agents \mapsto \agents'$. Let $\config_0'$ an initial configuration over $\agents \cup \agents'$ which is equal to $\config_0$ on every agent of $\agents$. 
The constructed "run" $\run'$ starts on configuration $\config_0'$ and is composed of two parts. 
The first part consists in mimicking the sequence of steps of $\run$ with agents of $\agents$, while agents in $\agents'$ remain idle. This defines a run $\run_p' : \config_0' \step{*}  \config_m'$.  
The second part consists in a path $\apath': \config_m' \step{*} \config_f'$ which replicates the sequence of actions of $\run$ but on agents of $\agents'$: if an agent $a \in \agents$ takes a transition at step $i$ of $\run$, we make $\psi(a)$ take that transition at step $i$ of $\apath'$.
In this second part, agents of $\agents$ are left idle. $\run'$ is obtained by concatenating $\run_p'$ and $\apath'$. Since agents of $\agents$ behave the same in $\run$ and $\run'$, for all $a \in \agents$, $\config_f(a) = \config_f'(a')$. Also, $\psi(a_m)$ took the same transitions as $a_m$ hence $\st{\config_f}(\psi(a_m)) = q$. Finally, thanks to Remark~\ref{rem:run_no_new_register_values}, the register values of agents of $\agents'$ are in the set $\data{\config_0'}(\agents')$ while the register values of agents of $\agents$ are in $\data{\config_0'}(\agents')$; both sets are disjoint as $\config_0'$ is initial and for all $a \in \agents, a' \in \agents'$, $\data{\config_f'}(a) \ne \data{\config_f'}(a'_q)$.
\end{proof}
\fi


We can also clone some agents with their values. 
Take any run in which one agent broadcast its initial value, each time it sends its value, any number of processes can receive it. If one process stores the value at one point, any number of agents can do the same. This way, we can create run with any arbitrary number of processes storing the same value of the initial agent. This idea is formalized in the following lemma.

\begin{lemma}[Strong copycat]\label{lem:strong_copycat}
Let $\run : \config_0 \step{*} \config_f$ a "run" over a set of agents $\agents$ with some distinguished agent $a_m \in \agents$ such that $\data{\config_f}(a_m) \ne \data{\config_0}(a_m)$. % note that this condition may be relaxed to : a_m's register value did not stay the same throughout all of $\run$. 
 There exist $\agents' \in \nats$ st. $\agents \cap \agents' = \emptyset$, $a_c \in \nats\setminus (\agents \cup \agents')$ and a "run" $\run': \config_0' \step{*} \config_f'$ over $\agents \cup \agents' \cup \set{a_c}$ such that:
\begin{itemize}
\item for all $a \in \agents$, $\config_f'(a) = \config_f(a)$ (agents of $\agents$ behave the same in $\run$ and $\run'$),
\item for all $a \in \agents$, $a' \in \agents'$, $\data{\config_f}(a) \ne \data{\config_f'}(a')$,
\item $\st{\config_f'}(a_c) = \st{\config_f'}(a_m)$ and $\data{\config_f'}(a_c) = \data{\config_f'}(a_m)$.
\end{itemize}
\end{lemma}

\ifproofs
\begin{proof}
Write $\aval := \data{\config_f}(a_m)$. Since $a_m$ does not start with $v$, $a_m$ does a $\quotemarks{\enregact}$ action in $\run$.
Decompose $\run$ into $\run_p: \config_0 \step{*} \config_{m,1}$, $\apath_i: \config_{m,1} \step{} \config_{m,2}$ and $\apath_s: \config_{m,2} \step{*} \config_f$ where $\apath_i$ is the step where $a_m$ does a $\quotemarks{\enregact}$ action for the last time; write $(q,\rec{\amessage}{\enregact},q') \in\transitions$ the transition taken by $a_m$ in $\apath_i$. By applying weak copycat on $\run_p : \config_0 \step{*} \config$, we add an agent $a_c$ on state $q$ at the cost of adding a set of agents $\agents'$ whose data is disjoint from  the one of agents in $\agents$. Agents of $\agents'$ are left idle in subsequent steps. We make $a_c$ take the $\quotemarks{\enregact}$ transition at the same time as $a_m$, so that there are both on $q'$ with value $v$. We finally make $a_c$ mimick $a_m$ throughout $\apath_s$: to do so, whenever $a_m$ takes a reception transition, we make $a_c$ take the same transition in the same step (which is possible as they have the same register value). When $a_m$ broadcasts, we duplicate the broadcast step to make $a_c$ broadcast immediatly after, although no agent receives $a_c$'s broadcast. We end up with $a_m$ and $a_c$ on the same state with the same register value, which concludes the proof. 
 \end{proof}
\fi

\begin{corollary}
\label{cor:removing_diseq_tests}
Let $(\prot, \query)$ an instance of the "query coverability problem". This instance is positive if and only if $(\tilde{\prot}, \query)$ is positive, where $\tilde{\prot}$ is equal to $\prot$ where every disequality test $\quotemarks{\diseqtestact}$ is replaced by dummy action $\quotemarks{\dummyact}$.  
\end{corollary}

\ifproofs
\begin{proof}
First, if $(\prot, \query)$ is positive then so is $(\tilde{\prot}, \query)$, as one can easily lift any "run" in $\prot$ to an equivalent "run" in $\tilde{\prot}$ (transitions are less guarded  in $\tilde{\prot}$ that in $\prot$). 

Suppose now that $(\tilde{\prot}, \query)$ is a positive instance of the "query coverability problem". There exists a "run" $\tilde{\run}: \tilde{\config}_0 \step{*} \tilde{\config}$ in $\tilde{\prot}$ that satisfies $\query$. We prove by induction on the length of $\tilde{\run}$ that there exists a "run" $\run$ reaching a configuration $\config$ such that $\tilde{\config} \lessthan \config$ (Remark~\ref{rem:bigger_config_query} then allows us to conclude). 

If $\config = \config_0$ then $\run = \tilde{\run}$ suffices. Suppose that $\tilde{\run}$ has length $k \geq 1$, and that the result if true for "runs" of length $k-1$. Decompose $\tilde{\run}$ into $\tilde{\run_{k-1}}: \tilde{\config_0} \step{*} \tilde{\config_{k-1}}$ of length $k-1$ and a final step $\tilde{\config_{k-1}} \step{} \tilde{\config_k}$. 
By induction hypothesis, there exists $\run_{k-1}: \config_0 \step{*} \config_{k-1}$ such that $\tilde{\config_{k-1}} \lessthan \config_{k-1}$: there exists an injective function $\pi : \tilde{\agents} \rightarrow \agents$
 such that, for all $a \in \tilde{\agents}$, $\tilde{\config_{k-1}}(a) = \config_{k-1}(\pi(a))$, where $\tilde{\agents} := \agentsof{\tilde{\run}}$ and $\agents := \agentsof{\run}$. If $\tilde{\config_{k-1}} \step{} \tilde{\config_k}$ involves no reception transition from $\tilde{\prot}$ whose corresponding transition in $\prot$ has action $\quotemarks{\diseqtestact}$, then we directly lift this step into a step appended at the end of $\run_{k-1}$ (making $\pi(a)$ take a transition whenever $a$ does so in $\tilde{\config_{k-1}} \step{} \tilde{\config_k}$). Otherwise, write $\tilde{\agents}_{\diseqtestact}$ the subset of $\tilde{\agents}$ corresponding to agents taking in $\tilde{\config_{k-1}} \step{} \tilde{\config_k}$ a reception transition from $\tilde{\prot}$ whose corresponding transition in $\prot$ has action $\quotemarks{\diseqtestact}$ . Write $(q, \brone{m}, q') \in \transitions$ the broadcast transition used in this step.  Using Lemma~\ref{lem:weak_copycat}, we add to $\config_{k-1}$ a fresh agent $a_{\mathsf{new}}$ with state $q$ and a register value that does not appear in $\config_{k-1}$. 
We first mimic this broadcast step at the end of $\run_{k-1}$, making any agent $\pi(a) \in \pi(\tilde{\agents} \setminus \tilde{\agents}_{\diseqtestact})$ take the transition that $a$ takes in $\tilde{\config_{k-1}} \step{} \tilde{\config_k}$. We then add a new step where $a_{\mathsf{new}}$ broadcasts using transition $(q, \brone{m}, q')$, and every agent $\pi(a) \in \pi(\tilde{\agents}_{\diseqtestact})$ takes the transition corresponding to the transition taken by $a$ in $\tilde{\config_{k-1}} \step{} \tilde{\config_k}$. Such a transition is a reception with action $\quotemarks{\diseqtestact}$ in $\prot$; however, because $a_{\mathsf{new}}$ does not share its register value with any process from $\tilde{\agents}$, all disequality conditions are satisfied and this step is valid. In this end, every agent $\pi(a) \in \pi(\tilde{\agents})$ has taken the transition in $\prot$ corresponding to the one $a$ took in $\tilde{\prot}$ in step $\tilde{\config_{k-1}} \step{} \tilde{\config_k}$, hence the configuration $\config_k$ reached by the constructed run is such that $\tilde{\config_k} \lessthan \config_k$. 
\end{proof}
\fi

Thanks to Corollary~\ref{cor:removing_diseq_tests}, we shall from now on suppose that all considered protocols have no disequality tests. 

\begin{definition}
	Given a set of states $K \subseteq Q$, $\config$ a "configuration" over a set of agents $\agents$ and $\aval \in \nats$, we say that the pair $\config, \aval$ satisfies $K$ if for all $q \in K$, there exists $a \in \agents$ such that $\config(a) = (q,\aval)$.
	A configuration $\config$ satisfies $K$ if there exists $\aval \in \nats$ such that $\config, \aval $ satisfies $K$. 
	A "run" satisfies $K$ if its final configuration does.
\end{definition}
Note that if a "run" satisfies $K$, then it covers every state in $K$. The converse is not true as, in order to satisfy $K$, one needs to have agents of every state of $K$ that all share the same register value. See for instance the example below.

\begin{example}
	Consider the "protocol" displayed in Figure~\ref{fig:no-clique}.
	We can obtain configurations satisfying $\set{1,2}$, $\set{2,3}$ or $\set{1,3}$, but we cannot obtain one satisfying $\set{1,2,3}$.
	
	\begin{figure}[h]
		\begin{tikzpicture}[xscale=0.5,AUT style,node distance=2.1cm,auto,>= triangle
	45]
	\tikzstyle{initial}= [initial by arrow,initial text=,initial
	distance=.7cm, initial above]
	%	\tikzstyle{accepting}= [accepting by arrow,accepting text=,accepting
	%	distance=.7cm,accepting where =right]
	
	\node[state,initial, minimum width=0.1pt] (0) at (0,0) {0};
	\node[state] [left of=0, xshift=-50] (x) {1};
	\node[state] [below of=0] (y) {2};
	\node[state] [right of=0, xshift=50] (z) {3};

	\path[->, bend left=10] 	
	(0) edge node[above] {$\brone{x}$} (x)
	(0) edge node {$\brone{z}$} (z)
	;
	\path[->, bend right=10] 	
	(0) edge node[above] {$\recone{y}{\enregact}$} (z)
	(0) edge node[above] {$\recone{z}{\enregact}$} (x)
	;
	\path[->, bend left=20] 	
	(0) edge node {$\brone{y}$} (y)
	;
	\path[->, bend right=20] 	
	(0) edge node[left] {$\recone{x}{\enregact}$} (y)
	;
\end{tikzpicture}
		\caption{An illustrating example}
		\label{fig:no-clique}
	\end{figure}
\end{example}


The rest of this subsection aims to prove that one can only consider "non-contradictory" queries. We formalize this term in the following definition.
\begin{definition}
We say that a "query" $\query$ is ""contradictory"" if either:
	\begin{itemize}
		\item there exists $\varz$ and $q \neq q' \in Q$ such that $\quotemarks{q(\varz)}, \quotemarks{q'(\varz)} \in \query$ or
		
		\item there exist $\var{z_1}, \ldots, \var{z_k} \in \varset$ with $k\geq 1$ such that $\quotemarks{R(\varz_{i}) = R(\varz_{i+1})} \in \query$ for all $i \in \set{1,\ldots,k-1}$ and $\quotemarks{R(\varz_k) \neq R(\varz_1)} \in \query$.
	\end{itemize}
\end{definition}

\begin{proposition}
A "query" $\query$ is "contradictory" if and only if there exists no configuration satisfying $\query$. Additionnaly, one may decide whether $\query$ is contradictory in time polynomial in $\size{\prot} + \size{\query}$. 
\end{proposition}

\ifproofs
\begin{proof}
Suppose first that $\query$ is "contradictory". If there exist $q \ne q'$ such that
 $\quotemarks{q(\varz)}, \quotemarks{q'(\varz)} \in \query$,  then a "configuration" satisfying $\query$ would need to have a single agent in two distinct states. If there exist $\var{z_1}, \ldots, \var{z_k}$ distinct variables with $k>1$ such that $\quotemarks{R(\varz_{i}) = R(\varz_{i+1})} \in \query$ for all $i \in \set{1,\ldots,k-1}$ and $\quotemarks{R(\varz_k) \neq R(\varz_1)} \in \query$, then in any configuration satisfying $\query$, $\varz_k$ and $\varz_1$ must have the same register value but also distinct register values, which is a contradiction. 

Suppose now that the $\query$ is not "contradictory" and show that there exists some configuration satisfying $\query$.

We consider the graph $G$ whose vertices are variables of $\varset$ appearing in $\query$ and with an edge between $\varz$ and $\varz'$ if and only if $R(\varz) = R(\varz') \in \query$. Since $\query$ is not "contradictory", for any $\varz, \varz'$ in the same connected component of $G$, $\quotemarks{R(\varz) \ne R(\varz')} \notin \query$. We construct a configuration satisfying $\query$ with an agent $a_\varz$ for every $\varz$ such that:
\begin{itemize}
\item if $\quotemarks{q(\varz)} \in \query$ then $a_\varz$ is in state $q$,
\item the agents of the variables of a connected component of $G$ have all the same register value.  
\end{itemize}
The built configuration satisfies $\query$, which concludes the proof. 
\end{proof}
\fi

\begin{lemma}
\label{lem:query-decomposition}
Let $\query$ a  "non-contradictory" "query". There exist $K_1, \ldots, K_p \subseteq Q$ with $p \leq \size{\query}$ with the following property: there exists a "run" satisfying $\query$ if and only if for all $i \in \set{1,\ldots, p}$, there is a run satisfying $K_i$. Moreover,such sets $K_1, \dots, K_p$ may be computed in time polynomial in $\size{\prot}$. 
\end{lemma}

\ifproofs
\begin{proof}

	We consider the same graph $G$ as in the previous proof. We decompose $G$ into connected components $C_1, \ldots, C_p$. 
For each $C_i$ we define $K_i = \set{q \in Q \mid \exists \varz \in C_i, q(\varz) \in \query}$. Note that $K_i$ may be empty when there is no constraints on the states of the variables in $C_i$. An empty set is satisfied by any "run". This is coherent with the fact that, when $K_i = \emptyset$, it suffices to assign all variables in $C_i$ to an agent that stays in $q_0$ for the whole "run".

	Suppose there is a "run" satisfying $\query$, let $\config$ be its final configuration, let $\nu$ be a valuation witnessing the satisfaction of $\query$ by $\config$.
	Let $i \in \set{1,\ldots, p}$, we show that $\config$ satisfies $K_i$. For all $\varz \in C_i$, let $(q_\varz,\aval_\varz) = \config(\nu(\varz))$. As $C_i$ is a connected component of $G$ and $\config$ satisfies $\query$, all $\nu(\varz)$ with $\varz \in C_i$ have the same register value, which we call $\aval$. 
	For all $q \in K_i$, there exists $\varz \in C_i$ such that $q(\varz) \in \query$ and thus $q_\varz =q$ and $\aval_\varz =\aval$, thus $C_i$ is satisfied.
	
	For the converse implication, suppose for all $i$ we have a run $\run_i$ over a set of agents $\agents_i$ satisfying $K_i$. We rename agents so that the $\agents_i$ are pairwise disjoint, we introduce fresh agents $a_1, \ldots, a_p$ and we set $\agents = \set{a_1, \ldots, a_p} \sqcup \bigsqcup_{i=1}^p \agents_i$. We consider $\run: \config_0 \step{*} \config$ a "run" in which the sequences of actions of each $\run_i$ are executed sequently by agents from $\agents_i$. In $\run$, agents $a_1$ to $a_p$ are left idle on $q_0$. We build the valuation $\nu$ witnessing that $\run$ satisfies $\query$ as follows. Let $i \in \nset{1}{p}$. If $K_i = \emptyset$ then $\nu$ assigns $a_i$ to every variable of $C_i$. Otherwise, we know that $\config|_{\agents_i}$ satisfies $K_i$ because the final configuration of $\run_i$ does: there exists $\aval_i \in \nats$ such that, for all $q \in K_i$, some agent $a_{i,q}$ has state $q$ and register value $\aval_i$ in $\config$. Let $\varz \in C_i$. If, for some $q \in Q$, $\quotemarks{q(\varz)} \in \query$, $\nu$ assigns agent $a_{i,q}$ to $\varz$ ($q$ is unique as $\query$ is not "contradictory"). If $\query$ has no constraint about the state of $\varz$, $\nu$ assigns to $\varz$ some $a_{i,q}$ with $q \in k_i$ arbitrary. 
	
	% We define a valuation $\nu$ as follows. Let $\varz \in \varset$ and $i$ so that $C_i$ is the connected component of $\varz$ in $G$. If  for some $q$, then this $q$ is unique as $\query$ is not "contradictory". We set $\nu(\varz) = a_q^i$.
	% If there is no $q(\varz)$ for any $q \in Q$ in $\query$ but there is one of the form $q'(\varz')$ with $\varz' \in C_i$, then we set $\nu(\varz) = \nu(\varz')$.
	% Otherwise we set $\nu(\varz) = a_i$.
	% This fully defines our valuation $\nu$.
	
	Clearly all formulas of the form $q(\varz)$ or $R(\varz) = R(\varz')$ are satisfied.
	For the formulas of the form $R(\varz) \neq R(\varz')$, we observe that as $\query$ is not "contradictory", $\varz$ and $\varz'$ are not in the same connected component of $G$. Because the runs $\run_i$ worked with disjoint sets of register values ($\config_0$ is initial), the agents assigned to $\varz$ and $\varz'$ end the run with distinct register values.
	Hence $\data{\config}(\nu(\varz)) \neq \data{\config}(\nu(\varz'))$. As a result, $\query$ is satisfied by $\run$.
\end{proof}
\fi


We are now ready to define our abstraction.

\subsubsection{Abstraction}
\label{sec:abstraction}

The abstraction should keep track of a particular state (the state of the process with the initial register value) and a set of states in which any number of processes can be on with the same non-initial value. We name this tuple a "gang" which we define formally below.

\begin{definition}
	Let $(Q,\transitions, q_0)$ be a protocol.

	A ""gang"" is a pair $\gang = (\boss, \clique) \in (Q \cup \set{\noboss}) \times \powerset{Q}$. The element $\boss$ is the ""boss"" and the set $\clique$ is the ""clique"" of the "gang". %We write $\gangconfigs$ the set of "gangs". 

	Let $\run = \config_0 \step{} \config_1 \step{} \cdots \step{} \config_k$ be a "run" and $\aval \in \valsof{\run}$. The "gang" of value $\aval$ in $\run$, written $\gangof{\aval}{\run}$, is the "gang" $(\boss, \clique)$ such that, 
	\begin{itemize}
	\item if there exists $a_0 \in \agentsof{\run}$ such that, 
	for every 
	$i \in \nset{0}{k}$, 
	$\data{\config_i}(a_0) = \aval$ then $\boss := \st{\config_k}(a_0)$, otherwise $\boss := \noboss$, 
%	\nico{changement de def: pour etre le boss il faut garder sa valeur tt le long de l'execution}
	\item  $\clique := \set{q \in Q \mid \exists i \leq k, \exists a \in \agents\setminus \set{a_0}, \config_i(a) = (q,\aval)}$ %\\
%´\nico{j'ai changé la def pour fque ça soit plus facile: ancienne def $\clique := \set{q \in Q \mid \exists a \in \agents\setminus \set{\boss}, \config_k(a) = (q,\aval)}$}
	\end{itemize}
Note that, if such an agent $a_0$ exists, then it is unique as $\config_0$ is initial hence this definition is sound. If there exists no agents with value $\aval$ in $\config_0$, then trivially $\gangof{\aval}{\run} = (\noboss, \emptyset)$. 
\end{definition} 

Intuitively, a "gang" corresponds to the set of agents with a given "register value". The "boss" $\boss$ represents the process that had this value at the beginning and the "clique" $\clique$ the set of states of processes who have received and stored this register value. If the original owner of this value no longer has it, then $\boss = \noboss$. Note that we define the clique as the set of states $q$ such that \emph{at some point in the run} some agent was in state $q$ with value $v$. This is because we can use the copycat principle to add a large amount of agents that are in state $q$ with value $v$, and thus we can assume that there is always one.

We now define an abstract semantics based on gangs: intuitively, an abstract configuration is composed of a gang along with a set of states which are known to be coverable.

\begin{definition}
\label{def:abstract-configuration}
An ""abstract configuration"" over $\agents$ is a tuple of $2^Q \times \gangset$ where $\gangset$ designates the set of all "gangs". We write $\aconfigs{\agents}$ the set of "abstract configurations" over $\agents$ and $\allaconfigs := \bigcup_{\agents \subseteq \nats \text{ finite }}\aconfigs{\agents}$ the set of all abstract configurations. 

Given two abstract configurations $\aconfig = (\covset, \boss, \clique)$ and $\aconfig' = (\covset', \boss', \clique')$, there is an ""abstract step"" from $\aconfig$ to $\aconfig'$, denoted $\aconfig \step{} \aconfig'$, when $\clique' \subseteq \covset'$, $\boss' \in \covset' \cup \set{\noboss}$ and one of the following cases is satisfied.
\begin{enumerate}
\item \emph{Broadcast from "clique":}
	\begin{enumerate}[i]
		\item\label{item:broadcast_from_clique_broadcast} There exist $\amessage \in \messages$ and $\statebr \in \clique, \statebr' \in \clique'$ s.t. $(\statebr, \brone{m}, \statebr') \in \transitions$. 
		
		\item\label{item:broadcast_from_clique_boss} Either $\boss = \boss'$ or there exists $(\boss, \rec{\amessage}{\anact}, \boss') \in \transitions$ for some action $\anact$.

		\item\label{item:broadcast_from_clique_clique}$(\clique \cup \set{\statebr'}) \subseteq \clique'$ and, for all $q' \in \clique' \setminus (\clique \cup \set{\statebr'})$, there exists $q$ s.t. $(q, \rec{\amessage}{\anact}, q') \in \transitions$ where:
		\begin{itemize}
			\item $\anact = \quotemarks{\eqtestact}$ or $\quotemarks{\dummyact}$ and $q \in \clique$, or
			\item $\anact= \quotemarks{\enregact}$ and $q \in \covset$.
		\end{itemize}
		
		\item\label{item:broadcast_from_clique_covset}$(\covset \cup \set{\statebr'}) \subseteq \covset'$ and, for all $q' \in \covset' \setminus (\covset \cup \set{\statebr'})$, there exists $q$ s.t. $(q, \rec{\amessage}{\anact}, q') \in \transitions$ where:
		\begin{itemize}
			\item  $\anact = \quotemarks{\eqtestact}$ and $q \in \clique$, or
			\item $\anact = \quotemarks{\enregact}$ or $\quotemarks{\dummyact}$ and $q \in \covset$.
		\end{itemize}
	\end{enumerate}


	\item \emph{Broadcast from "boss":}
	\begin{enumerate}[i]
		\item \label{item:broadcast_from_boss_broadcast} there exists $\amessage \in \messages$ such that $(\boss, \brone{m}, \boss') \in \transitions$
		
		\item\label{item:broadcast_from_boss_boss} $\boss, \boss' \ne \noboss$ (technically implied by \ref{item:broadcast_from_boss_broadcast} but written here to match other cases)

		\item\label{item:broadcast_from_boss_clique} 	$\clique \subseteq \clique'$ and, for all $q' \in \clique' \setminus \clique$, there exists $q$ s.t. $(q, \rec{\amessage}{\anact}, q') \in \transitions$ where:
		\begin{itemize}
			\item $\anact = \quotemarks{\eqtestact}$ or $\quotemarks{\dummyact}$ and $q \in \clique$, or
			\item $\anact= \quotemarks{\enregact}$ and $q \in \covset$.
		\end{itemize}
				
		\item\label{item:broadcast_from_boss_covset} $\covset \cup \set{\boss'} \subseteq \covset'$ and, for all $q' \in \covset' \setminus (\covset \cup \set{\boss'})$, there exists $q$ s.t. $(q, \rec{\amessage}{\anact}, q') \in \transitions$ where:
		\begin{itemize}
			\item  $\anact = \quotemarks{\eqtestact}$ and $q \in \clique$, or
			\item $\anact = \quotemarks{\enregact}$ or $\quotemarks{\dummyact}$ and $q \in \covset$.
		\end{itemize}
	\end{enumerate}


	\item \emph{External broadcast:}
	\begin{enumerate}[i]
		\item\label{item:external_broadcast_broadcast} There exists $\amessage \in \messages$ and $\statebr \in \covset, \statebr' \in \covset'$ s.t. $(\statebr, \brone{m}, \statebr') \in \transitions$. 
	
		\item\label{item:external_broadcast_boss}Either $\boss = \boss'$ or:
		\begin{itemize} 
			\item $\boss' \ne \noboss$ and there exists $(\boss, \rec{\amessage}{\dummyact}, \boss') \in \transitions$, or
			\item $\boss' = \noboss$ and there exists $(\boss, \rec{\amessage}{\enregact}, \boss') \in \transitions$.
		\end{itemize}

		\item\label{item:external_broadcast_clique}$\clique \subseteq \clique'$ and, for all $q' \in \clique' \setminus \clique$, there exists $q \in \clique$ s.t. $(q, \rec{\amessage}{\dummyact}, q') \in \transitions$.
		
		\item\label{item:external_broadcast_covset}$(\covset \cup \set{\statebr'}) \subseteq \covset'$ and, for all $q' \in \covset' \setminus (\covset \cup \set{\statebr'})$, there exists $q \in \covset$ s.t. $(q, \rec{\amessage}{\anact}, q') \in \transitions$ where $\anact = \quotemarks{\enregact}$ or $\anact = \quotemarks{\dummyact}$.
	\end{enumerate}
	\item \emph{""Gang reset"":} $S' = S$, $\clique' = \emptyset$ and $\boss'= q_0$
\end{enumerate}


Given a concrete run $\run: \config_0 \step{*} \config_k$, we write \AP  $\intro*\absproj{\aval}{\run}$ for the "abstract configuration" $(\covset, \gangof{\aval}{\run})$ where $\covset$ is the set of all states appearing in $\run$. 

%The set of \emph{initial abstract configurations} is $\aconfiginitset := \set{(\set{q_0}, \boss, \clique)  \mid \boss \in \set{q_0, \noboss}, \clique \subseteq \set{q_0}}$.
The \emph{initial abstract configuration} is $\aconfiginit := (\set{q_0}, q_0, \emptyset)$. 
As in the concrete case, an ""abstract run"" is a sequence $\arun = \aconfig_0, \dots, \aconfig_k$ such that $\aconfig_0 = \aconfiginit$ is the initial configuration and, for all $i$, $\aconfig_i \step{} \aconfig_{i+1}$. We denote such a run $\aconfig_0 \step{*} \aconfig_k$. Similarly, we denote by $\aconfig \step{*} \aconfig'$ the existence of a sequence of steps from $\aconfig$ to $\aconfig'$.
\end{definition}

The gang reset will help us doing the following, if one wants to check if one state is reachable, it should be the abstract run leading to it. Once it is done, the reachable state is added to $S$, we are now ready to check something else, for example, if another state is reachable. This way, it can do a "gang reset" step in order to restart with some new boss but it keeps in mind the states it knows to be reachable, so it can use it later on. \lu{c'est une première formulation un peu moche, il faut repasser dessus}

\begin{lemma}
	\label{lem:short-run}
For every $\aconfig \in \allaconfigs$ such that $\aconfiginit \step{*} \aconfig$, there exists an abstract run $\arun: \aconfig_0 \step{*} \aconfig$ of less that $(|Q|+2)^3$ steps.
% note: this bound is not optimal and is chosen to keep the proof simple
\end{lemma}

\ifproofs
\begin{proof}
Note that $\covset$ may never decrease along an abstract execution and that $\clique$ may only decrease at "gang resets".

We can hence enforce in the abstract semantics that, at least every $|Q|+2$ steps without "reset", either $\covset$ or $\clique$ has increased. Indeed, otherwise the configuration has looped as the boss may only take $|Q| +1$ values. We may also enforce that $\covset$ has strictly increased between two "resets", as otherwise one may remove anything that happened between the two "resets". Therefore, there are at most $|Q|-1$ "gang resets" in total, and each portion of the execution with no "reset" has at most $(|Q|+2)(|Q|+1)$ steps. This yields the bound of the proposition. 
\end{proof}
\fi

We now prove that our abstraction is sound and complete for our problem of interest. 

\subsection{Completeness}
This subsection is devoted to proving Lemma~\ref{lem:abstraction_complete}.

\begin{lemma}
\label{lem:abstraction_complete}
If $\run$ is a (concrete) "run" and $\aval \in \valsof{\run}$, then $\aconfiginit \step{*} \absproj{\aval}{\run}$. 
\end{lemma}


% The following lemma shall later be useful:
% \begin{lemma}
% \label{lem:adding_states_in_covset}
% If $\aconfig = (\covset, \gang), \aconfig' = (\covset', \gang') \in \allaconfigs$ are such that $\aconfig \step{*} \aconfig'$ then $\covset \subseteq \covset'$ and, for all $T \subseteq Q$, $(\covset \cup T, \gang) \step{*} (\covset' \cup T, \gang')$. 
% \end{lemma}
% \begin{proof}
% To prove the first statement, in suffices to observe that, in a step of the abstract semantics, the set $\covset$ may not decrease.
% To prove the second statement, it suffices to notice in the abstract semantics that a larger $\covset$ may never hinder a step. 
% \end{proof}

\begin{lemma}
\label{lem:proof_completeness_covset_constant}
For all "runs" $\run: \config_0 \step{*} \config$ and $\aval \in \valsof{\run}$, $(\statesin{\run}, q_0, \emptyset) \step{*} \absproj{\aval}{\run}$. 
\end{lemma}

\ifproofs
\begin{proof}
Let $\covset := \statesin{\run}$ and $\agents = \agentsof{\run}$.

Thanks to Remark~\ref{rem:run_no_new_register_values},  $\aval$ appears in $\config_0$; let $a_0$ be the (unique) agent such that $\data{\config_0}(a_0) = \aval$. We write $\run : \config_0 \step{} \config_1 \step{} \dots \step{} \config_k = \config$. For every $i \leq k$, let $\run_i : \config_0 \step{*} \config_i$ be the prefix of $\run$ of length $i$, and write $\aconfig^i := (\covset, \gangof{\aval}{\run_i})$. Note that $\gangof{\aval}{\run_0} = (q_0, \emptyset)$ hence $\aconfig^0 = (\covset, q_0, \emptyset)$. Also, we write $(\covset, \boss_i, \clique_i) := \aconfig^i$.

We prove by induction on $i$ that $\aconfig^0 \step{*} \aconfig^i$.
The statement is trivially true for $i =0$. 

Suppose now that $(\covset, \emptyset, \noboss) \step{*} \aconfig^i$. 
If suffices to prove that $\aconfig^i \step{} \aconfig^{i+1}$. First of all we clearly have, by definition, $K_{i+1} \subseteq S$ and $\boss_{i+1}\in S\cup\set{\noboss}$. We consider the last step of $\run_{i+1}$, which is referred to under the name $s_{i+1}$ in what follows; $s_{i+1}: \config_i \step{} \config_{i+1}$. Let $\agentbr$ the agent making the broadcast transition in $s_{i+1}$ and $A_{\recsymb}$ the set of agents receiving this broadcast in $s_{i+1}$. Let $(\statebr, \brone{\amessage}, \statebr') \in \transitions$ denote the transition taken by $\agentbr$ in $s_{i+1}$.

We now make the following case distinction to determine the type of the abstract step $\aconfig^i \step{} \aconfig^{i+1}$:
\begin{enumerate}
\item\label{proof_completeness:case_broadcast_clique} if $\data{\config_{i}}(\agentbr) = \aval$ but there exists $j<i$ such that $\data{\config_{j}}(\agentbr) \ne \aval$ then it is a ``broadcast from clique'',
\item\label{proof_completeness:case_broadcast_boss} if, for all $j \leq i$, $\data{\config_j}(\agentbr) = \aval$ then it is a ``broadcast from boss'',
\item\label{proof_completeness:case_external_broadcast} otherwise it is an ``external broadcast''. 
\end{enumerate}
Note that $\agentbr$ may not change its register value in $s_{i+1}$ hence $\data{\config_i}(\agentbr) = \data{\config_{i+1}}(\agentbr)$. 

Let $\agentboss$ the agent such that $\data{\config_0}(\agentboss) = \aval$. In case~\ref{proof_completeness:case_broadcast_boss}, $\agentboss = \agentbr$; in the other two cases, $\agentboss \ne \agentbr$. 

We now prove the other conditions:
\begin{enumerate}[i]
\item In case~\ref{proof_completeness:case_broadcast_clique}, since $\agentbr$ has value $\aval$ in $\config_i$ and $\config_{i+1}$ but not in every $\config_j$ for $j \leq i$, we directly have $\statebr \in \clique_i$ and $\statebr' \in \clique_{i+1}$. In case~\ref{proof_completeness:case_broadcast_boss}, it suffices to note that $\boss_i = \st{\config}(\agentbr)$ and $\boss_{i+1} = \st{\config_{i+1}}(\agentbr)$. In case~\ref{proof_completeness:case_external_broadcast}, it suffices to note that $\statebr, \statebr' \in \covset$ as both states appear in $\run$.
\item In case~\ref{proof_completeness:case_broadcast_boss}, this condition is automatically satisfied. In the other two cases, we look at what $\agentboss$ does in $s_{i+1}$. If it remains idle then we have $\boss_i = \boss_{i+1}$. Otherwise it takes a reception transition as $\agentbr \ne \agentboss$. 
In case~\ref{proof_completeness:case_external_broadcast}, this reception may not have action $\quotemarks{\eqtestact}$ as the broadcast is from an agent with register value that is not $\aval$ ($\data{\config_i}(\agentbr) \ne \aval$ by hypothesis). For the same reason, if this reception has action $\quotemarks{\enregact}$ then $\boss_{i+1}= \noboss$. If this reception has action $\quotemarks{\dummyact}$ and $\boss_i \ne \noboss$ then $\boss_{i+1} \ne \noboss$ as $\agentboss$ keeps value $\aval$.
\item We have $\clique_i \subseteq \clique_{i+1}$ because $\run_{i}$ is a prefix of $\run_{i+1}$; also, in case~\ref{proof_completeness:case_broadcast_clique}, $\statebr' \in \clique_{i+1}$ because of $\agentbr$. Let $q' \in \clique_{i+1} \setminus \clique_i$ with, in case~\ref{proof_completeness:case_broadcast_clique}, $q' \ne \statebr'$. There exist $q\in \covset$ and an agent $a$ that takes a reception transition $(q, \rec{\amessage}{\anact}, q')$ in $s_{i+1}$ and has value $\aval$ in $\config_{i+1}$. In cases~\ref{proof_completeness:case_broadcast_clique} and \ref{proof_completeness:case_broadcast_boss}, the broadcast has value $\aval$ hence if $\anact = \quotemarks{\eqtestact}$ or $\quotemarks{\dummyact}$ then $a$ has value $\aval$ in $\config_i$ and $q \in \clique_i$. In case~\ref{proof_completeness:case_external_broadcast}, the broadcast has value $\ne \aval$ hence $a$ may have value $\aval$ in $\config_{i+1}$ only when $\anact = \quotemarks{\dummyact}$ and $a$ had value $\aval$ in $\config_i$, which implies $q \in \clique_i$. 
\item It suffices to note that the first components of $\aconfig^i$ and $\aconfig^{i+1}$ are equal to $\covset$ and $\statebr' \in \covset$, and $\boss_{i+1} \in \covset \cup \set{\bot}$. 
\end{enumerate}
 
Overall, we have proven that $\aconfig^i \step{} \aconfig^{i+1}$, which concludes the induction step. Appyling the result with $i = k$ proves Lemma~\ref{lem:proof_completeness_covset_constant}. 

\end{proof}
\fi

\ifproofs
We may now prove Lemma~\ref{lem:abstraction_complete}. 

\begin{proof}[Proof of Lemma~\ref{lem:abstraction_complete}]
Let $\run$ a run.
We proceed by induction on the size of the set $\statesin{\run}$. 
First, if $\statesin{\run}$ is of size $1$ then $\statesin{\run} = \set{q_0}$. Applying Lemma~\ref{lem:proof_completeness_covset_constant} directly gives $\aconfiginit = (\set{q_0}, q_0, \noboss) \step{*} \absproj{\aval}{\run}$.


Suppose now that the statement is true for any $\run$ such that $\statesin{\run}$ is of size $k$, and suppose that we have a run $\run$ such that $\statesin{\run}$ is of size $k+1$. Let $\aval \in \valsof{\run}$. Let $\run_p: \config_0 \step{*}\config_p$ the longest suffix of $\run$ such that $\statesin{\run} \ne \statesin{\run_p}$. By induction hypothesis, we know that for all $\aval \in \valsof{\run}$, $\aconfiginit \step{*} \absproj{\aval}{\run_p}$. Write $s$ the step immediatly after $\run_p$ in $\run$. By maximality of $\run_p$, $\run_p s$ covers all states in $\statesin{\run} \setminus \statesin{\run_p}$. 

Write $\agentbr$ the agent broadcasting in $s$, $(\statebr, \brone{\amessage}, \statebr')$ the corresponding transition and $\avalbr$ the broadcast value. By induction hypothesis applied to $\run_p$ and $\avalbr$, $\aconfiginit \step{*} \absproj{\avalbr}{\run_p}$. Applying Lemma~\ref{lem:proof_completeness_covset_constant} on $\run$ and $\aval$ gives that $(\statesin{\run}, q_0, \emptyset) \step{*} \absproj{\aval}{\run}$. Therefore, it remains to prove that $\absproj{\aval}{\run_p} \step{*} (\statesin{\run}, q_0, \emptyset)$. It suffices to prove that there exists $\boss, \clique$ such that $\absproj{\avalbr}{\run_p} \step{} (\statesin{\run}, \boss,\clique)$, a "gang reset" allowing us to then reach $(\statesin{\run}, q_0, \emptyset)$. In other words, we prove that, from $\absproj{\avalbr}{\run_p}$, one may cover all states in $\statesin{\run}$ in just one abstract step.

We aim at proving that $\absproj{\avalbr}{\run_p} \step{} (\statesin{\run}, \boss,\clique)$ for well-chosen $\boss$ and $\clique$.
Just like in the proof of Lemma~\ref{lem:proof_completeness_covset_constant}, we make a case disjunction to prove that the broadcast in $s$ may be mimicked in the abstraction from $\absproj{\avalbr}{\run_p}$. The type obtained is either a "broadcast from clique" or a "broadcast from boss", because by hypothesis the agent broadcasting has value $\avalbr$ and therefore its state before the broadcast is either the boss or in the clique in $\absproj{\avalbr}{\run_p}$. 

Because we may choose $\boss$ and $\clique$ freely, the only challenging condition is \ref{item:broadcast_from_clique_covset}.
Let $q' \in \statesin{\run} \setminus \statesin{\run_p}$, $q' \ne \statebr'$. 
There exists an agent $a$ that takes a reception transition $(q,\rec{m}{\anact},q')$ in $s$. 
If $\anact = \quotemarks{\eqtestact}$, then agent $a$ has value $\avalbr$ at the end of $\run_p$ hence $q$ is in the clique of $\absproj{\avalbr}{\run_p}$ and condition \ref{item:broadcast_from_clique_covset} is satisfied. Otherwise, one has $q \in \statesin{\run_p}$ and condition \ref{item:broadcast_from_clique_covset} is satisfied.
In the end, there exist $\boss$, $\clique$ such that $\absproj{\avalbr}{\run_p} \step{} (\statesin{\run}, \boss,\clique)$. By a "gang reset", this implies that $\absproj{\avalbr}{\run_p} \step{*} (\statesin{\run}, q_0, \emptyset)$; we have proven that $\aconfiginit \step{*}\absproj{\avalbr}{\run_p} \step{*} (\statesin{\run}, q_0, \emptyset) \step{*} \absproj{\aval}{\run}$ which concludes the proof. 
\end{proof}
\fi

\subsection{Soundness}
\label{sec:soundness}

%\begin{lemma}[Strong copycat]\label{lem:strong-copycat}
%	Let $q \in Q$ and  $\run : \config_0 \step{*} \config_f$ an "initial run"  that covers $q$. Write $\agents := \agentsof{\run}$. There exists $\agents' \subseteq \nats$ finite, a bijection $\psi : \agents \to \agents'$ and a "run" $\run': \config_0' \step{*} \config_f'$ over $\agents \cup \agents'$ such that:
%	\begin{itemize}
%		\item for all $a \in \agents$, $\config_f'(a) = \config_f(a)$ (agents of $\agents$ behave the same in $\run$ and $\run'$),
%		\item for all $a' \in \agents'$, $\st{\config_f'}(a') = \st{\config_f}(\psi^{-1}(a'))$,
%		\item for all $a \in \agents$, either $\data{\config_f}(a) = \data{\config_0}(a)$ and $\data{\config_f'}(a') = \data{\config_0'}(a')$ or $\data{\config_f'}(\psi(a)) = \data{\config_f'}(a)$.
%	\end{itemize}
%\end{lemma}
%
%\begin{proof}
%	Let $\agents' \subseteq \nats$ such that $|\agents'|= |\agents|$ and $\agents \cap \agents' = \emptyset$. 
%	There exists a bijection $\psi: \agents \mapsto \agents'$. Let $M = 1+ \max\set{\data{\config_0}(a) \mid a \in \agents}$.
%	Let $\config_0'$ an initial configuration over $\agents \cup \agents'$ which is equal to $\config_0$ on every agent of $\agents$, and such that $\config'_0(\psi(a)) = \config_0(a) +M$ for all $a \in \agents$. 
%	The constructed "run" $\run'$ over $\agents \cup \agents'$ starts on configuration $\config_0'$. We define it by induction on $\run$.
%	
%	Let $\run = \config_0 \to \config_1 \to \cdots \to \config_n$, let $u = \config_1 \to \cdots \to \config_{n-1}$, suppose we have a run $u' = \config_0' \to \cdots \to \config_{k}'$ over $\agents \cup \agents'$ such that $u$ and $u'$ satisfy the conditions of the lemma.
%	
%	Let $a$ be the broadcasting agent in the last step of $\run$, $m$ the corresponding message.
%	We have $\config_{n-1}(a) = \config_{k}'(a)$.
%	
%	First, we extend $u'$ by making $\psi(a)$ broadcast $m$, and no agent receives it.
%	We obtain a configuration $\config'_{k+1}$, which is equal to $\config'_k$ except that $\psi(a)$ is now in state $\st{\config_{n}}(a)$.
%	
%	Second we make $a$ broadcast $m$ and for all $a_r \in \agents$ that receives the message in $\config_{n-1} \to \config_n$ with an operation $\alpha$, we make both $a_r$ and $\psi(a_r)$ receive it with the same operation.
%	We have to show that this is a correct transition for all $a_r$.
%	
%	\paragraph{First case: $\config_{n-1}(a_r) \neq \config'_{k}(\psi(a_r))$}  Then $a_r$ and $\psi(a_r)$ both still have their initial value, hence the action $\alpha$ cannot be an equality test
%	
%	
%	First we make $a$ broadcast the same message, and 
%	
%	$\psi(a)$ moves in $\run'$ instead, and if the broadcasting agent has register value $v$ in $\run$ then it has value $v+M$ in $\run'$. In this second part, agents of $\agents$ are left idle. Write $\config_f'$ the final configuration of $\run'$. Since agents of $\agents$ behave the same in $\run$ and $\run'$, for all $a \in \agents$, $\config_f(a) = \config_f'(a')$.
%	A straightforward induction shows that this is indeed a correct run, and that in the end of $\run'$ the states and values of agents of $\agents$ are the same as at the end of $\run$, and that for all $a \in \agents$ we have $\config_f'(\psi(a)) = (\st{\config_f}(a), \data{\config_f}(a) +M)$.
%	
%	From this we immediately infer the conditions of the lemma.
%\end{proof}


\begin{lemma}
	\label{lem:correctness-construction}
	
	\luin{donner de l'intuition sur lemma 39}
	Let $\sigma_0 \in \aconfiginitset$, and $\sigma_0 \to \sigma_1 \to \cdots \to \sigma_n$ an abstract run. For all $i$ let $(S_i, b_i, K_i) := \sigma_i$. Let $M = \size{\Delta}+1$.
	
	For all $i$, there exist a set of agents $\agents_i$, a configuration $\config_i$, a run $\run_i : \config_0 \step{*} \config_i$ over $\agents_i$, agents $a_0, \cdots, a_n \in \agents_i$ and values $v_0, \ldots, v_n \in \nats$ such that:
	\begin{itemize}
		\item for all $s \in S_i$, there are at least $M^{n-i}$ agents (different from $a_i$) in state $s$ 
		
		\item for all $s \in K_i$, there are at least $M^{n-i}$ agents (different from $a_i$) in state $s$ with value $v_i$
		
		\item if $b_i \neq \noboss$, then $a_i$ is in state $b_i$ with value $v_i$.
	\end{itemize}
\end{lemma}

\ifproofs
\begin{proof}

We proceed by induction on $i$.
We set $\agents_0 = \set{1, \ldots, M^n}$, and we set $\config_0(a) = (q_0, a)$ for all $a$. Clearly $\config_0$ satisfies the requirements with respect to $\sigma_0$, with $a_0 = v_0 \in \agents$.

Now assume we constructed $\config_0 \step{*} \cdots \step{*} \config_{i}$ over $\agents_i$ satisfying the conditions of the lemma, we construct $\config_{i+1}$ using a case distinction on the form of the transition $\sigma_i \to \sigma_{i+1}$.
For each $s \in S\setminus K$ we define $\agents_{i,s}$ as the set of agents in state $s$ in $\config_{i}$. We have $\size{\agents_{i,s}} \geq M^{n-i}$ thus we can extract $M = \size{\Delta}+1$ disjoint sets of agents $(\agents_{i,s}^d)_{d \in \Delta\cup\set{\epsilon}}$ from it.
Similarly, for each $s \in K$ we define $\agents_{i,s}$ as the set of agents in state $s$ \textbf{with value $\mathbf{v_i}$} in $\config_{i}$. We have $\size{\agents_{i,s}} \geq \size{\Delta}^{n-i}$ thus we can extract $\size{\Delta}+1$ disjoint sets of agents $(\agents_{i,s}^d)_{d \in \Delta\cup\set{\epsilon}}$ from it.
\\

\textbf{Case 1: } If $\sigma_i \to \sigma_{i+1}$ is a \emph{broadcast from the clique} $d = (q, \brone{m}, q')$ with $q \in K_i$, then we make all agents $a \in \agents_{i,q}^{d}$ (which all have value $v_i$) execute that transition one by one.
None of those broadcasts are received by any other agent, except for the last one:
If $b \neq b'$ then there is a transition $(b, \rec{m}{\alpha}, b')$ and we make $a_i$ execute it upon receiving the broadcast. We then set $a_{i+1} = a_i$.
For all $k' \in K_{i+1} \setminus K_i$ there exists a transition $d'=(k, \rec{m}{\alpha}, k')$ such that either $\alpha$ is $\eqtestact$ or $*$ and $k \in K_i$ or $\alpha$ is $\enregact$ and $k\in S$.
In both cases we make all agents of $\agents_{i,k}^{d'}$ take that transition.

For all $s' \in S_{i+1} \setminus (S_i \cup K_{i+1})$ there exists a transition $d'=(s,\rec{m}{*},s')$ (the operation cannot be $\enregact$ or $\eqtestact$ as otherwise $s$ would be in $K_{i+1}$). We then make all agents of $\agents_{i,s}^{d'}$ follow that transition. 

We set $v_{i+1} = v_i$.
\\

\textbf{Case 2: }If $\sigma_i \to \sigma_{i+1}$ is a \emph{broadcast from the boss} $d = (b_i, \brone{m}, b_{i+1})$, then we make $a_i$ (which has value $v_i$) execute that transition, and we set $a_{i+1} = a_i$.
The agents receiving that message are as follows:

For all $k' \in K_{i+1} \setminus K_i $ there exists a transition $d'=(k, \rec{m}{\alpha}, k')$ such that $\alpha$ is either $\eqtestact$ or $*$ and $k \in K_i$ or $\alpha$ is $\enregact$ and $k\in S$.
In both cases we make all agents of $\agents_k^{d'}$ take that transition.

For all $s' \in S_{i+1} \setminus (S_i \cup K_{i+1} \cup \set{b_{i+1}})$ there exists a transition $d'=(s,\rec{m}{*},s')$ (the operation cannot be $\enregact$ or $\eqtestact$ as otherwise $s$ would be in $K_{i+1}$). We then make all agents of $\agents_s^{d'}$ follow that transition. 

By definition of an "abstract run", we must have $b_i \in S_i$.
Hence we can make all agents of $\agents_{i,s}^{d}$ execute $d$, with no agent receiving the corresponding broadcasts.

We set $v_{i+1} = v_i$.
\\

\textbf{Case 3: } If $\sigma_i \to \sigma_{i+1}$ is an \emph{external broadcast} $d = (q, \brone{m}, q')$ , then we make all agents $a \in \agents_q^{d}$ execute that transition one by one. None of those broadcasts are received by any other agent, except for the last one:
If $b_i \neq b_{i+1}$ then there is a transition $(b, \rec{m}{\alpha}, b'')$ and either $b_{i+1} = b'' \neq \noboss$ and $\alpha = *$ or $b_{i+1} = \noboss$ and $\alpha=\enregact$. In both cases we make $a_i$ execute that transition, and we set $a_{i+1} = a_i$.

For all $k' \in K_{i+1} \setminus K_i$ there exists a transition $d'=(k, \rec{m}{*}, k')$ with $k \in K_i$. We make all agents of $\agents_k^{d'}$ take that transition.

For all $s' \in S_{i+1} \setminus (S_i \cup K_{i+1})$ there exists a transition $d'=(s,\rec{m}{\alpha},s')$ with $\alpha \in \set{*, \enregact}$. We then make all agents of $\agents_s^{d'}$ follow that transition. 

We set $v_{i+1} = v_i$.
\\

\textbf{Case 4: }  If $\sigma_i \to \sigma_{i+1}$ is a \emph{gang reset} then no agent moves and we select some $a_{i+1}$ in $\agents_{q_0}$ and set $v_{i+1}$ to be its value.
\\
%\paragraph{In all cases:} After applying the given transitions, we use the copycat property: we add to $\agents_i$ $M^{n-i-1}$ disjoint copies of itself to obtain $\agents_{i+1}$, and repeat the run constructed thus far over each copy separately. This is to ensure that there are $M^{n-i-1}$ agents in $b_{i+1}$ (if it is not $\noboss$) after this step.

Throughout the case distinction we have ensured that:
\begin{itemize}
	\item If $b_{i+1} \neq \noboss$ then $a_{i+1}$ is an agent of value $v_{i+1}$.
	
	\item For all $k \in K_{i}$, the agents of $\agents_{i,k}^\epsilon$ do not move between configurations $\config_{i}$ and $\config_{i+1}$, hence they have state $k$ and value $v_{i+1}$ in $\config_{i+1}$.
	
	\item If the step is not a gang reset, then $v_{i+1} = v_i$ and for all $k' \in K_{i+1} \setminus K_i$, there exists $d \in \Delta$ from some $k$ to $k'$ such that all agents of $\agents_{i,k}^d$ take that transition. Furthermore, if $d$ is of the form $(k,\rec{m}{\enregact},k')$ then the broadcasting process has value $v_i$, thus all those agents keep value $v_i = v_{i+1}$. 
%	As $\size{\agents_{i,k}^d}\geq M^{n-i-1}$, there are at least that many agents with value $v_{i+1} = v_i$ in $k'$.
	
	\item For all $s \in S_{i}$, the agents of $\agents_{i,s}^\epsilon$ do not move between configurations $\config_{i}$ and $\config_{i+1}$, hence they have state $s$ in $\config_{i+1}$.
	
	\item If the step is not a gang reset, for all $s' \in S_{i+1} \setminus (S_i \cup \set{b_{i+1}})$, there exists $d \in \Delta$ from some $s \in S_i$ to $s'$ such that all agents of $\agents_{i,s}^d$ take that transition.
	
	\item If the step is a gang reset, the conditions of the lemma hold trivially.
\end{itemize}

As a result, we have ensured that the conditions of the lemma were respected.
This concludes our induction.
\end{proof}
\fi

\begin{corollary}
	\label{cor:soundness}
%	For all $\sigma_0 \in \aconfiginitset$ and $\sigma = (S, b, K) \in \Sigma$ such that $\sigma_0 \step{*} \sigma$ there exists a reachable configuration $\gamma$ 
%	satisfying $K \cup \set{b}$ if $b \neq \noboss$ and $K$ otherwise.
	For all $\sigma_0 \in \aconfiginitset$ and $\sigma = (S, b, K) \in \Sigma$ such that $\sigma_0 \step{*} \sigma$, for all $s \in S$, there exists a reachable configuration $\gamma$ covering $s$.
%	satisfying $K \cup \set{b}$ if $b \neq \noboss$ and $K$ otherwise.
\end{corollary}

\ifproofs
\begin{proof}
	We simply apply Lemma~\ref{lem:correctness-construction} to an "abstract run" $\sigma_0 \to \cdots \sigma_n = \sigma$ from $\sigma_0$ to $\sigma$ by setting $i = n$.
%	We obtain (by setting $i = n$) that there exists a value $v_n$ in the final configuration of the constructed run, for all $s \in K$, there is an agent with value $v_n$ in state $s$. Furthermore, if $b \neq \noboss$, then there is an agent in state $b$ with value $v_n$.  
\end{proof}
\fi


\section{An \np algorithm}

\begin{proposition}
	\label{prop:sound-and-complete}
	Let $K$ be a set of states, there exists a reachable "configuration" satisfying $K$ if and only if there exists a reachable "abstract configuration" $(S,b,K')$ with either $K \subseteq K'$ or $b \neq \noboss$ and $K \subseteq K' \cup \set{b}$.  
\end{proposition}

\begin{proof}
	The right-to-left direction is given by Corollary~\ref{cor:soundness}.
	
	For the left-to-right direction, let $\run$ be a run ending in a configuration $\config$ satisfying $K$. Let $v \in \nats$ be such that $\config$ has some agents with value $v$ in every state of $K$.
	
	We construct a suitable "abstract run" as follows: by Lemma~\ref{lem:abstraction_complete} there exists an abstract run from some $\sigma_0 \in \aconfiginitset$ to an abstract configuration $\absproj{v}{\run} = (S, b, K')$.
	
	As $\config, v$ satisfies $K$, for all $s \in K$ either $s = b$ or $s \neq b$ and there exists an agent with state $s$ and value $v$ in $\config$. 	
	By definition of the abstraction $\absproj{v}{\run}$, we have $K \subseteq K'$ if $b=\noboss$ and $K \subseteq K' \cup \set{b}$ otherwise, proving the proposition.
\end{proof}

\begin{theorem}
	\label{thm:np-complete-query-cover}
	The "query coverability problem" is \np-complete.
\end{theorem}

\begin{proof}
	The lower bound is given by Proposition~\ref{prop:np-hard-query-cover}.
	For the upper bound, say we are given a "protocol" $\prot = (Q, \messages, \Delta, q_0)$ and a "query" $\phi$.
	We start by decomposing $\phi$ into cliques $K_1, \ldots, K_p$ in polynomial time, as in Lemma~\ref{lem:query-decomposition}.
	For each $i$, we have to verify that there exists a reachable configuration $\config_i$ satisfying $K_i$. By Proposition~\ref{prop:sound-and-complete}, it is the case if and only if there is an "abstract run" to an "abstract configuration" $(S,b, K)$ with either $K_i \subseteq K$ or $b\neq \noboss$ and $K_i \subseteq K\cup\set{b}$.
	Furthermore, by Lemma~\ref{lem:short-run} if there is such an "abstract run" then there is one with at most $(\size{Q}+2)^3$ steps. 
	Thus we can simply guess such an abstract run and verify it in polynomial time.
	As a result, the "query coverability problem" is in \np. 
\end{proof}
The following lemma formalizes the intuition that, because the only way to take a new register value is to receive another agent's value, no new register values are created in a "run".
\begin{lemma}
\label{lem:run_no_new_register_values}
For all "runs" $\run: \config_0 \step{*} \config_f$, for all $a \in \agentsof{\run}$, there exists $a' \in \agentsof{\run}$ such that $\data{\config_f}(a) = \data{\config_0}(a')$.  
\end{lemma}


A "run" $\run: \config_0 \step{*} \config_f$ ""covers"" a state $q \in Q$ when there exists $a \in \agents$ such that $\st{\config_f}(a) = q$. 

%\begin{definition}
%	Given two runs $\run, \run'$ over sets of agents $\agents, \agents'$ respectively, we define their ""parallel composition"" as the run over $\agents\times \set{0} \cup \agents'\times \set{1}$ defined as follows: 
%\end{definition}


\begin{lemma}[Copycat]\label{lem:copycat}
Let $q \in Q$ and  $\run : \config_0 \step{*} \config_f$ an "initial" "run"  that covers $q$. Write $\agents := \agentsof{\run}$. There exists $\agents' \subseteq \nats$ finite, a "run" $\run': \config_0' \step{*} \config_f'$ over $\agents \cup \agents'$ and $a' \in \agents'$ such that:
\begin{itemize}
\item for all $a \in \agents$, $\config_f'(a) = \config_f(a)$ (agents of $\agents$ behave the same in $\run$ and $\run'$),
\item $\st{\config_f'}(a') = q$,
\item for all $a \in \agents$, $\data{\config_f}(a) \ne \data{\config_f'}(a')$.
\end{itemize}
	% Let $\run$ be an initial run over a set of agents $\agents$, for all $n$ we can define an initial run $\run'$ over $\agents \times \set{1,\ldots, n}$ such that:
	% \begin{itemize}
	% 	\item   
		
	% 	\item
	% \end{itemize}
	
	
	
	% Let $\run, \run'$ be "initial runs" of a "protocol" over sets of agents $\agents, \agents'$, $\config, \config'$ their final "configurations".
	% Then there exist $X, X'$ disjoint sets of agents, bijections $g_x: X\to \agents, g_{X'} : X' \to \agents'$ and a run $\run$ over $X \sqcup X'$ whose final configuration $\config_f$ is such that $\st{\config}_f|_X = \st{\config} \circ g_X$, $\st{\config}_f|_{X'} = \st{\config'} \circ g_{X'}$ for all $x, y \in X$ (resp., $\in X'$), $\data{\config_f}(x) = \data{\config_f}(y)$ if and only if $\data{\config}(g_X(x)) = \data{\config}(g_X(y))$ (resp. $\data{\config'}(g_{X'}(x)) = \data{\config'}(g_{X'}(y))$), and for all $x \in X$, $x' \in X'$, $\data{\config_f}(x) \neq \data{\config_f}(x')$.
	% \cortoin{TODO : find a good formulation}
\end{lemma}

\begin{proof}
Let $\agents' \subseteq \nats$ such that $|\agents'|= |\agents|$ and $\agents \cap \agents' = \emptyset$. 
There exists a bijection $\psi: \agents \mapsto \agents'$. Let $\config_0'$ an initial configuration over $\agents \cup \agents'$ which is equal to $\config_0$ on every agent of $\agents$. 
The constructed "run" $\run'$ over $\agents \cup \agents'$ starts on configuration $\config_0'$ and is composed of two parts. 
The first part consists in mimicking the sequence of steps of $\run$ with agents of $\agents$, while agents in $\agents'$ remain idle. 
The second part consists in replicating this sequence of actions, but whenever an agent $z \in \agents$ moves in $\run$, 
$\psi(a)$ moves in $\run'$ instead. In this second part, agents of $\agents$ are left idle. Write $\config_f'$ the final configuration of $\run'$. Since agents of $\agents$ behave the same in $\run$ and $\run'$, for all $a \in \agents$, $\config_f(a) = \config_f'(a')$.
Since $\run$ covers $q$, 
there exists $a_q \in \agents$ such that $\st{\config_f}(a_q) = q$;
because $\psi(a_q)$ mimicks the behavior of $a_q$, $\st{\config_f'}(\psi(a_q)) = q$. Finally, thanks to Lemma~\ref{lem:run_no_new_register_values}, the register value of $a'_q := \psi(a_q)$ in $\config_f'$ was the register value of some agent of $\agents'$ in $\config_0'$. Meanwhile, for all $a \in \agents$, the register value of $a$ in $\config_f'$ was the register value of some agent of $\agents$ in $\config_0'$. Since $\config_0'$ is an initial configuration, no register value appears twice in $\config_0'$ hence for all $a \in \agents$, $\data{\config_f'}(a) \ne \data{\config_f'}(a'_q)$.
\end{proof}

\begin{corollary}
\label{cor:removing_diseq_tests}
Let $(\prot, \query)$ an instance of the "query reachability problem". This instance is positive if and only if $(\tilde{\prot}, \query)$ is positive, where $\tilde{\prot}$ is equal to $\prot$ where every disequality test $\quotemarks{\diseqtestact}$ is replaced by dummy action $\quotemarks{\dummyact}$.  
\end{corollary}
\begin{proof}
First, if $(\prot, \query)$ is positive then so is $(\tilde{\prot}, \query)$, as one can easily lift any "run" in $\prot$ to an equivalent "run" in $\tilde{\prot}$ (transitions are less guarded  in $\tilde{\prot}$ that in $\prot$). 

Suppose now that $(\tilde{\prot}, \query)$ is a positive instance of the "query reachability problem". There exists a "run" $\tilde{\run}: \tilde{\config}_0 \step{*} \tilde{\config}$ in $\tilde{\prot}$ that satisfies $\query$. We prove by induction on the length of $\tilde{\run}$ that there exists a "run" $\run$ reaching a configuration $\config$ such that $\tilde{\config} \lessthan \config$ (Remark~\ref{rem:bigger_config_query} then allows us to conclude). 

If $\config = \config_0$ then $\run = \tilde{\run}$ suffices. Suppose that $\tilde{\run}$ has length $k \geq 1$, and that the result if true for "runs" of length $k-1$. Decompose $\tilde{\run}$ into $\tilde{\run_{k-1}}: \tilde{\config_0} \step{*} \tilde{\config_{k-1}}$ of length $k-1$ and a final step $\tilde{\config_{k-1}} \step{} \tilde{\config_k}$. 
By induction hypothesis, there exists $\run_{k-1}: \config_0 \step{*} \config_{k-1}$ such that $\tilde{\config_{k-1}} \lessthan \config_{k-1}$: there exists an injective function $\pi : \tilde{\agents} \rightarrow \agents$
 such that, for all $a \in \tilde{\agents}$, $\tilde{\config_{k-1}}(a) = \config_{k-1}(\pi(a))$, where $\tilde{\agents} := \agentsof{\tilde{\run}}$ and $\agents := \agentsof{\run}$. If $\tilde{\config_{k-1}} \step{} \tilde{\config_k}$ involves no reception transition from $\tilde{\prot}$ whose corresponding transition in $\prot$ has action $\quotemarks{\diseqtestact}$, then we directly lift this step into a step appended at the end of $\run_{k-1}$ (making $\pi(a)$ take a transition whenever $a$ does so in $\tilde{\config_{k-1}} \step{} \tilde{\config_k}$). Otherwise, write $\tilde{\agents}_{\diseqtestact}$ the subset of $\tilde{agents}$ corresponding to agents taking in $\tilde{\config_{k-1}} \step{} \tilde{\config_k}$ a reception transition from $\tilde{\prot}$ whose corresponding transition in $\prot$ has action $\quotemarks{\diseqtestact}$ . Write $(q, \br{m}, q') \in \transitions$ the broadcast transition used in this step.  Using Lemma~\ref{lem:copycat}, we add to $\config_{k-1}$ a fresh agent $a_{\mathsf{new}}$ with state $q$ and a register value that do not appear in $\config_{k-1}$. 
We first mimick this broadcast step at the end of $\run_{k-1}$, making any agent $\pi(a) \in \pi(\tilde{\agents} \setminus \tilde{\agents}_{\diseqtestact})$ take the transition that $a$ takes in $\tilde{\config_{k-1}} \step{} \tilde{\config_k}$. We then add a new step where $a_{\mathsf{new}}$ broadcasts using transition $(q, \br{m}, q')$, and every agent $\pi(a) \in \pi(\tilde{\agents}_{\diseqtestact})$ takes the transition corresponding to the transition taken by $a$ in $\tilde{\config_{k-1}} \step{} \tilde{\config_k}$. Such a transition is a reception with action $\quotemarks{\diseqtestact}$ in $\prot$; however, because $a_{\mathsf{new}}$ does not share its register value with any process from $\tilde{\agents}$, all disequality conditions are satisfied and this step is valid. In this end, every agent $\pi(a) \in \pi(\tilde{\agents})$ has taken the transition in $\prot$ corresponding to the one $a$ took in $\tilde{\prot}$ in step $\tilde{\config_{k-1}} \step{} \tilde{\config_k}$, hence the configuration $\config_k$ reached by the constructed run is such that $\tilde{\config_k} \lessthan \config_k$. 
\end{proof}

Thanks to Corollary~\ref{cor:removing_diseq_tests}, we shall from now on suppose that all considered protocols have no disequality tests. 

\begin{definition}
	Given a set of states $K \subseteq Q$, we say that a "configuration" $\config$ over a set of agents $\agents$ satisfies $K$ if there exists $n \in \nats$ such that for all $q \in K$, there exists $a \in \agents$ such that $\config(a) = (q,n)$.
	A "run" satisfies $K$ if its final configuration does.
\end{definition}
Note that a "run" satisfies $K$, then it covers every state in $K$. The converse is not true as, in order to satisfy $K$, one needs to have agents of every state of $K$ that all share the same register value. 
\nicoin{interesting example}

\begin{definition}
We say that a "query" $\query$ is ""contradictory"" if either:
	\begin{itemize}
		\item there exists $\varz$ and $q \neq q' \in Q$ such that $\quotemarks{q(\varz)}, \quotemarks{q'(\varz)} \in \query$ or
		
		\item there exist $\var{z_1}, \ldots, \var{z_k} \in \varset$ with $k\geq 1$ such that $\quotemarks{R(\varz_{i}) = R(\varz_{i+1})} \in \query$ for all $i \in \set{1,\ldots,k-1}$ and $\quotemarks{R(\varz_k) \neq R(\varz_1)} \in \query$.
	\end{itemize}
\end{definition}

\begin{proposition}
A "query" $\query$ is "contradictory" if and only if there exists no configuration satisfying $\query$. Additionnaly, one may decide whether $\query$ is contradictory in time polynomial in $\size{\prot} + \size{\query}$. 
\end{proposition}
\begin{proof}
Suppose first that $\query$ is "contradictory". If there exist $q \ne q'$ such that
 $\quotemarks{q(\varz)}, \quotemarks{q'(\varz)} \in \query$,  then a "configuration" satisfying $\query$ would need to have a single agent in two distinct states. If there exist $\var{z_1}, \ldots, \var{z_k}$ distinct variables with $k>1$ such that $\quotemarks{R(\varz_{i}) = R(\varz_{i+1})} \in \query$ for all $i \in \set{1,\ldots,k-1}$ and $\quotemarks{R(\varz_k) \neq R(\varz_1)} \in \query$, then in any configuration satisfying $\query$, $\varz_k$ and $\varz_1$ must have the same register value but also distinct register values, which is a contradiction. 

Suppose now that the $\query$ is not "contradictory" and show that there exists some configuration satisfying $\query$.

We consider the graph $G$ whose vertices are variables of $\varset$ appearing in $\query$ and with an edge between $\varz$ and $\varz'$ if and only if $R(\varz) = R(\varz') \in \query$. Since $\query$ is not "contradictory", for any $\varz, \varz'$ in the same connected component of $G$, $\quotemarks{R(\varz) \ne R(\varz')} \notin \query$. We construct a configuration satisfying $\query$ with an agent $a_\varz$ for every $\varz$ such that:
\begin{itemize}
\item if $\quotemarks{q(\varz)} \in \query$ then $a_\varz$ is in state $q$,
\item the agents of the variables of a connected component of $G$ have all the same register value.  
\end{itemize}
The built configuration satisfies $\query$, which concludes the proof. 
\end{proof}

\begin{lemma}
Let $\query$ a  non-"contradictory" "query". There exist $K_1, \ldots, K_p \subseteq Q$ with $p \leq \size{\query}$ with the following property: there exists a "run" satisfying $\query$ if and only if for all $i \in \set{1,\ldots, p}$, there is a run satisfying $K_i$. Moreover,such sets $K_1, \dots, K_p$ may be computed in time polynomial in $\size{\prot}$. 
\end{lemma}

\begin{proof}

	We consider the same graph $G$ as in the previous proof. We decompose $G$ into connected components $C_1, \ldots, C_p$. 
For each $C_i$ we define $K_i = \set{q \in Q \mid \exists \varz \in C_i, q(\varz) \in \query}$. Note that $K_i$ may be empty when there is no constraints on the states of the variables in $C_i$. An empty set is satisfied by any "run". This is coherent with the fact that, when $K_i = \emptyset$, it suffices to assign all variables in $C_i$ to an agent that stays in $q_0$ for the whole "run".

	Suppose there is a "run" satisfying $\query$, let $\config$ be its final configuration, let $\nu$ be a valuation witnessing the satisfaction of $\query$ by $\config$.
	Let $i \in \set{1,\ldots, p}$, we show that $\config$ satisfies $K_i$. For all $\varz \in C_i$, let $(q_\varz,\aval_\varz) = \config(\nu(\varz))$. As $C_i$ is a connected component of $G$ and $\config$ satisfies $\query$, all $\nu(\varz)$ with $\varz \in C_i$ have the same register value, which we call $\aval$. 
	For all $q \in K_i$, there exists $\varz \in C_i$ such that $q(\varz) \in \query$ and thus $q_\varz =q$ and $\aval_\varz =\aval$, thus $C_i$ is satisfied.
	
	For the converse implication, suppose for all $i$ we have a run $\run_i$ over a set of agents $\agents_i$ satisfying $K_i$. We rename agents so that the $\agents_i$ are pairwise disjoint, we introduce fresh agents $a_1, \ldots, a_p$ and we set $\agents = \set{a_1, \ldots, a_p} \sqcup \bigsqcup_{i=1}^p \agents_i$. We consider $\run: \config_0 \step{*} \config$ a "run" in which the sequences of actions of each $\run_i$ are executed sequently by agents from $\agents_i$. In $\run$, agents $a_1$ to $a_p$ are left idle on $q_0$. We build the valuation $\nu$ witnessing that $\run$ satisfies $\query$ as follows. Let $i \in \nset{1}{p}$. If $K_i = \emptyset$ then $\nu$ assigns $a_i$ to every variable of $C_i$. Otherwise, we know that $\config|_{\agents_i}$ satisfies $K_i$ because the final configuration of $\run_i$ does: there exists $\aval_i \in \nats$ such that, for all $q \in K_i$, some agent $a_{i,q}$ has state $q$ and register value $\aval_i$ in $\config$. Let $\varz \in C_i$. If, for some $q \in Q$, $\quotemarks{q(\varz)} \in \query$, $\nu$ assigns agent $a_{i,q}$ to $\varz$ ($q$ is unique as $\query$ is not "contradictory"). If $\query$ has no constraint about the state of $\varz$, $\nu$ assigns to $\varz$ some $a_{i,q}$ with $q \in k_i$ arbitrary. 
	
	% We define a valuation $\nu$ as follows. Let $\varz \in \varset$ and $i$ so that $C_i$ is the connected component of $\varz$ in $G$. If  for some $q$, then this $q$ is unique as $\query$ is not "contradictory". We set $\nu(\varz) = a_q^i$.
	% If there is no $q(\varz)$ for any $q \in Q$ in $\query$ but there is one of the form $q'(\varz')$ with $\varz' \in C_i$, then we set $\nu(\varz) = \nu(\varz')$.
	% Otherwise we set $\nu(\varz) = a_i$.
	% This fully defines our valuation $\nu$.
	
	Clearly all formulas of the form $q(\varz)$ or $R(\varz) = R(\varz')$ are satisfied.
	For the formulas of the form $R(\varz) \neq R(\varz')$, we observe that as $\query$ is not "contradictory", $\varz$ and $\varz'$ are not in the same connected component of $G$. Because the runs $\run_i$ worked with disjoint sets of register values ($\config_0$ is initial), the agents assigned to $\varz$ and $\varz'$ end the run with distinct register values.
	Hence $\data{\config}(\nu(\varz)) \neq \data{\config}(\nu(\varz'))$. As a result, $\query$ is satisfied by $\run$.
\end{proof}



\begin{definition}
	Let $(Q,\transitions, q_0)$ be a protocol.

	A ""gang"" is a pair $\gang = (\boss, \clique) \in (Q \cup \set{\noboss}) \times \powerset{Q}$. The element $\boss$ is the ""boss"" and the set $\clique$ is the ""clique"" of the "gang". %We write $\gangconfigs$ the set of "gangs". 

	Let $\run = \config_0 \step{} \config_1 \step{} \cdots \step{} \config_k$ be a "run" and $\aval \in \data{\config_0}(\agents)$. The "gang" of value $\aval$ in $\run$, written $\gangof{\aval}{\run}$, is the "gang" $(\boss, \clique)$ such that, 
	\begin{itemize}
	\item if there exists $a_0 \in \agentsof{\run}$ such that $\data{\config_0}(a_0) = \data{\config_k}(a_0) = \aval$ then $\boss := a_0$, otherwise $\boss := \noboss$,
	\item  $\clique := \set{q \in Q \mid \exists i \leq k, \exists a \in \agents\setminus \set{\boss}, \config_i(a) = (q,\aval)}$ %\\
\nico{j'ai changé la def pour que ça soit plus facile: ancienne def $\clique := \set{q \in Q \mid \exists a \in \agents\setminus \set{\boss}, \config_k(a) = (q,\aval)}$}
	\end{itemize}
Note that, if such an agent $a_0$ exists, then it is unique as $\config_0$ is initial hence this definition is sound. If there exists no agents with value $\aval$ in $\config_0$, then trivially $\gangof{\aval}{\run} = (\noboss, \emptyset)$. 
\end{definition} 

Intuitively, a "gang" corresponds to the set of agents with a given "register value". The "boss" $\boss$ represents the process that had this value at the beginning and the "clique" $\clique$ the set of states of processes who receive and stored this register value. If the original owner of this value no longer has it, then $\boss = \noboss$.

We now define an abstract semantics based on gangs: intuitively, an abstract configuration is composed of a gang along with a set of states which are known to be coverable.

\begin{definition}
An ""abstract configuration"" over $\agents$ is a tuple of $2^Q \times \gangset$ where $\gangset$ designates the set of all "gangs". We write $\aconfigs{\agents}$ the set of "abstract configurations" over $\agents$ and $\allaconfigs := \bigcup_{\agents \subseteq \nats \text{ finite }}\aconfigs{\agents}$ the set of all abstract configurations. 

Given two abstract configurations $\aconfig = (\covset, \boss, \clique)$ and $\aconfig' = (\covset', \boss', \clique')$, there is an ""abstract step"" from $\aconfig$ to $\aconfig'$, denotes $\aconfig \step{} \aconfig'$, when one of the following cases is satisfied.
\begin{enumerate}
	\item[Broadcast from "boss":] $\boss, \boss' \ne \noboss$. There exists $\amessage \in \messages$ such that $(\boss, \br{m}, \boss') \in \transitions$. 

	$\clique \subseteq \clique'$ and, for all $q' \in \clique' \setminus \clique$, there exists $q$ s.t. $(q, \rec{\amessage}{\anact}, q') \in \transitions$ where:
	\begin{itemize}
		\item $\anact = \quotemarks{\eqtestact}$ or $\quotemarks{\dummyact}$ and $q \in \clique$, or
		\item $\anact= \quotemarks{\enregact}$ and $q \in \covset$.
	\end{itemize}
	
	$\covset \subseteq \covset'$ and, for all $q' \in \covset' \setminus \covset$, there exists $q$ s.t. $(q, \rec{\amessage}{\anact}, q') \in \transitions$ where:
	\begin{itemize}
		\item  $\anact = \quotemarks{\eqtestact}$ and $q \in \clique$, or
		\item $\anact = \quotemarks{\enregact}$ or $\quotemarks{\dummyact}$ and $q \in \covset$.
	\end{itemize}

	\item[Broadcast from "clique":] There exists $\amessage \in \messages$ and $q_{\brsymb} \in \clique, q'_{\brsymb} \in \clique'$ s.t. $(q_{\brsymb}, \br{m}, q'_{\brsymb}) \in \transitions$. 
	
	Either $\boss = \boss'$ or there exists $(\boss, \rec{\amessage}{\anact}, \boss') \in \transitions$ for some action $\anact$.

	$(\clique \cup \set{q'_{\brsymb}}) \subseteq \clique'$ and, for all $q' \in \clique' \setminus (\clique \cup \set{q'_{\brsymb}})$, there exists $q$ s.t. $(q, \rec{\amessage}{\anact}, q') \in \transitions$ where:
	\begin{itemize}
		\item $\anact = \quotemarks{\eqtestact}$ or $\quotemarks{\dummyact}$ and $q \in \clique$, or
		\item $\anact= \quotemarks{\enregact}$ and $q \in \covset$.
	\end{itemize}
	
	 $(\covset \cup \set{q'_{\brsymb}}) \subseteq \covset'$ and, for all $q' \in \covset' \setminus (\covset \cup \set{q'_{\brsymb}})$, there exists $q$ s.t. $(q, \rec{\amessage}{\anact}, q') \in \transitions$ where:
	\begin{itemize}
		\item  $\anact = \quotemarks{\eqtestact}$ and $q \in \clique$, or
		\item $\anact = \quotemarks{\enregact}$ or $\quotemarks{\dummyact}$ and $q \in \covset$.
	\end{itemize}
	\item[External broadcast:] There exists $\amessage \in \messages$ and $q_{\brsymb} \in \covset, q'_{\brsymb} \in \covset'$ s.t. $(q_{\brsymb}, \br{m}, q'_{\brsymb}) \in \transitions$. 
	
	Either $\boss = \boss'$ or:
	\begin{itemize} 
		\item $\boss' \ne \noboss$ and there exists $(\boss, \rec{\amessage}{\dummyact}, \boss') \in \transitions$, 
		\item $\boss' = \noboss$ and there exists $(\boss, \rec{\amessage}{\enregact}, \boss') \in \transitions$.
	\end{itemize}

	$\clique \subseteq \clique'$ and, for all $q' \in \clique' \setminus \clique$, there exists $q \in \clique$ s.t. $(q, \rec{\amessage}{\dummyact}, q') \in \transitions$.
	
	 $(\covset \cup \set{q'_{\brsymb}}) \subseteq \covset'$ and, for all $q' \in \covset' \setminus (\covset \cup \set{q'_{\brsymb}})$, there exists $q \in \covset$ s.t. $(q, \rec{\amessage}{\anact}, q') \in \transitions$ where $\anact = \quotemarks{\dummyact}$ or $\anact = \quotemarks{\dummyact}$.
	\item[Gang reset:] $S' = S$, $\clique' = \emptyset$, $\boss' \in S' \cup \set{\noboss}$
\end{enumerate}


Given a concrete run $\run: \config_0 \step{*} \config_k$, we write \AP  $\intro*\absproj{\aval}{\run}$ the "abstract configuration" $(\covset, \gangof{\aval}{\run})$ where $\covset$ is the set of all states appearing in $\run$. 

The set of "initial abstract configurations" is $\aconfiginitset := \set{(\set{q_0}, \boss, \clique)  \mid \boss \in \set{q_0, \noboss}, \clique \subseteq \set{q_0}}$.  
As in the concrete case, an "abstract run" is a sequence $\arun = \aconfig_0, \dots, \aconfig_k$ such that $\aconfig_0 \in \aconfiginitset$ and, for all $i$, $\aconfig_i \step{} \aconfig_{i+1}$. We denote such a run $\aconfig_0 \step{*} \aconfig_k$. Simiarly, we denote by $\aconfig \step{*} \aconfig'$ the existence of a sequence of steps from $\aconfig$ to $\aconfig'$.
\end{definition}


\begin{proposition}
For every $\aconfig_0 \in \aconfiginitset$, $\aconfig' \in \allaconfigs$ such that $\aconfig \step{*} \aconfig'$, there exists an abstract run $\arun: \aconfig \step{*} \aconfig'$ of less that $(|Q|+2)^3$ steps.
% note: this bound is not optimal and is chosen to keep the proof simple
\end{proposition}
\begin{proof}
Note that $\covset$ may never decrease along an abstract execution and that $\clique$ may only decrease at "gang resets".

We can hence enforce in the abstract semantics that, at least every $|Q|+2$ steps without "reset", either $\covset$ or $\clique$ has increased. Indeed, otherwise the configuration has looped as the boss may only take $|Q| +1$ values. We may also enforce that $\covset$ has strictly increased between two "resets", as otherwise one may remove anything that happened between the two "resets". Therefore, there are at most $|Q|-1$ "gang resets" in total, and each portion of the execution with no "reset" has at most $(|Q|+2)(|Q|+1)$ steps. This yields the bound of the proposition. 
\end{proof}

We now prove that our abstraction is sound and complete for our problem of interest. 

\subsection{Completeness}

\begin{lemma}
\label{lem:abstraction_complete}
Let $\run: \config_0 \step{} \config_1 \step{} \dots \step{} \config_k$ and $\aval \in \nats$. There exists $\aconfig_0 \in \aconfiginitset$ such that $\aconfig_0 \step{*} \absproj{\aval}{\run}$. 
\end{lemma}
The rest of this subsection is devoted to proving Lemma~\ref{lem:abstraction_complete}.

The following lemma shall later be useful:
\begin{lemma}
\label{lem:adding_states_in_covset}
If $\aconfig = (\covset, \gang), \aconfig' = (\covset, \gang') \in \allaconfigs$ are such that $\aconfig \step{*} \aconfig'$ then $\covset \subseteq \covset'$ and, for all $T \subseteq Q$, $(\covset \cup T, \gang) \step{*} (\covset' \cup T, \gang')$. 
\end{lemma}
\begin{proof}
To prove the first statement, in suffices to observe that, in a step of the abstract semantics, the set $\covset$ may not decrease.
To prove the second statement, it suffices to notice in the abstract semantics that a larger $\covset$ may never hinder a step. 
\end{proof}

\begin{lemma}
For all $\run: \aconfig_0 \step{*} \aconfig$ and $\aval \in \nats$, one has $(\statesin{\run}, \emptyset, \noboss) \step{*} \absproj{\aval}{\run}$. 
\end{lemma}
\begin{proof}
Let $\covset := \statesin{\run}$ and $\agents = \agentsof{\run}$.
First, if $\aval$ does not appear in $\aconfig_0$ then $\absproj{\aval}{\run} = (\covset, \emptyset, \noboss)$ and the statement is true.

Suppose now that $\aval$ does appear in $\aconfig_0$; let $a_0$ the (unique) agent such that $\data{\aconfig_0}{a_0} = \aval$. We write $\run : \aconfig_0 \step{} \aconfig_1 \step{} \dots \step{} \aconfig_k = \aconfig$. For every $i \leq k$, let $\run_i : \config_0 \step{*} \config_i$ the prefix of $\run$ of length $i$, and write $\aconfig_i := (\covset, \gangof{\aval}{\run_i})$. 

We prove by induction on $i$ that $(\covset, \emptyset, \noboss) \step{*} \aconfig_i$.

First, $\gangof{\aval}{\run_0} = (\st{\config_0}(a_0), \emptyset)$ and $\st{\config_0}(a_0) \in \covset$ by definition of $\covset$ therefore $\aconfig_i$ may be reached from $(\covset, \emptyset, \noboss)$ by a simple "gang reset".

Suppose now that $(\covset, \emptyset, \noboss) \step{*} \aconfig_i$. 
If suffices to prove that $\aconfig_i \step{} \aconfig_{i+1}$. Consider the last step of $\run_{i+1}$, which is referred to under the name $s_{i+1}$ in the following. Also, we write $(\covset, \boss_i, \clique_i) := \aconfig_i$. Let $a_{\brsymb}$ the agent making the broadcast transition in $s_{i+1}$ and $A_{\recsymb}$ the set of agents receiving this broadcast in $s_{i+1}$. 

We now make the following case distinction to decide of the type of the abstract step $\aconfig_i \step{} \aconfig_{i+1}$:
\begin{enumerate}
\item\label{proof_soundess:case_broadcast_boss} if $\data{\aconfig_0}(a_{\brsymb}) = \data{\aconfig_{i}}(a_{\brsymb}) = \aval$ then it is a ``broadcast from boss'',
\item\label{proof_soundess:case_broadcast_clique} if $\data{\aconfig_{i}}(a_{\brsymb}) = \aval$ but $\data{\aconfig_{i}}(a_{\brsymb}) \ne \aval$ then it is a ``broadcast from clique'',
\item\label{proof_soundess:case_external_broadcast} otherwise it is an ``external broadcast''. 
\end{enumerate}

Case~\ref{proof_soundess:case_broadcast_boss}:  We claim that $\aconfig_i \step{} \aconfig_{i+1}$ is a valid ``broadcast from boss'' step in the abstract semantics.
$a_{\brsymb}$ may not change its register value in $s_{i+1}$ as it is broadcasting. Since $\data{\aconfig_0}(a_{\brsymb}) = \data{\aconfig_{i}}(a_{\brsymb}) = \data{\aconfig_{i+1}}(a_{\brsymb}) = \aval$, 
one has $\boss_i = \st{\config_i}(a_{\brsymb})$ and $\boss_{i+1} =  \st{\config_{i+1}}(a_{\brsymb})$. Write $(\boss_i, \br{\amessage}, \boss_{i+1})$ the transition taken by $a_{\brsymb}$ in $s_{i+1}$. 
One has $\clique_i \subseteq \clique_{i+1}$ as $\run_{i+1}$ is a prefix of $\run_i$. Moreover, let $q \in \clique_{i+1} \setminus \clique_i$. 
There exists $a \in \agents \setminus a_{\brsymb}$ such that $\st{\config_{i+1}}(a) = q$, $\st{\config_{i}}(a) \ne q$ and $\data_{\config_{i+1}}(a) = \aval$. \nico{to be continued}
\end{proof}

\section{Proofs of section \ref{sec:cover-1BNRA}}
\luin{Plan : 1) diseq test, 2) Completeness, 3) Soundness, 4) conclusion (abstraction sound and complete, NP hardness, NP-complete)}

\label{app:cover-one-reg}
\subsection{Removing disequality tests}
\begin{corollary}
	\label{cor:removing_diseq_tests}
	Let $(\prot, q_f)$ an instance of the "coverability problem". This instance is positive if and only if $(\tilde{\prot}, q_f)$ is positive, where $\tilde{\prot}$ is equal to $\prot$ where every disequality test $\quotemarks{\diseqtestact}$ is replaced by dummy action $\quotemarks{\dummyact}$.  
\end{corollary}

\ifproofs
\begin{proof}
	First, if $(\prot, q_f)$ is positive then so is $(\tilde{\prot}, q_f)$, as one can easily lift any "run" in $\prot$ to an equivalent "run" in $\tilde{\prot}$ (transitions are less guarded  in $\tilde{\prot}$ that in $\prot$). 
	
	Suppose now that $(\tilde{\prot}, q_f)$ is a positive instance of the "coverability problem". There exists a "run" $\tilde{\run}: \tilde{\config}_0 \step{*} \tilde{\config}$ in $\tilde{\prot}$ that covers $q_f$. We prove by induction on the length of $\tilde{\run}$ that there exists a "run" $\run$ reaching a configuration $\config$ such that $\tilde{\config} \lessthan \config$ (Remark~\ref{rem:bigger_config_query} then allows us to conclude). 
	
	If $\config = \config_0$ then $\run = \tilde{\run}$ suffices. Suppose that $\tilde{\run}$ has length $k \geq 1$, and that the result if true for "runs" of length $k-1$. Decompose $\tilde{\run}$ into $\tilde{\run_{k-1}}: \tilde{\config_0} \step{*} \tilde{\config_{k-1}}$ of length $k-1$ and a final step $\tilde{\config_{k-1}} \step{} \tilde{\config_k}$. 
	By induction hypothesis, there exists $\run_{k-1}: \config_0 \step{*} \config_{k-1}$ such that $\tilde{\config_{k-1}} \lessthan \config_{k-1}$: there exists an injective function $\pi : \tilde{\agents} \rightarrow \agents$
	such that, for all $a \in \tilde{\agents}$, $\tilde{\config_{k-1}}(a) = \config_{k-1}(\pi(a))$, where $\tilde{\agents} := \agentsof{\tilde{\run}}$ and $\agents := \agentsof{\run}$. If $\tilde{\config_{k-1}} \step{} \tilde{\config_k}$ involves no reception transition from $\tilde{\prot}$ whose corresponding transition in $\prot$ has action $\quotemarks{\diseqtestact}$, then we directly lift this step into a step appended at the end of $\run_{k-1}$ (making $\pi(a)$ take a transition whenever $a$ does so in $\tilde{\config_{k-1}} \step{} \tilde{\config_k}$). Otherwise, write $\tilde{\agents}_{\diseqtestact}$ the subset of $\tilde{\agents}$ corresponding to agents taking in $\tilde{\config_{k-1}} \step{} \tilde{\config_k}$ a reception transition from $\tilde{\prot}$ whose corresponding transition in $\prot$ has action $\quotemarks{\diseqtestact}$ . Write $(q, \brone{m}, q') \in \transitions$ the broadcast transition used in this step.  Using the copycat principle, we add to $\config_{k-1}$ a fresh agent $a_{\mathsf{new}}$ with state $q$ and a register value that does not appear in $\config_{k-1}$. 
	We first mimic this broadcast step at the end of $\run_{k-1}$, making any agent $\pi(a) \in \pi(\tilde{\agents} \setminus \tilde{\agents}_{\diseqtestact})$ take the transition that $a$ takes in $\tilde{\config_{k-1}} \step{} \tilde{\config_k}$. We then add a new step where $a_{\mathsf{new}}$ broadcasts using transition $(q, \brone{m}, q')$, and every agent $\pi(a) \in \pi(\tilde{\agents}_{\diseqtestact})$ takes the transition corresponding to the transition taken by $a$ in $\tilde{\config_{k-1}} \step{} \tilde{\config_k}$. Such a transition is a reception with action $\quotemarks{\diseqtestact}$ in $\prot$; however, because $a_{\mathsf{new}}$ does not share its register value with any process from $\tilde{\agents}$, all disequality conditions are satisfied and this step is valid. In this end, every agent $\pi(a) \in \pi(\tilde{\agents})$ has taken the transition in $\prot$ corresponding to the one $a$ took in $\tilde{\prot}$ in step $\tilde{\config_{k-1}} \step{} \tilde{\config_k}$, hence the configuration $\config_k$ reached by the constructed run is such that $\tilde{\config_k} \lessthan \config_k$. 
\end{proof}
\fi

\subsection{Completeness}
This subsection is devoted to proving Lemma~\ref{lem:abstraction_complete}.

\begin{lemma}
	\label{lem:abstraction_complete}
	If $\run$ is a (concrete) "run" and $\aval \in \valsof{\run}$, then $\aconfiginit \step{*} \absproj{\aval}{\run}$. 
\end{lemma}


% The following lemma shall later be useful:
% \begin{lemma}
	% \label{lem:adding_states_in_covset}
	% If $\aconfig = (\covset, \gang), \aconfig' = (\covset', \gang') \in \allaconfigs$ are such that $\aconfig \step{*} \aconfig'$ then $\covset \subseteq \covset'$ and, for all $T \subseteq Q$, $(\covset \cup T, \gang) \step{*} (\covset' \cup T, \gang')$. 
	% \end{lemma}
% \begin{proof}
	% To prove the first statement, in suffices to observe that, in a step of the abstract semantics, the set $\covset$ may not decrease.
	% To prove the second statement, it suffices to notice in the abstract semantics that a larger $\covset$ may never hinder a step. 
	% \end{proof}

\begin{lemma}
	\label{lem:proof_completeness_covset_constant}
	For all "runs" $\run: \config_0 \step{*} \config$ and $\aval \in \valsof{\run}$, $(\statesin{\run}, q_0, \emptyset) \step{*} \absproj{\aval}{\run}$. 
\end{lemma}

\ifproofs
\begin{proof}
	Let $\covset := \statesin{\run}$ and $\agents = \agentsof{\run}$.
	
	Thanks to Remark~\ref{rem:run_no_new_register_values},  $\aval$ appears in $\config_0$; let $a_0$ be the (unique) agent such that $\data{\config_0}(a_0) = \aval$. We write $\run : \config_0 \step{} \config_1 \step{} \dots \step{} \config_k = \config$. For every $i \leq k$, let $\run_i : \config_0 \step{*} \config_i$ be the prefix of $\run$ of length $i$, and write $\aconfig^i := (\covset, \gangof{\aval}{\run_i})$. Note that $\gangof{\aval}{\run_0} = (q_0, \emptyset)$ hence $\aconfig^0 = (\covset, q_0, \emptyset)$. Also, we write $(\covset, \boss_i, \clique_i) := \aconfig^i$.
	
	We prove by induction on $i$ that $\aconfig^0 \step{*} \aconfig^i$.
	The statement is trivially true for $i =0$. 
	
	Suppose now that $(\covset, \emptyset, \noboss) \step{*} \aconfig^i$. 
	If suffices to prove that $\aconfig^i \step{} \aconfig^{i+1}$. First of all we clearly have, by definition, $K_{i+1} \subseteq S$ and $\boss_{i+1}\in S\cup\set{\noboss}$. We consider the last step of $\run_{i+1}$, which is referred to under the name $s_{i+1}$ in what follows; $s_{i+1}: \config_i \step{} \config_{i+1}$. Let $\agentbr$ the agent making the broadcast transition in $s_{i+1}$ and $A_{\recsymb}$ the set of agents receiving this broadcast in $s_{i+1}$. Let $(\statebr, \brone{\amessage}, \statebr') \in \transitions$ denote the transition taken by $\agentbr$ in $s_{i+1}$.
	
	We now make the following case distinction to determine the type of the abstract step $\aconfig^i \step{} \aconfig^{i+1}$:
	\begin{enumerate}
		\item\label{proof_completeness:case_broadcast_clique} if $\data{\config_{i}}(\agentbr) = \aval$ but there exists $j<i$ such that $\data{\config_{j}}(\agentbr) \ne \aval$ then it is a ``broadcast from clique'',
		\item\label{proof_completeness:case_broadcast_boss} if, for all $j \leq i$, $\data{\config_j}(\agentbr) = \aval$ then it is a ``broadcast from boss'',
		\item\label{proof_completeness:case_external_broadcast} otherwise it is an ``external broadcast''. 
	\end{enumerate}
	Note that $\agentbr$ may not change its register value in $s_{i+1}$ hence $\data{\config_i}(\agentbr) = \data{\config_{i+1}}(\agentbr)$. 
	
	Let $\agentboss$ the agent such that $\data{\config_0}(\agentboss) = \aval$. In case~\ref{proof_completeness:case_broadcast_boss}, $\agentboss = \agentbr$; in the other two cases, $\agentboss \ne \agentbr$. 
	
	We now prove the other conditions:
	\begin{enumerate}[i]
		\item In case~\ref{proof_completeness:case_broadcast_clique}, since $\agentbr$ has value $\aval$ in $\config_i$ and $\config_{i+1}$ but not in every $\config_j$ for $j \leq i$, we directly have $\statebr \in \clique_i$ and $\statebr' \in \clique_{i+1}$. In case~\ref{proof_completeness:case_broadcast_boss}, it suffices to note that $\boss_i = \st{\config}(\agentbr)$ and $\boss_{i+1} = \st{\config_{i+1}}(\agentbr)$. In case~\ref{proof_completeness:case_external_broadcast}, it suffices to note that $\statebr, \statebr' \in \covset$ as both states appear in $\run$.
		\item In case~\ref{proof_completeness:case_broadcast_boss}, this condition is automatically satisfied. In the other two cases, we look at what $\agentboss$ does in $s_{i+1}$. If it remains idle then we have $\boss_i = \boss_{i+1}$. Otherwise it takes a reception transition as $\agentbr \ne \agentboss$. 
		In case~\ref{proof_completeness:case_external_broadcast}, this reception may not have action $\quotemarks{\eqtestact}$ as the broadcast is from an agent with register value that is not $\aval$ ($\data{\config_i}(\agentbr) \ne \aval$ by hypothesis). For the same reason, if this reception has action $\quotemarks{\enregact}$ then $\boss_{i+1}= \noboss$. If this reception has action $\quotemarks{\dummyact}$ and $\boss_i \ne \noboss$ then $\boss_{i+1} \ne \noboss$ as $\agentboss$ keeps value $\aval$.
		\item We have $\clique_i \subseteq \clique_{i+1}$ because $\run_{i}$ is a prefix of $\run_{i+1}$; also, in case~\ref{proof_completeness:case_broadcast_clique}, $\statebr' \in \clique_{i+1}$ because of $\agentbr$. Let $q' \in \clique_{i+1} \setminus \clique_i$ with, in case~\ref{proof_completeness:case_broadcast_clique}, $q' \ne \statebr'$. There exist $q\in \covset$ and an agent $a$ that takes a reception transition $(q, \rec{\amessage}{\anact}, q')$ in $s_{i+1}$ and has value $\aval$ in $\config_{i+1}$. In cases~\ref{proof_completeness:case_broadcast_clique} and \ref{proof_completeness:case_broadcast_boss}, the broadcast has value $\aval$ hence if $\anact = \quotemarks{\eqtestact}$ or $\quotemarks{\dummyact}$ then $a$ has value $\aval$ in $\config_i$ and $q \in \clique_i$. In case~\ref{proof_completeness:case_external_broadcast}, the broadcast has value $\ne \aval$ hence $a$ may have value $\aval$ in $\config_{i+1}$ only when $\anact = \quotemarks{\dummyact}$ and $a$ had value $\aval$ in $\config_i$, which implies $q \in \clique_i$. 
		\item It suffices to note that the first components of $\aconfig^i$ and $\aconfig^{i+1}$ are equal to $\covset$ and $\statebr' \in \covset$, and $\boss_{i+1} \in \covset \cup \set{\bot}$. 
	\end{enumerate}
	
	Overall, we have proven that $\aconfig^i \step{} \aconfig^{i+1}$, which concludes the induction step. Appyling the result with $i = k$ proves Lemma~\ref{lem:proof_completeness_covset_constant}. 
	
\end{proof}
\fi

\ifproofs
We may now prove Lemma~\ref{lem:abstraction_complete}. 

\begin{proof}[Proof of Lemma~\ref{lem:abstraction_complete}]
	Let $\run$ a run.
	We proceed by induction on the size of the set $\statesin{\run}$. 
	First, if $\statesin{\run}$ is of size $1$ then $\statesin{\run} = \set{q_0}$. Applying Lemma~\ref{lem:proof_completeness_covset_constant} directly gives $\aconfiginit = (\set{q_0}, q_0, \noboss) \step{*} \absproj{\aval}{\run}$.
	
	
	Suppose now that the statement is true for any $\run$ such that $\statesin{\run}$ is of size $k$, and suppose that we have a run $\run$ such that $\statesin{\run}$ is of size $k+1$. Let $\aval \in \valsof{\run}$. Let $\run_p: \config_0 \step{*}\config_p$ the longest suffix of $\run$ such that $\statesin{\run} \ne \statesin{\run_p}$. By induction hypothesis, we know that for all $\aval \in \valsof{\run}$, $\aconfiginit \step{*} \absproj{\aval}{\run_p}$. Write $s$ the step immediatly after $\run_p$ in $\run$. By maximality of $\run_p$, $\run_p s$ covers all states in $\statesin{\run} \setminus \statesin{\run_p}$. 
	
	Write $\agentbr$ the agent broadcasting in $s$, $(\statebr, \brone{\amessage}, \statebr')$ the corresponding transition and $\avalbr$ the broadcast value. By induction hypothesis applied to $\run_p$ and $\avalbr$, $\aconfiginit \step{*} \absproj{\avalbr}{\run_p}$. Applying Lemma~\ref{lem:proof_completeness_covset_constant} on $\run$ and $\aval$ gives that $(\statesin{\run}, q_0, \emptyset) \step{*} \absproj{\aval}{\run}$. Therefore, it remains to prove that $\absproj{\aval}{\run_p} \step{*} (\statesin{\run}, q_0, \emptyset)$. It suffices to prove that there exists $\boss, \clique$ such that $\absproj{\avalbr}{\run_p} \step{} (\statesin{\run}, \boss,\clique)$, a "gang reset" allowing us to then reach $(\statesin{\run}, q_0, \emptyset)$. In other words, we prove that, from $\absproj{\avalbr}{\run_p}$, one may cover all states in $\statesin{\run}$ in just one abstract step.
	
	We aim at proving that $\absproj{\avalbr}{\run_p} \step{} (\statesin{\run}, \boss,\clique)$ for well-chosen $\boss$ and $\clique$.
	Just like in the proof of Lemma~\ref{lem:proof_completeness_covset_constant}, we make a case disjunction to prove that the broadcast in $s$ may be mimicked in the abstraction from $\absproj{\avalbr}{\run_p}$. The type obtained is either a "broadcast from clique" or a "broadcast from boss", because by hypothesis the agent broadcasting has value $\avalbr$ and therefore its state before the broadcast is either the boss or in the clique in $\absproj{\avalbr}{\run_p}$. 
	
	Because we may choose $\boss$ and $\clique$ freely, the only challenging condition is \ref{item:broadcast_from_clique_covset}.
	Let $q' \in \statesin{\run} \setminus \statesin{\run_p}$, $q' \ne \statebr'$. 
	There exists an agent $a$ that takes a reception transition $(q,\recone{m}{\anact},q')$ in $s$. 
	If $\anact = \quotemarks{\eqtestact}$, then agent $a$ has value $\avalbr$ at the end of $\run_p$ hence $q$ is in the clique of $\absproj{\avalbr}{\run_p}$ and condition \ref{item:broadcast_from_clique_covset} is satisfied. Otherwise, one has $q \in \statesin{\run_p}$ and condition \ref{item:broadcast_from_clique_covset} is satisfied.
	In the end, there exist $\boss$, $\clique$ such that $\absproj{\avalbr}{\run_p} \step{} (\statesin{\run}, \boss,\clique)$. By a "gang reset", this implies that $\absproj{\avalbr}{\run_p} \step{*} (\statesin{\run}, q_0, \emptyset)$; we have proven that $\aconfiginit \step{*}\absproj{\avalbr}{\run_p} \step{*} (\statesin{\run}, q_0, \emptyset) \step{*} \absproj{\aval}{\run}$ which concludes the proof. 
\end{proof}
\fi


\subsection{Soundness}

\begin{lemma}
	\label{cor:soundness}
	%	For all $\sigma_0 \in \aconfiginitset$ and $\sigma = (S, b, K) \in \Sigma$ such that $\sigma_0 \step{*} \sigma$ there exists a reachable configuration $\gamma$ 
	%	satisfying $K \cup \set{b}$ if $b \neq \noboss$ and $K$ otherwise.
	For all $\sigma_0 \in \aconfiginitset$ and $\sigma = (S, b, K) \in \Sigma$ such that $\sigma_0 \step{*} \sigma$, for all $s \in S$, there exists a reachable configuration $\gamma$ covering $s$.
	%	satisfying $K \cup \set{b}$ if $b \neq \noboss$ and $K$ otherwise.
\end{lemma}

\ifproofs
\begin{proof}

We in fact prove the stronger lemma first.

\begin{lemma}
	\label{lem:correctness-construction}
	
	
	Let $\sigma_0 \in \aconfiginitset$, and $\sigma_0 \to \sigma_1 \to \cdots \to \sigma_n$ an abstract run. For all $i$ let $(S_i, b_i, K_i) := \sigma_i$. Let $M = \size{\Delta}+1$.
	
	For all $i$, there exists a set of agents $\agents_i$, a configuration $\config_i$, a run $\run_i : \config_0 \step{*} \config_i$ over $\agents_i$, agents $a_0, \cdots, a_n \in \agents_i$ and values $v_0, \ldots, v_n \in \nats$ such that:
	\begin{itemize}
		\item for all $s \in S_i$, there are at least $M^{n-i}$ agents (different from $a_i$) in state $s$ 
		
		\item for all $s \in K_i$, there are at least $M^{n-i}$ agents (different from $a_i$) in state $s$ with value $v_i$
		
		\item if $b_i \neq \noboss$, then $a_i$ is in state $b_i$ with value $v_i$.
	\end{itemize}
\end{lemma}

\ifproofs
\begin{proof}
	
	We proceed by induction on $i$.
	We set $\agents_0 = \set{1, \ldots, M^n}$, and we set $\config_0(a) = (q_0, a)$ for all $a$. Clearly $\config_0$ satisfies the requirements with respect to $\sigma_0$, with $a_0 = v_0 \in \agents$.
	
	Now assume we constructed $\config_0 \step{*} \cdots \step{*} \config_{i}$ over $\agents_i$ satisfying the conditions of the lemma, we construct $\config_{i+1}$ using a case distinction on the form of the transition $\sigma_i \to \sigma_{i+1}$.
	For each $s \in S\setminus K$ we define $\agents_{i,s}$ as the set of agents in state $s$ in $\config_{i}$. We have $\size{\agents_{i,s}} \geq M^{n-i}$ thus we can extract $M = \size{\Delta}+1$ disjoint sets of agents $(\agents_{i,s}^d)_{d \in \Delta\cup\set{\epsilon}}$ from it, each set having $M^{n-i-1}$ agents.
	Similarly, for each $s \in K$ we define $\agents_{i,s}$ as the set of agents in state $s$ \textbf{with value $\mathbf{v_i}$} in $\config_{i}$. We have $\size{\agents_{i,s}} \geq M^{n-i}$ thus we can extract $\size{\Delta}+1$ disjoint sets of agents $(\agents_{i,s}^d)_{d \in \Delta\cup\set{\epsilon}}$ from it each set having $M^{n-i-1}$ agents.
	\\
	
	\textbf{Case 1: } If $\sigma_i \to \sigma_{i+1}$ is a \emph{broadcast $d = (q, \brone{m}, q')$ from the clique} with $q \in K_i$, then we make all agents $a \in \agents_{i,q}^{d}$ (which all have value $v_i$) execute that transition one by one.
	None of those broadcasts are received by any other agent, except for the last one:
	If $b \neq b'$ then there is a transition $(b, \recone{m}{\alpha}, b')$ and we make $a_i$ execute it upon receiving the broadcast. We then set $a_{i+1} = a_i$.
	For all $k' \in K_{i+1} \setminus K_i$ there exists a transition $d'=(k, \recone{m}{\alpha}, k')$ such that either $\alpha$ is $\eqtestact$ or $*$ and $k \in K_i$ or $\alpha$ is $\enregact$ and $k\in S$.
	In both cases we make all agents of $\agents_{i,k}^{d'}$ take that transition.
	
	For all $s' \in S_{i+1} \setminus (S_i \cup K_{i+1})$ there exists a transition $d'=(s,\recone{m}{*},s')$ (the operation cannot be $\enregact$ or $\eqtestact$ as otherwise $s$ would be in $K_{i+1}$). We then make all agents of $\agents_{i,s}^{d'}$ follow that transition. 
	
	We set $v_{i+1} = v_i$.
	\\
	
	\textbf{Case 2: }If $\sigma_i \to \sigma_{i+1}$ is a \emph{broadcast $d = (b_i, \brone{m}, b_{i+1})$ from the boss}, then we make $a_i$ (which has value $v_i$) execute that transition, and we set $a_{i+1} = a_i$.
	The agents receiving that message are as follows:
	
	For all $k' \in K_{i+1} \setminus K_i $ there exists a transition $d'=(k, \recone{m}{\alpha}, k')$ such that $\alpha$ is either $\eqtestact$ or $*$ and $k \in K_i$ or $\alpha$ is $\enregact$ and $k\in S$.
	In both cases we make all agents of $\agents_k^{d'}$ take that transition.
	
	For all $s' \in S_{i+1} \setminus (S_i \cup K_{i+1} \cup \set{b_{i+1}})$ there exists a transition $d'=(s,\recone{m}{*},s')$ (the operation cannot be $\enregact$ or $\eqtestact$ as otherwise $s$ would be in $K_{i+1}$). We then make all agents of $\agents_s^{d'}$ follow that transition. 
	
	By definition of an "abstract run", we must have $b_i \in S_i$.
	Hence we can make all agents of $\agents_{i,s}^{d}$ execute $d$, with no agent receiving the corresponding broadcasts.
	
	We set $v_{i+1} = v_i$.
	\\
	
	\textbf{Case 3: } If $\sigma_i \to \sigma_{i+1}$ is an \emph{external broadcast} $d = (q, \brone{m}, q')$ , then we make all agents $a \in \agents_q^{d}$ execute that transition one by one. None of those broadcasts are received by any other agent, except for the last one:
	If $b_i \neq b_{i+1}$ then there is a transition $(b, \recone{m}{\alpha}, b'')$ and either $b_{i+1} = b'' \neq \noboss$ and $\alpha = *$ or $b_{i+1} = \noboss$ and $\alpha=\enregact$. In both cases we make $a_i$ execute that transition, and we set $a_{i+1} = a_i$.
	
	For all $k' \in K_{i+1} \setminus K_i$ there exists a transition $d'=(k, \recone{m}{*}, k')$ with $k \in K_i$. We make all agents of $\agents_k^{d'}$ take that transition.
	
	For all $s' \in S_{i+1} \setminus (S_i \cup K_{i+1})$ there exists a transition $d'=(s,\recone{m}{\alpha},s')$ with $\alpha \in \set{*, \enregact}$. We then make all agents of $\agents_s^{d'}$ follow that transition. 
	
	We set $v_{i+1} = v_i$.
	\\
	
	\textbf{Case 4: }  If $\sigma_i \to \sigma_{i+1}$ is a \emph{gang reset} then no agent moves and we select some $a_{i+1}$ in $\agents_{q_0}$ and set $v_{i+1}$ to be its value.
	\\
	%\paragraph{In all cases:} After applying the given transitions, we use the copycat property: we add to $\agents_i$ $M^{n-i-1}$ disjoint copies of itself to obtain $\agents_{i+1}$, and repeat the run constructed thus far over each copy separately. This is to ensure that there are $M^{n-i-1}$ agents in $b_{i+1}$ (if it is not $\noboss$) after this step.
	
	Throughout the case distinction we have ensured that:
	\begin{itemize}
		\item If $b_{i+1} \neq \noboss$ then $a_{i+1}$ is an agent of value $v_{i+1}$.
		
		\item For all $k \in K_{i}$, the agents of $\agents_{i,k}^\epsilon$ do not move between configurations $\config_{i}$ and $\config_{i+1}$, hence they have state $k$ and value $v_{i+1}$ in $\config_{i+1}$.
		
		\item If the step is not a gang reset, then $v_{i+1} = v_i$ and for all $k' \in K_{i+1} \setminus K_i$, there exists $d \in \Delta$ from some $k$ to $k'$ such that all agents of $\agents_{i,k}^d$ take that transition. Furthermore, if $d$ is of the form $(k,\recone{m}{\enregact},k')$ then the broadcasting process has value $v_i$, thus all those agents keep value $v_i = v_{i+1}$. 
		%	As $\size{\agents_{i,k}^d}\geq M^{n-i-1}$, there are at least that many agents with value $v_{i+1} = v_i$ in $k'$.
		
		\item For all $s \in S_{i}$, the agents of $\agents_{i,s}^\epsilon$ do not move between configurations $\config_{i}$ and $\config_{i+1}$, hence they have state $s$ in $\config_{i+1}$.
		
		\item If the step is not a gang reset, for all $s' \in S_{i+1} \setminus (S_i \cup \set{b_{i+1}})$, there exists $d \in \Delta$ from some $s \in S_i$ to $s'$ such that all agents of $\agents_{i,s}^d$ take that transition.
		
		\item If the step is a gang reset, the conditions of the lemma hold trivially.
	\end{itemize}
	
	As a result, we have ensured that the conditions of the lemma were respected.
	This concludes our induction.
\end{proof}
\fi

	We simply apply Lemma~\ref{lem:correctness-construction} to an "abstract run" $\sigma_0 \to \cdots \sigma_n = \sigma$ from $\sigma_0$ to $\sigma$ by setting $i = n$.
%	We obtain (by setting $i = n$) that there exists a value $v_n$ in the final configuration of the constructed run, for all $s \in K$, there is an agent with value $v_n$ in state $s$. Furthermore, if $b \neq \noboss$, then there is an agent in state $b$ with value $v_n$.  
\end{proof}
\fi

\subsection{Conclusion}


\begin{proposition}
	\label{prop:sound-and-complete}
	Let $q_f$ be a state, there exists a reachable "configuration" covering $q_f$ if and only if there exists a reachable "abstract configuration" $(S,b,K)$ with $q_f \in S$.
	%either $q_f \in K$ or $b \neq \noboss$ and $q_f = b$.  
\end{proposition}

\begin{proof}
	The right-to-left direction is given by Corollary~\ref{cor:soundness}.
	For the left-to-right direction, let $\run$ be a run ending in a configuration $\config$ covering $q_f$. Let $v \in \nats$ be such that $\config$ has some agents with value $v$ in state $q_f$.
	
	We construct a suitable "abstract run" as follows: by Lemma~\ref{lem:abstraction_complete} there exists an abstract run from some $\sigma_0 \in \aconfiginitset$ to an abstract configuration $\absproj{v}{\run} = (S, b, K')$ for some $S,b,K$ such that $q_f \in S$ and $q_f \in \set{b} \cup K$.
	%	
	%	As $\config, v$ covers $q_f$, either $q_f = b$ or $q_f \neq b$ and there exists an agent with state $q_f$ and value $v$ in $\config$. 	
	%	By definition of the abstraction $\absproj{v}{\run}$, we have $q_f \in K$ if $b=\noboss$ and $q_f = b$ otherwise, proving the proposition.
	
	%	For the left-to-right direction, let $\run$ be a run ending in a configuration $\config$ satisfying $K$. Let $v \in \nats$ be such that $\config$ has some agents with value $v$ in every state of $K$.
	%	
	%	We construct a suitable "abstract run" as follows: by Lemma~\ref{lem:abstraction_complete} there exists an abstract run from some $\sigma_0 \in \aconfiginitset$ to an abstract configuration $\absproj{v}{\run} = (S, b, K')$.
	%	
	%	As $\config, v$ satisfies $K$, for all $s \in K$ either $s = b$ or $s \neq b$ and there exists an agent with state $s$ and value $v$ in $\config$. 	
	%	By definition of the abstraction $\absproj{v}{\run}$, we have $K \subseteq K'$ if $b=\noboss$ and $K \subseteq K' \cup \set{b}$ otherwise, proving the proposition.
\end{proof}




\textbf{NP-hardness.} We present here a reduction of the 3SAT problem to the "cover problem" in 1-BNRAs.
%\label{sec:lower-bounds}

%\ifproofs
%\begin{proof}
\begin{figure}[h]
	\begin{tikzpicture}[xscale=0.5,AUT style,node distance=2cm,auto,>= triangle
	45]
	\tikzstyle{initial}= [initial by arrow,initial text=,initial
	distance=.7cm]
	%	\tikzstyle{accepting}= [accepting by arrow,accepting text=,accepting
	%	distance=.7cm,accepting where =right]
	
	\node[state,initial, minimum width=0.1pt] (0) at (0,0) {0};
	
	\node[state] [right of=0] (1) {1};
	
	\node [right of=1, xshift=-20pt] (token1) {};
	\node [right of=token1, xshift=-50pt] (dots) {\huge $\cdots$};
	\node [right of=dots, xshift=-50pt] (token2) {};
	
	
	\node[state] [right of=token2, xshift=-20pt] (N-1) {N-1};
	\node[state] [right of=N-1] (N) {N};
	
	\node[state] [right of=N] (1') {1'};
	
	\node [right of=1', xshift=-20pt] (token3) {};
	\node [right of=token3, xshift=-50pt] (dots2) {\huge $\cdots$};
	\node [right of=dots2, xshift=-20pt] (token4) {};
	
	
	\node[state] [right of=token4, xshift=-40pt] (m-1) {m-1'};
	\node[state] [right of=m-1] (m) {m'};
	
	\coordinate[below of=1] (stop);
	\node[state] [below right of=stop] (r1) {$x_1$};
	\node [above left of=r1, yshift=-20pt, xshift=10pt] (r1') {$\rec(x_1, \enreg)$};
	\node[state] [right of=r1] (r1b) {$\neg x_1$};
	\node [above left of=r1b, yshift=-20pt, xshift=10pt] (r1b') {$\rec(\neg x_1, \enreg)$};
	\node [right of= r1b] (dots3) {\huge $\cdots$};
	\node[state] [right of=dots3] (rn) {$x_n$};
	\node [above left of=rn, yshift=-20pt, xshift=10pt] (rn') {$\rec(\neg x_n, \enreg)$};
	\node[state] [right of=rn] (rnb) {$\neg x_n$};
	\node [above left of=rnb, yshift=-20pt, xshift=10pt] (rnb') {$\rec(\neg x_n, \enreg)$};
	
	\draw (0) .. controls +(0,-2) and +(-1,0) .. (stop);
	\draw[->] (stop) .. controls +(2,0) and +(0,1) .. (r1);
	\draw[->] (stop) .. controls +(6,0) and +(0,1) .. (r1b);
	\draw[->] (stop) .. controls +(14,0) and +(0,1) .. (rn);
	\draw[->] (stop) .. controls +(18,0) and +(0,1) .. (rnb);
	\path[->, bend left=20]
	(0) edge node[above] {$\br(x_1)$} (1)
	(N-1) edge node[above] {$\br(x_N)$} (N) 	
	;
	\path[->, bend left=40]
	(N) edge node[above] {$\rec(\ell_1^1, \testeq)$} (1')
	(m-1) edge node[above] {$\rec(\ell_m^1, \testeq)$} (m) 	
	;
	\path[->, bend right=20] 
	(0) edge node[below] {$\br(\neg x_1)$} (1)
	(N-1) edge node[below] {$\br(\neg x_N)$} (N) 
	;
	\path[->, bend right=30] 
	(N) edge node[below] {$\rec(\ell_1^3, \testeq)$} (1')
	(m-1) edge node[below] {$\rec(\ell_m^3, \testeq)$} (m)	
	;
	\path[->]
	(N) edge node[above] {$\rec(\ell_1^2, \testeq)$} (1')
	(m-1) edge node[above] {$\rec(\ell_m^2, \testeq)$} (m) 	
	;
	\path[->, loop below]
	(r1) edge node[below] {$\br(x_1)$} (r1)
	(r1b) edge node[below] {$\br(\neg x_1)$} (r1b) 	
	(rn) edge node[below] {$\br(x_n)$} (rn)
	(rnb) edge node[below] {$\br(\neg x_n)$} (rnb) 	
	;
\end{tikzpicture}
	\caption{The "protocol" used for the NP-hardness proof.}
	\label{fig:np-hard}
\end{figure}

%We present here a reduction of the 3SAT problem to our problem.
Let $x_1, \ldots, x_n$ be variables and $\query = \bigwedge_{j=1}^m C_j$ with, for all $j$, $C_j = \ell_j^1 \lor \ell_j^2 \lor \ell_j^3$ and $\ell_j^1, \ell_j^2, \ell_j^3 \in \set{x_i, \neg x_i \mid 1 \leq i \leq n}$. 

Consider the "protocol" displayed in Figure~\ref{fig:np-hard}.
Our alphabet of messages is the set of literals $\set{x_i, \neg x_i \mid 1 \leq i \leq n}$.
Each agent may either receive a message, and repeat it forever or it may broadcast one of $x_i, \neg x_i$ for each $i$ and then try to receive a message one of $\ell_j^1, \ell_j^2, \ell_j^3$ for each $j$, with its own register value.

Suppose $\query$ is satisfiable, let $\nu$ be a satisfying assignment, then we set $\agents = \set{a, a_1, \ldots, a_n}$ as our set of agents. First for each $i$ we make $a$ broadcast $x_i$ if $\nu(x_i)= \top$ and $\neg x_i$ otherwise, this broadcast shall be received only by agent $a_i$. Agent $a_i$ will store $a$'s register value.
Then for each $j$ we select some $\ell_j^p$ satisfied by $\nu$. There exists $i$ such that $a_i$ is in state $\ell_j^p$. It broadcasts $\ell_j^p$ along with the initial register value of $a$, allowing $a$ to go to the next state.

As a result, there is an execution in which agent $a$ reaches $m'$.

Now suppose there is an execution $\run$ over some set of agents $\agents$ such that some agent $a \in \agents$ is in state $m'$ in the final configuration.
For each $i$, $a$ has broadcast either $x_i$ or $\neg x_i$, but not both.
Let $\nu$ be the valuation assigning $\top$ to $x_i$ if and only if $a$ has broadcast it.
The register of $a$ cannot have changed its value throughout the run. 
For each $j$ it has received one of $\ell_j^1, \ell_j^2, \ell_j^3$ along with its own initial register value (which we call $r$). Let $p_j$ be such that $a$ has received $\ell_j^{p_j}$.
Hence for all $j$ there exists an agent $a_j \in \agents$ such that at some point in the run the register value of $a_j$ is $r$ and $a_j$ broadcasts $\ell_j^{p_j}$.
This agent $a_j$ must be in state $\ell_j^{p_j}$ after receiving the broadcast $\brone{\ell_j^{p_j}}$ from $a$ (as all agents start with different register values).
Hence $\ell_j^{p_j}$ is satisfied by $\nu$. 

As a result, $\nu$ satisfies a literal of each clause of $\query$, and thus satisfies $\query$. This concludes our reduction.

%\end{proof}
%\fi

\begin{proposition}
	\label{prop:np-hard-query-cover}
	The "cover problem" is NP-hard.
\end{proposition}


\begin{proof}
	The lower bound is given by Proposition~\ref{prop:np-hard-query-cover}.
	For the upper bound, say we are given a "protocol" $\prot = (Q, \messages, \Delta, q_0, \regnum)$ and a state $q_f$. We have to verify that there exists a reachable configuration $\config_f$ convering $q_f$. By Proposition~\ref{prop:sound-and-complete}, it is the case if and only if there is an "abstract run" to an "abstract configuration" $(S,b, K)$ with $q_f \in S$.
	Furthermore, by Lemma~\ref{lem:short-run} if there is such an "abstract run" then there is one with at most $(\size{Q}+2)^3$ steps. 
	Thus we can simply guess such an abstract run and verify it in polynomial time.
	As a result, the "cover problem" is in \NP. 
\end{proof}
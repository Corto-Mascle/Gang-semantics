\section{Proofs of Section~\ref{sec:decidability-shortening-branches}}

\lemIncreasingBosses*

\begin{proof}
	Let $\tree_{\node}$, $\tree_{\node'}$ be the subtrees rooted in $\node$, $\node'$ respectively. 
	Let $\tree'$ be the tree obtained by replacing $\tree_{\node}$ with $\tree_{\node'}$. The size of $\tree'$ is smaller than the one of $\tree$, as $\tree_{\node'}$ is a strict subtree of $\tree_{\node}$.
	
	If $\node$ is the root of $\tree$, then $\tree'$ is a "tree unfolding" with $\node'$ as root. As $\node$ is a "boss node" and $\tree$ satisfies $\spec$, $\spec$ is a "boss specification", and a subword of $\bosslabel{\node}$.
	As $\bosslabel{\node}$ is a subword of $\bosslabel{\node'}$, $\spec$ is a subword of $\bosslabel{\node'}$ and thus $\tree'$ satisfies $\spec$.
	
	If $\node$ is not the root of $\tree$ then let $\node''$ be its father. We have to check that $\tree'$ is a "tree unfolding". 
	All nodes other than $\node''$ have the same label and children as before, thus the conditions of "tree unfoldings" are still respected for them.
	As for $\node''$, it has the same "follower" children, hence conditions \ref{unfoldingC1} and \ref{unfoldingC3.1} are respected. Condition \ref{unfoldingC2} only depends on its label, which hasn't changed.
	
	Finally, let $v'$ be a value that is not initial in $\localrunlabel{\node''}$. Either $\node''$ has a "boss" child other than $\node'$ such that $\vinput{v'}{\localrunlabel{\node''}}$ is a subword of its label, or $\vinput{v'}{\localrunlabel{\node''}}$ is a subword of $\bosslabel{\node}$, which is a subword of $\bosslabel{\node'}$, hence condition \ref{unfoldingC3.2} is satisfied. 
	
	As a result, in both cases $\tree'$ is a "tree unfolding" smaller than $\tree$ that satisfies $\spec$. 
\end{proof}

\lemIncreasingFollowers*

\begin{proof}
	Let $\tree_{\node}$, $\tree_{\node'}$ be the subtrees rooted in $\node$, $\node'$ respectively. 
	Let $\tree'$ be the tree obtained by replacing $\tree_{\node}$ with $\tree_{\node'}$. The size of $\tree'$ is smaller than the one of $\tree$, as $\tree_{\node'}$ is a strict subtree of $\tree_{\node}$.
	
	If $\node$ is the root of $\tree$, then $\tree'$ is a "tree unfolding" with $\node'$ as root. As $\node$ is a "follower node" and $\tree$ satisfies $\spec$, $\spec$ is a "follower specification". Let $(\followwordspec, \followmessagespec) = \spec$, we have $\followlabelword{\node} \subword \followwordspec$ and $\followlabelmessage{\node} = \followmessagespec$. 
	Hence we have $\followlabelword{\node'} \subword \followlabelword{\node} \subword \followwordspec $ and $\followlabelmessage{\node'} = \followlabelmessage{\node} = \followmessagespec$, thus $\tree'$ satisfies $\spec$.
	
	If  $\node$ is not the root of $\tree$ then let $\node''$ be the father of $\node$. We have to check that $\tree'$ is a "tree unfolding". 
	All nodes other than $\node''$ have the same children as before, thus the conditions of "tree unfoldings" are still respected for them.
	As for $\node''$, it has the same "boss" children, hence condition \ref{unfoldingC3.2} is respected. Condition \ref{unfoldingC2} only depends on its label, which hasn't changed.
	
	For conditions \ref{unfoldingC1} and \ref{unfoldingC3.1}, we can check that they are still satisfied by observing that since $\followlabelword{\node'} \subword \followlabelword{\node}$, for all "decomposition" $\decsymb$, if $\followlabelword{\node} \in \langdecdown{\decsymb}$ then $\followlabelword{\node'} \in \langdecdown{\decsymb}$. One can see that both conditions are satisfied by using the same "decompositions" and applying this fact.
	
	As a result, in both cases $\tree'$ is a "tree unfolding" smaller than $\tree$ that satisfies $\spec$. 
\end{proof}


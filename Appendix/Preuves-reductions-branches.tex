\section{Proofs of Lemma~\ref{lem:shortening-branches}}
\label{app:proofs-reduction-branches}

\lemShorteningBranches*

\begin{proof}
	Let $\tree_{\node}$, $\tree_{\node'}$ be the subtrees rooted in $\node$, $\node'$ respectively. 
	Let $\tree'$ be the tree obtained by replacing $\tree_{\node}$ with $\tree_{\node'}$. The size of $\tree'$ is smaller than the one of $\tree$, as $\tree_{\node'}$ is a strict subtree of $\tree_{\node}$.
	
	If $\node$ is the root of $\tree$, then $\tree'$ is a "unfolding tree" because it is the subtree of $\tree$ rooted in $\node'$. It satisfies "specification" $\speclabel{\node'}$ hence also $\speclabel{\node}$. 
	
	If $\node$ is not the root of $\tree$ then let $\node''$ be its father. We have to check that $\tree'$ is a "unfolding tree"; all node satisfy automatically satisfy the condition except $\node''$. The only condition that may no longer be satisfied at  

	Since $\tree$ is an "unfolding tree", for every non-initial value in $\localrunlabel{\node''}$, $\vinput{v'}{\localrunlabel{\node''}}$ was the subword of the "boss specification" of a child of $\node''$ in $\tree$; the only problematic case is when that child was $\node$. In that case, since $\vinput{v'}{\localrunlabel{\node''}}$ is a subword of $\bosslabel{\node}$, it is also a subword of $\bosslabel{\node'}$ and condition \ref{item:condition1_non_initial_value} is satisfied.
	Moreover, since $\node''$ has the same "follower" children in $\tree$ and $\tree'$ condition \ref{item:condition2_initial_value} and \ref{item:condition4_boss_node} are satisfied. Condition \ref{item:condition3_follower_node} is automatically satisfied as it does not speak about the children of $\node''$. 
	As a result, in both cases $\tree'$ is a "unfolding tree" smaller than $\tree$ that satisfies $\spec$. 

	We now prove it for "follower specifications". 

	Let $\tree_{\node}$, $\tree_{\node'}$ be the subtrees rooted in $\node$, $\node'$ respectively. 
	Let $\tree'$ be the tree obtained by replacing $\tree_{\node}$ with $\tree_{\node'}$. The size of $\tree'$ is smaller than the one of $\tree$, as $\tree_{\node'}$ is a strict subtree of $\tree_{\node}$.
	
	If $\node$ is the root of $\tree$, then $\tree'$ is a "unfolding tree" with $\node'$ as root. As $\node$ is a "follower node" and $\tree$ satisfies $\spec$, $\spec$ is a "follower specification". Let $(\followwordspec, \followmessagespec) = \spec$, we have $\followlabelword{\node} \subword \followwordspec$ and $\followlabelmessage{\node} = \followmessagespec$. 
	Hence we have $\followlabelword{\node'} \subword \followlabelword{\node} \subword \followwordspec $ and $\followlabelmessage{\node'} = \followlabelmessage{\node} = \followmessagespec$, thus $\tree'$ satisfies $\spec$.
	
	If  $\node$ is not the root of $\tree$ then let $\node''$ be the father of $\node$. We have to check that $\tree'$ is a "unfolding tree". 
	All nodes other than $\node''$ have the same children as before, thus the conditions of "unfolding trees" are still respected for them.
	As for $\node''$, it has the same "boss" children, hence condition \ref{unfoldingC3.2} is respected. Condition \ref{unfoldingC2} only depends on its label, which hasn't changed.
	
	For conditions \ref{unfoldingC1} and \ref{unfoldingC3.1}, we can check that they are still satisfied by observing that since $\followlabelword{\node'} \subword \followlabelword{\node}$, for all "decomposition" $\decsymb$, if $\followlabelword{\node} \in \langdec{\decsymb}$ then $\followlabelword{\node'} \in \langdec{\decsymb}$. One can see that both conditions are satisfied by using the same "decompositions" and applying this fact.
	
	As a result, in both cases $\tree'$ is a "unfolding tree" smaller than $\tree$ that satisfies $\spec$. 
\end{proof}


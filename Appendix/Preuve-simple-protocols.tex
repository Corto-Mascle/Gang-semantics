\section{Proof of Lemma~\ref{lem:simple-reduction}}
\label{sec:proof-simple}

\SimpleReduction*

\ifproofs
\begin{proof}
	
	We construct $\prot'$ as follows: It has one more register than $\prot$, but stores in its state the current state of $\prot$ and a mapping $map : [1,r] \to [1,r+1]$ from registers of $\prot$ to its own registers.
	
	Intuitively, it is going to use its registers to store the different values that are in the registers of $\prot$, and the mapping to keep track of which registers contain which values.
	When a value is received it guesses whether it is one of the existing values or a fresh one, and updates the registers and the mapping accordingly.
	The extra register of $\prot'$ is not necessary, but simplifies the proof.
	We will first define a system with $\varepsilon$-transitions (transitions with no effect), which we will then eliminate. 
	
	
	Formally, we set $\prot = (Q, \messages, \Delta, q_0)$.
	The set of states of $\prot'$ is $Q \times ([1,r] \to [1,r])$, its initial state is $(q_0, id)$ where $id(i) = i$ for all $i \in [1,r]$. 
	
	There is a transition $(q, map) \xrightarrow{op} (q', map')$ in the following cases:
	
	\begin{itemize}
		\item $op = \br{m}{j}$ 
		$map = map'$ and $\prot$ has a transition $q \xrightarrow{\br{m}{i}} q'$ with $map(i)=j$.
		
		\item $op=\rec{m}{j}{\enregact}$ and $map^{-1}(j) =\emptyset$ and either
		\begin{itemize}
			\item there is a transition $q \xrightarrow{\rec{m}{i_0}{\enregact}} q'$ in $\prot$ such that for all $i \in [1,r]$, $map'(i) = map(i)$ if $i \neq i_0$ and $map'(i_0) = j$.
			
			\item or there is a transition $q \xrightarrow{\rec{m}{i_0}{\dummyact}} q'$ in $\prot$ and for all $i \in [1,r]$, $map'(i) = map(i)$.
		\end{itemize}
		
		\item $op = \rec{m}{j}{\eqtestact}$ and either
		\begin{itemize}
			\item there is a transition $q \xrightarrow{\rec{m}{i}{\eqtestact}} q'$ in $\prot$ such that $map(i)=j$ and $map'=map$.
			
			\item there is a transition $q \xrightarrow{\rec{m}{i_0}{\enregact}} q'$ in $\prot$ and $map'(i_0)=j$ and $map'(i) = map(i)$ for all $i \neq i_0$.
		\end{itemize} 
		
		\item $op = \rec{m}{j}{\diseqtestact}$ and there is a transition $q \xrightarrow{\rec{m}{i}{\diseqtestact}} q'$ in $\prot$ such that $map(i)=j$
		
		\item $op=\loc{j_1}{j_2}{\diseqtestact}$ and there is a transition $q \xrightarrow{\loc{i_1}{i_2}{\diseqtestact}} q'$ in $\prot$ such that $map(i_1)=j_1$ and $map(i_2 = j_2)$.
	\end{itemize}
	
	
	We define the relation $\xrightarrow{\varepsilon}$ as $(q, map) \xrightarrow{\varepsilon} (q', map')$ if and only if $map=map'$ and there is a transition $q \xrightarrow{\loc{i_1}{i_2}{\eqtestact}} q'$ in $\prot$ such that $map(i_1)=j_1$ and $map(i_2) = j_2$.
	Let $\xrightarrow{\varepsilon}^*$ be its transitive closure.
	
	For all transitions $(q,map) \xrightarrow{op} (q', map')$ we add transitions $(q,map) \xrightarrow{op} (q'', map'')$ for all $q'', map''$ such that $(q', map') \xrightarrow{\varepsilon}^* (q'', map'')$, to obtain $\prot'$.
	
	We prove by induction on $\size{u}$ that for all "local run" $u$ of $\prot$ there exists a "local run" $u'$ of $\prot'$ with the same "trace" and such that if the last configuration of $u$ is $(q,\nu)$ and the last configuration of $u'$ is $((q',map), \nu')$ then $(q', map') \xrightarrow{\varepsilon}^* (q, map')$, for all $i$, $\nu(i) = \nu'(map(i))$ and for all $i_1, i_2$, if $\nu(i_1)=\nu_{i_2}$ then $map(i_1) = map(i_2)$.  
	
	If $\size{u} = 0$, let $(q_0, \nu_0)$ be its initial configuration, then we simply set $u'$ as an empty "local run" with the initial configuration $((q_0, id), \nu'_0)$ where $\nu'_0(i) = \nu_0(i)$ for all $i \in [1,r]$ and $\nu_0(r+1)$ is a fresh value.
	
	If $\size{u}>0$, let $\Tilde{u}$ be $u$ without its last step. Let $(q, \nu)$ be the last "local configuration" of $u$, $(\Tilde{q}, \Tilde{\nu})$ the one of $\Tilde{u}$.
	
	By induction hypothesis there exists $\Tilde{u}'$ a local run of $\prot'$ with the same "trace" to a "local configuration" $((\Tilde{q}', \Tilde{map}'), \Tilde{\nu}')$ such that $(\Tilde{q}', \Tilde{map}') \xrightarrow{\varepsilon}^* (\Tilde{q}, \Tilde{map}')$ and for all $i \in [1,r]$, $\Tilde{\nu}'(\Tilde{map}'(i)) = \Tilde{\nu}(i)$.
	
	\begin{itemize}
		\item 
		If the last step of $u$ is $(\Tilde{q}, \Tilde{\nu}) \xrightarrow{\br{m}{i}} (q, \nu)$ then $\Tilde{\nu} = \nu$, and there is a transition $(\Tilde{q}, \Tilde{map}') \xrightarrow{\br{m}{j}} (q, \Tilde{map}')$ with $j=\Tilde{map}'(i)$.
		
		We add a step $((\Tilde{q}', \Tilde{map}'), \Tilde{\nu}') \xrightarrow{\br{m}{j}} ((q', map'), \nu')$ at the end of $\Tilde{u}'$ to obtain $u'$, where $q' = q$, $map' = \Tilde{map}'$ and $\nu' = \Tilde{\nu}'$. 
		This step can be taken as $(\Tilde{q}', \Tilde{map}') \xrightarrow{\varepsilon}^* (\Tilde{q}, \Tilde{map}')$ and $(\Tilde{q}, \Tilde{map}') \xrightarrow{\br{m}{j}} (q, \Tilde{map}')$.   
		
		The resulting $u'$ satisfies the conditions as $(q', map') = (q, map')$, $\Tilde{map'} = map'$, the final configurations of $u$ and $u'$ are the same as the ones of $\Tilde{u}$ and $\Tilde{u}'$, and $\trace{u'} = \trace{\Tilde{u}'} (m,\Tilde{\nu}'(\Tilde{map}'(i)),out) = \trace{\Tilde{u}} (m,\Tilde{\nu}(i),out) = \trace{u}$.
		
		\item 
		If the last step of $u$ is an "external message" $(\Tilde{q}, \Tilde{\nu}) \extbr{m, v} (q, \nu)$ then let $j\in [1,r+1]$ be such that $\Tilde{map}'^{-1}(j) = \emptyset$ (such a $j$ exists as $\size{map([1,r])} = r < r+1$). There is a transition $\Tilde{q} \xrightarrow{\rec{m}{i_0}{\alpha}} q$ in $\prot$ such that $\nu(i)=\Tilde{\nu}(i)$ for all $i\neq i_0$ and we are in one of the following cases:
		
		\begin{itemize}
			\item $\alpha=\enregact$, $\nu(i_0) = v$ and there exists $i_1 \neq i_0$ such that $\nu(i_1) = v$. Let $j_1=\Tilde{map}'(i_1)$
			There is a transition $(\Tilde{q}, \Tilde{map}') \xrightarrow{\rec{m}{j_1}{\eqtestact}} (q, map')$ in $\prot'$ with $map'(i) = \Tilde{map}'(i)$ for all $i\neq i_0$ and $map'(i_0) = j_1$.
			
			We add a step $((\Tilde{q}', \Tilde{map}'), \Tilde{\nu}') \extbr{m, v} ((q', map'), \nu')$ at the end of $\Tilde{u}'$ to obtain $u'$, where $q' = q$ and $\nu' = \Tilde{\nu}'$.
			
			This step can be taken as $(\Tilde{q}', \Tilde{map}') \xrightarrow{\varepsilon}^* (\Tilde{q}, \Tilde{map}')$ and $(\Tilde{q}', \Tilde{map}') \xrightarrow{\rec{m}{j_1}{\eqtestact}} (q', map')$. 
			
			We have $(q', map') = (q, map')$, $\nu'(map'(i_0)) = v = \nu(i_0)$ and $\nu'(map'(i)) = \Tilde{\nu}'(map'(i)) = \Tilde{\nu}(i) = \nu(i)$, and $\trace{u'} = \trace{\Tilde{u}'} (m,v,in) = \trace{\Tilde{u}} (m,v,in) = \trace{u}$.
			Furthermore for all $i', i'' \in [1,r]\setminus\set{i_0}$, if $\nu(i') = \nu(i'')$ then $map'(i') = \Tilde{map}'(i') = \Tilde{map}'(i'') = map'(i'')$, and for all $i' \in [1,r]$, if $\nu(i')=\nu(i_0)$ then $\nu(i')=\nu(i_1)$ and thus $map'(i') = map'(i_1) = map'(i_0)$.
			Thus the resulting $u'$ satisfies the conditions
			
			\item $\alpha=\enregact$, $\nu(i_0) = v$ and for all $i_1 \neq i_0$, $\nu(i_1) \neq v$. 
			There is a transition $(\Tilde{q}, \Tilde{map}') \xrightarrow{\rec{m}{j}{\enregact}} (q, map')$ in $\prot'$ with $map'(i) = \Tilde{map}'(i)$ for all $i\neq i_0$ and $map'(i_0) = j$.
			
			We add a step $((\Tilde{q}', \Tilde{map}'), \Tilde{\nu}') \extbr{m, v} ((q', map'), \nu')$ at the end of $\Tilde{u}'$ to obtain $u'$, where $q' = q$ and $\nu'(j') = \Tilde{\nu}'(j')$ for all $j' \neq j$ and $\nu'(j) = v$.
			
			This step can be taken as $(\Tilde{q}', \Tilde{map}') \xrightarrow{\varepsilon}^* (\Tilde{q}, \Tilde{map}')$ and $(\Tilde{q}', \Tilde{map}') \xrightarrow{\rec{m}{j}{\enregact}} (q', map')$. 
			
			The resulting $u'$ satisfies the conditions as $(q', map') = (q, map')$, we have $\nu'(map'(i_0)) = v = \nu(i_0)$ and $\nu'(map'(i)) = \Tilde{\nu}'(map'(i)) = \Tilde{\nu}(i) = \nu(i)$, and $\trace{u'} = \trace{\Tilde{u}'} (m,v,in) = \trace{\Tilde{u}} (m,v,in) = \trace{u}$.
			
			\item $\alpha = \eqtestact$, $\nu(i_0) = \Tilde{\nu}(i_0) = v$. Then there is a transition $(\Tilde{q}, \Tilde{map}') \xrightarrow{\rec{m}{j}{\eqtestact}} (q, map')$ in $\prot'$ with $map' = \Tilde{map}'$ and $j = \Tilde{map}'(i_0)$.
			
			We add a step $((\Tilde{q}', \Tilde{map}'), \Tilde{\nu}') \extbr{m, v} ((q', map'), \nu')$ at the end of $\Tilde{u}'$ to obtain $u'$, where $q' = q$ and $\nu' = \Tilde{\nu}'$.
			
			This step can be taken as $\Tilde{\nu}'(\Tilde{map}'(i)) = \Tilde{\nu}(i) = v$ and $(\Tilde{q}', \Tilde{map}') \xrightarrow{\varepsilon}^* (\Tilde{q}, \Tilde{map}')$ and $(\Tilde{q}', \Tilde{map}') \xrightarrow{\rec{m}{j}{\eqtestact}} (q', map')$.
			
			The resulting $u'$ satisfies the conditions as $(q', map') = (q, map')$, the final configurations of $u$ and $u'$ are the same as the ones of $\Tilde{u}$ and $\Tilde{u}'$, and $\trace{u'} = \trace{\Tilde{u}'} (m,v,in) = \trace{\Tilde{u}} (m,v,in) = \trace{u}$.
			
			\item $\alpha = \diseqtestact$ and $\nu(i_0) = \Tilde{\nu}(i_0) \neq v$. Then there is a transition $(\Tilde{q}, \Tilde{map}') \xrightarrow{\rec{m}{j}{\diseqtestact}} (q, map')$ in $\prot'$ with $map' = \Tilde{map}'$ and $j = \Tilde{map}'(i_0)$.
			
			We add a step $((\Tilde{q}', \Tilde{map}'), \Tilde{\nu}') \extbr{m, v} ((q', map'), \nu')$ at the end of $\Tilde{u}'$ to obtain $u'$, where $q' = q$ and $\nu' = \Tilde{\nu}'$.
			
			This step can be taken as $\Tilde{\nu}'(\Tilde{map}'(i)) = \Tilde{\nu}(i) \neq v$ and $(\Tilde{q}', \Tilde{map}') \xrightarrow{\varepsilon}^* (\Tilde{q}, \Tilde{map}')$ and $(\Tilde{q}', \Tilde{map}') \xrightarrow{\rec{m}{j}{\diseqtestact}} (q', map')$.
			
			The resulting $u'$ satisfies the conditions as $(q', map') = (q, map')$, the final configurations of $u$ and $u'$ are the same as the ones of $\Tilde{u}$ and $\Tilde{u}'$, and $\trace{u'} = \trace{\Tilde{u}'} (m,v,in) = \trace{\Tilde{u}} (m,v,in) = \trace{u}$.
			
			\item $\alpha = \dummyact$ and $\nu(i_0) = \Tilde{\nu}(i_0)$. Then there is a transition $(\Tilde{q}, \Tilde{map}') \xrightarrow{\rec{m}{j}{\enregact}} (q, map')$ in $\prot'$ with $map' = \Tilde{map}'$.
			
			We add a step $((\Tilde{q}', \Tilde{map}'), \Tilde{\nu}') \extbr{m, v} ((q', map'), \nu')$ at the end of $\Tilde{u}'$ to obtain $u'$, where $q' = q$ and $\nu' = \Tilde{\nu}'$.
			
			This step can be taken as $(\Tilde{q}', \Tilde{map}') \xrightarrow{\varepsilon}^* (\Tilde{q}, \Tilde{map}')$ and $(\Tilde{q}', \Tilde{map}') \xrightarrow{\rec{m}{j}{\enregact}} (q', map')$.
			
			The resulting $u'$ satisfies the conditions as $(q', map') = (q, map')$, the final configurations of $u$ and $u'$ are the same as the ones of $\Tilde{u}$ and $\Tilde{u}'$, and $\trace{u'} = \trace{\Tilde{u}'} (m,v,in) = \trace{\Tilde{u}} (m,v,in) = \trace{u}$. 
		\end{itemize}
		
		\item 
		If the last step of $u$ is $(\Tilde{q}, \Tilde{\nu}) \xrightarrow{\loc{i_1}{i_2}{\eqtestact}} (q, \nu)$ then $\Tilde{\nu} = \nu$, and $(\Tilde{q}, \Tilde{map}') \xrightarrow{\varepsilon} (q, \Tilde{map}')$.
		
		We set $u' = \Tilde{u}'$. The last configuration of $u'$ is $((\Tilde{q}', \Tilde{map}'), \Tilde{\nu}')$. It satisfies the conditions as $(\Tilde{q}', \Tilde{map}') \xrightarrow{\varepsilon}^* (\Tilde{q}, \Tilde{map}') \xrightarrow{\varepsilon} (q, \Tilde{map}')$ and the final configurations of $u$ and $u'$ are the same as the ones of $\Tilde{u}$ and $\Tilde{u}'$.
		
		\item 
		If the last step of $u$ is $(\Tilde{q}, \Tilde{\nu}) \xrightarrow{\loc{i_1}{i_2}{\diseqtestact}} (q, \nu)$ then $\Tilde{\nu} = \nu$, and there is a transition $(\Tilde{q}, \Tilde{map}') \xrightarrow{\loc{j_1}{j_2}{\diseqtestact}} (q, \Tilde{map}')$ with $j_1=\Tilde{map}'(i_1)$ and $j_2=\Tilde{map}'(i_2)$.
		
		We add a step $((\Tilde{q}', \Tilde{map}'), \Tilde{\nu}') \xrightarrow{\loc{j_1}{j_2}{\diseqtestact}} ((q', map'), \nu')$ at the end of $\Tilde{u}'$ to obtain $u'$, where $q' = q$, $map' = \Tilde{map}'$ and $\nu' = \Tilde{\nu}'$. 
		This step can be taken as $\Tilde{\nu}'(j_1) = \Tilde{\nu}(i_1) \neq \Tilde{\nu}(i_2) = \Tilde{\nu}'(j_2)$ and $(\Tilde{q}', \Tilde{map}') \xrightarrow{\varepsilon}^* (\Tilde{q}, \Tilde{map}')$ and $(\Tilde{q}, \Tilde{map}') \xrightarrow{\loc{j_1}{j_2}{\diseqtestact}} (q, \Tilde{map}')$.   
		
		The resulting $u'$ satisfies the conditions as $(q', map') = (q, map')$ and the "traces" and final configurations of $u$ and $u'$ are the same as the ones of $\Tilde{u}$ and $\Tilde{u}'$.
	\end{itemize}
	
	This concludes our first induction.
	
	We now prove by induction on $\size{u'}$ that for all "local run" $u'$ of $\prot'$ there exists a "local run" $u$ of $\prot$ with the same "trace" and such that if the last configuration of $u$ is $(q,\nu)$ and the last configuration of $u'$ is $((q',map), \nu')$ then $q=q'$ and for all $i$, $\nu(i) = \nu'(map(i))$.  
	
	If $\size{u'} = 0$, let $(q_0, \nu'_0)$ be its initial configuration, then we  set $u$ as an empty "local run" with the initial configuration $((q_0, id), \nu_0)$ where $\nu_0(i) = \nu'_0(i)$ for all $i \in [1,r]$.
	
	If $\size{u}>0$, let $\Tilde{u}'$ be $u'$ without its last step. Let $((q', map'), \nu')$ be the last "local configuration" of $u'$, $((\Tilde{q}', \Tilde{map}'), \Tilde{\nu}')$ the one of $\Tilde{u}'$.
	
	By induction hypothesis there exists $\Tilde{u}$ a local run of $\prot$ with the same "trace" as $u'$ to a "local configuration" $(\Tilde{q}, \Tilde{\nu})$ such that $\Tilde{q}'=\Tilde{q}$ and for all $i \in [1,r]$, $\Tilde{\nu}'(\Tilde{map}'(i)) = \Tilde{\nu}(i)$.
	
	\begin{itemize}
		\item If the last step of $u'$ is $((\Tilde{q}', \Tilde{map}'), \Tilde{\nu}') \xrightarrow{\br{m}{j}} ((q', map'), \nu')$, then $\Tilde{map}' = map'$ and there exists $q \in Q$ and $i \in [1,r]$ such that $map'(i)=j$ and $(\Tilde{q}, map') \xrightarrow{\varepsilon}^* (q, map')$ and $q \xrightarrow{\br{m}{i}} q'$ is a transition of $\prot$.
		
		As $(\Tilde{q}, map') \xrightarrow{\varepsilon}^* (q, map')$ and $map'=\Tilde{map}'$, there is a path in $\prot$ from $\Tilde{q}$ to $q$ consisting only of transitions labelled by operations of the form $\loc{i_1}{i_2}{\eqtestact}$ with $\Tilde{map}'(i_1) = \Tilde{map}'(i_2)$ and thus $\Tilde{\nu}(i_1) = \Tilde{\nu}(i_2)$.
		As a result, we can extend $\Tilde{u}$ into a "local run" to $(q, \Tilde{\nu})$ with the same  "trace" as $\Tilde{u}$.
		
		We then append a step $(q, \Tilde{\nu}) \xrightarrow{\br{m}{i}} (q', \nu)$ at the end of that local run to obtain $u$. As $j = map(i)$, we have $\Tilde{\nu}(i) = \Tilde{\nu}'(j)$ and thus $\trace{u} = \trace{\Tilde{u}}(m, \Tilde{\nu}(i), out) = \trace{\Tilde{u}'}(m, \Tilde{\nu}'(j), out) = \trace{u'}$.
		
		\item If the last step of $u'$ is $((\Tilde{q}', \Tilde{map}'), \Tilde{\nu}') \extbr{m, v} ((q', map'), \nu')$, then there exists a transition  $(\Tilde{q}', \Tilde{map}') \xrightarrow{\rec{m}{j_0}{\alpha}} (q', map')$ and $i_0 \in [1,r]$ such that $map'(i) = \Tilde{map}'(i)$ for all $i\neq i_0$ and we are in one of the following cases:
		
		\begin{itemize}
			\item $\alpha = \enregact$ and $\nu'(j_0)=v$. Then there exists $q \in Q$ and $i \in [1,r]$ such that $(\Tilde{q}, map') \xrightarrow{\varepsilon}^* (q, map')$ and $q \xrightarrow{\rec{m}{i_0}{\beta}} q'$ is a transition of $\prot$
			
			Like before, as $(\Tilde{q}, map') \xrightarrow{\varepsilon}^* (q, map')$ and $map'=\Tilde{map}'$, we can extend $\Tilde{u}$ into a "local run" to $(q, \Tilde{\nu})$ with the same  "trace" as $\Tilde{u}$.
			
			We have either
			\begin{itemize}
				\item $\beta=\enregact$ and $map'(i_0) = j_0$. Then we add a step $(q, \Tilde{\nu}) \extbr{m,v} (q', \nu)$ at the end of the constructed local run to obtain $u$, with $\nu(i_0)=v$ and $\nu(i) = \Tilde{\nu}(i)$ for all $i \neq i_0$.
				
				\item $\beta=\dummyact$ and $map' = \Tilde{map}'$. Then we add a step $(q, \Tilde{\nu}) \extbr{m,v} (q', \nu)$ at the end of the constructed local run to obtain $u$, with $\nu = \Tilde{\nu}$.
			\end{itemize}
			
			\item $\alpha = \eqtestact$ and $map'(i_0)=\Tilde{map}'(i_0)$ and $\nu'(j_0) = \Tilde{\nu}'(j_0) = v$. Then there exists $q \in Q$ such that $(\Tilde{q}, map') \xrightarrow{\varepsilon}^* (q, map')$ and $q \xrightarrow{\rec{m}{i_0}{\beta}} q'$ is a transition of $\prot$.
			
			Like before, as $(\Tilde{q}, map') \xrightarrow{\varepsilon}^* (q, map')$ and $map'=\Tilde{map}'$, we can extend $\Tilde{u}$ into a "local run" to $(q, \Tilde{\nu})$ with the same  "trace" as $\Tilde{u}$.
			
			We have either
			\begin{itemize}
				\item $\beta=\eqtestact$ and $map' = \Tilde{map}'$ and $\Tilde{\nu}(i_0) = v$. Then we add a step $(q, \Tilde{\nu}) \extbr{m,v} (q', \nu)$ at the end of the constructed local run to obtain $u$, with $\nu = \Tilde{\nu}$.
				
				\item $\beta=\enregact$. Then we add a step $(q, \Tilde{\nu}) \extbr{m,v} (q', \nu)$ at the end of the constructed local run to obtain $u$, with $\nu(i_0) = v$ and $\nu(i)=\Tilde{\nu}(i)$ for all $i \neq i_0$.
			\end{itemize}
			
			\item $\alpha = \diseqtestact$ and $map'(i_0)=\Tilde{map}'(i_0) \neq v$. Then there exists $q \in Q$ and $i_0 \in [1,r]$ such that $(\Tilde{q}, map') \xrightarrow{\varepsilon}^* (q, map')$ and $q \xrightarrow{\rec{m}{i_0}{\diseqtestact}} q'$ is a transition of $\prot$.
			
			Like before, as $(\Tilde{q}, map') \xrightarrow{\varepsilon}^* (q, map')$ and $map'=\Tilde{map}'$, we can extend $\Tilde{u}$ into a "local run" to $(q, \Tilde{\nu})$ with the same "trace" as $\Tilde{u}$.
			
			We add a step $(\Tilde{q}, \Tilde{\nu}) \extbr{m,v} (q, \nu)$ at the end to obtain $u$, with $\nu = \nu'$.
		\end{itemize}
		
		In all cases, the conditions are respected, as $\trace{u} = \trace{\Tilde{u}} (m,v, in) = \trace{\Tilde{u}'} (m,v, in) = \trace{u'}$.
		
		\item If the last step of $u'$ is $((\Tilde{q}', \Tilde{map}'), \Tilde{\nu}') \xrightarrow{\loc{j_1}{j_2}{\diseqtestact}} ((q', map'), \nu')$ then $map'=\Tilde{map}'$, $\nu' = \Tilde{\nu}'$, $\nu'(j_1) \neq \nu'(j_2)$ and there exists $q \in Q$ and $i_1, i_2 \in [1,r]$ such that $map'(i_1)=j_1$, $map'(i_2)=j_2$ and $(\Tilde{q}, map') \xrightarrow{\varepsilon}^* (q, map')$ and $q \xrightarrow{\loc{i_1}{i_2}{\diseqtestact}} q'$ is a transition of $\prot$.
		
		Like before, as $(\Tilde{q}, map') \xrightarrow{\varepsilon}^* (q, map')$ and $map'=\Tilde{map}'$, we can extend $\Tilde{u}$ into a "local run" to $(q, \Tilde{\nu})$ with the same "trace" as $\Tilde{u}$.
		
		
		We append a step $(q, \Tilde{\nu}) \xrightarrow{\loc{i_1}{i_2}{\diseqtestact}} (q', \nu)$ at the end of that local run to obtain $u$, with $\nu = \Tilde{\nu}$. We have $\Tilde{\nu}(i_1) = \Tilde{\nu}'(j_1) \neq \Tilde{\nu}'(j_2) = \Tilde{\nu}(i_2)$ and thus $\trace{u} = \trace{\Tilde{u}} = \trace{\Tilde{u}'} = \trace{u'}$.
	\end{itemize}
	
	We have shown that $\prot$ and $\prot'$ produce the same set of "traces" from their "local runs". As the "input" and "output" of a local run depend only on its "trace", we obtain the result by Lemma~\ref{lem:local-to-global}.
\end{proof}
\fi

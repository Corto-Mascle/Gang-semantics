\section{Proof of Proposition~\ref{prop:trees-sound-complete}}
\label{app:trees-sound-complete}

We here prove that "unfolding trees" are sound and complete for the \COVER problem for \BNRA{}s. 
We start with a few definitions. 

A "run" $\run$ satisfies a boss specification $\bossspec$ if there exists $\aval \in \nats$ such that $\bossspec$ is a subword of the sequence of messages sent with value $\aval$ in $\run$.
A "run" $\run$ satisfies a follower specification $(\followwordspec, \followmessagespec)$ if there exists a value $\aval$ and an agent $a$ such that $\aval$ is not an initial value of $a$, the $\aval$-input of $a$ in $\run$ is a subword of $\followwordspec$ and agent $a$ broadcasts $\followmessagespec$ with value $\aval$ at some point.

An "unfolding tree" satisfies a boss specification $\bossspec$ if its root $\node$ is a "boss node" and $\bossspec$ is a subword of its specification label $\bosslabel{\node}$.
An "unfolding tree" satisfies a follower specification $(\followwordspec, \followmessagespec)$ if its root $\node$ is a "follower node" such that $\followmessagespec=\followlabelmessage{\node}$ and  $\followlabelword{\node}$ is a subword of $\followwordspec$.

Moreover, given a run $\run: \config_0 \step{} \config_1 \step{} \cdot \step{}\config_n$ and a agent $a$, the "local run" of $a$ in $\run$ is the "local run" whose sequence of "local configurations" of $a$ in $\run$ and whose sequence of "local steps" are the local version for $a$ of the steps of $\run$. Formally:
\begin{itemize}
\item if $\config_i \step{} \config_{i+1}$ does not involve $a$, then both configurations are projected to a single 
\end{itemize}
% Also note that the interpretation of the "follower nodes" changes between the two directions: when building the tree a follower node labelled $(\followwordspec, \followmessagespec)$ means that there exists a global run in which some agent receives the sequence of messages $\followwordspec$ and broadcasts $\followmessagespec$, all with a value $\aval$ that is not one of its initial ones.

% When building the run a follower node means that there exists a "partial run" which receives a sequence of external messages which forms the word $fw$, with a value $v$ that it initially does not contain, and eventually broadcasts $m$ with that same value.

Note that one can modify the system so that the \COVER problem becomes the problem of the existence of a "run" satisfying a "boss specification"; to do so, it suffices to add a broadcast loop on $q_f$ that broadcasts a special message type. Therefore, it suffices to prove that "runs" of $\prot$ and "unfolding trees" satisfy the same "boss specifications". To do so, we prove the two following implications:

\begin{restatable}{lemma}{LemRunToTree}
	\label{lem:run-to-tree}
	If there exists a "run" $\run$ of $\prot$ satisfying some "specification" $\spec$ then there exists a finite "unfolding tree" $\tree$ over $\prot$ satisfying $\spec$.
\end{restatable}

\begin{restatable}{lemma}{LemTreeToRun}
	\label{lem:tree-to-run}
	If there exists an "unfolding tree" over $\prot$ satisfying a "boss specification" $bw \in \messages^*$ then there exists a finite "run" $\run$ of $\prot$ satisfying $bw$.
\end{restatable}
\section{Proof of Lemma~\ref{lem:run-to-tree}}

\LemRunToTree*

\begin{proof}
	
	We proceed by strong induction on the lexicographic order on $\nats \times \set{boss, follower}$ with the length of $\run$ as the first component and the type of specification as the second, ``boss'' being considered higher than ``follower''. 
	
	Let $\run$ be a run, $\spec \in \messages^* \cup \messages^* \times \messages$ a specification, assume the property is satisfied for all lower pairs of runs and specifications. We construct a root and attach subtrees to it so that conditions \ref{unfoldingC1} to \ref{unfoldingC3} are satisfied 
	
	We split our construction in four parts. The first part applies if $\spec$ is a "boss specification", the second one if $spec$ is a "follower specification". 
	Those parts construct a root labelled with a local run $u$ and a value $v$, as well as some children so that conditions \ref{unfoldingC1} and \ref{unfoldingC2} are satisfied.
	
	The third and fourth part add children so that \ref{unfoldingC3.1} and \ref{unfoldingC3.2} are satisfied, respectively.
	
	\textbf{Case 1: The specification is a "boss specification"}
	
	Assume $\spec$ is a "boss specification" $\spec = \bossspec \in \messages^*$.
	There exists a value $v$ such that $\bossspec$ is a subword of the sequence of messages broadcast with value $v$ through $\run$. Let $w$ be the latter sequence.
	If $\bossspec$ is empty then the tree decomposition with one node labelled with an empty local run, any value and $\bossspec$ satisfies $\bossspec$.	
	
	Otherwise let $a$ be the agent which has $v$ as an initial value, let $u$ be its local run in $\run$. We set $u$, $v$ and $w$ to be the labels of the root of the tree unfolding we are constructing.
	We decompose $w$ as $(w_0, m_1, w_1, \ldots, m_\ell, w_\ell)$, where $w_0\cdots w_\ell$ is the sequence of messages broadcast by $a$ with value $v$, and the $(m_j)_{1\leq j\leq \ell}$ are the elements of $\messages$, placed at the times at which they are first broadcast by an agents other than $a$ with value $v$. If there is no such broadcast of a message $m$, then it does not appear as an $m_j$. This forms a "decomposition" of $w'$.
	
	For all $1 \leq j \leq \ell$ let us write $\decsymb_j$ for the "decomposition" $\decsymb = (w_0, m_1, w_1, \ldots, m_{j-1}, w_{j-1})$. Let $\run_j$ be the prefix of $\run$ up until the first broadcast of $m_j$ with $v$ by some agent $a_j$ that does not have $v$ as an initial value, and let $\run'_j$ be $\run_j$ without that last step. By definition of $\decsymb$, the sequence of broadcasts with value $v$ in $\run'_j$ "decomposes as" $\decsymb_j$.
	In particular, the $v$-input $w'_j$ of $a_j$ before it broadcasts $m_j$ with $v$ must decompose as $\decsymb_j$.
	
	Hence $(w'_j, m_j)$ is a "follower specification" satisfied by $\run_j$, which has a length smaller or equal to the one of $\run$. By induction hypothesis (recall that we make our induction on the length of $\run$ and on the type of specification), there is a "tree unfolding" satisfying it.
	
	We put that "tree unfolding" as a child of our root, hence we satisfy \ref{unfoldingC1}. We satisfy \ref{unfoldingC2} as the root we constructed is not a "follower node".
	
	\textbf{Case 2: The specification is a "follower specification"} 
	
	Assume $\spec$ is a "follower specification" $\spec = (\followwordspec, \followmessagespec) \in \messages^* \times \messages$. 
	There exists a value $v$ and an agent $a$ with a local run $u$ whose $v$-input $w$ is a subword of $\followwordspec$ and whose output contains $\followmessagespec$. 
	
	We set our root to have as labels $u$, $v$ and $w$, thus satisfying \ref{unfoldingC2}. We satisfy \ref{unfoldingC1} as the root is not a "boss node".
	
	\textbf{In both cases}
	
	We have constructed an agent $a$ and a value $v$ and set our root to be labelled by the local run $u$ of $a$ and $v$, and added children to that root so that conditions \ref{unfoldingC1} and \ref{unfoldingC2} are satisfied. We will now add some more children to satisfy condition \ref{unfoldingC3}.
	
	We can assume that the last step of $\run$ is a broadcast of a message with value $v$, as otherwise we can erase its last step to get a run that still satisfies the specification, but has a smaller length than $\run$, and apply the induction hypothesis.
	
	Let $v' \neq v$ be a value broadcast or received in $u$. 
	
	\begin{itemize}
		\item If $v'$ is an initial value of $u$ then let $w'$ be the $v'$-output of $u$. Let $\decsymb' = (w'_0, m_1, w_1, \ldots, w'_k)$ be such that $w' = w'_0 \cdots w'_k$ and the $m_i$ mark the first time another agent sends each message of $\messages$ with value $v'$ in $\run$, if that happens.  
		
		For all $j \in \nset{1}{k}$, all receptions of $m_j$ with value $v'$ in $u$ must happen after another agent has managed to broadcast it, hence after broadcasts of $w'_0 \cdots w'_{j-1}$ with value $v'$ have happened in $u$, by definition of $\decsymb'$.
		
		Let $j \in \nset{1}{k}$, let $a_j$ be the first agent other than $a$ to broadcast $m_j$ with value $v'$, and let $\run_j$ be the prefix of $\run$ up until that broadcast. The $v'$-input $w''_j$ of $a_j$ in $\run_j$ is a subword of the sequence of messages broadcast with value $v'$ in $\run_j$, which decomposes as $\decsymb'_j = (w'_0, m_1, w_1, \ldots, w'_{j-1})$, hence $w''_j \in \langdecdown{\decsymb'_j}$.
		
		As the last step of $\run$ is a broadcast of $v$ and $v' \neq v$, all $\run_j$ are shorter than $\run$. Furthermore for all $j$, $\run_j$ satisfies the specification $(\decsymb_j, m_j)$.		
		
		By induction hypothesis, for all $j$ there exists a "tree unfolding" satisfying $(w''_j, m_j)$. We add all those trees as children of our root.
		
		\item If $v'$ is not an initial value of $u$ then let $w'$ be the $v'$-input of $u$. 
		As $\run$ ends with a broadcast of $v$ and $v' \neq v$, we can remove the last step of $\run$ to obtain a run $\run'$. As $w'$ is a subword of the sequence of messages broadcast in $\run$ with value $v'$, it is also the case for $\run'$.
		Hence $\run'$ satisfies $w'$ as a specification, and has smaller length than $\run$. By induction hypothesis, there is a "tree unfolding" satisfying $w'$. We add it as a child of our root.
	\end{itemize}
	
	We added some "tree unfoldings" as children of the root so that conditions \ref{unfoldingC1} to \ref{unfoldingC3} are satisfied. We obtain a "tree unfolding" satisfying the specification.
\end{proof}


	\section{Proof of Lemma~\ref{lem:tree-to-run}}

\begin{definition}
	We extend the notion of "local run" to describe the behaviour of a subset of agents in a "run".
	
	An ""internal test@global"" from $\config$ to $\config'$, denoted $\config \inttest \config'$, is defined when there is a "step" $\config \step{} \config'$ in which an agent does a "local test". 
	
	An ""internal message@global"" from $\config$ to $\config'$, denoted $\config \intmessage{m}{v} \config'$, is defined when there is a "step" $\config \step{} \config'$ in which an agent broadcasts a message $m$ with value $v$. 
	
	An ""external message@global"" from $\config$ to $\config'$, denoted $\config \extmessage{m}{v} \config'$ is defined when for all agents $a$, either there is a "local step" $\config(a) \extbr{m}{v} \config'(a)$ or $\config(a) = \config'(a)$.
	
	A ""partial run"" is a sequence of configurations $\config_0 \cdots \config_k$  such that for all $i \in [1, k]$, either $\config_{i-1} \inttest \config_{i}$ or $\config_{i-1} \intmessage{m, v} \config_{i}$ or $\config_{i-1} \extmessage{m, v} \config_{i}$ for some $m \in \messages$, $v\in \nats$. 
	
	We define the $v$-projection $\vproj{\aval}{\run}$ of $\run$, which is the word $(b_0, x_0) \cdots (b_\ell, x_\ell) \in (\messages \times \set{in, out})^*$ obtained from $\run$ by mapping every "external message@@global" $\config \extmessage{m}{v} \config'$ to $(m, in)$, every "internal message" $\config \intmessage{m}{v} \config'$ to $(m, out)$, and every "internal step" to $\varepsilon$.
	
	The ""input@partial"" $\vinput{v}{\run}$ (resp. ""output@partial"" $\voutput{v}{\run}$) of a "partial run" $\run$ is the sequence $m_0 \cdots m_k$ such that the projection on $\messages \times \set{in}$ (resp. $\messages\times \set{out}$) of $\vproj{v}{\run}$ is $(m_0, in) \cdots (m_k, in)$ (resp. $(m_0, out)\cdots(m_k, out)$).
	
	Note that a "local run" can be seen as a "partial run" with a single agent.
\end{definition}


\begin{lemma}
	\label{lem:boss-composition}
	Let $\run$ be a " partial run", $\run'$ a "run".
	If there exist $v, v' \in \nats$ such that $\vinput{v}{\run} \subword \voutput{v'}{\run'}$ and $v'$ is an initial value in $\run'$ but $v$ is not initial in $\run$, then there exists a "partial run" $\Tilde{\run}$ such that 
	\begin{itemize}
		\item $\vinput{v}{\Tilde{\run}} = \varepsilon$ 
		
		\item for all $v'' \neq v$, if $\vproj{v''}{\run} \neq \varepsilon$ then $\vproj{v''}{\Tilde{\run}} = \vproj{v''}{\run}$
		
		\item for all $v'' \neq v$, if $\vproj{v''}{\run} = \varepsilon$ then $\vinput{v''}{\Tilde{\run}} = \varepsilon$.
	\end{itemize}
\end{lemma}

\begin{proof}
	Let $\agents, \agents'$ be the sets of agents of $\run$ and $\run'$, respectively.
	We can rename agents so that those two sets are disjoint.
	
	Let $m_1 \cdots m_k = \vinput{v}{\run}$, $\run$ can be split into $\run = \config_0 \step{\run_0} \overline{\config}_0 \extbr{m_1,v} \config_{1} \step{\run_1} \cdots \extbr{m_k,v} \config_{k} \step{\run_k} \overline{\config}_k$ so that for all $j\in [0,k]$, $\vinput{v}{\run_j} = \varepsilon$.
	
	Similarly, as $m_1\cdots m_k \subword \voutput{v}{\run'}$, $\run'$ can be split into $\run' = \config'_0 \step{\run'_0} \overline{\config}'_0 \step{m_1, v} \config'_{1} \step{\run'_1} \cdots \step{m_k,v'} \config'_{k} \step{\run'_k} \overline{\config}'_k$.
	
	For all configurations $\config$ over $\agents$ and $\config'$ over $\agents'$, we write $\config \sqcup \config'$ for the configuration over $\agents \cup \agents'$ such that $\config \sqcup \config'(a)$ is $\config(a)$ if $a \in \agents$ and $\config'(a)$ if $a \in \agents'$.
	
	We apply a renaming to the values of $\run'$ so that $v'$ is mapped to $v$ and all other values are mapped to distinct values that do not appear in $\run$.
	Hence the only value appearing in both runs is $v$, and it is initial only in $\run'$. As a consequence, the configuration $\config_0 \sqcup \config'_0$ is an "initial configuration".
	
	We then construct the desired run $\Tilde{\run}$ over $\agents \cup \agents'$ by matching the "external broadcasts" in $\run$ with the broadcasts in $\run'$ and executing the rest of the run in parallel.
	
	We have $\Tilde{\run} = \config_0 \sqcup \config'_0 \step{\run_0} \overline{\config}_0 \sqcup \config'_0 \step{\run'_0} \overline{\config}_0 \sqcup \overline{\config}'_0 \step{m_1, v} \config_{1} \sqcup \config'_1 \step{\run_1} \cdots \step{m_k,v} \config_{k}  \sqcup \config'_{k} \step{\run_k}  \overline{\config}_k \sqcup \config'_k \step{\run'_k} \overline{\config}_k \sqcup \overline{\config}'_k$.
	
	We indeed have $\overline{\config}_{j-1} \sqcup \overline{\config}'_{j-1} \step{m_j, v} \config_{j} \sqcup \config'_j$ as we can make all agents in $\agents$ that receive an "external broadcast" with message $m_j$ and value $v$ receive the broadcast made in $\agents'$ instead.
	
	Hence we obtain a run with 
	\begin{itemize}
		\item $\vinput{v}{\Tilde{\run}} = \varepsilon$ 
		
		\item for all $v'' \neq v$, if $\vproj{v''}{\run} \neq \varepsilon$ then $v''$ does not appear in $\run'$ after the renaming hence $\vproj{v''}{\Tilde{\run}} = \vproj{v''}{\run}$
		
		\item for all $v'' \neq v$, if $\vproj{v''}{\run} = \varepsilon$ then $\vinput{v''}{\run} = \varepsilon$ and $\vinput{v''}{\run'} = \varepsilon$ as $\run'$ is a "run" and not a "partial run", hence $\vinput{v''}{\Tilde{\run}} = \varepsilon$.
	\end{itemize}
	This concludes our proof.
\end{proof}

%\begin{lemma}
%	Let $M \subseteq \messages$, $m \in M$, $w \in \messages^*$ and $w_{-m}$ its projection on $\messages\setminus \set{m}$, suppose there exists a run $\run$ and a value $v$ such that $\vinput{v}{\run} \in M^*$ and $w_{-m} \subword \voutput{v}{\run}$. 
%	
%	Suppose also that there exists a run $\run'$ and a value $v'$ such that the $\vinput{}$ 
%	
%	$\config_0, \ldots, \config_{\ell+1}$ configurations and $v$ a value such that for all $j \in [1,\ell+1]$ there exists a "partial run" $\run_j$ from $\config_{j-1}$ to $\config_{j}$ with $\voutput{v}{\run_j} = w_j$, $\vinput{v}{\run_j} \in \set{m_1,\ldots, m_{j-1}}^*$.
%	
%	Suppose that for all $j \in [1,\ell]$ there exists a "partial run" $\run'_j$ and a value $v_j$ such that $\vinput{v_j}{\run'_j}$ "decomposes as" $\decsymb_j = (w_0, m_1, \ldots, w_{j-1})$, $\voutput{v_j}{\run'_j}$ contains $m_j$ and $\vinput{v'}{\run'_j} = \varepsilon$ for all $v' \neq v_j$.
%	
%	Then there exists a "partial run" $\Tilde{\run}$ such that 
%	\begin{itemize}
%		\item $w \subword \voutput{v}{\Tilde{\run}}$, 
%		
%		\item $\vinput{v}{\run} = \varepsilon$,
%		
%		\item for all $v' \neq v$, if $\vproj{v'}{\run} \neq \varepsilon$ then $\vproj{v'}{\Tilde{\run}} = \vproj{v'}{\run}$
%		
%		\item for all $v' \neq v$, if $\vproj{v'}{\run} = \varepsilon$ then $\vinput{v'}{\Tilde{\run}} = \varepsilon$
%	\end{itemize}
%\end{lemma}

\begin{lemma}
	\label{lem:follower-composition}
	Let $\decsymb = (w_0, m_1, \ldots, w_\ell)$ be a "decomposition", $w \in \langdecdown{\decsymb}$, $\config_0, \ldots, \config_{\ell+1}$ configurations and $v$ a value such that for all $j \in [1,\ell+1]$ there exists a "partial run" $\run_j$ from $\config_{j-1}$ to $\config_{j}$ with $w_j \subword \voutput{v}{\run_j}$ and $\vinput{v}{\run_j} \in \set{m_1,\ldots, m_{j-1}}^*$.
	
	Suppose that for all $j \in [1,\ell]$ there exists a "partial run" $\run'_j$ and a value $v_j$ such that $\vinput{v_j}{\run'_j}$ $\vinput{v_j}{\run'_j} \in \langdecdown{\decsymb_j}$ where $\decsymb_j = (w_0, m_1, \ldots, w_{j-1})$, $\vinput{v'}{\run'_j} = \varepsilon$ for all $v' \neq v_j$ and the last step of $\run'_j$ is an internal message $\intmessage{m}{v_j}$.
	
	Then there exists a "partial run" $\Tilde{\run}$ such that 
	\begin{itemize}
		\item $w \subword \voutput{v}{\Tilde{\run}}$, 
		
		\item $\vinput{v}{\run} = \varepsilon$,
		
		\item for all $v' \neq v$, if $\vproj{v'}{\run} \neq \varepsilon$ then $\vproj{v'}{\Tilde{\run}} = \vproj{v'}{\run}$
		
		\item for all $v' \neq v$, if $\vproj{v'}{\run} = \varepsilon$ then $\vinput{v'}{\Tilde{\run}} = \varepsilon$
\end{itemize}
\end{lemma}


\begin{proof}
	Let $\decsymb = (w_0, m_1, \ldots, w_\ell)$ be a "decomposition", we prove the property for all $w \in \langdecdown{\decsymb}$ by strong induction on $(\size{\vinput{v}{\run}}_{m_\ell} + \size{w}_{m_\ell}, \ldots, \size{\vinput{v}{\run}}_{m_1}+ \size{w}_{m_1})$, with the lexicographic ordering, $\run$ being the run obtained by concatenating $\run_1, \ldots, \run_\ell$.
	
	\textbf{Case 1: $w \subword w_0\cdots w_\ell \subword \voutput{v}{\run}$}
	
	If $\vinput{v}{\run} = \epsilon$ then we can set $\Tilde{\run} = \run$ to obtain the result.
	
	If $\vinput{v}{\run} \neq \epsilon$, then let $j$ be such that $\vinput{v}{\run_j} \neq \epsilon$, then we can decompose $\run_j$ as $\run_j = \config_{j-1} \step{\run_j^-} \config_j^- \extmessage{m_{j'}}{v} \config_j^+ \step{\run_j^+} \config_j$ with $j'\leq j$. 
	
	There exists a "partial run" $\run'_{j'}$ and a value $v_{j'}$ such that $\vinput{v_{j'}}{\run'_{j'}} \in \langdecdown{\decsymb_{j'}}$,  $\vinput{v'}{\run'_{j'}} = \varepsilon$ for all $v' \neq v_j$  and the last step of $\run'_{j'}$ is an internal message $\intmessage{m}{v_{j'}}$.
	
	Hence $\run'_j$ can be decomposed as $\config_{{j'},0} \step{\run_{{j'},1}'} \config_{{j'},1} \cdots \step{\run_{{j'},\ell}'} \config_{{j'},\ell} \intmessage{m_{j'}}{v} \config'_{j'}$ with, for all $i \in [1, \ell]$, the projection of $\vinput{v_{j'}}{\run_{{j'},i}'}$ on ${\messages\setminus\set{m_1, \ldots, m_{{i}-1}}}$ is a subword of $w_j$.
	
	We then proceed in a similar way as in Lemma~\ref{lem:boss-composition}.
	We can rename agents so that $\run$ and $\run'_j$ are on disjoint sets of agents, and apply a renaming of values in $\run_j'$ so that $v_j$ is mapped to $v$ and all other values are mapped to values that do not appear in $\run$.
	We then construct a run by executing in parallel $\run_{j',i}'$ and $\run_i$ for all $i\in [1,\ell]$. As $w_j \subword \voutput{v}{\run_j}$, we can match the "internal messages" of $\run_j$ with some "external messages" of $\run'_{j',i}$ so that the remaining "external messages" with value $v$ in $\run'_{j',i}$ are all in $\set{m_1, \ldots, m'_{j'}}$
	
	Finally, we match the last step of $\run'_{j'}$ broadcasting $m_{j'}$ with value $v$ with the "external message" between $\run_{j}^-$ and $\run_{j}^+$ in $\run_j$.
	
	
\end{proof}

\LemTreeToRun*


	We prove the following statement by strong induction on the "tree unfolding". \cortoin{notation for t.u.}
	
	For all "tree unfolding" $\tau$, 
	
	\begin{itemize}
		\item if $\tau$ satisfies a "boss specification" $w \in \messages^*$, then there exists a run $\run$ satisfying $w$.
		
		\item if $\tau$ satisfies a "follower specification" $(\decsymb, m)$ then there exists a word $w \in \langdecdown{\decsymb}$, a "partial run" $\run$ and a value $v$ such that $\Input{\run} \in (\messages\times \set{v})^*$, $\vinput{v}{\run} \in \langdecdown{\decsymb}$, and $\voutput{v}{\run}$ contains a message $m$.
	\end{itemize}
	
	Let $\tau$ be a "tree unfolding", let $\node$ be its root.
	
	We see $\localrunlabel{\node}$ as a "partial run" with a single agent.
	We are going to compose $\localrunlabel{\node}$ with some runs given by the children of $\node$ to construct a run satisfying the properties above.
	Let $V$ be the set of values received or broadcast in $\localrunlabel{\node}$ and $V_{init}$ be the set of initial values of $\localrunlabel{\node}$.
	
	
	\subsection{Step 1: Non-initial values}
	
	We show the following statement by reverse induction on $\size{V}$.
	
	For all $V' \subseteq V\setminus V_{init}$, there exists a "partial run" $\run$ such that for all $v \in \nats$:
	\begin{itemize}
		\item If $v \in V' \cup V_{init}$ then $\vinput{v}{\run} = \vinput{v}{u}$
		
		\item If $v \notin V' \cup V_{init}$ then $\vinput{v}{\run} = \epsilon$
	\end{itemize}  
	
	If $V' = V\setminus (V_{init} \cup \set{\valuelabel{\mu}})$ then this is clear as we can simply take $\run = u$.
	
	Now suppose there exists $v \in V \setminus (V_{init} \cup V')$. Let $V'' = V' \setminus \set{v}$. By induction hypothesis (on $\size{V}$) there exists a "partial run" $\run'$ such that for all $v' \in \nats$:
	\begin{itemize}
		\item If $v' \in V'' \cup V_{init}$ then $\vinput{v'}{\run'} = \vinput{v'}{u}$
		
		\item If $v' \notin V'' \cup V_{init}$ then $\vinput{v'}{\run'} = \varepsilon$
	\end{itemize}

	If $v \neq \valuelabel{\node}$ then by condition \ref{unfoldingC3.2} $\node$ has a child $\node'$ such that $\vinput{v}{\localrunlabel{\mu}} \subword  \bosslabel{\node'}$.
	By induction hypothesis (on the "tree unfolding"), as the subtree rooted in $\mu'$ satisfies the "boss specification" $\vinput{v}{\mu}$, there exists a run $\run_v$ satisfying $\vinput{v}{\localrunlabel{\mu}}$.
	
	Let $v' \in \nats$ be such that $\vinput{v}{\localrunlabel{\node}} \subword \voutput{v'}{\run_v}$. 
	
	We apply an injective renaming $\sigma : \nats \to \nats$ on the values in $\run_v$ to obtain a "partial run" $\run'_v$ so that $\sigma(v') = v$ and there is no other value appearing in both $\run'$ and $\run'_v$. 
	We then compose $\run'_v$ and $\run'$ as follows: Let $m_1\cdots m_k = \vinput{v}{\localrunlabel{\node}}$, 
	
	
	
	
	
	
	
	
	
	
%	
%	
%	\textbf{Case 1: The specification is a "boss specification" $spec \in \messages^*$}
%	
%	Then $\mu$ is a "boss node", labelled by $u$ and by a word $w$ of which $spec$ is a subword. 
%	By definition of a "tree unfolding", there exists a value $v$ and a decomposition $\decsymb = (w_0, m_1, \ldots, w_\ell) \in \decset{\messages}$ such that $\voutput{v}{u} = w_0\cdots w_\ell$ and $w \in \langdecdown{\decsymb}$.
%	%	Furthermore for all $j \in [1,\ell]$, $\mu$ has a child labelled $(\decsymb_j, m_j)$ with $\decsymb_j = (w_0, m_1, \ldots, w_{j-1})$. 	By induction hypothesis, there exists a partial run $\run_{v, \decsymb_j}$ in which there is a broadcast of $m_j$ with value $v$ 
%	
	\cortoin{IN PROGRESS}



\section{Proof of Proposition~\ref{prop:reduction-LCS}}
\label{app:reduction-lcs}

\propReductionLCS*

\begin{proof}
	It is sufficient to prove that it is as hard as reachability for lossy channel systems with a single channel, which corresponds to a single finite-state machine that has the ability to buffer symbols in a lossy FIFO queue \cite{Schnoebelen2002verifying}. We present here a polynomial-time reduction.
	Let $\los := (\lstates,\Sigma)$ be a "lossy channel system", where $\lstates$ is a finite set of locations, $\Sigma$ is a finite set of symbols and $\ltransitions \subseteq \lstates \times \Sigma^* \times \set{!, ?} \times \lstates$; $\quotemarks{!}$ corresponds to a "push" (writing at the end of the channel) and $\quotemarks{?}$ to a "pop" (reading at the beginning the channel). A configuration of $\los$ is a pair of $\lstates \times \Sigma^*$ denoting the location of the system and the content of the channel. There exists a step from $(\lstate,w)$ to $(\lstate',w')$ using transition $\ltrans \in \ltransitions$, denoted $(\lstate,w) \lstep{\ltrans} (\lstate',w')$, when
	\begin{itemize}
		\item $\ltrans = (\lstate,u,!,\lstate')$ for some $u \in \Sigma^*$ and $w' \subword w \cdot u$ (a ""push""),
		\item $\ltrans = (\lstate,u,?,\lstate')$ for some $u \in \Sigma^*$ and $u \cdot w' \subword w$ (a ""pop"")
	\end{itemize}
	where $\subword$ denotes the "subword" order.
	This subword order encodes the lossiness of the channel (a non-lossy channel would have equalities instead); intuitively, it expresses the fact that letters in the channel may get lost. 
	
	The existence of such a transition for some $\ltrans \in \transitions$ is denoted $(\lstate,w) \lstep{} (\lstate',w')$, and its transitive closure is denoted $\lstep{*}$. The ""reachability problem@@lcs"" asks, given $\los$ and two locations $\lstate_i, \lstate_f \in \lstates$, whether there exists an execution $(\lstate_i,\epsilon) \lstep{*} (\lstate_f, w)$ for some $w$. 
	
	We aim at constructing a protocol $\prot$ with two registers with a distinguished state $q_f$ such that $q_f$ may be covered if and only if $(\los, q_i, q_f)$ is a positive instance of the "reachability problem@@lcs". 
	The intuition of $\prot$ is the following. Agents will organize in chains so that each agent of a chain knows the identifier of its predecessor in the chain, hence is able to check that messages received come from it. Each agent of the chain is meant to encode a step of the execution in the "lossy channel system". 
	An agent of the chain will only listen to their predecessor in the chain, from which they will obtain a location of the system and the content of the channel. As it receives those, it will broadcast the new location of the system and the new content of the channel to the next agent of the chain. 
	
	Note that the content of the channel might become big in a "lossy channel system", therefore agents will not store it but rather rebroadcast it letter by letter. An agent only applies a small modification at the beginning of the channel if it decides to encode a "pop" transition and at the end of the channel if it decides to encode a "push" transition. Messages might get lost, which is why we are able to encode "lossy channel systems" but not non-lossy ones.
	
	\AP In some initial phase, agents decide whether they are ""root"" (at the beginning of their chain) or ""link"". A "root" agent receives no message; it simply broadcasts its identifiers and the initial configuration of $\los$, $(\lstate_0,\epsilon)$. To encode this option, in $\prot$, from the initial state $q_0$ one has the possibility to move to a part where the only sequence of transitions possible is $\br{\mathsf{init}}{1}, \br{\mathsf{q_0}}{1}, \br{\mathsf{\#}}{1}$ which gets to $\finstate{\lstate_0} \in Q$. The symbol $\mathsf{\#} \in \Sigma$ is the final symbol meaning that the channel was fully broadcast. 
	
	A "link" agent first receives a broadcast with an identifier which it decides to be the one of its predecessor. It then broadcasts its own identifier. This construction guarantees that "link" agents have exactly one predecessor. It does not guarantee, however, than all agents are in the same chain or that any agent is the predecessor of at most one agent. Concretely, there is a sequence of two transitions from $q_0$ labeled by $\rec{\mathsf{init}}{2}{\enregact}, \br{\mathsf{init}}{1}$ that gets to $\waitstate \in Q$. 
	
	From $\waitstate$, there is, for every $\lstate \in \lstates$, a transition labeled by $\rec{\lstate}{2}{\eqtestact}$ that goes to state $\startstate{\lstate} \in Q$. 
	For every transition $\ltrans = (\lstate, \op, \lstate') \in \ltransitions$ in $\los$ (\emph{i.e.}, every transition of $\ltransitions$ whose source is location $\lstate$), there is a transition in $\prot$ labeled by $\br{\mathsf{\lstate'}}{1}$ that goes from $\startstate{\lstate}$ to $\transstateone{\ltrans} \in Q$. 
	Transitions from $\transstateone{\ltrans}$ in $\prot$ depend on $\ltrans$:
	\begin{itemize}
		\item If $\ltrans=(\lstate, u,?, \lstate')$ is a "pop" then $\prot$ has a sequence of transitions from $\transstateone{\ltrans}$ to $\transstatetwo{\ltrans} \in Q$ labeled by $\rec{\mathsf{u_1}}{2}{\eqtestact}, \rec{\mathsf{u_2}}{2}{\eqtestact}, \dots, \rec{\mathsf{u_k}}{2}{\eqtestact}$ where $u = u_1 u_2 \cdots u_k$. Moreover, there is, for every $x \in \Sigma$, a loop on $\transstatetwo{\ltrans}$ labeled with sequence of actions $\rec{\mathsf{x}}{2}{\eqtestact}, \br{\mathsf{x}}{1}$. There is also a sequence of  transitions from $\transstatetwo{\ltrans}$ to $\finstate{\lstate'}$ labeled by $\rec{\mathsf{\#}}{2}{\eqtestact}, \br{\mathsf{\#}}{1}$.
		\item If $\ltrans= (\lstate, u,!, \lstate')$ is a "push" then there is, for every $x \in \Sigma$, a loop on $\transstateone{\ltrans}$ labeled with sequence of actions $\rec{\mathsf{x}}{2}{\eqtestact}, \br{\mathsf{x}}{1}$. There also is a sequence of transitions from $\transstateone{\ltrans}$ to $\transstatetwo{\ltrans}$ labeled by $\br{\mathsf{u_1}}{1}, \br{\mathsf{u_2}}{1}, \dots, \br{\mathsf{u_k}}{1}$ where $u = u_1 \cdot u_2 \cdots u_k$. From $\transstatetwo{\ltrans}$, there is  a sequence of transitions going to $\finstate{\lstate'}$ labeled by $\rec{\mathsf{\#}}{2}{\eqtestact}, \br{\mathsf{\#}}{1}$.
	\end{itemize}
	Finally, the objective state of our system is $q_f := \finstate{\lstate_f}$.
	
	Note that, in this protocol, register $1$ is broadcast-only, therefore acting as a signature for message, and register $2$ is reception-only, therefore used to check the signature of messages.
	
	We claim that $(\prot, q_f)$ is a positive instance of \COVER if and only if $(\los, \lstate_f)$ is a positive instance of the reachability problem for "lossy channel systems".
	First, suppose that there exists $w \in \Sigma^*$ such that $(\lstate_0, \epsilon) \lstep{*} (\lstate_f, w)$. Decompose the witness into $(\lstate_0, w_0) \lstep{} (\lstate_1, w_1) \lstep{} (\lstate_2, w_2) \cdots \lstep{} (\lstate_n,w_n)$ with $\lstate_n = \lstate_f$ and $w_n =w$. 
	We build an execution of $\prot$ that covers $q_f$ as follows. It has set of agents $\agents := \set{0,\dots, n}$. Agent $0$ becomes the "root" and for all $i \geq 1$, agent $i$ becomes a "link" with predecessor agent $i-1$. By induction on $i$, we build an execution using agents $0$ to $i$ such that agent $i$ ends on state $\finstate{\ltrans_i}$ and the sequence of messages sent by agent $i$ admits as subword $\mathsf{init} \cdot \mathsf{\lstate_i} \cdot w_i \cdot \mathsf{\#}$. For $i=0$, this condition is met as agent $0$ becomes "root". When the construction has been done up until agent $i$, we make agent $i+1$ do the following. It first places itself after agent $i$ in the chain and goes to $\waitstate$. It then received from agent $i$ state $l_i$ and goes to $\startstate{l_i}$. It then moves to $\transstateone{\ltrans}$ where $\ltrans = (\lstate_i, \op, \lstate_{i+1})$ is the transition of step $(\lstate_i, w_i) \lstep{} (\lstate_{i+1}, w_{i+1})$. Doing so, it broadcasts $\mathsf{\lstate_{i+1}}$. Its behavior then depends on $\ltrans$.
	\begin{itemize}
		\item if $\ltrans =(\lstate_i, u,!, \lstate_{i+1})$ is a "push" then $w_{i+1} \subword w_i \cdot u$; write $w_{i+1} = w_{i}' \cdot u'$ where $w_i' \subword w_i$ and $u' \subword u$. Agent $i+1$ receives every message in $w_i'$ and rebroadcasts it while looping on $\transstateone{\ltrans}$. It then broadcasts $u$ to get to $\transstatetwo{\ltrans}$. It finally receives $\mathsf{\#}$ from agent $i$ and rebroadcasts it, going to state $\finstate{\lstate_{i+1}}$. Overall the word broadcast by agent $i+1$ is $\mathsf{init} \cdot \mathsf{\lstate_i} \cdot w'_i \cdot u \cdot \mathsf{\#}$ which admits as subword $\mathsf{init} \cdot \mathsf{\lstate_i} \cdot w_{i+1} \cdot \mathsf{\#}$.
		\item if $\ltrans =(\lstate_i, u,?, \lstate_{i+1})$ is a "pop" then $u \cdot w_{i+1} \subword w_i$; write $w_{i} = u' \cdot w_{i+1}'$ where $w_{i+1} \subword w_{i+1}'$ and $u \subword u'$. Overall, agent $i+1$ receives from agent $i$ sequence of messages $\mathsf{init} \cdot \mathsf{\lstate_i} \cdot u \cdot w_{i+1}' \cdot \mathsf{\#}$ (some messages get lost). 
		From $\transstateone{\ltrans}$, agent $i+1$ receives $u$ and goes to $\transstatetwo{\ltrans}$. It then receives every message in $w_{i+1}'$ and rebroadcasts it while looping on $\transstatetwo{\ltrans}$. It finally received $\mathsf{\#}$ from agent $i$ and rebroadcasts it, going to state $\finstate{\lstate_{i+1}}$. Overall the word broadcast by agent $i+1$ is $\mathsf{init} \cdot \mathsf{\lstate_i} \cdot w_{i+1}'  \cdot \mathsf{\#}$ which admits as subword $\mathsf{init} \cdot \mathsf{\lstate_i} \cdot w_{i+1} \cdot \mathsf{\#}$.
	\end{itemize}
	This concludes the induction step.
	When applied to $i=n$, this builds an execution where agent $n$ ends on $\finstate{\lstate_n}$, which is a witness that $(\prot, q_f)$ is positive.
	
	Suppose now that $(\prot, q_f)$ is positive. Let $\run: \aconfig_0 \step{*} \aconfig_f$ where $\config_f$ covers $q_f$. 
	Write $\agents$ the set of agents in $\run$. Observe that, in $\prot$, one may never change the value of register $1$. 
	Moreover, any agent either keeps its original value in register $2$ or changes it exactly once to take the value of some other agent's register $1$. Therefore, we define a function $\predfun: \agents \rightarrow \agents \cup \set{\bot}$ that associates to every agent $a$ the agent $\agentbr$ whose initial value in register $1$ eventually becomes the value in register $2$ of $a$ (and $\bot$ if such an agent does not exist). Consider $a_f$ the agent that first covers $q_f$ in $\config_f$. Let $a_1, \dots ,a_n$ the sequence of agents such that $a_f = a_n$, $\predfun(a_{i+1}) = a_i$ for all $i \in \nset{0}{n-1}$ and $\predfun(a_0) = \bot$. 
	Because $a_n$ covers $\finstate{\lstate_f}$, for all $i$, agent $i$ eventually broadcasts $\mathsf{\#}$ and it ends on some $\finstate{l_i}$ for $l_i \in \lstates$. Because $\predfun(a_0) = \bot$, $a_0$ has taken the "root" branch and therefore broadcasts sequence $\mathsf{init} \cdot \mathsf{\lstate_0} \cdot \mathsf{\#}$. 
	By structure of the protocol, because agent $i$ ends of $\finstate{\lstate_i}$, its broadcast sequence is of the form $\mathsf{init} \cdot \mathsf{\lstate_i} \cdot \mathsf{w_i} \cdot \mathsf{\#}$ where $w_i \in \Sigma^*$. 
	We prove by induction that, for every $i$, $(\lstate_i, w_i) \lstep{\ltrans_{i+1}} (\lstate_{i+1}, w_{i+1})$ where $\ltrans_{i+1}$ is such that agent $a_{i+1}$ goes through states $\transstateone{\ltrans_{i+1}}$ and $\transstatetwo{\ltrans_{i+1}}$ ($\ltrans_{i+1}$ is unique by construction of $\prot$).
	By construction, $\ltrans_{i+1}$ has source $\lstate_i$ (it is the value broadcast by $a_i$) and destination $\lstate_{i+1}$ ($a_{i+1}$ ends on $\finstate{\lstate_{i+1}$}). 
	Moreover, in order to get through $\transstateone{\ltrans_{i+1}}$ and $\transstatetwo{\ltrans_{i+1}}$, $a_{i+1}$ must indeed apply on the channel the "push" or "pop" that corresponds to transition $\ltrans_{i+1}$; because it receives stack $w_i$ from $a_i$, we have $(\lstate_i, w_i) \lstep{\ltrans_{i+1}} (\lstate_{i+1}, w_{i+1})$. 
	Since $w_0 = \epsilon$, we have proven that $(\lstate_0, \epsilon) \lstep{*} (\lstate_f, w_n)$ and the instance of $\los$ is positive. 
\end{proof}
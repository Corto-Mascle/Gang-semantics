\section{Proofs of Section~\ref{sec:local-bounds}}

\lemShortLocalRuns*

\begin{proof}
	Given a "local run" $\localrun$, register $i$ is ""active"" in $\localrun$ if at least one $\quotemarks{\enregact}$ step on register $i$ is performed in $\localrun$. Also, let $n:=\size{\prot}$.
	
	We prove the ""shortening property"": \\
	Let a "local run" $\localrun: (q_i, \localdata_i) \step{*} (q_f, \localdata_f)$ with $k$ "active registers" such that $\length{\localrun} > \towerfun(n,k)$ and let $V \subseteq \nats$ finite that contains every message value appearing in $\localrun$. We claim that $\localrun$ can be shortened into a local run $\localrun': (q_i, \localdata_i) \step{*} (q_f, \localdata_f)$ with $k$ active registers such that $\length{\localrun'} < \length{\localrun}$ and:
	\begin{itemize}
		\item for all $\aval' \in \nats$, there exists $\aval \in \nats$ such that $\vinput{\aval'}{\localrun'}$ is a subword of $\vinput{\aval}{\localrun}$,
		\item for all $\aval \in V$, $\vinput{\aval}{\localrun'}$ is a subword of $\vinput{\aval}{\localrun}$. 
	\end{itemize}
	
	We proceed by induction on the number $k$ of "active registers" in the "local run". If $k=0$, register values do not change in $\localrun$. Therefore, if $\localrun$ goes through the same state twice, all steps in between may be removed and it suffices to take $\towerfun(\size{\prot}, 0) := |\prot| + 1 \geq |Q| +1$.  
	
	Suppose that the property is true for any protocol with $\leq k$ "active registers", and consider a run $\localrun: (q_i,\localdata_i) \step{*} (q_f,\localdata_f)$ with $k+1$ "active registers" such that $\length{\localrun} > \towerfun(n)(k+1)$.
	
	First, if there exists an infix "local run" $\localrun_i$ of $\localrun$ of length $\towerfun(n)(k)+1$ with only $k$ active registers, then it suffices to apply the induction hypothesis on $u_i$ and $V$.
	
	Suppose now that there exists no such infix "local run".
	Let $I \subseteq \nset{1}{r}$ the set of "active registers" in $\localrun$, $|I| = k+1$. Let $M:= \towerfun(n)(k){+}1$. 
	In any sequence of $M$ "local steps" in a row in $\localrun$, 
	there is a $\quotemarks{\enregact}$ transition on every register in $I$. Let $\towerfun(n)(k+1) := M (n^{2M}+1)$ and suppose that $\length{\localrun} > \towerfun(n)(k+1)$. Additionally, suppose that $\localrun$ no local configuration appears twice in $\localrun$ (otherwise $\localrun$ may easily be shortened). For every $i$, let $\atrans_i$ the $i$-th transition in $\localrun$.
	For every $i \in \nset{0}{n^{2M}}$, we write $s_i$ the sequence $\delta_{2  M \cdot i+1}, \delta_{2  M \cdot i+2}, \cdots, \delta_{2 M \cdot i+2M}$.
	By Pigeonhole principle two of the sequences $s_i$ are equal (there are $|\transitions|^{2M}$ distinct such sequences and $|\transitions| \leq n$). 
	There exist two infix "local runs" $\localrun_a: (q_1, \localdata_1) \step{*} (q_2, \localdata_2)$, $\localrun_b: (q_3, \localdata_3) \step{*} (q_4, \localdata_4)$ in $\localrun$ such that $(q_2,\localdata_2)$ appears strictly before $(q_3,\localdata_3)$ in $\localrun$ and $\localrun_a$ and $\localrun_b$ both have sequence of transitions $s$.
	
	Although $\localrun_a$ and $\localrun_b$ have the same sequence of transitions, their "traces" may differ because their "external messages" may have different values.
	We build a "trace" $\atrace$ such that $(q_1,\localdata_1) \step{\atrace} (q_4, \localdata_4)$ where the underlying sequence of transitions of $\atrace$ is $s$.
	
	For every active register $i \in I$, let $e_i \in \nset{1}{2M}$ denote the index of the first $\quotemarks{\enregact}$ on register $i$ in $s$ and $f_i \in \nset{1}{2M}$ the index of the last $\quotemarks{\enregact}$ on register $i$ in $s$. By hypothesis, because $s$ is of length $2M$, it contains at least two $\quotemarks{\enregact}$ on register $i$, one in the fist half and one in the second half, hence $e_i \leq M < M +1 \leq f_i$. 
	
	For every $j \in \nset{1}{2M}$, let $\delta_j$ denote the $j$-th transition of $s$. First, if  $\delta_j$ is a "broadcast" or a "local test", we define the $j$-th "local step" of $\atrace$ as $\intlabel{\delta_j}$. 
	Suppose now that $\delta_j$ is a "reception" of the form $\rec{\amessage}{i}{\anact}$. The $j$-th "local step" of $\localrun_a$ (resp. $\localrun_b$) has underlying transition $\delta_j$ hence is an "external message" of the form $\extlabel{\delta_j}{\aval_a}$ for some $\aval_a \in V$ (resp. $\extlabel{\delta_j}{\aval_b}$ for some $\aval_b \in V$).
	Because $V$ is finite, there exists a function $\phi: V \rightarrow \nats \setminus V$ injective.
	We define the $j$-th "local step" of $\atrace$ to be $\extlabel{\delta_j}{\aval}$ where:
	\begin{itemize} 
		\item if $j < e_i$, $\aval = \aval_a$,
		\item if $e_i \leq j < f_i$, $\aval = \phi(\aval_a)$,
		\item if $f_i \leq j$, $\aval = \aval_b$.
	\end{itemize}
	We now claim that $(q_1,\localdata_1) \step{\atrace} (q_4,\localdata_4)$. 
	First, for every active register $i$, the last $\quotemarks{\enregact}$ step on register $i$ has value $\localdata_4(i)$ in $\atrace$ (as we are in the case $f_i \leq j$). Hence if every "local step" is valid then the final "local configuration" is $(q_4, \localdata_4)$.
	For every $l \in \nset{0}{2M}$, let $\atrace_l$ denote the prefix of $\atrace$ of length $l$.
	We prove by induction on $l$ that $\atrace_l$ is valid from $(q_1, \localdata_1)$. It is trivially true for $l =0$. Assume that we have $(q_1, \localdata_1) \step{\atrace_l} (q, \localdata)$ and let $\locallabel$ such that $\atrace_l \cdot \locallabel = \atrace_{l+1}$. Let $\atrans$ the underlying transition of $\locallabel$.
	First, $q$ is the initial state of $\atrans$ because $\locallabel$ is valid at step $l+1$ of $\localrun_a$ (and $\localrun_b$). Hence if $\atrans$ is a "broadcast" then $\locallabel$ is valid from $(q,\localdata)$.
	
	For every $i \in I$, let $\localdata_a, \localdata_b$ the content of register $i$ after the $l$-th step in $\localrun_a$ and $\localrun_b$ respectively.
	
	\noindent \textbf{$\locallabel$ is an "external message"} \\
	If $\locallabel$ is an "external message" of the form $\extlabel{\atrans}{\aval}$, then $\atrans$ has action $\rec{\amessage}{i}{\anact}$. Let $\aval_a$, $\aval_b$ the value of the corresponding "external message" in $\localrun_a$ and $\localrun_b$ respectively. The only problematic cases are when $\anact = \quotemarks{\diseqtestact}$ or $\anact = \quotemarks{\eqtestact}$.
	
	In this case, we prove that $\binrel{\aval}{\anact}{\localdata(i)}$ with $\anact \in \set{\quotemarks{\eqtestact}, \quotemarks{\diseqtestact}}$. First, because the corresponding step is valid in $\localrun_a$ and $\localrun_b$, we have $\binrel{\localdata_a(i)}{\anact}{\aval_a}$ and $\binrel{\localdata_b(i)}{\anact}{\aval_b}$. We distinguish cases depending on the value of $l+1$:
	\begin{itemize}
		\item $l+1<e_i$: $\localdata(i)= \localdata_a(i)$, $\aval = \aval_a$ and $\binrel{\localdata_a(i)}{\anact}{\aval_a}$. 
		\item $e_i \leq l+1 < f_i$: We have $\aval = \phi(\aval_a)$. 
		Moreover, because $e_i < l+1$, there is at least one $\quotemarks{\enregact}$ on register $i$ in $\atrace_l$. Consider the last such "transition" in $\atrace_l$; its index $j$ satisfies $e_i \leq j < f_i$ by definition of $e_i$, hence the value of the corresponding "external message" in $\localrun_a$ is $\localdata_a(i)$ and its value in $\atrace_{l}$ is $\phi(\localdata_a(i))$. One has $\binrel{\localdata_a(i)}{\anact}{\aval_a}$ therefore (by injectivity of $\phi$ for $\anact = \quotemarks{\diseqtestact}$)
		$\binrel{\phi(\localdata_a(i))}{\anact}{\phi(\aval_a)}$.
		\item$f_i \leq l+1$: $\localdata(i)= \localdata_b(i)$ and $\aval = \aval_b$, and because the "internal step" is valid in $\localrun_b$ we have $\binrel{\localdata_b(i)}{\anact}{\aval_b}$. 
	\end{itemize}
	
	\noindent \textbf{$\locallabel$ is an "internal step"} \\
	The only problematic case is when $\locallabel =: \loc{i}{i'}{\diseqtestact}$ with $i,i' \in \nset{1}{\regnum}$ (we have no equality tests thanks to Lemma~\ref{prop:loc-eq-test-elimination}). We have to prove that $\localdata(i) \ne \localdata(i')$. 
	Because the "local test" is satisfied in $\localrun_a$ and $\localrun_b$, we have $\localdata_a(i) \ne \localdata_a(i')$ and $\localdata_b(i) \ne \localdata_b(i')$.
	
	If $i \notin I$ and $i' \notin I$ (none are "active registers") then  $\localdata(i) = \localdata_a(i) \ne \localdata_a(i') = \localdata(i')$.
	
	If $i \in I$ and $i' \notin I$, then $\localdata(i') = \localdata_a(i') = \localdata_b(i')$ and:
	\begin{itemize}
		\item if $l+1<e_i$ then $\localdata(i) = \localdata_a(i) \ne \localdata_a(i')$,
		\item if $e_i \leq l+1 < f_i$ then $\localdata(i) = \phi(\localdata_a(i)) \notin V$ and  $\localdata(i') \in V$ therefore $\localdata(i) \ne \localdata(i')$,
		\item if $f_i \leq l+1$ then $\localdata(i) = \localdata_b(i) \ne \localdata_b(i')$.
	\end{itemize}
	We treat the case $i \notin I$ and $i' \in I$ symmetrically.
	
	If $i \in I$ and $i' \in I$, recall that $e_i \leq M < f_{i'}$ and $e_{i'} \leq M < f_i$ thus $e_i < f_{i'}$ and $e_{i'}< f_i$. We again make a case disjunction on the value of $l+1$:
	\begin{itemize}
		\item $l+1<e_i$ and $l+1 <e_{i'}$: $\localdata(i)= \localdata_a(i)$, $\localdata(i)= \localdata_a(i)$ and $\localdata_a(i) \ne \localdata_a(i')$. 
		\item $e_i \leq l+1 < f_i$, $l+1<e_{i'}$: $\localdata(i) = \phi(\localdata_a(i)) \notin V$, $\localdata(i') = \localdata_a(i') \in V$ therefore $\localdata(i') \ne \localdata(i)$.
		\item $e_{i'} \leq l+1 < f_{i'}$, $l+1<e_{i}$: symmetric to the previous case.
		\item $e_i \leq l+1 < f_i$, $e_{i'} \leq l+1 < f_{i'}$: $\localdata(i) = \phi(\localdata_a(i))$, $\localdata(i') = \phi(\localdata_a(i'))$; however $\localdata_a(i) \ne \localdata_a(i')$ and $\phi$ is injective hence $\localdata(i') \ne \localdata(i)$.
		\item $e_i \leq l+1 < f_i$, $f_{i'} \leq l+1$: $\localdata(i') = \localdata_b(i') \in V$, $\localdata(i) = \phi(\localdata_a(i)) \notin V$ hence $\localdata(i) \ne \localdata(i')$.
		\item $f_i \leq l+1$, $e_{i'} \leq l+1 < f_{i'}$: symmetric to the previous case.
		\item $f_i \leq l+1$, $f_{i'} \leq l+1$: $\localdata(i)= \localdata_b(i)$, $\localdata(i)= \localdata_b(i)$ and $\localdata_b(i) \ne \localdata_b(i')$. 
	\end{itemize}
	
	This proves that $\locallabel$ is valid from $(q,\localdata)$ which concludes the induction. 
	We have proven that $(q_1,\localdata_1) \step{\atrace} (q_4, \localdata_4)$; however $\atrace$ is of length $2M$ and there are at least $2M+1$ steps between $(q_1,\localdata_1)$ and $(q_4, \localdata_4)$ in $\localrun$. Therefore, replacing this part of $\localrun$ with $(q_1,\localdata_1) \step{\atrace} (q_4, \localdata_4)$ yields a "local run" $\localrun': (q_i, \localdata_i) \step{*} (q_f,\localdata_f)$ with strictly less steps that $\localrun$. 
	
	It remains to prove the conditions on the "$v$-inputs" of $\localrun'$. If suffices to prove the condition for the part between $(q_1, \localdata_1)$ to $(q_4, \localdata_4)$,
	because the rest of $\localrun$ is left untouched. 
	
	Let $(q_m, \localdata_m)$ the "local configuration" after $M$ steps of $\atrace$ from $(q_1, \localdata_1)$; write $\localrun_1$ the local run from $(q_1,\localdata_1)$ to $(q_m, \localdata_m)$ corresponding to the first $M$ steps of $\atrace$ in $\localrun'$, and $\localrun_2$ the "local run" from $(q_m, \localdata_m)$ to $(q_4, \localdata_4)$ corresponding to the last $M$ steps of $\atrace$ in $\localrun'$.  
	
	Let $\aval \in V$. We claim that $\vinput{\aval}{\localrun_{1}}$ is a subword of $\vinput{\aval}{\localrun_a}$ and $\vinput{\aval}{\localrun_{2}}$ is a subword of $\vinput{\aval}{\localrun_b}$. Indeed, in the construction of $\atrace$, the "external message" steps in the first $M$ steps were those of $\localrun_a$ except that some values were replaced with fresh values in $\nats \setminus V$, and similarly with $\localrun_b$ and the last $M$ steps. Overall, this proves that $\vinput{\aval}{\localrun'}$ is a subword of $\vinput{\aval}{\localrun}$ for every $\aval \in V$ and values of $V$ satisfy conditions \ref{item:shorterrun_anyvalue} and \ref{item:shorterrun_oldvalues}. 
	
	Let $\aval' \in \nats \setminus V$; $\aval$ does not appear in $\localrun$. Either $\aval'$ does not appear in $\atrace$ in which case the desired property is true, or there exists $\aval \in V$ such that $\aval' = \phi(\aval)$. But then $\vinput{\aval'}{\localrun_{1}}$ is a subword of $\vinput{\aval}{\localrun_a}$ and $\vinput{\aval'}{\localrun_{2}}$ is a subword of $\vinput{\aval}{\localrun_b}$. Indeed, in $\localrun_1$, the "external message" steps with value $\phi(\aval)$ correspond to "external message" steps in $\localrun_a$ with value $\aval$, and similarly for $\localrun_2$ and $\localrun_b$. This proves condition \ref{item:shorterrun_anyvalue} for every $\aval' \in \nats \setminus V$.
	Overall, we have proven the existence of a "local run" $\localrun': (q_i,\localdata_i) \step{*} (q_f,\localdata_f)$ that satisfies conditions \ref{item:shorterrun_anyvalue} and \ref{item:shorterrun_oldvalues} and that is strictly shorter that $\localrun$, which proves the "shortening property".
	
	We build a "local run" of length strictly less that $\towerfun(\size{\prot})(\regnum)$ as follows. We start with $\localrun^{(0)} := \localrun$ and $V^{(0)}$ the set of values of messages appearing in $\localrun^{(0)}$. For every $k$ such that $\length{\localrun^{(0)}} > \towerfun(\size{\prot})(\regnum)$, we apply the "shortening property" on $\localrun^{(k)}$ and $V^{(k)}$ to obtain $\localrun^{(k+1)}$ and define $V^{(k+1)}$ by $V^{(k)} \cup W$ where $W$ is the set of values of messages in $\localrun^{(k+1)}$, which is finite.
	The construction stops when $\length{\localrun^{(k)}} \leq \towerfun(\size{\prot})(\regnum)$, which concludes the proof of Lemma~\ref{lem:short-local-runs}. 
\end{proof}

\begin{lemma}
	\label{lem:short-dec}
	Let $\decsymb = (w_0, m_1, \ldots, w_\ell)$ be a decomposition, and for all $i\in \nset{0}{\ell}$ let $\decsymb_i = (w_0, m_1, \ldots, w_i)$.
	Let $w'_0, \ldots, w'_\ell$ be such that $w'_i \in \langdec{\decsymb_i}$ for all $i$.
	
	Then there exists $\widehat{\decsymb} = (\widehat{w}_0, m_1, \ldots, \widehat{w}_\ell)$ such that, for all $i$, $\widehat{w}_i \subword w_i$ and $w'_i \in \langdec{\widehat{\decsymb}_i}$ (where $\widehat{\decsymb}_i = (\widehat{w}_0, m_1, \ldots, \widehat{w}_\ell)$), and $\sum_{i=0}^{\ell} \size{\widehat{w}_i} \leq \sum_{i=0}^{\ell} \size{w'_i}$. 
\end{lemma}

\begin{proof}
	For all $i$, as $w'_i \in \langdec{\decsymb_i}$, we can split $w'_i$ into $v_{i,0} \cdots v_{i,i}$ where, for all $j$, $\wordproj{v_{i,j}}{\messages\setminus\set{m_1, \ldots, m_{j}}}$ is a subword of $w_j$. \corto{define pi as word projection}
	
	For all $j \in \nset{0}{\ell}$, for all $i\geq j$, $\wordproj{v_{i,j}}{\messages\setminus\set{m_1, \ldots, m_{j}}}$ is a subword of $w_j$.
	Hence we can find a subword $\widehat{w}_j$ of $w_j$ of size at most $\sum_{i=0}^{j} \size{\wordproj{v_{i,j}}{\messages\setminus\set{m_1, \ldots, m_{j}}}}$ such that $\wordproj{v_{i,j}}{\messages\setminus\set{m_1, \ldots, m_{j}}}$ is a subword of $\widehat{w}_j$ for all $i$, by considering for each $i$ a set of positions in $w_j$ that forms $\wordproj{v_{i,j}}{\messages\setminus\set{m_1, \ldots, m_{j}}}$, and then setting $\widehat{w}_j$ as the word formed by the union of those sets of positions.
	
	We then define $\widehat{\decsymb} = (\widehat{w}_0, m_1, \ldots, \widehat{w}_\ell)$ and for all $i$ $\widehat{\decsymb}_i = (\widehat{w}_0, m_1, \ldots, \widehat{w}_\ell)$.
	
	We obtain that for all $i$, $w'_i = v_{i,0} \cdots v_{i,i}$ with, for all $j \in \nset{0}{i}$, $\wordproj{v_{i,j}}{\messages\setminus\set{m_1, \ldots, m_j}} \subword \widehat{w}_j$. Hence we have $w'_i \in \langdec{\decsymb_i}$.
	
	Furthermore we have $\sum_{i=0}^{\ell} \size{\widehat{w}_i} \leq \sum_{i=0}^{\ell} \sum_{j=0}^i \size{v_{i,j}} = \sum_{i=0}^{\ell} \size{w'_i}$.
\end{proof}

\lemShortRunOutput*

\begin{proof}
	
	We proceed by induction on $\size{w_{in}}$.
	
	If $w_{in} = \epsilon$ we simply apply Lemma~\ref{lem:short-local-runs}.
	
	Otherwise let $w'_{in} (m,v) = w_{in}$.
	As $w_{in} \subword \Output{\localrun}$ we can split $\localrun$ in three $\localrun_0: (q, \localdata) \step{*} (q_0, \nu_0)$, $(q_0, \localdata_0) \intstep{\delta} (q_1, \localdata_1)$, $\localrun_1 :(q_1, \localdata_1) \step{*} (q', \localdata')$ where $\delta$ applies a broadcast operation $\br{m}{i}$ with $\nu'_{m} (i) = v$.
	
	Let $V \subseteq \nats$ finite that contains every message value appearing in $\localrun$, and thus in particular every message value appearing in $\localrun_0$.
	By induction hypothesis, there exists a "local run" $\localrun_0': (q, \localdata) \step{*} (q_0, \localdata_0)$ such that $\length{\localrun_0'} \leq (\towerfun(\size{\prot})(r)+1)\size{w'_{in}}$, $w'_{in} \subword \Output{\localrun_0'}$ and:
	
	\begin{enumerate}
		\item for all $\aval' \in \nats$, there exists $\aval \in \nats$ such that $\vinput{\aval'}{\localrun_0'}$ is a subword of $\vinput{\aval}{\localrun_0}$,
		\item for all $\aval \in V$, $\vinput{\aval}{\localrun_0'}$ is a subword of $\vinput{\aval}{\localrun_0}$. 
	\end{enumerate}
	
	We set $V'$ as the set of values that either are in $V$ or appear in $\localrun_0$.
	By Lemma~\ref{lem:short-local-runs}, there exists a "local run" $\localrun_1': (q_1, \localdata_1) \step{*} (q', \localdata')$ such that $\length{\localrun_1'} \leq \towerfun(\size{\prot})(r)$ and:
	\begin{enumerate}
		\item for all $\aval' \in \nats$, there exists $\aval \in \nats$ such that $\vinput{\aval'}{\localrun_1'}$ is a subword of $\vinput{\aval}{\localrun_1}$,
		\item for all $\aval \in V$, $\vinput{\aval}{\localrun_1'}$ is a subword of $\vinput{\aval}{\localrun_1}$. 
	\end{enumerate}
	
	Let $\localrun'$ be the concatenation of $\localrun'_0$, $(q_0, \localdata_0) \intstep{\delta} (q_1, \localdata_1)$ and $\localrun'_1$. As $w'_{in} \subword \Output{\localrun'_0}$, we have $w_{in} \subword \Output{\localrun}$.
	The length of $\localrun'$ is $\length{\localrun'_0} + 1 + \length{\localrun'_1}$, which is at most $(\towerfun(\size{\prot})(r)+1)\size{w'_{in}} + 1 + \towerfun(\size{\prot})(r) = (\towerfun(\size{\prot})(r)+1)\size{w_{in}}$.
	
	Moreover, for all $v' \in \nats$, 
	\begin{itemize}
		\item either $v'\in V$, and then $\vinput{v'}{\localrun_0'} \subword \vinput{v'}{\localrun_0}$ and $\vinput{v'}{\localrun_1'} \subword \vinput{v'}{\localrun_1}$, thus $\vinput{v'}{\localrun'} \subword \vinput{v'}{\localrun}$.
		
		\item or $v' \in V' \setminus V$ and then $v'$ does not appear in $\localrun$, and there exists $v \in \nats$ such that $\vinput{v'}{\localrun_0'} \subword \vinput{v}{\localrun_0}$. Further, as $v' \in V'$ we have $\vinput{v'}{\localrun_1'} \subword \vinput{v'}{\localrun_1} = \epsilon$, hence $\vinput{v'}{\localrun'} \subword \vinput{v}{\localrun}$.
		
		\item or $v' \notin V'$, thus by definition of $V'$ we have $\vinput{v'}{\localrun_0'} = \epsilon$, and there exists $v$ such that $\vinput{v'}{\localrun_1'} \subword \vinput{v}{\localrun_1}$, hence $\vinput{v'}{\localrun'} \subword \vinput{v}{\localrun}$.
	\end{itemize}
	This concludes our proof.
\end{proof}

\lemBoundSuccessorHeight*

\begin{proof}
	Let $\node$ be a node of $\tree$. \corto{properly define trees and nodes}
	Let $V$ be the set of "initial values" being broadcast or received in $\localrunlabel{\node}$. We have $\size{V}\leq r$ as there is at most one such value for each register.  
	
	For each $v \in V$, if $v \neq \valuelabel{\node}$ (resp. if $v = \valuelabel{\node}$), let $\decsymb_v = (w_{v,0}, m_{v,1}, \ldots, w_{v,\ell_v})$ be a "decomposition" and $\node_{v,1}, \ldots, \node_{v, \ell_v}$ "follower" children of $\node$ such that condition \ref{unfoldingC3.1} (resp. \ref{unfoldingC1}) is respected.
	
	We can delete all other "follower" children of $\node$ and still have a "unfolding tree" satisfying $w$ as the $\node_{v,i}$ suffice to satisfy conditions \ref{unfoldingC1} and \ref{unfoldingC3.1} and the other conditions do not depend on "follower" children.
	As we assumed $\tree$ to be of minimal size, $\node$ does not have "follower" children besides the $\node_{v,i}$. As $\size{V} \leq r$ and $\ell_v \leq \size{\messages}$ for all $v$ (by definition of a "decomposition", as the $(m_{v,i})_{1\leq i \leq \ell_v}$ are all distinct), $\node$ has at most $\size{\messages}r$ "follower" children, proving property 1.
	
	Property 2 follows immediately from the fact that $\tree$ satisfies a "boss specification" $w$, which by definition implies that its root is a "boss node".
	
	Let $\node$ be a "boss node". If $\node$ is the root, then as $\tree$ satisfies $w$ we have $w \subword \bosslabel{\node}$. If $w \neq \bosslabel{\node}$, we can replace $\bosslabel{\node}$ with $w$: As $w \subword \bosslabel{\node}$, for all "decomposition" $\decsymb$, if $\bosslabel{\node} \in \langdec{\decsymb}$ then $w \in \langdec{\decsymb}$, hence condition~\ref{unfoldingC1} is still satisfied, and the other conditions of a "unfolding tree" are unaffected by this change.
	Furthermore the resulting "unfolding tree" still satisfies $w$. This contradicts the minimality of $\tree$, hence we have $\bosslabel{\node} = w$.
	
	If $\node$ is not the root, then let $\node'$ be its father. 
	Let $V'$ be the set of non-initial values $v'$ of $\node'$ such that $\vinput{v'}{\localrunlabel{\node'}} \subword \bosslabel{\node}$.
	
	We can construct a subword $\bossspec'$ of $\bosslabel{\node}$ of length at most $\sum_{v' \in V'} \size{\vinput{v'}{\localrunlabel{\node'}}}$ such that all $\vinput{v'}{\localrunlabel{\node'}}$ are subwords of $\bossspec'$. It suffices to select for each $v'$ a set of positions in $\bosslabel{\node}$ forming $\vinput{v'}{\localrunlabel{\node'}}$, and then taking the word formed by the union of those sets of positions.
	Observe that we necessarily have $\sum_{v' \in V'} \size{\vinput{v'}{\localrunlabel{\node'}}} \leq \size{\localrunlabel{\node'}}$, thus $\size{\bossspec'} \leq \size{\localrunlabel{\node'}}$.
	
	Like in the previous case, as $\bossspec'$ is a subword of $\bosslabel{\node}$, for all "decomposition" $\decsymb$, if $\bosslabel{\node} \in \langdec{\decsymb}$ then $\bossspec' \in \langdec{\decsymb}$. As a result, condition~\ref{unfoldingC1} is still satisfied after switching the label $\bosslabel{\node}$ to $\bossspec'$. Furthermore, by definition of $V'$ and $\bossspec'$, for all non-initial value $v'$ in $\localrunlabel{\node'}$, if $\vinput{v'}{\localrunlabel{\node'}} \subword \bosslabel{\node}$ then  $\vinput{v'}{\localrunlabel{\node'}} \subword \bossspec'$, hence condition~\ref{unfoldingC3.2} is still satisfied as well. Other conditions are unaffected by this change.
	
	Hence if $\size{\bosslabel{\node}} > \size{\localrunlabel{\node'}}$ we can obtain a "unfolding tree" smaller than $\tree$ and satisfying $w$, contradicting the minimality of $\tree$. As a result, $\size{\bosslabel{\node}} \leq \size{\localrunlabel{\node'}}$.
	
	We have shown the first two points of property 3. For the third one, we isolate the broadcasts of $\localrunlabel{\node}$ that are necessary to satisfy the "unfolding tree conditions" and then bound the length of $\localrunlabel{\node}$ using Lemma~\ref{lem:short-run-for-output}.
	
	Let $v$ be an initial value of $\localrunlabel{\node}$. If $v = \valuelabel{\node}$ then by condition~\ref{unfoldingC1} there exists a "decomposition" $\decsymb_v = (w_{v,0}, m_{v,1}, \ldots, w_{v,\ell_v})$ and "follower" children $\node_1, \ldots, \node_{\ell_v}$ such that
	\begin{itemize}
		\item $\bosslabel{\node} \in \langdec{\decsymb_v}$
		\item $u$ can be split into $u_{v,0}, \ldots, u_{v,\ell_v}$ so that for all $i$, $w_{v,i} \subword \voutput{v}{u_{v,i}}$ and $\vinput{v}{u_{v,i}} \in \set{m_1, \ldots, m_{i-1}}^*$
		
		\item for all $i$, $\followlabelword{\node_i} \in \langdec{\decsymb_{v,i}}$, with $\decsymb_{v,i}= (w_{v,0}, m_{v,1}, \ldots, w_{v,i-1})$, and $\followlabelmessage{\node_i} = m_i$.
	\end{itemize}
	
	Similarly, if $v \neq \valuelabel{\node}$, we set $\decsymb_v = (w_{v,0}, m_{v,1}, \ldots, w_{v,\ell_v})$, "follower" children $\node_1, \ldots, \node_{\ell_v}$ and $u_{v,0}, \ldots, u_{v,\ell_v}$ such that Condition~\ref{unfoldingC3.1} is satisfied.
	
	In both cases by Lemma~\ref{lem:short-dec}, we can assume that $\sum_{i=0}^{\ell_v} \size{w_{v,i}} \leq \sum_{i=0}^{\ell_v} \size{\followlabelword{\node_i}}$. \corto{A preciser}
	
	There exists a subword $x = (m_1, v_1) \cdots (m_k, v_k)$ of $\Output{\localrunlabel{\node}}$ such that for all
	
	\cortoin{TO FINISH} 
\end{proof}


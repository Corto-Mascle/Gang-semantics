\section{Proof of Lemma~\ref{lem:tree-to-run}}
\label{app:tree-to-run}

\begin{definition}
	We extend the notion of "local run" to describe the behaviour of a subset of agents in a "run".
	
	An ""internal test@global"" from $\config$ to $\config'$, denoted $\config \inttest \config'$, is defined when there is a "step" $\config \step{} \config'$ in which an agent does a "local test". 
	
	An ""internal message@global"" from $\config$ to $\config'$, denoted $\config \intmessage{m}{v} \config'$, is defined when there is a "step" $\config \step{} \config'$ in which an agent broadcasts a message $m$ with value $v$. 
	
	An ""external message@global"" from $\config$ to $\config'$, denoted $\config \extmessage{m}{v} \config'$ is defined when for all agents $a$, either there is a "local step" $\config(a) \extbr{m}{v} \config'(a)$ or $\config(a) = \config'(a)$.
	
	A ""partial run"" is a sequence of configurations $\config_0 \cdots \config_k$  such that for all $i \in [1, k]$, either $\config_{i-1} \inttest \config_{i}$ or $\config_{i-1} \intmessage{m, v} \config_{i}$ or $\config_{i-1} \extmessage{m, v} \config_{i}$ for some $m \in \messages$, $v\in \nats$. 
	
	We define the $v$-projection $\vproj{\aval}{\run}$ of $\run$, which is the word $(b_0, x_0) \cdots (b_\ell, x_\ell) \in (\messages \times \set{in, out})^*$ obtained from $\run$ by mapping every "external message@@global" $\config \extmessage{m}{v} \config'$ to $(m, in)$, every "internal message" $\config \intmessage{m}{v} \config'$ to $(m, out)$, and every "internal step" to $\varepsilon$.
	
	The ""input@partial"" $\vinput{v}{\run}$ (resp. ""output@partial"" $\voutput{v}{\run}$) of a "partial run" $\run$ is the sequence $m_0 \cdots m_k$ such that the projection on $\messages \times \set{in}$ (resp. $\messages\times \set{out}$) of $\vproj{v}{\run}$ is $(m_0, in) \cdots (m_k, in)$ (resp. $(m_0, out)\cdots(m_k, out)$).
	
	Note that a "local run" can be seen as a "partial run" with a single agent.
\end{definition}


\begin{lemma}
	\label{lem:boss-composition}
	Let $\run$ be a " partial run", $\run'$ a "run".
	If there exist $v, v' \in \nats$ such that $\vinput{v}{\run} \subword \voutput{v'}{\run'}$ and $v'$ is an initial value in $\run'$ but $v$ is not initial in $\run$, then there exists a "partial run" $\Tilde{\run}$ such that 
	\begin{itemize}
		\item $\vinput{v}{\Tilde{\run}} = \varepsilon$ 
		
		\item for all $v'' \neq v$, if $\vproj{v''}{\run} \neq \varepsilon$ then $\vproj{v''}{\Tilde{\run}} = \vproj{v''}{\run}$
		
		\item for all $v'' \neq v$, if $\vproj{v''}{\run} = \varepsilon$ then $\vinput{v''}{\Tilde{\run}} = \varepsilon$.
	\end{itemize}
\end{lemma}

\begin{proof}
	Let $\agents, \agents'$ be the sets of agents of $\run$ and $\run'$, respectively.
	We can rename agents so that those two sets are disjoint.
	
	Let $m_1 \cdots m_k = \vinput{v}{\run}$, $\run$ can be split into $\run = \config_0 \step{\run_0} \overline{\config}_0 \extbr{m_1,v} \config_{1} \step{\run_1} \cdots \extbr{m_k,v} \config_{k} \step{\run_k} \overline{\config}_k$ so that for all $j\in [0,k]$, $\vinput{v}{\run_j} = \varepsilon$.
	
	Similarly, as $m_1\cdots m_k \subword \voutput{v}{\run'}$, $\run'$ can be split into $\run' = \config'_0 \step{\run'_0} \overline{\config}'_0 \step{m_1, v} \config'_{1} \step{\run'_1} \cdots \step{m_k,v'} \config'_{k} \step{\run'_k} \overline{\config}'_k$.
	
	For all configurations $\config$ over $\agents$ and $\config'$ over $\agents'$, we write $\config \sqcup \config'$ for the configuration over $\agents \cup \agents'$ such that $\config \sqcup \config'(a)$ is $\config(a)$ if $a \in \agents$ and $\config'(a)$ if $a \in \agents'$.
	
	We apply a renaming to the values of $\run'$ so that $v'$ is mapped to $v$ and all other values are mapped to distinct values that do not appear in $\run$.
	Hence the only value appearing in both runs is $v$, and it is initial only in $\run'$. As a consequence, the configuration $\config_0 \sqcup \config'_0$ is an "initial configuration".
	
	We then construct the desired run $\Tilde{\run}$ over $\agents \cup \agents'$ by matching the "external messages" in $\run$ with the broadcasts in $\run'$ and executing the rest of the run in parallel.
	
	We have $\Tilde{\run} = \config_0 \sqcup \config'_0 \step{\run_0} \overline{\config}_0 \sqcup \config'_0 \step{\run'_0} \overline{\config}_0 \sqcup \overline{\config}'_0 \step{m_1, v} \config_{1} \sqcup \config'_1 \step{\run_1} \cdots \step{m_k,v} \config_{k}  \sqcup \config'_{k} \step{\run_k}  \overline{\config}_k \sqcup \config'_k \step{\run'_k} \overline{\config}_k \sqcup \overline{\config}'_k$.
	
	We indeed have $\overline{\config}_{j-1} \sqcup \overline{\config}'_{j-1} \step{m_j, v} \config_{j} \sqcup \config'_j$ as we can make all agents in $\agents$ that receive an "external messages" with message $m_j$ and value $v$ receive the broadcast made in $\agents'$ instead.
	
	Hence we obtain a run with 
	\begin{itemize}		
		\item $\vinput{v}{\Tilde{\run}} = \varepsilon$ 
		
		\item for all $v'' \neq v$, if $\vproj{v''}{\run} \neq \varepsilon$ then $v''$ does not appear in $\run'$ after the renaming hence $\vproj{v''}{\Tilde{\run}} = \vproj{v''}{\run}$
		
		\item for all $v'' \neq v$, if $\vproj{v''}{\run} = \varepsilon$ then $\vinput{v''}{\run} = \varepsilon$ and $\vinput{v''}{\run'} = \varepsilon$ as $\run'$ is a "run" and not a "partial run", hence $\vinput{v''}{\Tilde{\run}} = \varepsilon$.
	\end{itemize}
	This concludes our proof.
\end{proof}

%\begin{lemma}
%	Let $M \subseteq \messages$, $m \in M$, $w \in \messages^*$ and $w_{-m}$ its projection on $\messages\setminus \set{m}$, suppose there exists a run $\run$ and a value $v$ such that $\vinput{v}{\run} \in M^*$ and $w_{-m} \subword \voutput{v}{\run}$. 
%	
%	Suppose also that there exists a run $\run'$ and a value $v'$ such that the $\vinput{}$ 
%	
%	$\config_0, \ldots, \config_{\ell+1}$ configurations and $v$ a value such that for all $j \in [1,\ell+1]$ there exists a "partial run" $\run_j$ from $\config_{j-1}$ to $\config_{j}$ with $\voutput{v}{\run_j} = w_j$, $\vinput{v}{\run_j} \in \set{m_1,\ldots, m_{j-1}}^*$.
%	
%	Suppose that for all $j \in [1,\ell]$ there exists a "partial run" $\run'_j$ and a value $v_j$ such that $\vinput{v_j}{\run'_j}$ "decomposes as" $\decsymb_j = (w_0, m_1, \ldots, w_{j-1})$, $\voutput{v_j}{\run'_j}$ contains $m_j$ and $\vinput{v'}{\run'_j} = \varepsilon$ for all $v' \neq v_j$.
%	
%	Then there exists a "partial run" $\Tilde{\run}$ such that 
%	\begin{itemize}
%		\item $w \subword \voutput{v}{\Tilde{\run}}$, 
%		
%		\item $\vinput{v}{\run} = \varepsilon$,
%		
%		\item for all $v' \neq v$, if $\vproj{v'}{\run} \neq \varepsilon$ then $\vproj{v'}{\Tilde{\run}} = \vproj{v'}{\run}$
%		
%		\item for all $v' \neq v$, if $\vproj{v'}{\run} = \varepsilon$ then $\vinput{v'}{\Tilde{\run}} = \varepsilon$
%	\end{itemize}
%\end{lemma}

\begin{lemma}
	\label{lem:follower-composition-completion}
	Let $\decsymb = (w_0, m_1, \ldots, w_\ell)$ be a "decomposition", $\config_0, \ldots, \config_{\ell+1}$ configurations and $v$ a value such that for all $j \in [0,\ell]$ there exists a "partial run" $\run_j$ from $\config_{j}$ to $\config_{j+1}$ with $w_j \subword \voutput{v}{\run_j}$ and $\vinput{v}{\run_j} \in \set{m_1,\ldots, m_{j-1}}^*$.
	
	Suppose that for all $j \in [1,\ell]$ there exists a "partial run" $\run'_j$ and a value $v_j$ such that $\vinput{v_j}{\run'_j} \in \langdecdown{\decsymb_j}$ where $\decsymb_j = (w_0, m_1, \ldots, w_{j-1})$, $\vinput{v'}{\run'_j} = \varepsilon$ for all $v' \neq v_j$ and the last step of $\run'_j$ is an internal message $\intmessage{m}{v_j}$.
	
	Then, there exist "configurations" $\Tilde{\config}_0, \ldots, \Tilde{\config}_{\ell+1}$ such that for all $j \in [0,\ell]$, there is a "partial run" $\Tilde{\run}_j$ from $\Tilde{\config}_j$ to $\Tilde{\config}_{j+1}$ and
	\begin{itemize}
		\item $\voutput{v}{\run_j} \subword \voutput{v}{\Tilde{\run}_j}$, 
		
		\item $\vinput{v}{\Tilde{\run}_j} = \varepsilon$,
		
		\item for all $v' \neq v$, if $\vproj{v'}{\run_j} \neq \varepsilon$ then $\vproj{v'}{\Tilde{\run}_j} = \vproj{v'}{\run_j}$
		
		\item for all $v' \neq v$, if $\vproj{v'}{\run_j} = \varepsilon$ then $\vinput{v'}{\Tilde{\run}_j} = \varepsilon$
\end{itemize}
\end{lemma}


\begin{proof}
	
	We prove the property by strong induction on $(\size{\vinput{v}{\run}}_{m_\ell}, \ldots, \size{\vinput{v}{\run}}_{m_1})$, with the lexicographic ordering, $\run$ being the run obtained by concatenating $\run_1, \ldots, \run_\ell$.
	
	If $\vinput{v}{\run} = \epsilon$ then we can set $\Tilde{\run} = \run$ to obtain the result.
	
	If $\vinput{v}{\run} \neq \epsilon$, then let $j$ be such that $\vinput{v}{\run_j} \neq \epsilon$, then we can decompose $\run_j$ as $\run_j = \config_{j} \step{\run_j^-} \config_j^- \extmessage{m_{j'}}{v} \config_j^+ \step{\run_j^+} \config_{j+1}$ with $j'\leq j$. 
	
	There exists a "partial run" $\run'_{j'}$ and a value $v_{j'}$ such that $\vinput{v_{j'}}{\run'_{j'}} \in \langdecdown{\decsymb_{j'}}$,  $\vinput{v'}{\run'_{j'}} = \varepsilon$ for all $v' \neq v_j$  and the last step of $\run'_{j'}$ is an internal message $\intmessage{m}{v_{j'}}$.
	
	Hence $\run'_{j'}$ can be decomposed as $\config_{{j'},0} \step{\run_{{j'},1}'} \config_{{j'},1} \cdots \step{\run_{{j'},j'-1}'} \config_{{j'},j'} \intmessage{m_{j'}}{v} \config'_{j'}$ where, for all $i \in [1, j'-1]$, the projection of $\vinput{v_{j'}}{\run_{{j'},i}'}$ on ${\messages\setminus\set{m_1, \ldots, m_{{i}-1}}}$ is a subword of $w_i$.
	
	We then proceed in a similar way as in Lemma~\ref{lem:boss-composition}.
	We can rename agents so that $\run$ and $\run'_j$ are on disjoint sets of agents, and apply a renaming of values in $\run_j'$ so that $v_j$ is mapped to $v$ and all other values are mapped to values that do not appear in $\run$.
	We then construct a sequence of runs by executing in parallel $\run_{j',i}'$ and $\run_i$ for all $i\in [1,\ell]$. As $w_i \subword \voutput{v}{\run_i}$, we can match the "internal messages" of $\run_j$ with some "external messages" of $\run'_{j',i}$ so that the remaining "external messages" with value $v$ in $\run'_{j',i}$ are all in $\set{m_1, \ldots, m'_{j'}}$. 
	We obtain, for each $i \in [1,j'-1]$, a run $\run''_i$ from $\config_{i-1} \sqcup \config_{j', i-1}$ to $\config_{i} \sqcup \config_{j', i}$ with $\vinput{v}{\run''_i} \in \set{m_1, \ldots, m_{i-1}}^*$. 
	
	For each $i \in [j'+1, j-1]$, let $\run''_i$ be the run that goes from $\config_{i-1} \sqcup \config_{j', j'}$ to $\config_{i} \sqcup \config_{j', j'}$ by executing $\run_i$ on the first part and staying idle on the second one.
	
	Let $\run''_j$ be the run from $\config_{j-1} \sqcup \config_{j', j'}$ to $\config_{j} \sqcup \config'_{j'}$ by executing $\run_{j}^-$ on the first part, then matching the $\config_j^- \extmessage{m_{j'}}{v} \config_j^+$ step in $\run$ with the last step of $\run'_{j'}$ to obtain a  step $\config_j^- \sqcup \config_{j', j'} \intmessage{m_{j'}}{v} \config_j^+ \sqcup \config'_{j'}$, and then executing $\run_{j}^+$ on the first part.
	
	For each $i \in [j+1, \ell]$, let $\run''_i$ be the run that goes from $\config_{i-1} \sqcup \config'_{j'}$ to $\config_{i} \sqcup \config'_{j'}$ by executing $\run_i$ on the first part and staying idle on the second one.
	
	Note that for all $i \in [0, \ell]$, for all $v' \neq v$, if $\vproj{v'}{\run_i} \neq \epsilon$ then $\vproj{v'}{\run''_i} = \vproj{v'}{\run_i}$ (as we renamed values so that $\run$ and $\run'_j$ use disjoint sets of values apart from $v$) and if $\vproj{v'}{\run_i} = \epsilon$ then $\vinput{v'}{\run''_i} = \vinput{v'}{\run'_{j',i}} = \epsilon$.
	Let $\run''$ be the concatenation of the $\run''_i$.
	We can apply the induction hypothesis to the runs $\run''_i$ constructed above, as $\size{\vinput{v}{\run''}}_{m_{j'}} < \size{\vinput{v}{\run}}_{m_{j'}}$ and $\size{\vinput{v}{\run''}}_{m_{j''}} = \size{\vinput{v}{\run}}_{m_{j''}}$ for all $j''>j'$.
	
	We obtain a family of runs $\Tilde{\run}_0, \ldots, \Tilde{\run}_\ell$ such that, for all $i$, $\voutput{v}{\run''_i} \subword \voutput{v}{\Tilde{\run}_i}$, $\vinput{v}{\Tilde{\run}_i} = \epsilon$, and for all $v' \neq v$, if $\vproj{v'}{\run''_i} \neq \epsilon$ then $\vproj{v'}{\run''_i} = \vproj{v'}{\Tilde{\run}_i}$ and if $\vproj{v'}{\run''_i} = \epsilon$ then $\vinput{v'}{\Tilde{\run}_i} = \epsilon$.
	
	As a consequence, we have, for all $i\in [0,\ell]$:
	\begin{itemize}
		\item $\voutput{v}{\run_i} \subword \voutput{v}{\run''_i} \subword \voutput{v}{\Tilde{\run}_i}$
		
		\item $\vinput{v}{\Tilde{\run}_i} = \epsilon$
		
		\item for all $v' \neq v$, if $\vproj{v'}{\run_i} \neq \epsilon$ then $\vproj{v'}{\run''_i} = \vproj{v'}{\run_i} \neq \epsilon$ and thus $\vproj{v'}{\run_i} = \vproj{v'}{\run''_i} = \vproj{v'}{\Tilde{\run}_i}$
		
		\item for all $v' \neq v$, if $\vproj{v'}{\run_i} = \epsilon$ then $\vinput{v'}{\run''_i} = \epsilon$ and either $\vproj{v'}{\run''_i} = \vproj{v'}{\Tilde{\run}_i}$, hence in particular $\vinput{v'}{\Tilde{\run}_i} = \vinput{v'}{\run''_i} = \epsilon$, or $\vinput{v'}{\Tilde{\run}_i} = \epsilon$.
	\end{itemize}

All the conditions are satisfied, the lemma is proven.
\end{proof}

\begin{lemma}
	\label{lem:follower-composition-output}
	Let $\decsymb = (w_0, m_1, \ldots, w_\ell)$ be a "decomposition", $w'_0, \ldots, w'_\ell \in \messages^*$ such that for all $i \in [0, \ell]$, the projection of $w'_i$ on $\messages\setminus \set{m_1, \ldots, m_{i-1}}$ is a subword of $w_i$.
	
	Let $\config_0, \ldots, \config_{\ell+1}$ be configurations and $v$ a value such that for all $j \in [1,\ell+1]$ there exists a "partial run" $\run_j$ from $\config_{j-1}$ to $\config_{j}$ with $w_j \subword \voutput{v}{\run_j}$ and $\vinput{v}{\run_j} \in \set{m_1,\ldots, m_{j-1}}^*$.
	
	Suppose that for all $j \in [1,\ell]$ there exists a "partial run" $\run'_j$ and a value $v_j$ such that $\vinput{v_j}{\run'_j} \in \langdecdown{\decsymb_j}$ where $\decsymb_j = (w_0, m_1, \ldots, w_{j-1})$, $\vinput{v'}{\run'_j} = \varepsilon$ for all $v' \neq v_j$ and the last step of $\run'_j$ is an internal message $\intmessage{m}{v_j}$.
	
	Then, there exist "configurations" $\Tilde{\config}_0, \ldots, \Tilde{\config}_{\ell+1}$ such that for all $j \in [0,\ell]$, there is a "partial run" $\Tilde{\run}_j$ from $\Tilde{\config}_j$ to $\Tilde{\config}_{j+1}$ and
\begin{itemize}
	\item $w'_j \subword \voutput{v}{\Tilde{\run}_j}$, 
	
	\item $\vinput{v}{\Tilde{\run}_j} = \varepsilon$,
	
	\item for all $v' \neq v$, if $\vproj{v'}{\run_j} \neq \varepsilon$ then $\vproj{v'}{\Tilde{\run}_j} = \vproj{v'}{\run_j}$
	
	\item for all $v' \neq v$, if $\vproj{v'}{\run_j} = \varepsilon$ then $\vinput{v'}{\Tilde{\run}_j} = \varepsilon$
\end{itemize}
\end{lemma}

\begin{proof}
	Let $w' = w'_0 \cdots w'_\ell$.
	We proceed by strong induction on $\size{w'}$.
	
	If for all $j$ we have $w'_j \subword w_j$  (and thus $w'_j \subword \voutput{v}{\run_j}$), the result is a direct consequence of Lemma~\ref{lem:follower-composition-completion}.
	
	Suppose it is not the case. Then there exists $j, j'$ such that $j' \leq j$ and $m_{j'}$ appears in $w'_j$. Let $w'_j = w^-_j m_{j'} w^+_j$ and let $w''_j$ be the word $w^-_jw^+_j$.
	
	Then, there exist "configurations" $\widehat{\config}_0, \ldots, \widehat{\config}_{\ell+1}$ such that for all $i \in [0,\ell]$, there is a "partial run" $\widehat{\run}_i$ from $\widehat{\config}_i$ to $\widehat{\config}_{i+1}$ and
\begin{itemize}
	\item $w'_i \subword \voutput{v}{\widehat{\run}_i}$ if $i \neq j$,
	
	\item $w''_j \subword \voutput{v}{\widehat{\run}_j}$ 
	
	\item $\vinput{v}{\widehat{\run}_i} = \varepsilon$,
	
	\item for all $v' \neq v$, if $\vproj{v'}{\run_i} \neq \varepsilon$ then $\vproj{v'}{\widehat{\run}_i} = \vproj{v'}{\run_i}$
	
	\item for all $v' \neq v$, if $\vproj{v'}{\run_j} = \varepsilon$ then $\vinput{v'}{\widehat{\run}_i} = \varepsilon$
\end{itemize}

From these runs we infer a sequence of "partial runs" that output $w'_0, \ldots, w'_\ell$ but have a non-empty $v$-input. We then use Lemma~\ref{lem:follower-composition-completion} to obtain the desired runs with an empty $v$-input.

We proceed in a similar manner as in Lemma~\ref{lem:follower-composition-completion}.
There exists a "partial run" $\run'_{j'}$ and a value $v_{j'}$ such that $\vinput{v_{j'}}{\run'_{j'}} \in \langdecdown{\decsymb_{j'}}$,  $\vinput{v'}{\run'_{j'}} = \varepsilon$ for all $v' \neq v_j$  and the last step of $\run'_{j'}$ is an internal message $\intmessage{m}{v_{j'}}$.

Hence $\run'_{j'}$ can be decomposed as $\config_{{j'},0} \step{\run_{{j'},1}'} \config_{{j'},1} \cdots \step{\run_{{j'},j'-1}'} \config_{{j'},j'} \intmessage{m_{j'}}{v} \config'_{j'}$ where, for all $i \in [1, j'-1]$, the projection of $\vinput{v_{j'}}{\run_{{j'},i}'}$ on ${\messages\setminus\set{m_1, \ldots, m_{{i}-1}}}$ is a subword of $w_i$.

We can rename agents so that the $\widehat{\run}_i$ and $\run'_{j',i}$ are on disjoint sets of agents, and apply a renaming of values in $\run'_{j'}$ so that $v_{j'}$ is mapped to $v$ and all other values are mapped to values that do not appear in $\run$.
We then construct a sequence of runs by executing in parallel $\run_{j',i}'$ and $\widehat{\run}_i$ for all $i\in [1,\ell]$. As $w_i \subword \voutput{v}{\widehat{\run}_i}$, we can match the "internal messages" of $\widehat{\run}_i$ with some "external messages" of $\run'_{j',i}$ so that the remaining "external messages" with value $v$ in $\run'_{j',i}$ are all in $\set{m_1, \ldots, m'_{j'}}$. 
We obtain, for each $i \in [1,j'-1]$, a run $\run''_i$ from $\widehat{\config}_{i-1} \sqcup \config_{j', i-1}$ to $\widehat{\config}_{i} \sqcup \config_{j', i}$ with $\vinput{v}{\run''_i} \in \set{m_1, \ldots, m_{i-1}}^*$. 

For each $i \in [j'+1, j-1]$, let $\run''_i$ be the run that goes from $\widehat{\config}_{i-1} \sqcup \config_{j', j'}$ to $\widehat{\config}_{i} \sqcup \config_{j', j'}$ by executing $\widehat{\run}_i$ on the first part and staying idle on the second one.

We can split $\widehat{\run}$ into $\widehat{\config}_{j} \step{\widehat{\run}^-} \widehat{\config}^- \step{\widehat{\run}^+} \widehat{\config}_{j+1}$
Let $\run''_j$ be the run from $\widehat{\config}_{j-1} \sqcup \config_{j', j'}$ to $\widehat{\config}_{j} \sqcup \config'_{j'}$ obtained by executing $\widehat{\run}_{j}^-$ on the first part, then executing the last step of $\run'_{j'}$ to broadcast $m_{j'}$ with value $v$ and then executing $\run_{j}^+$ on the first part.

For each $i \in [j+1, \ell]$, let $\run''_i$ be the run that goes from $\widehat{\config}_{i-1} \sqcup \config'_{j'}$ to $\widehat{\config}_{i} \sqcup \config'_{j'}$ by executing $\run_i$ on the first part and staying idle on the second one. \corto{notation sqcup}

Note that for all $i \in [0, \ell]$, for all $v' \neq v$, if $\vproj{v'}{\run_i} \neq \epsilon$ then $\vproj{v'}{\run''_i} = \vproj{v'}{\run_i}$ (as we renamed values so that $\run$ and $\run'_j$ use disjoint sets of values apart from $v$) and if $\vproj{v'}{\run_i} = \epsilon$ then $\vinput{v'}{\run''_i} = \vinput{v'}{\run'_{j',i}} = \epsilon$.

Note that for all $i$ we have $w'_i \subword \voutput{v}{\widehat{\run}_i}$
We can apply Lemma~\ref{lem:follower-composition-completion} to the $\widehat{\run}_i$ and the $w'_i$.

We obtain a sequence of runs $\Tilde{\run}_0, \ldots, \Tilde{\run}_\ell$ such that, for all $i\in [0,\ell]$:
\begin{itemize}
	\item $\voutput{v}{\widehat{\run}_i} \subword \voutput{v}{\Tilde{\run}_i}$
	
	\item $\vinput{v}{\Tilde{\run}_i} = \epsilon$
	
	\item for all $v' \neq v$, if $\vproj{v'}{\widehat{\run}_i} \neq \epsilon$ then $\vproj{v'}{\widehat{\run}_i} = \vproj{v'}{\Tilde{\run}_i}$
	
	\item for all $v' \neq v$, if $\vproj{v'}{\widehat{\run}_i} = \epsilon$ then  $\vinput{v'}{\Tilde{\run}_i} = \epsilon$.
\end{itemize}

As a result, we have the following properties for all $i$:
\begin{itemize}
	\item $w'_i \subword \voutput{v}{\widehat{\run}_i} \subword \voutput{v}{\Tilde{\run}_i}$
	
	\item $\vinput{v}{\Tilde{\run}_i} = \epsilon$
	
	\item for all $v' \neq v$, if $\vproj{v'}{\run_i} \neq \epsilon$ then $\vproj{v'}{\run_i} = \vproj{v'}{\widehat{\run}_i} = \vproj{v'}{\Tilde{\run}_i}$
	
	\item for all $v' \neq v$, if $\vproj{v'}{\run_i} = \epsilon$ then either $\vproj{v'}{\widehat{\run}_i} = \epsilon$ and thus $\vinput{v'}{\Tilde{\run}_i} = \epsilon$, or $\vproj{v'}{\widehat{\run}_i} \neq \epsilon$ and then $\vproj{v'}{\Tilde{\run}_i} = \vproj{v'}{\widehat{\run}_i}$, thus in particular $\vinput{v'}{\Tilde{\run}_i} = \vinput{v'}{\widehat{\run}_i} = \epsilon$. 
\end{itemize}

Hence the $\Tilde{\run}_i$ satisfy all the required conditions.
\end{proof}

\LemTreeToRun*


	We prove the following statement by strong induction on the "tree unfolding". \cortoin{notation for t.u.}
	
	For all "tree unfolding" $\tau$, 
	
	\begin{itemize}
		\item if $\tau$ satisfies a "boss specification" $w \in \messages^*$, then there exists a run $\run$ satisfying $w$.
		
		\item if $\tau$ satisfies a "follower specification" $(\followwordspec, \followmessagespec)$ then there exists a "partial run" $\run$ and a value $v$ such that $\Input{\run} \in (\messages\times \set{v})^*$, $\vinput{v}{\run} \subword \followwordspec$, and $\voutput{v}{\run}$ contains $\followmessagespec$.
	\end{itemize}
	
	Let $\tau$ be a "tree unfolding", let $\node$ be its root.
	
	We see $\localrunlabel{\node}$ as a "partial run" with a single agent.
	We are going to compose $\localrunlabel{\node}$ with some runs given by the children of $\node$ to construct a run satisfying the properties above.
	Let $V$ be the set of values received or broadcast in $\localrunlabel{\node}$ and $V_{init}$ be the set of initial values of $\localrunlabel{\node}$.
	
	
	\subsection{Step 1: Non-initial values}
	\label{sec:tree-to-run-step-one}
	
	We show the following statement by induction on $\size{V'}$.
	
For all $V' \subseteq V \setminus (V_{init}\cup \set{\valuelabel{\node}})$, there exists a "partial run" $\run$ such that for all $v \in \nats$:
\begin{itemize}
	\item If $v \in V \setminus V'$ then $\vproj{v}{\run} = \vproj{v}{u}$
	
	\item If $v \in V' \cup \nats \setminus V$ then $\vinput{v}{\run} = \epsilon$
\end{itemize}  
	
	If $V' = \emptyset$ then this is clear as we can simply take $\run$.
	
	Now suppose there exists $v \in V \setminus (V_{init}\cup \set{\valuelabel{\node}} \cup V')$. Let $V'' = V' \cup \set{v}$. By induction hypothesis (on $\size{V'}$) there exists a "partial run" $\run'$ such that for all $v' \in \nats$:
	\begin{itemize}
		\item If $v' \in V \setminus V''$ then $\vproj{v'}{\run} = \vproj{v'}{u}$
		
		\item If $v' \in (\nats \setminus V) \cup V''$ then $\vinput{v'}{\run} = \epsilon$
	\end{itemize}

	As $v \neq \valuelabel{\node}$, by condition~\ref{unfoldingC3.2} $\node$ has a child $\node'$ such that $\vinput{v}{\localrunlabel{\mu}} \subword  \bosslabel{\node'}$.
	By induction hypothesis (on the "tree unfolding"), as the subtree rooted in $\mu'$ satisfies the "boss specification" $\vinput{v}{\mu}$, there exists a run $\run_v$ satisfying $\vinput{v}{\localrunlabel{\mu}}$.
	
	Let $v' \in \nats$ be such that $\vinput{v}{\localrunlabel{\node}} \subword \voutput{v'}{\run_v}$. 
	
	We apply Lemma~\ref{lem:boss-composition} to obtain a run $\Tilde{\run}$ such that 
		\begin{itemize}			
		\item $\vinput{v}{\Tilde{\run}} = \varepsilon$ 
		
		\item for all $v'' \neq v$, if $\vproj{v''}{\run} \neq \varepsilon$ then $\vproj{v''}{\Tilde{\run}} = \vproj{v''}{\run}$
		
		\item for all $v'' \neq v$, if $\vproj{v''}{\run} = \varepsilon$ then $\vinput{v''}{\Tilde{\run}} = \varepsilon$.
	\end{itemize}
	
	As a result, $\Tilde{\run}$ is a partial run satisfying the three requirements for $V''$.  This concludes our induction.
	
	In particular, with $V' = V\setminus (V_{init}\cup \set{\valuelabel{\node}})$, we obtain that there exists a run $\run$ such that 
	for all $v \in \nats$:
	\begin{itemize}
		\item $\voutput{v}{u} \subword \voutput{v}{\run}$
		
		\item If $v \in (V_{init}\cup \set{\valuelabel{\node}})$ then $\vproj{v}{\run} = \vproj{v}{u}$
		
		\item If $v \notin (V_{init}\cup \set{\valuelabel{\node}})$ then $\vinput{v}{\run} = \epsilon$
	\end{itemize}  
	
	\subsection{Step 2: Initial values}
	\label{sec:tree-to-run-step-two}
	
	We show the following statement by induction on $\size{V'}$.
	
	For all $V'$ such that $V \setminus (V_{init} \cup \set{\valuelabel{\node}}) \subseteq V' \subseteq V \setminus \set{\valuelabel{\node}}$, there exists a "partial run" $\run'$ such that for all $v \in \nats$:
	\begin{itemize}		
		\item If $v \in V \setminus V'$ then $\vproj{v}{\run'} = \vproj{v}{u}$
		
		\item If $v \in V' \cup (\nats \setminus V)$ then $\vinput{v}{\run'} = \epsilon$
	\end{itemize}  

If $V' = V \setminus (V_{init} \cup \set{\valuelabel{\node}})$ then this is clear as we can simply take the run $\run$ constructed in Section~\ref{sec:tree-to-run-step-one}.

Now suppose there exists $v \in V \setminus (\set{\valuelabel{\node}} \cup V')$. Let $V'' = V' \cup \set{v}$. By induction hypothesis (on $\size{V'}$) there exists a "partial run" $\run'$ such that for all $v' \in \nats$:
\begin{itemize}
	\item If $v' \in V \setminus V''$ then $\vproj{v'}{\run} = \vproj{v'}{u}$
	
	\item If $v' \in (\nats \setminus V) \cup V''$ then $\vinput{v'}{\run} = \epsilon$
\end{itemize}

As $v \neq \valuelabel{\node}$, by condition~\ref{unfoldingC3.1} there exists a "decomposition" $\decsymb = (w_0, m_1, \ldots, m_\ell, w_\ell)$ such that $u$ can be split into $u_0, \cdots, u_\ell$ so that for all $i \in [1, \ell]$, $w_i \subword \voutput{v}{u_i}$ and $\vinput{v}{u_i} \in \set{m_0, \ldots, m_{i-1}}^*$ and $\node$ has a child $\node_i$ such that $\followlabelmessage{\mu_i} = m_i$ and $\followlabelword{\node_i} \in \langdecdown{\decsymb_i}$, with $\decsymb_i = (w_0, m_1, \ldots, w_{i-1})$.

By induction hypothesis (on the "tree unfolding"), for each $i \in [1,\ell]$ there exists a run $\run'_i$ and a value $v'_i$ such that $\vinput{v''}{\run'_i} = \epsilon$ for all $v'' \neq v'$, $\vinput{v'_i}{\run'_i} \subword \followlabelword{\node_i}$ and $\voutput{v'_i}{\run'_i}$ contains $\followlabelmessage{\node_i}$. \corto{fix follower spec dec,m vs fw, fm}

For each $i$ we consider the shortest prefix $\run''_i$ of $\run'_i$ ending with an "internal message" $\intmessage{\followlabelmessage{\node_i}}{v'_i}$, we have that $\vinput{v'_i}{\run''_i} \subword \vinput{v'_i}{\run'_i} \subword \followlabelword{\node_i}$.

Furthermore, as $u$ can be split into $u_0, \cdots, u_\ell$ so that for all $i \in [1, \ell]$, $w_i \subword \voutput{v}{u_i}$ and $\vinput{v}{u_i} \in \set{m_0, \ldots, m_{i-1}}^*$, we can do the same for $\run$ as $v \in V_{init}$ and thus by definition of $\run$ we have $\vproj{v}{\run} = \vproj{v}{u}$.
Hence we obtain a sequence of runs $\run_0, \ldots, \run_\ell$ such that for all $i \in [1, \ell]$, $w_i \subword \voutput{v}{\run_i}$ and $\vinput{v}{\run_i} \in \set{m_0, \ldots, m_{i-1}}^*$.

We apply Lemma~\ref{lem:follower-composition-completion} to obtain a sequence of runs $\Tilde{\run}_0, \ldots, \Tilde{\run}_\ell$ such that, for all $i$, 
\begin{itemize}	
	\item $\vinput{v}{\Tilde{\run}_i} = \varepsilon$ 
	
	\item for all $v' \neq v$, if $\vproj{v}{\run_i} \neq \varepsilon$ then $\vproj{v'}{\Tilde{\run}} = \vproj{v'}{\run}$
	
	\item for all $v' \neq v$, if $\vproj{v}{\run} = \varepsilon$ then $\vinput{v''}{\Tilde{\run}} = \varepsilon$.
\end{itemize}

As a result, $\Tilde{\run}$ is a partial run satisfying the three requirements for $V''$.  This concludes our induction.

	In particular, with $V' = V\setminus \set{\valuelabel{\node}}$, we obtain that there exists a run $\widehat{\run}$ such that 
for all $v \in \nats$:
\begin{itemize}
	\item If $v = \valuelabel{\node}$ then $\vproj{v}{\widehat{\run}} = \vproj{v}{\run} = \vproj{v}{u}$
	
	\item If $v \neq \valuelabel{\node}$ then $\vinput{v}{\widehat{\run}} = \vinput{v}{\run} = \epsilon$
\end{itemize}  

\subsection{Step 3: $\valuelabel{\node}$}
	
In Section~\ref{sec:tree-to-run-step-two} we constructed a run $\widehat{\run}$ such that for all $v \in \nats$:
\begin{itemize}
	\item If $v = \valuelabel{\node}$ then $\vproj{v}{\widehat{\run}} = \vproj{v}{u}$
	
	\item If $v \neq \valuelabel{\node}$ then $\vinput{v}{\widehat{\run}} = \epsilon$
\end{itemize}  

We have two cases: either $\mu$ is a "boss node" or a "follower node".

If $\mu$ is a "boss node" then there exists a decomposition $\decsymb=(w_0, m_1, \ldots, w_\ell)$ such that $\bosslabel{\node} \in \langdecdown{\decsymb}$ and $u$ can be split into $u_0, \cdots, u_\ell$ so that for all $i \in [1, \ell]$, $w_i \subword \voutput{v}{u_i}$ and $\vinput{v}{u_i} \in \set{m_0, \ldots, m_{i-1}}^*$ and $\node$ has a child $\node_i$ such that $\followlabelmessage{\mu_i} = m_i$ and $\followlabelword{\node_i} \in \langdecdown{\decsymb_i}$, with $\decsymb_i = (w_0, m_1, \ldots, w_{i-1})$.

As $\bosslabel{\node} \in \langdecdown{\decsymb}$, we have $\bosslabel{\node} = w'_0 \cdots w'_\ell$ where for all $i$ the projection of $w'_i$ on $\messages\setminus\set{m_1, \ldots, m_{i-1}}$ is a subword of $w_i$. 

We repeat the arguments from Section~\ref{sec:tree-to-run-step-two}, but we use Lemma~\ref{lem:follower-composition-output} instead of Lemma~\ref{lem:follower-composition-completion} to obtain a run $\Tilde{\run}$ such that,
\begin{itemize}	
	\item $\vinput{v}{\Tilde{\run}} = \epsilon$ for all $v \in \nats$
	
	\item $\bosslabel{\node} \subword \voutput{\valuelabel{\node}}{\Tilde{\run}}$
\end{itemize}

If $\tau$ satisfies a "boss specification" $w$  then $w \subword \bosslabel{\node} $ and thus $w \subword \voutput{\valuelabel{\node}}{\Tilde{\run}}$.

As a result, $\Tilde{\run}$ satisfies $w$ as well.

If $\node$ is a "follower node" then the run $\widehat{\run}$ previously constructed is such that  

\begin{itemize}	
	\item $\vinput{v}{\Tilde{\run}} = \epsilon$ for all $v \neq \valuelabel{\node}$
	
	\item $\vinput{\valuelabel{\node}}{\widehat{\run}} = \voutput{\valuelabel{\node}}{\localrunlabel{\node}} \subword \followlabelword{\node}$
	
	\item $\voutput{\valuelabel{\node}}{\widehat{\run}} = \voutput{\valuelabel{\node}}{\localrunlabel{\node}}$ contains $\followlabelmessage{\node}$
\end{itemize}

Hence $\widehat{\run}$ satisfies the required properties.
This concludes our induction.
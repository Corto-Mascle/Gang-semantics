\section{Introduction}

\cortoin{plagiat}
We introduce a formal model of data-sensitive distributed protocols, called Broad-
cast Networks of Register Automata (BNRA), aimed at modelling both the local
knowledge of distributed nodes as well as their interaction via broadcast com-
munication. 


A network is modelled via a finite graph where each node runs an
instance of a common protocol. A protocol is specified via a register automaton,
an automaton equipped with a finite set of registers [20]. Each register assumes
values taken from the set of natural numbers. Node interaction is specified via
broadcast communication, well-suited to model scenarios in which individual
nodes have partial information aboutDST-RP13 the network topology. Messages are al-
lowed to carry data, that can be assigned to or tested against the local registers
of receivers. 

Dynamic updates of the current configuration are modelled via
non-deterministic reconfigurations of the underlying connectivity graph. A node
may disconnect from its neighbours and connect to other ones at any time of the
execution. This behaviour models in a natural way unexpected power-off and
dynamic movement of devices. The resulting model can be used to reason about
core parts of client-server protocols as well as of routing protocols, e.g. route
maintainance as in Link Reversal Routing.
\cortoin{plagiat}

This model was introduced in \cite{DST2013}, as a natural extension of Reconfigurable Broadcast Networks. That paper claimed that the coverability problem, i.e., the problem of finding a run from an initial configuration to one where at least one agent is in a designated state, was decidable and even \PSPACE-complete, but the proof turned out to be wrong. As we will see, the complexity of that problem is actually much higher.

In this paper we establish the decidability of the coverability problem. We even prove its completeness for the HyperAckermannian complexity class $\Falpha{\omega^\omega}$, thereby establishing that the problem requires a non-multiply recursive amount of time, and that it is in some sense as hard as lossy channel systems \corto{ref}. 
We contrast it with the undecidability of the target problem (asking for a run at the end of which all agents are in a given state) on one side, and with the \NP-completeness of the cover ability problem when each agent has only one register. 

\paragraph*{Related work}


\paragraph*{Overview}



	\textbf{Broadcast protocols}

	\cite{emerson1998model} -> Introduction of broadcast protocols (no reconfiguration, reliable broadcasts)
	
	\cite{EsparzaFM1999verification} -> Undecidability of liveness and safety for this model

	\textbf{Ad hoc Networks}

	\cite{Godskesen2007calculus}, \cite{Merro2007observational} -> From Arnaud's paper

	\textbf{RBN}
	
	\cite{DelzannoSZ2010Adhoc} -> Parametrized Model for Ad Hoc networks (undecidable), then switch to RBN
	
	\cite{Delzanno2012complexity} -> Complexity of RBN
	
	\cite{BouyerMRSS2016} -> ASMS almost-sure reachability
	
	\cite{fortin2017model} -> ASMS with stacks and leaders
	
	\cite{BalaW2021} -> Equivalence ASMS <-> RBN
	
	And some more ! \cite{BalasubramanianBM2018parameterized}, \cite{BalasubramanianGW2022parameterized}, 	\cite{ChiniMS2019liveness} 
	
	\textbf{Lossy channel systems}
	
	\cite{AbdullaJ1996verif} -> Introduction of LCS
	
	\cite{AbdullaJ1996undec} -> Büchi conditions are undecidable for LCS 
	
	\cite{SchmitzS2011upperHigman} -> Upper bounds for Higman's lemma
	
	\cite{ChambartS2008ordinal} -> Upper bound for LCS coverability 
	
	\cite{Schnoebelen2002verifying} -> Lower bound for LCS coverability 
	

	\textbf{Register automata}
	
	\cite{kaminski1994finite} -> Introduction of register automata
	
	\cite{segoufin2006automata} -> survey of results for register automata (didn't find anything more recent)
	
	\textbf{Data nets}
	
	\cite{lazic2007nets} -> Introduction of data nets
	
	\cite{ROSAVELARDO201741} -> Unordered data nets in $F_{\omega^\omega}$
	
	\textbf{Dynamic register automata}
	
	La thèse d'Othmane Rezine \cite{Rezine2017verification}, et des papiers associés 
	\cite{AbdullaAKR2014verification},
	\cite{AbdullaAKR2015verification}
	
	\textbf{Systems communicating via (lossy) channels}
	
	\cite{Aiswarya2021network} -> Survey, with some relevant papers 	\cite{Aiswarya2015model},
	\cite{AbdullaAA2016data}
	
	\textbf{Of interest}
	
	\cite{BertrandDGGG2018controlling} -> Controlling a population $\heartsuit$
	
	\cite{baldoni2009implementing} -> Implementation stuff, probably irrelevant
	 	
	\cite{DelzannoSZ2011cliquesAdhoc} -> More ad hoc networks
	
	\cite{DBLP:journals/tcs/AbdullaDRST16} -> Even more
	
	\cortoin{J'ai pris les refs de Scholar, mais il vaut mieux les refaire depuis DBLP}
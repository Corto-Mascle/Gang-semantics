\section{Introduction}

We study a model representing networks of processes communicating by broadcast. We make no assumption on the evolution of the communication graph, i.e., when a process broadcasts a message, each other process, may receive it or not, independently. This models unreliable systems, with unexpected crashes and disconnections, in a pessimistic way. 
Furthermore, each agent has access to some local registers, containing natural numbers (initially, identifiers). Each broadcast from an agent consists in a message from a finite alphabet and a value from one of their registers. The agents receiving it may store the value in their registers or compare it with its register contents. 
This allows an agent, for instance, to check that a sequence of messages all come from the same agent by storing the value of the first message and then checking that all the following ones have the same value.
This also allows agents to get confirmations that a message was received, by broadcasting it with their identifier and then waiting for a response with that same value.

This model was introduced in \cite{DelzannoST13}, as a natural extension of Reconfigurable Broadcast Networks~\cite{DelzannoSZ2010Adhoc}. That first paper claimed that the coverability problem, i.e., the problem of finding a run from an initial configuration to one where at least one agent is in a designated state, was decidable and even \PSPACE-complete, but the proof turned out to be wrong. As we will see, the complexity of that problem is actually much higher.

In this paper we establish the decidability of the coverability problem. We even prove its completeness for the hyper-Ackermannian complexity class $\Fcomplexity{\omega^\omega}$, thereby showing that the problem requires a non-multiply recursive amount of time, and that it is in some sense as hard as reachability for lossy channel systems (transition systems which can store some letters in an unreliable FIFO memory, in which any letter may be erased at any time~\cite{AbdullaJ1996verif}). 
We contrast it with the undecidability of the target problem (asking for a run at the end of which all agents are in a given state) on one side, and with the \NP-completeness of the coverability problem when each agent has only one register. 

\paragraph*{Related work}

Broadcast protocols are a widely studied class of systems, in which processes are organised in a graph and can send messages to their neighbours. There are however many choices to make when designing a model for those systems: how individual processes are represented, whether the number of communication graph is fixed or can change, the type of messages they can send... 
A model where messages ranged over a finite alphabet was presented in~\cite{emerson1998model}, over a fully connected communication graph. It was rapidly shown that many basic problems were undecidable over that model~\cite{EsparzaFM1999verification}. This negative results, along with similar ones for Ad Hoc networks, lead to Reconfigurable Broadcast Networks (RBN), a new formalism where each broadcast can be received by an arbitrary subset of agents~\cite{DelzannoSZ2010Adhoc}.

The complexity of parameterized verification problems over RBN have been the subject of an extensive study in recent years~\cite{DelzannoSTZ12, Balasubramanian18, BalasubramanianGW22, DBLP:journals/computing/ChiniMS22}. In~\cite{DelzannoST13}, RBN were extended with registers allowing processes to exchange identifiers: this is the model we consider here. It is inspired by the success of register automata, which offer a convenient formalism to express properties of words over an infinite alphabet; see~\cite{Segoufin06} for a survey on the subject.

Other approaches exist to define parameterized models with registers~\cite{BolligRS21}, for instance Dynamic register automata, in which processes are allowed to spawn other processes with new identifiers and communicate integers values~\cite{AbdullaAKR14}. While basic problems on these models are in general undecidable, some restrictions on communications allow to recover decidability~\cite{AbdullaAKR15, Rezine17}.

Such parameterized verification problems often relate to the theory of well quasi-orders and well-structured transition systems, and the associated high complexities obtained from bounds on so-called ``bad sequences'' in ordered sets. In particular, our model is linked to two classical models from this field: First data nets, which are Petri nets in which tokens are labelled with natural numbers, and can exchange and compare their labels~\cite{LazicNORW08}. In general, inequality tests are allowed, but data nets with only equality have also been studied~\cite{Rosa-Velardo17}. They do not subsume BNRA as processes can only carry one integer, and not several register values (problems on models of data nets where tokens have tuples of integers as labels are typically undecidable).
The other closely related model is lossy channel systems (LCS)\cite{AbdullaJ1996undec}, which we will simulate using BNRA. We will also show that BNRA coverability and LCS reachability are complete for the same complexity class $\Fcomplexity{\omega^\omega}$~\cite{ChambartS08ordinal, Schnoebelen2002verifying}.
As a matter of fact, LCS are derived from distributed models with processes communicating through pairwise channels, which are a rich field of study on their own~\cite{Aiswarya2015model,Aiswarya2020networks}.

\paragraph*{Overview}


We start with the model definition and some preliminary results in Section~\ref{sec:preliminaries}. We prove the main decidability result in Section~\ref{sec:cover-decidability}, starting with some basic definitions in Section~\ref{sec:decidability-set-up}, then presenting a sound and complete abstraction, unfolding trees, in~\ref{sec:decidability-tree-unfoldings}, and bounding the size of witnesses in that abstraction in Sections~\ref{sec:tree-bounds}. We establish the coverability decidability theorem in Section~\ref{sec:decidability-end} and the undecidability of a closely-related problem in Section~\ref{sec:undec-target}. Finally, we prove the \NP-completeness of the coverability problem with one register per process in Section~\ref{sec:cover-1BNRA}.
Due to space constraints, we only give proof sketches in the main part of the paper, while the formal proofs can be found in the appendix.

%
%	\textbf{Broadcast protocols}
%
%	\cite{emerson1998model} -> Introduction of broadcast protocols (no reconfiguration, reliable broadcasts)
%	
%	\cite{EsparzaFM1999verification} -> Undecidability of liveness and safety for this model
%
%	\textbf{Ad hoc Networks}
%
%	\cite{Godskesen2007calculus}, \cite{Merro2007observational} -> From Arnaud's paper
%
%	\textbf{RBN}
%	
%	\cite{DelzannoSZ2010Adhoc} -> Parametrized Model for Ad Hoc networks (undecidable), then switch to RBN
%	
%	\cite{Delzanno2012complexity} -> Complexity of RBN
%	
%	\cite{BouyerMRSS2016} -> ASMS almost-sure reachability
%	
%	\cite{fortin2017model} -> ASMS with stacks and leaders
%	
%	\cite{BalaW2021} -> Equivalence ASMS <-> RBN
%	
%	And some more ! \cite{BalasubramanianBM2018parameterized}, \cite{BalasubramanianGW2022parameterized}, 	\cite{ChiniMS2019liveness} 
%	
%	\textbf{Lossy channel systems}
%	
%	\cite{AbdullaJ1996verif} -> Introduction of LCS
%	
%	\cite{AbdullaJ1996undec} -> Büchi conditions are undecidable for LCS 
%	
%	\cite{SchmitzS2011upperHigman} -> Upper bounds for Higman's lemma
%	
%	\cite{ChambartS2008ordinal} -> Upper bound for LCS coverability 
%	
%	\cite{Schnoebelen2002verifying} -> Lower bound for LCS coverability 
%	
%%
%%	\textbf{Register automata}
%%	
%%%	\cite{kaminski1994finite} -> Introduction of register automata
%%	
%%	\cite{segoufin2006automata} -> survey of results for register automata (didn't find anything more recent)
%%	
%	\textbf{Data nets}
%	
%	\cite{lazic2007nets} -> Introduction of data nets
%	
%	\cite{ROSAVELARDO201741} -> Unordered data nets in $F_{\omega^\omega}$
%	
%	\textbf{Systems communicating via (lossy) channels}
%	
%	\cite{Aiswarya2021network} -> Survey, with some relevant papers 	\cite{Aiswarya2015model},
%	\cite{AbdullaAA2016data}
%
%	\cite{BRS21} -> not really register automata, but systems where each agent has a private register and there are as many shared registers as agents; an agent may change its local value, write it in the global memory and test the number of occurrences of its values. Control-state reachability is \PSPACE-complete.
	
%	\cortoin{J'ai pris les refs de Scholar, mais il vaut mieux les refaire depuis DBLP}
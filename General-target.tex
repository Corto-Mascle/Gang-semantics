\subsection{Undecidability of \TARGET}
\label{sec:undec-target}

The natural next problem, after \COVER, is the "target reachability problem" (\TARGET).  
While our \COVER procedure produces a lot of waste, in the sense that we can add agents at no cost, for \TARGET we need to guarantee that those agents can later reach the target state, which makes the problem harder. 
We in fact show that \TARGET is undecidable, which suggests that while we can obtain decidability of \COVER thanks to some monotonicity properties of the problem, we cannot analyse more precisely the set of runs of such systems.

\begin{restatable}{proposition}{propTargetUndecidable}
\label{prop:target-undec}
\TARGET is undecidable for \BNRA with two registers.
\end{restatable}

\begin{proof}[Proof sketch]
We simulate a Minsky machine with two counters. Like for the LCS encoding, the first register is never modified and plays the role of a permanent identifier. We start with some initialisation phase where each agent store some other agent's identifier in their second register, which will be their ``predecessor''; it then only accepts messages from its predecessor. As there are finitely many agents, every agent is in a cycle in the predecessor graph. 

In a given cycle, we exploit the fact that \emph{all} agents must reach state $q_f$ to simulate faithfully a Minsky machine: we make agents alternate between receptions and broadcasts so that in the end they have received and sent the same number of messages, implying that no messages has been lost along the cycle.
With this guarantee, we simulate the machine by having an agent (the leader) choose transitions of the machine and the other ones simulate the counter values by memorizing a counter ($1$ or $2$) and a value ($0$ or $1$). For instance, an increment of counter $1$, initiated by the leader, takes the form of a message propagate in the cycle until it finds an agent simulating counter $1$ and having bit $0$. This agent switches to $1$ and sends an acknowledgment that eventually propagate back to the leader.
\end{proof}

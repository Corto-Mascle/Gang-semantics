\subsection{Undecidability of \TARGET}
\label{sec:undec-target}

The "target reachability problem" is a natural next step after coverability. 
It requires a more precise description of runs: while for \COVER we can produce a lot of waste, in the sense that we can add as many agents as we need to ensure that one of them reaches the given state, for \TARGET we need to guarantee that those agents can later reach the target state.

We show that \TARGET is undecidable for BNRAs, which suggests that while we can obtain decidability of \COVER thanks to the monotonicity of the problem, we cannot analyse precisely the set of runs of such systems.


\begin{restatable}{proposition}{propTargetUndecidable}
\label{prop:target-undec}
	The "target reachability problem" (asking whether there is a run in which all agents reach a state $q_f$) is undecidable for \BNRA with two registers.
\end{restatable}

\begin{proof}[Proof sketch]
We simulate a Minsky machine with two counters. In an initialisation phase, all agents record a value  sent by another agent in their second register, which will be their ``predecessor''. Throughout the run they will only listen to broadcasts from that agent, and only broadcast with the initial value in their first register. As there are finitely many agents, there must be a cycle in the predecessor graph.

We exploit the fact that \textbf{all} agents must reach state $q_f$ to simulate faithfully a Minsky machine: we make agents alternate between receptions and broadcasts so that in the end they have received and sent the same number of messages, implying that all messages have been received along the cycle.
Once we guarantee that no messages are lost, we can simulate the machine by having an agent (the leader) choose transitions of the machine and the other ones simulate the counter values by memorizing a counter 1 or 2 and a bit 0 or 1. For instance, an increment of counter 1 is done by having the leader send a message around the cycle, which eventually finds an agent simulating counter 1 and with bit 0, which switches to 1 and sends a confirmation message to confirm to the leader that the operation has been executed.	
\end{proof}

\section{Lower bounds}
\label{sec:lower-bounds}

\begin{ex}
	In some cases a protocol may require a "run" of exponential length (in the number of states) to reach a configuration satisfying a "query".
	
	\begin{figure}[h]
		\begin{tikzpicture}[xscale=0.5,AUT style,node distance=2.1cm,auto,>= triangle
	45]
	\tikzstyle{initial}= [initial by arrow,initial text=,initial
	distance=.7cm]
	%	\tikzstyle{accepting}= [accepting by arrow,accepting text=,accepting
	%	distance=.7cm,accepting where =right]
	
	\node[state,initial, minimum width=0.1pt] (0a) at (0,0) {0};
	\node[state] [below of=0a] (0b) {};
	\node[state] [below of=0b] (0c) {$q_0$};
	\node[state] [right of=0a] (0') {};
	
	\node[state] [right of=0'] (1a) {1};
	\node[state] [below of=1a] (1b) {};
	\node[state] [below of=1b] (1c) {$q_1$};
	
	\node [right of=1a, xshift=-30pt] (token1) {};
	\node [right of=token1, xshift=-40pt] (dots) {\huge $\cdots$};
	\node [right of=dots, xshift=-40pt] (token2) {};
	
	\node[state] [right of=token2, xshift=-30pt] (Na) {N};
	\node[state] [below of=Na] (Nb) {};
	\node[state] [below of=Nb] (Nc) {$q_N$};
	
	\node[state] [right of=Na] (N') {};
	\node[state] [right of=N'] (end) {N+1};
	
	\path[->, bend left=20] 	
	;
	\path[->, bend right=20] 
	;
	\path[->]
	(0a) edge node[right] {$\rec(m_0, \enregact)$} (0b)
	(0b) edge node[right] {$\br(m'_0)$} (0c)
	
	(1a) edge node[right] {$\rec(m_1, \enregact)$} (1b)
	(1b) edge node[right] {$\br(m'_1)$} (1c)
	
	(Na) edge node[right] {$\rec(m_N, \enregact)$} (Nb)
	(Nb) edge node[right] {$\br(m'_N)$} (Nc)
	
	(0a) edge node {$\br(m_0)$} (0')
	(0') edge node {$\rec(m'_0, \eqtestact)$} (1a)
	
	(Na) edge node {$\br(m_N)$} (N')
	(N') edge node {$\rec(m'_N, \eqtestact)$} (end)
	
	(1a) edge node {} (token1)
	(token2) edge node {} (Na)
	;
\end{tikzpicture}
		\label{fig:exp-run}
		\caption{Example of a "protocol" with a state that requires an exponentially long "run" to be covered.}
	\end{figure}

	Consider the "protocol" displayed in Figure~\ref{fig:exp-run}. 
	For all $n \in \set{0, \ldots, N+1}$ we call \emph{future} of $n$ the set of states accessible from state $n$ by a path in the protocol.
	
	There is an initial run covering $N+1$: we start with $2^{N+1}$ agents, which we pair two by two. We make each agent broadcast $m_0$ while its partner receives it, storing its register value and broadcasts $m'_0$, which the first one receives. After doing that with all pairs of agents we have $2^{N}$ agents in state $1$.
	We repeat this process to put $2^{N+1-n}$ agents in state $n$ for each $n$, one after the other. In the end we have an agent in state $N+1$.
	
	We show by induction on $n$ that for all $k \in \nats$, any reachable configuration with at least $k$ agents in states of the future of $n$ has at least $k(2^n-1)$ agents in states $\set{q_i \mid i<n}$.
	
	For $n=0$ this is immediate. Let $n>0$, suppose the property true for $n-1$, then let $k \in \nats$ and let $\config$ be a configuration with at least $k$ agents in the future of $n$. Let $\agents_+$ be the set of those agents, let $a_+ \in \agents_+$, $a_+$ has taken the transitions broadcasting $m_{n-1}$ and receiving $m'_{n-1}$ with its own register value. As it never changed its identifier, the reception must contain its initial value $r$. Hence there exists an agent $a_-$ which broadcasts $m'_{n-1}$ at some point while having value $r$. All such $a_-$ are distinct as they all broadcast only once with $m'_{n-1}$ and the $a_+$ have distinct initial values.
	Hence there are at least $k$ agents in state $q_{n-1}$, hence at least $2k$ in the future of $n-1$. By induction hypothesis this means there are $2k(2^{n-1}-1) = k(2^n-2)$ in $\set{q_i \mid i<n-1}$, thus at least $k(2^n-1)$ in $\set{q_i \mid i<n}$.
	
	As a consequence, any initial run covering $N+1$ needs to send at least $2^N-1$ agents to the $q_i$ states, hence it needs to be exponentially long as each of those agents needs to execute at least one broadcast.
\end{ex}

\begin{proposition}
	\label{prop:np-hard-query-cover}
	The "query coverability problem" is \np-hard.
\end{proposition}

\begin{proof}
	\begin{figure}[h]
		\begin{tikzpicture}[xscale=0.5,AUT style,node distance=2cm,auto,>= triangle
	45]
	\tikzstyle{initial}= [initial by arrow,initial text=,initial
	distance=.7cm]
	%	\tikzstyle{accepting}= [accepting by arrow,accepting text=,accepting
	%	distance=.7cm,accepting where =right]
	
	\node[state,initial, minimum width=0.1pt] (0) at (0,0) {0};
	
	\node[state] [right of=0] (1) {1};
	
	\node [right of=1, xshift=-20pt] (token1) {};
	\node [right of=token1, xshift=-50pt] (dots) {\huge $\cdots$};
	\node [right of=dots, xshift=-50pt] (token2) {};
	
	
	\node[state] [right of=token2, xshift=-20pt] (N-1) {N-1};
	\node[state] [right of=N-1] (N) {N};
	
	\node[state] [right of=N] (1') {1'};
	
	\node [right of=1', xshift=-20pt] (token3) {};
	\node [right of=token3, xshift=-50pt] (dots2) {\huge $\cdots$};
	\node [right of=dots2, xshift=-20pt] (token4) {};
	
	
	\node[state] [right of=token4, xshift=-40pt] (m-1) {m-1'};
	\node[state] [right of=m-1] (m) {m'};
	
	\coordinate[below of=1] (stop);
	\node[state] [below right of=stop] (r1) {$x_1$};
	\node [above left of=r1, yshift=-20pt, xshift=10pt] (r1') {$\rec(x_1, \enreg)$};
	\node[state] [right of=r1] (r1b) {$\neg x_1$};
	\node [above left of=r1b, yshift=-20pt, xshift=10pt] (r1b') {$\rec(\neg x_1, \enreg)$};
	\node [right of= r1b] (dots3) {\huge $\cdots$};
	\node[state] [right of=dots3] (rn) {$x_n$};
	\node [above left of=rn, yshift=-20pt, xshift=10pt] (rn') {$\rec(\neg x_n, \enreg)$};
	\node[state] [right of=rn] (rnb) {$\neg x_n$};
	\node [above left of=rnb, yshift=-20pt, xshift=10pt] (rnb') {$\rec(\neg x_n, \enreg)$};
	
	\draw (0) .. controls +(0,-2) and +(-1,0) .. (stop);
	\draw[->] (stop) .. controls +(2,0) and +(0,1) .. (r1);
	\draw[->] (stop) .. controls +(6,0) and +(0,1) .. (r1b);
	\draw[->] (stop) .. controls +(14,0) and +(0,1) .. (rn);
	\draw[->] (stop) .. controls +(18,0) and +(0,1) .. (rnb);
	\path[->, bend left=20]
	(0) edge node[above] {$\br(x_1)$} (1)
	(N-1) edge node[above] {$\br(x_N)$} (N) 	
	;
	\path[->, bend left=40]
	(N) edge node[above] {$\rec(\ell_1^1, \testeq)$} (1')
	(m-1) edge node[above] {$\rec(\ell_m^1, \testeq)$} (m) 	
	;
	\path[->, bend right=20] 
	(0) edge node[below] {$\br(\neg x_1)$} (1)
	(N-1) edge node[below] {$\br(\neg x_N)$} (N) 
	;
	\path[->, bend right=30] 
	(N) edge node[below] {$\rec(\ell_1^3, \testeq)$} (1')
	(m-1) edge node[below] {$\rec(\ell_m^3, \testeq)$} (m)	
	;
	\path[->]
	(N) edge node[above] {$\rec(\ell_1^2, \testeq)$} (1')
	(m-1) edge node[above] {$\rec(\ell_m^2, \testeq)$} (m) 	
	;
	\path[->, loop below]
	(r1) edge node[below] {$\br(x_1)$} (r1)
	(r1b) edge node[below] {$\br(\neg x_1)$} (r1b) 	
	(rn) edge node[below] {$\br(x_n)$} (rn)
	(rnb) edge node[below] {$\br(\neg x_n)$} (rnb) 	
	;
\end{tikzpicture}
		\label{fig:np-hard}
		\caption{The "protocol" used for the \np-hardness proof.}
	\end{figure}
	
	We reduce from the 3SAT problem.
	Let $x_1, \ldots, x_n$ be variables and $\query = \bigwedge_{j=1}^m C_j$ with, for all $j$, $C_j = \ell_j^1 \lor \ell_j^2 \lor \ell_j^3$ and $\ell_j^1, \ell_j^2, \ell_j^3 \in \set{x_i, \neg x_i \mid 1 \leq i \leq n}$. 

	Consider the "protocol" displayed in Figure~\ref{fig:np-hard}.
	Our alphabet of messages is the set of literals $\set{x_i, \neg x_i \mid 1 \leq i \leq n}$.
	Each agent may either receive a message, and repeat it forever or it may broadcast one of $x_i, \neg x_i$ for each $i$ and then try to receive a message one of $\ell_j^1, \ell_j^2, \ell_j^3$ for each $j$, with its own register value.
	
	Suppose $\query$ is satisfiable, let $\nu$ be a satisfying assignment, then we set $\agents = \set{a, a_1, \ldots, a_n}$ as our set of agents. First for each $i$ we make $a$ broadcast $x_i$ if $\nu(x_i)= \top$ and $\neg x_i$ otherwise, and $a_i$ only receive it (and go to the corresponding state).
	Then for each $j$ we select some $\ell_j^p$ satisfied by $\nu$. There exists $i$ such that $a_i$ is in state $\ell_j^p$. It broadcasts $\ell_j^p$ along with the initial register value of $a$, allowing $a$ to go to the next state.
	
	As a result, there is an execution in which agent $a$ reaches $m'$.
	
	Now suppose there is an execution $\run$ over some set of agents $\agents$ such that some agent $a \in \agents$ is in state $m'$ in the final configuration.
	For each $i$, $a$ has broadcast either $x_i$ or $\neg x_i$, but not both.
	Let $\nu$ be the valuation assigning $\top$ to $x_i$ if and only if $a$ has broadcast it.
	The register of $a$ cannot have changed its value throughout the run. 
	For each $j$ it has received one of $\ell_j^1, \ell_j^2, \ell_j^3$ along with its own initial register value (which we call $r$). Let $p_j$ be such that $a$ has received $\ell_j^{p_j}$.
	Hence for all $j$ there exists an agent $a_j \in \agents$ such that at some point in the run the register value of $a_j$ is $r$ and $a_j$ broadcasts $\ell_j^{p_j}$.
	This agent $a_j$ must be in state $\ell_j^{p_j}$ after receiving the broadcast $\br{\ell_j^{p_j}}$ from $a$ (as all agents start with different register values).
	Hence $\ell_j^{p_j}$ is satisfied by $\nu$. 
	
	As a result, $\nu$ satisfies a literal of each clause of $\query$, and thus satisfies $\query$. This concludes our reduction
\end{proof}

\begin{itemize}
\item def tower and repexp
\item s'assurer de la coherence step / transition
\item definir proprement "target" et "cover"
\luin{proposition de definition après query coverability problem }
\item mettre au bon endroit la definition de "partial run": 
	% A ""partial run"" is a sequence of configurations $\config_0 \cdots \config_k$  such that for all $i \in [1, k]$, either $\config_{i-1} \step{} \config_{i}$ or $\config_{i-1} \extbr{m, v} \config_{i}$ for some $m \in \messages$, $v\in \nats$. 
    \item passer $\regnum$ dans le tuple du protocole
    \item reflechir nomenclature internal broadcast et autres -> external message
    \item decompose as -> peut créer des ambiguités: machintruc decomposition 
    \item reduction query vers cover un registre
    \item reduction query vers cover dans le cas d'un registre à l'aide d'une habile remarque
    \item figure lemme tower
    \item[nico] encodage lcs
    \item[nico] remarque buchi
    \item[nico] remarque query 
    \item[nico] remarque on a besoin de tower
    \item remarques automates à pile    
    \item harmoniser bnra - protocol
    \item remplacer schedule par trace
    \item modifier le lemme 21 pour qu'il facilite la preuve du lemme 23
    \item verifier que la notation de $r$-BNRA est uniforme et bien definie
    \item  verifier broadcasted -> broadcast
\end{itemize}

A discuter:
\begin{itemize}
\item toutes les valeurs dans les différents registres sont différentes ? Cela me semble étrange vis-à-vis de la notion d'identifiant
\end{itemize}
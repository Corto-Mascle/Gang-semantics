\begin{itemize}
	\item s'assurer de la coherence step / transition
%	\item mettre au bon endroit la definition de "partial run": 
	% A ""partial run"" is a sequence of configurations $\config_0 \cdots \config_k$  such that for all $i \in \nset{1}{k}$, either $\config_{i-1} \step{} \config_{i}$ or $\config_{i-1} \extbr{m, v} \config_{i}$ for some $m \in \messages$, $v\in \nats$. 
%    \item passer $\regnum$ dans le tuple du protocole
    \item reduction query vers cover un registre
    \item reduction query vers cover dans le cas d'un registre à l'aide d'une habile remarque
    \item[nico] remarque query 
    \item[nico] remarque on a besoin de tower
    \item remarques automates à pile
    \item modifier le lemme 21 pour qu'il facilite la preuve du lemme 23
    \item verifier que la notation de $r$-BNRA est uniforme et bien definie
    \item  verifier broadcasted -> broadcast
    \item definir $\nset{1}{n}$
    \item run pas toujours initial dans les preuvves :-) + local run qui n'est pas initial -> il faudrait peut-être changer run pour qu'il ne soit pas forcément initial
    \item mentionner que si plusieurs valeurs par message, c'est indécidable d'après le papier d'Arnaud
    \item message et message type à distinguer 
    \item citer le site web d'Arnaud et expliquer que le papier il est tout cassé
    \item trouver une bonne façon que les conditions (ii) soient citées avec la parenthèse
\end{itemize}

A discuter:
\begin{itemize}
\item external message -> ambiguite entre message et step 
\end{itemize}
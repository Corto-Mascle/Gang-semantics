\begin{itemize}
\item def tower and repexp
\item definir local step
\item s'assurer de la coherence step / transition
\item ecrire quelque part qu'un run local inclut sa séquence de messages car un seul message convient par réception sous l'hypothèse de s'être débarassé des $\dummyact$ et des $\diseqtestact$: faire une remarque \label{rem:local_transition_unique_message}
\item definir proprement "target" et "cover"
\item mettre au bon endroit la definition de "partial run": 
	% A ""partial run"" is a sequence of configurations $\config_0 \cdots \config_k$  such that for all $i \in [1, k]$, either $\config_{i-1} \step{} \config_{i}$ or $\config_{i-1} \extbr{m, v} \config_{i}$ for some $m \in \messages$, $v\in \nats$. 
    \item choisir une seule macro de $\val$ et $\aval$
    \item passer $\regnum$ dans le tuple du protocole
    \item reflechir nomenclature internal broadcast et autres -> external message
    \item decompose as -> peut créer des ambiguités: machintruc decomposition 
\end{itemize}

A discuter:
\begin{itemize}
\item toutes les valeurs dans les différents registres sont différentes ? Cela me semble étrange vis-à-vis de la notion d'identifiant


\end{itemize}
\section{Definitions}

\begin{definition}
	A ""protocol"" is a tuple $(Q, \D, q_0)$ over an alphabet $\S$ with $Q$ a finite set of states, $q_0 \in Q$ an initial state, and $\D \subseteq Q \times \set{\br, \rec(*), \rec(\cp), \rec(\testeq)} \times Q$.
\end{definition}

\begin{definition}
	Let $(Q,\D, q_0)$ be a protocol.
	A ""configuration"" is a pair $(A, L)$ with $A$ a finite set of \emph{agents}   and $L : A \to Q \times \nats$  a labelling function, associating a state and a register value to each agent. We write $L_Q$ for the state component of $L$ and $L_R$ for its register component.
	An \emph{initial configuration} $(A, L)$ is one where for all $a \in A$, $L_Q(a) = q_0$.
	
	\AP Given two configurations $\g = (A, L)$ and $\g = (A', L')$, a ""transition"" $\g 
	\act{m} \g'$ between two configurations is defined when $A = A'$ and there exists $a_0 \in A$ such that $L_Q(a_0) \act{\br} L'_Q(a_0) \in \D$, $L_R(a_0) = L'_R(a_0)$ and for all $a \in A\setminus \set{a_0}$,  
	\begin{itemize}
		\item either $L'(a) = L(a)$
		
		\item or $L'(a)$ is such that $L_Q(a) \act{\rec(\a)} L'_Q(a) \in \D$ with either
		\begin{itemize}
			\item $\a = *$ and $L'_R(a) = L_{R}(a)$
			\item $\a = \cp$ and $L'_R(a) = m$
			\item $\a = \testeq$ and $L'_R(a) = L_R(a) =m$
		\end{itemize}
	\end{itemize}

	\AP A ""run"" is a sequence of transitions $\g_0 \act{m_0} \g_1 \act{m_1} \cdots \act{m_k} g_k$. We say that it is \emph{initial} if $g_0$ is an initial configuration.
\end{definition}

\begin{definition}
	A ""query"" is a conjunction of formulas of the form $q(z)$, $R(z) = R(z')$, $R(z) \neq R(z')$, with $z, z'$ taken in a set of variables $Var$.
	It is \emph{satisfied} by a "configuration" $(A, L)$ if there is a valuation $\nu : Var \to A$ such that:
	\begin{itemize}
		\item for all conjuncts $q(z)$, $L_Q(\nu(z)) = q$
		
		\item for all conjuncts $R(z) = R(z')$, $L_R(\nu(z)) = L_R(\nu(z'))$
		
		\item for all conjuncts $R(z) \neq R(z')$, $L_R(\nu(z)) \neq L_R(\nu(z'))$
	\end{itemize}

	\AP The ""query coverability problem"" is to determine, given a "protocol" and a "query", whether there is an "initial run" of this protocol whose last "configuration" satisfies the query.
\end{definition}

\begin{ex}
	In some cases a protocol may require a "run" of exponential length (in the number of states) to reach a configuration satisfying a "query".
	
	\begin{tikzpicture}[xscale=0.5,AUT style,node distance=2.1cm,auto,>= triangle
	45]
	\tikzstyle{initial}= [initial by arrow,initial text=,initial
	distance=.7cm]
	%	\tikzstyle{accepting}= [accepting by arrow,accepting text=,accepting
	%	distance=.7cm,accepting where =right]
	
	\node[state,initial, minimum width=0.1pt] (0a) at (0,0) {0};
	\node[state] [below of=0a] (0b) {};
	\node[state] [below of=0b] (0c) {$q_0$};
	\node[state] [right of=0a] (0') {};
	
	\node[state] [right of=0'] (1a) {1};
	\node[state] [below of=1a] (1b) {};
	\node[state] [below of=1b] (1c) {$q_1$};
	
	\node [right of=1a, xshift=-30pt] (token1) {};
	\node [right of=token1, xshift=-40pt] (dots) {\huge $\cdots$};
	\node [right of=dots, xshift=-40pt] (token2) {};
	
	\node[state] [right of=token2, xshift=-30pt] (Na) {N};
	\node[state] [below of=Na] (Nb) {};
	\node[state] [below of=Nb] (Nc) {$q_N$};
	
	\node[state] [right of=Na] (N') {};
	\node[state] [right of=N'] (end) {N+1};
	
	\path[->, bend left=20] 	
	;
	\path[->, bend right=20] 
	;
	\path[->]
	(0a) edge node[right] {$\rec(m_0, \enregact)$} (0b)
	(0b) edge node[right] {$\br(m'_0)$} (0c)
	
	(1a) edge node[right] {$\rec(m_1, \enregact)$} (1b)
	(1b) edge node[right] {$\br(m'_1)$} (1c)
	
	(Na) edge node[right] {$\rec(m_N, \enregact)$} (Nb)
	(Nb) edge node[right] {$\br(m'_N)$} (Nc)
	
	(0a) edge node {$\br(m_0)$} (0')
	(0') edge node {$\rec(m'_0, \eqtestact)$} (1a)
	
	(Na) edge node {$\br(m_N)$} (N')
	(N') edge node {$\rec(m'_N, \eqtestact)$} (end)
	
	(1a) edge node {} (token1)
	(token2) edge node {} (Na)
	;
\end{tikzpicture}
	
	\cortoin{TODO}
\end{ex}

\begin{proposition}
	The "query coverability problem" is \np-hard.
\end{proposition}

\begin{proof}
	\cortoin{TODO}
\end{proof}
 
\section{Definitions}

\begin{definition}
	A ""protocol"" is a tuple $\prot = (Q, \messages, \transitions, q_0)$  with $Q$ a finite set of states, $q_0 \in Q$ an initial state, $\messages$ a finite set of ""messages""  and $\transitions \subseteq Q \times \set{\br{\amessage}, \rec{\amessage}{\dummyact}, \rec{\amessage}{\enregact}, \rec{\amessage}{\eqtestact}, \rec{\amessage}{\diseqtestact} \mid \amessage \in \messages} \times Q$.
	
	Transitions labelled with $\brsymb$ are ""broadcasts"" and transitions labelled with $\recsymb$ are ""receptions"".
Given a reception $(q,\rec{\amessage}{\anact}, q') \in \transitions$, $\anact$ is the ""action"" of the reception.
\end{definition}

\begin{definition}
	Let $(Q,\messages, \transitions, q_0)$ be a protocol, and $\agents \subseteq \nats$ a finite set of \emph{agents}.
	A ""configuration"" over a set of agents $\agents$ is a labelling function $\config : \agents \to Q \times \nats$, associating a state and a ""register value"" to each agent. We write $\st{\config}$ for the state component of $\config$ and $\data{\config}$ for its register component. 
	An \emph{initial configuration} $\config$ is one where for all $a \in \agents$, $\st{\config}(a) = q_0$ and for all $a, a' \in \agents$, if $\data{\config}(a) = \data{\config}(a')$ then $a=a'$.
	
	\AP We denote by $\configs{\agents}$ the set of configurations over $\agents$, and by $\allconfigs := \cup_{\agents \subseteq \nats \text{ finite }} \configs{\agents}$ the set of all configurations. Given a configuration $\config$, we denote by $\agentsof{\config}$ the set of agents of $\config$.

	\AP Given two configurations $\config, \config' \in \configs{\agents}$, a ""step"" $\config \step{} \config'$ is defined when there exists $\amessage \in \messages$ and $a_0 \in \agents$ such that $(\st{\config}(a_0),\br{\amessage}, \st{\config'}(a_0)) \in \transitions$, $\data{\config}(a_0) = \data{\config'}(a_0)$ and for all $a \in \agents\setminus \set{a_0}$,  
	\begin{itemize}
		\item either $\config'(a) = \config(a)$
		
		\item or $\config'(a)$ is such that $(\st{\config}(a), \rec{\amessage}{\anact}, \st{\config'}(a)) \in \transitions$ with either
		\begin{itemize}
			\item $\anact = \quotemarks{\dummyact}$ 
			and $\data{\config'}(a) = \data{\config}(a)$
			\item $\anact = \quotemarks{\enregact}$ and $\data{\config'}(a) = \data{\config'}(a_0)$
			\item $\anact = \quotemarks{\eqtestact}$ and $\data{\config'}(a) = \data{\config}(a) =\data{\config}(a_0)$
			\item $\anact = \quotemarks{\diseqtestact}$ and $\data{\config'}(a) = \data{\config}(a) \ne \data{\config}(a_0)$
		\end{itemize}
	\end{itemize}

	\AP A ""path"" is a sequence of steps $\path : \config_0 \step{} \config_1 \step{} \cdots \step{} \config_k$. 
We write $\config_0 \step{*} \config_k$ the existence of such a "path".
 A ""run"" is a "path" whose first configuration is an "initial configuration"; runs are denoted using letter $\run$.  
Given such a "run" $\run$, we write $\agentsof{\run}$ its set of agents, $\statesin{\run}:= \set{q \in Q \mid \exists i, \exists a, \st{\config_i}(a) = q}$ the set of states appearing in $\run$ and $\valsof{\run} := \set{\aval \in \nats \mid \exists i, \exists a, \data{\config_i}(a) = \aval}$.  
% with current definition, every run is initial

A "run" $\run: \config_0 \step{*} \config_f$ ""covers"" a state $q \in Q$ when there exists $a \in \agents$ such that $\st{\config_f}(a) = q$. 

%We say that it is ""initial@@run"" if $\config_0$ is an initial configuration.
\end{definition}

The following lemma formalizes the intuition that, because the only way to take a new register value is to receive another agent's value, no new register values are created in a "run".
\begin{proposition}
\label{prop:run_no_new_register_values}
For all $\run: \config_0 \step{*} \config_f$, $\valsof{\run} = \set{\aval \in \nats \mid \exists a, \, \data{\config_0}(a) = \aval}$.
\end{proposition}

\begin{definition}
We define a preorder over the set of configurations $\allconfigs$ as follows: $\config \lessthan \config'$ if there exists an injective function $\pi: \agentsof{\config} \rightarrow \agentsof{\config'}$ such that, for all $a \in \agentsof{\config}$, $\config(a) = \config'(\pi(a))$. 
\end{definition}

\begin{definition}
	A ""query"" $\query$ is a finite set of formulas of the form $\quotemarks{q(\var{z})}$, $\quotemarks{R(\var{z}) = R(\var{z}')}$, $\quotemarks{R(\var{z}) \neq R(\var{z}')}$, with $\var{z}, \var{z}'$ taken in a set of variables $\varset$.
	It is \emph{satisfied} by a "configuration" $\config$ if there is a valuation $\nu : \varset \to \agentsof{\config}$ such that:
	\begin{itemize}
		\item for all $\quotemarks{q(\var{z})} \in \query$, $\st{\config}(\nu(\var{z})) = q$,
		
		\item for all $\quotemarks{R(\var{z}) = R(\var{z'})} \in \query$, $\data{\config}(\nu(\var{z})) = \data{\config}(\nu(\var{z'}))$,
		
		\item for all $\quotemarks{R(\var{z}) \neq R(\var{z'})} \in \query$, $\data{\config}(\nu(\var{z})) \neq \data{\config}(\nu(\var{z'}))$.
	\end{itemize}

	\AP The ""query coverability problem"" is to determine, given a "protocol" and a "query", whether there is a "run" of this protocol whose last "configuration" satisfies the query.
\end{definition}

\begin{remark}
\label{rem:bigger_config_query}
If some "configuration" $\config$ satisfies a "query" $\query$, then every configuration $\config'$ such that $\config \lessthan \config'$ satisfies $\query$. 
\end{remark}


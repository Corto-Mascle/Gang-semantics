\section{Definitions}

\begin{definition}
	A ""protocol"" is a tuple $(Q, \D, q_0)$ over an alphabet $\S$ with $Q$ a finite set of states, $q_0 \in Q$ an initial state, and $\D \subseteq Q \times \set{\br, \rec(*), \rec(\cp), \rec(\testeq)} \times Q$.
\end{definition}

\begin{definition}
	Let $(Q,\D, q_0)$ be a protocol, and $A$ a finite set of \emph{agents}.
	A ""configuration"" is a labelling function $\g : A \to Q \times \nats$, associating a state and a register value to each agent. We write $\g^Q$ for the state component of $\g$ and $\g^R$ for its register component.
	An \emph{initial configuration} $\g$ is one where for all $a \in A$, $\g^Q(a) = q_0$ and for all $a, a' \in A$, if $\g(a) = \g(a')$ then $a=a'$.
	
	\AP Given two configurations $\g$ and $\g'$, a ""transition"" $\g 
	\act{m} \g'$ between two configurations is defined when there exists $a_0 \in A$ such that $\g^Q(a_0) \act{\br} \g'^Q(a_0) \in \D$, $\g^R(a_0) = \g'^R(a_0)$ and for all $a \in A\setminus \set{a_0}$,  
	\begin{itemize}
		\item either $\g'(a) = \g(a)$
		
		\item or $\g'(a)$ is such that $\g^Q(a) \act{\rec(\a)} \g'^Q(a) \in \D$ with either
		\begin{itemize}
			\item $\a = *$ and $\g'^R(a) = \g^{R}(a)$
			\item $\a = \cp$ and $\g'^R(a) = m$
			\item $\a = \testeq$ and $\g'^R(a) = \g^R(a) =m$
		\end{itemize}
	\end{itemize}

	\AP A ""run"" is a sequence of transitions $\g_0 \act{m_0} \g_1 \act{m_1} \cdots \act{m_k} g_k$. We say that it is \emph{initial} if $g_0$ is an initial configuration.
\end{definition}

\begin{definition}
	A ""query"" is a set of formulas of the form $q(z)$, $R(z) = R(z')$, $R(z) \neq R(z')$, with $z, z'$ taken in a set of variables $Var$.
	It is \emph{satisfied} by a "configuration" $\g$ over a set of agents $A$ if there is a valuation $\nu : Var \to A$ such that:
	\begin{itemize}
		\item for all $q(z) \in \phi$, $\g_Q(\nu(z)) = q$
		
		\item for all $R(z) = R(z') \in \phi$, $\g_R(\nu(z)) = \g_R(\nu(z'))$
		
		\item for all $R(z) \neq R(z') \in \phi$, $\g_R(\nu(z)) \neq \g_R(\nu(z'))$
	\end{itemize}

	\AP The ""query coverability problem"" is to determine, given a "protocol" and a "query", whether there is an "initial run" of this protocol whose last "configuration" satisfies the query.
\end{definition}

\begin{ex}
	In some cases a protocol may require a "run" of exponential length (in the number of states) to reach a configuration satisfying a "query".
	
	\begin{figure}[h]
		\begin{tikzpicture}[xscale=0.5,AUT style,node distance=1.8cm,auto,>= triangle
	45]
	\tikzstyle{initial}= [initial by arrow,initial text=,initial
	distance=.7cm]
	%	\tikzstyle{accepting}= [accepting by arrow,accepting text=,accepting
	%	distance=.7cm,accepting where =right]
	
	\node[state,initial, minimum width=0.1pt] (0a) at (0,0) {0};
	\node[state] [below of=0a] (0b) {};
	\node[state] [below of=0b] (0c) {};
	\node[state] [right of=0a] (0') {};
	
	\node[state] [right of=0'] (1a) {1};
	\node[state] [below of=1a] (1b) {};
	\node[state] [below of=1b] (1c) {};
	
	\node [right of=1a, xshift=-20pt] (token1) {};
	\node [right of=token1, xshift=-30pt] (dots) {\huge $\cdots$};
	\node [right of=dots, xshift=-30pt] (token2) {};
	
	\node[state] [right of=token2, xshift=-20pt] (Na) {N};
	\node[state] [below of=Na] (Nb) {};
	\node[state] [below of=Nb] (Nc) {};
	
	\node[state] [right of=Na] (N') {};
	\node[state] [right of=N'] (end) {end};
	
	\path[->, bend left=20] 	
	;
	\path[->, bend right=20] 
	;
	\path[->]
	(0a) edge node[right] {$\rec(m_0, *)$} (0b)
	(0b) edge node[right] {$\br(m'_0)$} (0c)
	
	(1a) edge node[right] {$\rec(m_1, *)$} (1b)
	(1b) edge node[right] {$\br(m'_1)$} (1c)
	
	(Na) edge node[right] {$\rec(m_N, *)$} (Nb)
	(Nb) edge node[right] {$\br(m'_N)$} (Nc)
	
	(0a) edge node {$\br(m_0)$} (0')
	(0') edge node {$\rec(m'_0)$} (1a)
	
	(Na) edge node {$\br(m_N)$} (N')
	(N') edge node {$\rec(m'_N)$} (end)
	
	(1a) edge node {} (token1)
	(token2) edge node {} (Na)
	;
\end{tikzpicture}
		\label{fig:exp-run}
		\caption{Example of a "protocol" with a state that requires an exponentially long "run" to be covered.}
	\end{figure}
	
	\cortoin{TODO}
\end{ex}

\begin{proposition}
	The "query coverability problem" is \np-hard.
\end{proposition}

\begin{proof}
	\begin{figure}[h]
		\begin{tikzpicture}[xscale=0.5,AUT style,node distance=2cm,auto,>= triangle
	45]
	\tikzstyle{initial}= [initial by arrow,initial text=,initial
	distance=.7cm]
	%	\tikzstyle{accepting}= [accepting by arrow,accepting text=,accepting
	%	distance=.7cm,accepting where =right]
	
	\node[state,initial, minimum width=0.1pt] (0) at (0,0) {0};
	
	\node[state] [right of=0] (1) {1};
	
	\node [right of=1, xshift=-20pt] (token1) {};
	\node [right of=token1, xshift=-50pt] (dots) {\huge $\cdots$};
	\node [right of=dots, xshift=-50pt] (token2) {};
	
	
	\node[state] [right of=token2, xshift=-20pt] (N-1) {N-1};
	\node[state] [right of=N-1] (N) {N};
	
	\node[state] [right of=N] (1') {1'};
	
	\node [right of=1', xshift=-20pt] (token3) {};
	\node [right of=token3, xshift=-50pt] (dots2) {\huge $\cdots$};
	\node [right of=dots2, xshift=-20pt] (token4) {};
	
	
	\node[state] [right of=token4, xshift=-40pt] (m-1) {m-1'};
	\node[state] [right of=m-1] (m) {m'};
	
	\coordinate[below of=1] (stop);
	\node[state] [below right of=stop] (r1) {$x_1$};
	\node [above left of=r1, yshift=-20pt, xshift=10pt] (r1') {$\rec(x_1, \enregact)$};
	\node[state] [right of=r1] (r1b) {$\neg x_1$};
	\node [above left of=r1b, yshift=-20pt, xshift=10pt] (r1b') {$\rec(\neg x_1, \enregact)$};
	\node [right of= r1b] (dots3) {\huge $\cdots$};
	\node[state] [right of=dots3] (rn) {$x_n$};
	\node [above left of=rn, yshift=-20pt, xshift=10pt] (rn') {$\rec(\neg x_n, \enregact)$};
	\node[state] [right of=rn] (rnb) {$\neg x_n$};
	\node [above left of=rnb, yshift=-20pt, xshift=10pt] (rnb') {$\rec(\neg x_n, \enregact)$};
	
	\draw (0) .. controls +(0,-2) and +(-1,0) .. (stop);
	\draw[->] (stop) .. controls +(2,0) and +(0,1) .. (r1);
	\draw[->] (stop) .. controls +(6,0) and +(0,1) .. (r1b);
	\draw[->] (stop) .. controls +(14,0) and +(0,1) .. (rn);
	\draw[->] (stop) .. controls +(18,0) and +(0,1) .. (rnb);
	\path[->, bend left=20]
	(0) edge node[above] {$\br(x_1)$} (1)
	(N-1) edge node[above] {$\br(x_N)$} (N) 	
	;
	\path[->, bend left=40]
	(N) edge node[above] {$\rec(\ell_1^1, \eqtestact)$} (1')
	(m-1) edge node[above] {$\rec(\ell_m^1, \eqtestact)$} (m) 	
	;
	\path[->, bend right=20] 
	(0) edge node[below] {$\br(\neg x_1)$} (1)
	(N-1) edge node[below] {$\br(\neg x_N)$} (N) 
	;
	\path[->, bend right=30] 
	(N) edge node[below] {$\rec(\ell_1^3, \eqtestact)$} (1')
	(m-1) edge node[below] {$\rec(\ell_m^3, \eqtestact)$} (m)	
	;
	\path[->]
	(N) edge node[above] {$\rec(\ell_1^2, \eqtestact)$} (1')
	(m-1) edge node[above] {$\rec(\ell_m^2, \eqtestact)$} (m) 	
	;
	\path[->, loop below]
	(r1) edge node[below] {$\br(x_1)$} (r1)
	(r1b) edge node[below] {$\br(\neg x_1)$} (r1b) 	
	(rn) edge node[below] {$\br(x_n)$} (rn)
	(rnb) edge node[below] {$\br(\neg x_n)$} (rnb) 	
	;
\end{tikzpicture}
		\label{fig:np-hard}
		\caption{The "protocol" used for the \np-hardness proof.}
	\end{figure}
	
	We reduce from the 3SAT problem.
	Let $x_1, \ldots, x_n$ be variables and $\phi = \bigwedge_{j=1}^m C_j$ with, for all $j$, $C_j = \ell_j^1 \lor \ell_j^2 \lor \ell_j^3$ and $\ell_j^1, \ell_j^2, \ell_j^3 \in \set{x_i, \neg x_i \mid 1 \leq i \leq n}$. 
	
	\cortoin{TODO}
\end{proof}



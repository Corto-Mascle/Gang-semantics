

	We shall see that in this case, the "cover problem" is much easier than in the general case. Intuitively, it suffices to keep track of the state of one boss agent\lu{please remove if the term boss is not used in the general case} (the one who broadcast its initial value) and a set of states in which any number of processes with the same register's value can be. This way, we can (i) compute the set of reachable states and (ii) decide if a state is coverable or not. In this section, we prove that the "cover problem" for 1-BNRAs is NP-complete.

	We simplify our notations for "protocols" with one register.
	We do not consider "local tests" in this part, as with a single register every local equality test is satisfied and every local disequality test is not.
	Hence we can delete transitions with disequality tests and replace equality tests with $\varepsilon$-transitions, which can be eliminated with a quadratic blow-up as in non-deterministic finite automata.
	
	Furthermore, the register argument in receptions and broadcasts is always $1$, hence we remove it.
	Our new set of operations is 
	\[
	Op^{\messages} = \set{\brone{\amessage}, \recone{\amessage}{\dummyact}, \recone{\amessage}{\enregact}, \recone{\amessage}{\eqtestact}, \recone{\amessage}{\diseqtestact} \mid \amessage \in \messages}
	\]
	
	Given a "configuration" $\config$, we write $\data{\config}(a)$ for $\data{\config}(a,1)$. 

	




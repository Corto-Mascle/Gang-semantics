\subsection{Undecidability of \TARGET}
\label{sec:undec-target}

A natural next problem, after \COVER, is the "target problem" (\TARGET).  
Our \COVER procedure heavily relies on the ability to add agents at no cost; for \TARGET we need to guarantee that those agents can then reach the target state, making the problem harder. 
We show that \TARGET is undecidable, which suggests that while we can obtain decidability of \COVER thanks to the monotonicity of the problem, we cannot analyze precisely the set of runs of such systems.

\begin{restatable}{proposition}{propTargetUndecidable}
\label{prop:target-undec}
\TARGET is undecidable for \BNRA with two registers.
\end{restatable}

\begin{proof}[Proof sketch]
We simulate a Minsky machine with two counters. Like for the "LCS" encoding, the first register is never modified and plays the role of a permanent identifier. We start with some initialisation phase where each agent stores some other agent's identifier in its second register, which will be its ``predecessor''; it then only accepts messages from its predecessor. As there are finitely many agents, there is a cycle in the predecessor graph. 

In a cycle, we use the fact that \emph{all} agents must reach state $q_f$ to simulate faithfully a Minsky machine: we make agents alternate between receptions and broadcasts so that, in the end, they have received and sent the same number of messages, implying that no messages have been lost along the cycle.
We then simulate the machine by having an agent (the leader) choose transitions and the other ones simulate the counter values by memorizing a counter ($1$ or $2$) and a binary value ($0$ or $1$). For instance, an increment of counter $1$, initiated by the leader, takes the form of a message propagated in the cycle until it finds an agent simulating counter $1$ and having bit $0$. This agent switches to $1$ and sends an acknowledgment that eventually propagates back to the leader. See Appendix~\ref{app:target} for the full proof. 
\end{proof}

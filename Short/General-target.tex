\subsection{Undecidability of \TARGET}
\label{sec:undec-target}

A natural next problem, after \COVER, is the "target problem" (\TARGET).  
Our \COVER procedure heavily relies on the ability to add agents at no cost. For \TARGET we need to guarantee that those agents can then reach the target state, which makes the problem harder. 
We show that \TARGET is undecidable, which shows that our model lies at the frontier of decidability. \nico{modif ici}

\begin{restatable}{proposition}{propTargetUndecidable}
\label{prop:target-undec}
\TARGET is undecidable for "signature \BNRA" with two registers.
\end{restatable}

\begin{proof}[Proof sketch]
We simulate a Minsky machine with two counters using a "signature protocol". As in Proposition~\ref{prop:reduction-LCS},  we start with an initialisation phase where each agent stores some other agent's identifier, which will be its ``predecessor''. It then only accepts messages from its predecessor. As there are finitely many agents, there is a cycle in the predecessor graph. 

In a cycle, we use the fact that \emph{all} agents must reach state $q_f$ to simulate faithfully a Minsky machine: we make agents alternate between receptions and broadcasts so that, in the end, they have received and sent the same number of messages, implying that no messages have been lost along the cycle.
We then simulate the machine by having an agent (the leader) choose transitions and the other ones simulate the counter values by memorizing a counter ($1$ or $2$) and a binary value ($0$ or $1$). For instance, an increment of counter $1$, initiated by the leader, takes the form of a message propagated in the cycle until it finds an agent simulating counter $1$ and having bit $0$. This agent switches to $1$ and sends an acknowledgment that eventually propagates back to the leader. See Appendix~\ref{app:target} for the full proof. 
\end{proof}

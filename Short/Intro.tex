\section{Introduction}
We consider Broadcast Networks of Register Automata (BNRA), a model for networks of agents communicating by broadcasts. These systems are composed of an arbitrary number of agents whose behavior is specified with a finite automaton. This automaton is equipped with a finite set of private registers that contain values from an infinite unordered set. Initially, registers all contain distinct values, so these values can be used as identifiers. 
A broadcast message is composed of a symbol from a finite alphabet along with the value of one of the sender's registers. When an agent broadcasts a message, any subset of agents may receive it; this models unreliable systems with unexpected crashes and disconnections. Upon reception, an agent may store the received value or test it for equality with one of its register values. For example, an agent can check that several received messages have the same value.
% This allows an agent, for instance, to check the origin of received messages: to do so, the broadcasting agent signs messages with its identifier, and the receiving agent stores this identifier then checks that all incoming messages have this value. This also allows an agent to obtain acknowledgement that its message was received by someone, by broadcasting with its identifier and waiting for a response with that same identifier.

This model was introduced in \cite{DelzannoST13}, as a natural extension of Reconfigurable Broadcast Networks~\cite{DelzannoSZ2010Adhoc}. In \cite{DelzannoST13}, the authors established that coverability is undecidable if the agents are allowed to send two values per message. They moreover claimed that, with one value per message, the "coverability problem" was decidable and even \PSPACE-complete; however, the proof was wrong \cite{ArnaudErratum}. As we will see, the complexity of that problem is in fact much higher.

In this paper we establish the decidability of the "coverability problem" and its completeness for the hyper-Ackermannian complexity class $\Fcomplexity{\omega^\omega}$, showing that the problem has nonprimitive recursive complexity. The lower bound comes from "lossy channel systems", which consist (in their simplest version) of a finite automaton that uses an unreliable FIFO memory from which any letter may be erased at any time \cite{AbdullaJ1996verif, Schnoebelen2002verifying,ChambartS08ordinal}. 
We further establish that our model lies at the frontier of decidability by showing undecidability of the target problem (where all agents must synchronize in a given state). We contrast these results with the \NP-completeness of the "coverability problem" if each agent has only one register. 

\paragraph*{Related work} 
Broadcast protocols are a widely studied class of systems in which processes are represented by nodes of a graph and can send messages to their neighbors in the graph. There are many versions depending on how one models processes, the communication graph, the shape of messages... 
A model with a fully connected communication graph and messages ranging over a finite alphabet was presented in~\cite{emerson1998model}. When working with parameterized questions over this model (\emph{i.e.}, working with systems of arbitrary size), many basic problems are undecidable~\cite{EsparzaFM1999verification}; similar negative results were found for Ad Hoc Networks where the communication graph is fixed but arbitrary \cite{DelzannoSZ2010Adhoc}. This lead the community to consider Reconfigurable Broadcast Networks (RBN) where a broadcast can be received by an arbitrary subset of agents~\cite{DelzannoSZ2010Adhoc}.

Parameterized verification problems over RBN have been the subject of extensive study in recent years, concerning for instance reachability questions~\cite{DelzannoSTZ12, BalasubramanianGW22}, liveness~\cite{DBLP:journals/computing/ChiniMS22} or alternative communication assumptions~\cite{Balasubramanian18}; however, RBN have weak expressivity, in particular because agents are anonymous. In~\cite{DelzannoST13}, RBN were extended to BNRA, the model studied in this article, by the addition of registers allowing processes to exchange identifiers. 
%This extension was inspired by the success of register automata, which offer a convenient formalism to express properties of words over an infinite alphabet; see~\cite{Segoufin06} for a survey on the subject.

Other approaches exist to define parameterized models with registers~\cite{BolligRS21}, such as dynamic register automata in which processes are allowed to spawn other processes with new identifiers and communicate integers values~\cite{AbdullaAKR14}. While basic problems on these models are in general undecidable, some restrictions on communications allow to obtain decidability~\cite{AbdullaAKR15, Rezine17}.

Parameterized verification problems often relate to the theory of well quasi-orders
and the associated high complexities obtained from bounds on the length of sequences with no increasing pair (see for example \cite{WSTS}). 
% of length $2$
%\nico{pas très satisfait mais je n'arrive pas à résumer ``bad sequence'' de façon concise}
In particular, our model is linked to data nets, a classical model connected to well-quasi-orders. Data nets are Petri nets in which tokens are labeled with natural numbers and can exchange and compare their labels using inequality tests \cite{LazicNORW08}; in this model, the "coverability problem" is $\Fcomplexity{\omega^{\omega^{\omega}}}$-complete \cite{datanetsinequalityfomegaomegaomega}. When one restricts data nets to only equality tests, the "coverability problem" becomes $\Fomegaomega$-complete~\cite{Rosa-Velardo17}. Data nets with equality tests do not subsume BNRA: in data nets, each process can only carry one integer at a time (problems on models of data nets where tokens carry tuples of integers are typically undecidable)\cite{Lasota16}.
%The second closely related model is lossy channel systems (LCS)\cite{AbdullaJ1996verif}. LCS are derived from distributed models where processes communicate through pairwise channels; this model is a rich field of study in itself~\cite{Aiswarya2015model,Aiswarya2020networks}. LCS reachability is complete for the complexity class $\Fcomplexity{\omega^\omega}$~\cite{ChambartS08ordinal, Schnoebelen2002verifying}; we show that the same is true for BNRA coverability and that LCS can be simulated in BNRA.
% As a matter of fact, LCS are derived from distributed models with processes communicating through pairwise channels, which are a rich field of study on their own~\cite{Aiswarya2015model,Aiswarya2020networks}.

\paragraph*{Overview}
We start with the model definition and some preliminary results in Section~\ref{sec:preliminaries}. As our decidability proof is quite technical, we start by proving decidability of the coverability problem in a subcase called \emph{signature protocols} in Section~\ref{sec:cover-decidability}.
We then rely on the intuitions built in that subcase to generalize the proof to the general case in Section~\ref{sec:cover-general-case}. We also show the undecidability of the closely-related "target problem".
Finally, we prove the \NP-completeness of the coverability problem for protocols with one register in Section~\ref{sec:cover-1BNRA}.
Due to space constraints, most proofs are postponed to the appendix.

%
%	\textbf{Broadcast protocols}
%
%	\cite{emerson1998model} -> Introduction of broadcast protocols (no reconfiguration, reliable broadcasts)
%	
%	\cite{EsparzaFM1999verification} -> Undecidability of liveness and safety for this model
%
%	\textbf{Ad hoc Networks}
%
%	\cite{Godskesen2007calculus}, \cite{Merro2007observational} -> From Arnaud's paper
%
%	\textbf{RBN}
%	
%	\cite{DelzannoSZ2010Adhoc} -> Parametrized Model for Ad Hoc networks (undecidable), then switch to RBN
%	
%	\cite{Delzanno2012complexity} -> Complexity of RBN
%	
%	\cite{BouyerMRSS2016} -> ASMS almost-sure reachability
%	
%	\cite{fortin2017model} -> ASMS with stacks and leaders
%	
%	\cite{BalaW2021} -> Equivalence ASMS <-> RBN
%	
%	And some more ! \cite{BalasubramanianBM2018parameterized}, \cite{BalasubramanianGW2022parameterized}, 	\cite{ChiniMS2019liveness} 
%	
%	\textbf{Lossy channel systems}
%	
%	\cite{AbdullaJ1996verif} -> Introduction of LCS
%	
%	\cite{AbdullaJ1996undec} -> Büchi conditions are undecidable for LCS 
%	
%	\cite{SchmitzS2011upperHigman} -> Upper bounds for Higman's lemma
%	
%	\cite{ChambartS2008ordinal} -> Upper bound for LCS coverability 
%	
%	\cite{Schnoebelen2002verifying} -> Lower bound for LCS coverability 
%	
%%
%%	\textbf{Register automata}
%%	
%%%	\cite{kaminski1994finite} -> Introduction of register automata
%%	
%%	\cite{segoufin2006automata} -> survey of results for register automata (didn't find anything more recent)
%%	
%	\textbf{Data nets}
%	
%	\cite{lazic2007nets} -> Introduction of data nets
%	
%	\cite{ROSAVELARDO201741} -> Unordered data nets in $F_{\omega^\omega}$
%	
%	\textbf{Systems communicating via (lossy) channels}
%	
%	\cite{Aiswarya2021network} -> Survey, with some relevant papers 	\cite{Aiswarya2015model},
%	\cite{AbdullaAA2016data}
%
%	\cite{BRS21} -> not really register automata, but systems where each agent has a private register and there are as many shared registers as agents; an agent may change its local value, write it in the global memory and test the number of occurrences of its values. Control-state reachability is \PSPACE-complete.
	
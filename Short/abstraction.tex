
%To prove the \NP upper bound, we present an abstraction on configurations and runs. 
%The main ingredient of our abstraction is an extended version of the "copycat principle" presented in Remark~\ref{rem:copycat-principle}. 
%It is twofold: 
%\begin{itemize}
%	\item First, as explained in Remark~\ref{rem:copycat-principle}, if there is a run $\run$ sending an agent to a state $q$, then we can construct a run $\run'$ executing run over disjoint sets of agents (with different values) as many times as we want to obtain as many agents as we need in $q$.
%	
%	\item  Furthermore, if in a run $\run$ an agent $a$ gets in a state $q$ with a value $\aval$ that is not its initial one, it means that at some point in the run, it was in a state $q'$ and executed a transition in which it received a message with value $v$, stored it in its register and went to a state $q''$. 
%	As mentioned before, we can copy the run up to that point to have an unlimited supply of agents in $q'$, and thus an unlimited supply of agents in $q''$ with value $\aval$. We can then make all those agents copy the broadcasts of $a$ and receive the same messages so that they all reach $q$ with value $\aval$. Hence if we have an agent in a state $q$ with a value $\aval$ that is not its initial one, we can assume that we have as many agents in $q$ with value $\aval$ as we need.	
%\end{itemize}







%
%The proof can be found in Appendix~\ref{sec:one-diseq-tests}.
%\begin{example}
%	Consider the "protocol" displayed in Figure~\ref{fig:no-clique}.
%	We can obtain configurations satisfying $\set{1,2}$, $\set{2,3}$ or $\set{1,3}$, but we cannot obtain one satisfying $\set{1,2,3}$.
%	
%	\begin{figure}[h]
%		\begin{tikzpicture}[xscale=0.5,AUT style,node distance=2.1cm,auto,>= triangle
	45]
	\tikzstyle{initial}= [initial by arrow,initial text=,initial
	distance=.7cm, initial above]
	%	\tikzstyle{accepting}= [accepting by arrow,accepting text=,accepting
	%	distance=.7cm,accepting where =right]
	
	\node[state,initial, minimum width=0.1pt] (0) at (0,0) {0};
	\node[state] [left of=0, xshift=-50] (x) {1};
	\node[state] [below of=0] (y) {2};
	\node[state] [right of=0, xshift=50] (z) {3};

	\path[->, bend left=10] 	
	(0) edge node[above] {$\brone{x}$} (x)
	(0) edge node {$\brone{z}$} (z)
	;
	\path[->, bend right=10] 	
	(0) edge node[above] {$\recone{y}{\enregact}$} (z)
	(0) edge node[above] {$\recone{z}{\enregact}$} (x)
	;
	\path[->, bend left=20] 	
	(0) edge node {$\brone{y}$} (y)
	;
	\path[->, bend right=20] 	
	(0) edge node[left] {$\recone{x}{\enregact}$} (y)
	;
\end{tikzpicture}
%		\caption{An illustrating example}
%		\label{fig:no-clique}
%	\end{figure}
%\end{example}

%
%We are now ready to define our abstraction. As mentioned before, it should keep track of a particular state (the state of the process with the initial register value) and a set of states in which any number of processes can be on with the same non-initial value. We name this tuple a "gang" which we define formally below.

%
%Intuitively, a "gang" corresponds to the set of agents with a given "register value". The "boss" $\boss$ represents the process that had this value at the beginning and the "clique" $\clique$ the set of states of processes who have received and stored this register value. If the original owner of this value no longer has it, then $\boss = \noboss$. Note that we define the clique as the set of states $q$ such that \emph{at some point in the run} some agent was in state $q$ with value $v$. This is because we can use the copycat principle to add a large amount of agents that are in state $q$ with value $v$, and thus we can assume that there is always one.
%
%In a concrete "run" of our system, gangs of distinct values may only interact with one another by covering states $q$ which are needed by the other gang (in the form of a broadcast or of a $\quotemarks{\enregact}$ reception from $q$); therefore our abstraction also keeps track of the set of coverable states, which may only grow. However, it only needs to keep track of one gang at a time.
%
%This leads us to a natural abstract semantics based on gangs. An abstract configuration consists of a set of states $S$ (states covered so far by some agents) and a gang $(b, K)$ (the original owner of the value $v$ we are keeping track of and the states reached by other agents with that value). If the original owner of $v$ stores a new value we set $b = \bot$.
%Abstract transitions are defined by applying transitions of the protocol while assuming that we have unlimited supplies of agents in every state of $S$ and of agents with value $v$ in states of $K$.
%At any time, we can apply a \emph{gang reset}, which maintains $S$ but reinitializes $(b,K)$ to $(q_0, \emptyset)$ (we track a new value). 
%We define formally this abstraction and show its soundness and completeness in Appendix~\ref{app:cover-one-reg}.
%To bound the length of relevant abstract runs, we impose that $S$ should grow between two gang resets (otherwise they reset to the same abstract configuration) and that there may be at most $O(\size{Q}^2)$ "abstract steps" between two resets (as $K$ can only increase and there are only $\size{Q}+1$ possibilities for $b$). This means that if there is an abstract run covering a state, there is one of size $O(\size{Q}^3)$, proving the \NP upper bound. 

The \NP lower bound follows from a reduction from 3SAT. An agent $a$ sends a sequence of messages representing a valuation, with its identifier, to other agents who play the role of an external memory by broadcasting back the valuation. This then allows $a$ to check the satisfaction of a 3SAT formula. \nico{petite modif, notamment plus de parenthese}
%
%These results yield the main theorem of this section:


\begin{restatable}{theorem}{thmNPComplete}
	\label{thm:np-complete-query-cover}
	The "coverability problem" for 1-BNRA is \NP-complete.
\end{restatable}


\section{Definitions and Notations of \cref{sec:cover-general-case}}
\label{app:def-trees}
%\subsubsection{Notations and Definitions}
\label{sec:decidability-defs}
% decompositions , admits decompositions

We will first define the notion of decomposition, which is a formalization of the observation in Example~\ref{ex:decomposition}. 
A ""decomposition"" is a tuple $\decsymb = (w_0, m_1, \ldots, m_\ell, w_\ell)$ with $w_0, \ldots, w_\ell \in \messages^*$, and $m_1, \cdots, m_\ell \in \messages$, with $m_i \neq m_j$ for all $i\neq j$. In particular we have $\ell \leq \size{\messages}$. 
A word $w \in \messages^*$ ""admits decomposition"" $\decsymb = (w_0, m_1, \ldots, m_\ell, w_\ell)$ if $w \subword w'_0 w'_1 \cdots w'_\ell$ where for all $j$, $w'_j$ can be obtained from $w_j$ by adding letters from $\set{m_1, \ldots, m_{j}}$. 
We denote by $\intro*\langdec{\decsymb}$ the language of words that "admit decomposition" $\decsymb$. 

We are now ready to formally define "unfolding trees" in the general case.

\begin{definition}
	\label{def:unfolding_tree}
	\AP An ""unfolding tree"" $\tree$ over $\prot$ is
	a finite tree where nodes $\node$ have three labels:
	\begin{itemize}
		\item a "local run" of $\prot$, written $\reintro*\localrunlabel{\node}$, starting in the initial state with distinct register values;
		
		\item a value in $\nats$, written $\reintro*\valuelabel{\node}$;
		
		\item a \reintro{specification} $\reintro*\speclabel{\node}$, which is either a word $\intro*\bosslabel{\node} \in \messages^*$ (""boss specification"") or a pair $(\intro*\followlabelword{\node}, \intro*\followlabelmessage{\node}) \in \messages^* \times \messages$ (""follower specification""). In the first case we say that the node is a ""boss node"", otherwise it is a ""follower node"".
	\end{itemize} 
	Moreover, all nodes $\node$ must satisfy the four following conditions:
	\begin{enumerate}[label= (\roman*), ref=(\roman*)]
		\item \label{item:condition1_non_initial_value} For each "non-initial value" $\aval \ne \valuelabel{\node}$ of $\localrunlabel{\node}$, $\node$ has a child $\node'$ which is a "boss node" such that $\vinput{\aval}{\localrunlabel{\node}}$ is a subword of $\bosslabel{\node'}$.
		
		\item \label{item:condition2_initial_value} For each "initial value" $\aval$ in $\localrunlabel{\node}$, there is a "decomposition" \\ $\decsymb = (w_0, m_1, w_1, \ldots, m_{\ell}, w_{\ell})$~s.t.:
		\begin{itemize}
			\item $\localrunlabel{\node}$ may be split into successive "local runs" $\localrun_0, \dots, \localrun_{\ell}$ where, for all $i \in \nset{1}{\ell}$, $w_i \subword \voutput{\aval}{\localrun_i}$ and $\vinput{\aval}{\localrun_i} \in \set{m_1, \dots, m_{i-1}}^*$,
			\item for all $i \in [1,\ell]$, $\node$ has a child $\node_i$ which is a "follower node" such that $\followlabelmessage{\node_i} = m_i$ and $\followlabelword{\node_i} \in\langdec{\decsymb_i}$ where $\decsymb_i = (w_0, m_1, w_1, \ldots, m_{i-1}, w_{i-1})$.	\end{itemize}
		
		\item \label{item:condition3_follower_node} If $\node$ is a "follower node" then $\valuelabel{\node}$ is not an "initial value" of $\localrun$, $\vinput{\valuelabel{\node}}{\localrun} \subword \followlabelword{\node}$ and 
		$\voutput{\valuelabel{\node}}{\localrun}$ contains $\followlabelmessage{\node}$.
		
		\item \label{item:condition4_boss_node} If $\node$ is a "boss node", then $\valuelabel{\node}$ is an "initial value" of $\localrunlabel{\node}$ and the "decomposition" $\decsymb$ of \ref{item:condition2_initial_value} for $\valuelabel{\node}$ satisfies that $\bosslabel{\node} \in \langdec{\decsymb}$.
	\end{enumerate}
	
	\AP Lastly, as before, given $\tree$ an "unfolding tree", we define its ""size@@tree"" by $\size{\tree} := \sum_{\node \in \tree} \size{\localrunlabel{\node}} + \size{\speclabel{\node}} + 1$. %Note that the "size@@tree" of $\tree$ takes into account the size of its nodes, so that a tree $\tree$ can be stored in space polynomial in $\size{\tree}$ (renaming the values appearing in $\tree$ if needed). 
\end{definition}

We now explain this definition. Let $\node$ be a node of an "unfolding tree" $\tree$ and let $\localrun := \localrunlabel{\node}$. 
As before, $\localrun$ encodes the "local run" of a given agent, $\speclabel{\node}$ encodes the specification that this "local run" carries out and $\valuelabel{\node}$ encodes the value for which the "specification" is carried out.

Conditions \ref{item:condition1_non_initial_value} and \ref{item:condition2_initial_value} state that the "specifications" of the children of $\node$ are witnesses that messages received in the "local run" $\localrunlabel{\node}$ can be broadcast by other agents. 
Conditions \ref{item:condition3_follower_node} and \ref{item:condition4_boss_node} state that $\node$ is a witness that its "specification" is carried out. 

As before, condition \ref{item:condition1_non_initial_value} expresses that, for every "non-initial value" $\aval$ of $\localrun$, $\node$ must have a "boss" child witnessing that $\vinput{v}{\localrun}$ can indeed be broadcast. 
Because $\aval$ was first stored by a "reception step" of $\localrun$, any other (fresh) value with sequence of message types containing $\vinput{v}{\localrun}$ also works and we do not impose the value label of this child to be $v$. 

We now explain condition \ref{item:condition2_initial_value}, which states the existence of a "decomposition" for each "initial value", which serves as a summary of the broadcasts made with that value. Let $v$ be an "initial value" of $\localrun$. Consider a "run" where $\localrun$ is the "local run" of agent $a$. If another agent broadcasts with value $v$, it has first received and stored $\aval$. By duplicating agents, we may afterwards assume that we have an unlimited supply of messages $(m,v)$. Therefore the crucial information is the times at which each "message type" is first broadcast with $v$ by an agent other than $a$.

The "decomposition" $\decsymb = (w_0, m_1, w_1, \ldots, m_\ell, w_\ell)$ should be understood as follows: In the "run" we are representing, $a$ first broadcasts messages of $w_0$ with value $v$, after which other agents are able to broadcast $(m_1,v)$. Then $a$ broadcasts $w_1$ (and may receive $m_1$), after which other agents can broadcast $(m_2, v)$, and so on.
To sustain that description, we need to be able to split $\localrun$ into $u_0,\dots,u_\ell$, with each $u_i$ broadcasting $w_i$ and only receiving messages $m_1,\ldots, m_{i-1}$ over value $v$. We also need a "follower" child for each $m_i$ to witness that other agents may broadcast it.
For every $i$, the sequence of messages available with value $v$ during $u_i$ is $\voutput{v}{u_i}$ expanded by freely adding symbols from $\set{m_1,\dots, m_{i-1}}$. Therefore, the "follower" child $\node_i$ responsible for the broadcast of $(m_i,v)$ may first receive with value $v$ a subword of $w_0' \cdot w_1' \cdots w_{i-1}'$ where, for all $j \leq i-1$, $w_j$ is obtained from $\voutput{v}{u_i}$ by adding symbols from $\set{m_1, \dots, m_{j-1}}$, which we state as $\followlabelword{\node_i} \in \langdec{\decsymb_i}$.   

Condition~\ref{item:condition3_follower_node} directly states that a "follower node" $\node$ receives word $\followlabelword{\node}$ with value $\valuelabel{\node}$ and broadcasts message $(\followlabelmessage{\node}, \valuelabel{\node})$. Condition \ref{item:condition4_boss_node} expresses that a "boss node" witnesses the broadcast of a sequence of messages $\bosslabel{\node}$ with a single value; whereas in the "signature protocol" case, in this sequence, some messages may come from auxiliary agents encoded in "follower" children, which is why we have the condition that $\bosslabel{\node} \in \langdec{\decsymb}$ and not simply $ \voutput{\valuelabel{\node}}{\localrun} \subword \bosslabel{\node}$. 

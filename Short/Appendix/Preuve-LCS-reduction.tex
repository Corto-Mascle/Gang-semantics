\section{Proof of Proposition~\ref{prop:reduction-LCS}}
\label{app:reduction-lcs}

\propReductionLCS*
\begin{proof}
	We provide here a polynomial-time reduction from reachability for lossy channel systems with a single channel. A "lossy channel system" with a single channel is a finite-state machine that has the ability to buffer symbols in a lossy FIFO queue \cite{Schnoebelen2002verifying}. 
	Let $\los := (\lstates,\Sigma, \ltransitions)$ be a "lossy channel system", where $\lstates$ is a finite set of locations, $\Sigma$ is a finite set of symbols and $\ltransitions \subseteq \lstates \times \set{\popact{x}, \pushact{x} \mid x \in \Sigma} \times \lstates$; $\pushact{x}$ corresponds to writing $x$ at the end of the channel and $\popact{x}$ to reading $x$ at the beginning the channel. A configuration of $\los$ is a pair in $\lstates \times \Sigma^*$ denoting the location and the content of the channel. There exists a step from $(\lstate,w)$ to $(\lstate',w')$ using $\ltrans \in \ltransitions$, denoted $(\lstate,w) \lstep{\ltrans} (\lstate',w')$, when
	\begin{itemize}
		\item $\ltrans = (\lstate,\pushact{x},\lstate')$ for some $x \in \Sigma$ and $w' \subword w \cdot x$,
		\item $\ltrans = (\lstate,\popact{x},\lstate')$ for some $x \in \Sigma$ and $x \cdot w' \subword w$.
	\end{itemize}
	where $\subword$ denotes the "subword" order, which encodes the lossiness of the channel. 
	The existence of a step is denoted $(\lstate,w) \lstep{} (\lstate',w')$, and its transitive closure is denoted $\lstep{*}$. The (control-state) ""reachability problem@@lcs"" asks, given $\los$ and two locations $\lstate_s, \lstate_f \in \lstates$, whether $(\lstate_s,\epsilon) \lstep{*} (\lstate_f, w)$ for some $w$. 
	
	We construct a " signature protocol" $\prot$ with two registers (register $1$ is "broadcast-only", register $2$ is "reception-only") and a state $q_f$ that may be covered in $\prot$ if and only if $(\los, q_i, q_f)$ is a positive instance of the "reachability problem@@lcs". 
	In $\prot$, agents will organize in chains.
	An agent $a$ stores the identifier of the previous agent in the chain then tests for equality so that messages from other agents are ignored. An agent in the chain will encode one step of the "lossy channel system". 
	The predecessor will send a location of the system and the content of the channel to $a$, $a$ will in turn broadcast to the next agent in the chain the new location of the system and the new content of the channel; $a$ rebroadcasts on-the-fly the content of the channel letter by letter as it receives it, with only a small modification corresponding to the "push" or the "pop" that it applies. Messages might get lost, which is why we are not able to encode non-lossy channel systems.
	

	We construct a "signature protocol" $\prot$ with two registers (register $1$ is "broadcast-only", register $2$ is reception-only) and a state $q_f$ that may be covered if and only if $(\los, q_i, q_f)$ is a positive instance of the "reachability problem@@lcs". In $\prot$, agents will organize in chains where each agent encodes a step of the execution in the "lossy channel system". An agent $a$ stores the identifier of the previous agent in the chain then tests for equality so that messages from other agents are ignored. The predecessor will send a location of the system and (a subword of) the content of the channel. $a$ will in turn broadcast to the next agent in the chain, sending the new location of the system and the new content of the channel which $a$ rebroadcasts on-the-fly letter by letter as it receives it. $a$ only modifies a bit the beginning of the channel if it decides to encode a "pop" and the end of the channel if it decides to encode a "push". Messages might get lost, which is why we are not able to encode non-lossy channel systems.
	
	\AP From the initial state $q_0$ of $\prot$, agents non-deterministically decide whether they are ""root"" (at the beginning of their chain) or ""link"". 
	
	A "link" agent will first pick its predecessor by storing its identifier, then broadcast its own identifier.
	A "root" agent does not store anyone else's identifier: it is at the beginning of its chain. It simply broadcasts the initial configuration $(\lstate_s,\epsilon)$ of $\los$; special character $\#$ encodes the end of the broadcast of the channel's content.  A depiction of this initial phase can be found in Figure~\ref{fig:lcs-choice}.  
	% A "link" agent first receives a broadcast with an identifier which it stores as the one of its predecessor, then broadcasts its own identifier. This construction guarantees that "link" agents have exactly one predecessor. 
	Observe that it only guarantees that agents has at most one predecesor; it does not guarantee that all agents are in the same chain or that any agent is the predecessor of at most one agent. This is fine because the information only propagates forward in a chain, and never propagates in between chains. 
	% Concretely, there is a sequence of two transitions from $q_0$ labeled by $\rec{\mathsf{init}}{2}{\enregact}, \br{\mathsf{init}}{1}$ that gets to $\waitstate \in Q$. 
	% From $\waitstate$, there is, for every $\lstate \in \lstates$, a transition labeled by $\rec{\mathsf{\lstate}}{2}{\eqtestact}$ that goes to state $\startstate{\lstate} \in Q$.

	From state $\waitstate$, an agents will receive the location $\lstate$ from its predecessor and go to state $\startstate{\lstate}$; from there, it will non-deterministically pick a transition of $\los$ and apply it. See Figure~\ref{fig:lcs-trans} for a depiction of transitions from some state $\startstate{\lstate_1}$ (all $\startstate{\lstate}$, $\lstate \ne \lstate_1$, and corresponding transitions are omitted in the figure). 
	Finally, the objective state of our system is $q_f := \finstate{\lstate_f}$.

	% For every transition $\ltrans = (\lstate, \op, \lstate') \in \ltransitions$ in $\los$ (\emph{i.e.}, every transition of $\ltransitions$ whose source is location $\lstate$), there is a transition in $\prot$ labeled by $\br{\mathsf{\lstate'}}{1}$ that goes from $\startstate{\lstate}$ to $\transstateone{\ltrans} \in Q$. 
	% Transitions from $\transstateone{\ltrans}$ in $\prot$ depend on $\ltrans$. For a "pop" transition $(\lstate, \popact{u}, \lstate')$, the agent first must receive $u$ then the channel which it rebroadcasts, whereas for a "push" transition $\ltrans= (\lstate, u,!, \lstate')$, it first receives and rebroadcasts the channel then also broadcasts $u$. Either way, it goes to $\finstate{\lstate'}$ which is a deadlock. 
	% Formally:
	% \begin{itemize}
	% 	\item If $\ltrans=(\lstate, u,?, \lstate')$ is a "pop" then $\prot$ has a sequence of transitions from $\transstateone{\ltrans}$ to $\transstatetwo{\ltrans} \in Q$ labeled by $\rec{\mathsf{u_1}}{2}{\eqtestact}, \rec{\mathsf{u_2}}{2}{\eqtestact}, \dots, \rec{\mathsf{u_k}}{2}{\eqtestact}$ where $u = u_1 u_2 \cdots u_k$. Moreover, there is, for every $x \in \Sigma$, a loop on $\transstatetwo{\ltrans}$ labeled with the sequence of actions $\rec{\mathsf{x}}{2}{\eqtestact}, \br{\mathsf{x}}{1}$. There is also a sequence of  transitions from $\transstatetwo{\ltrans}$ to $\finstate{\lstate'}$ labeled by $\rec{\mathsf{\#}}{2}{\eqtestact}, \br{\mathsf{\#}}{1}$.
	% 	\item If $\ltrans= (\lstate, u,!, \lstate')$ is a "push" then there is, for every $x \in \Sigma$, a loop on $\transstateone{\ltrans}$ labeled with sequence of actions $\rec{\mathsf{x}}{2}{\eqtestact}, \br{\mathsf{x}}{1}$. There also is a sequence of transitions from $\transstateone{\ltrans}$ to $\transstatetwo{\ltrans}$ labeled by $\br{\mathsf{u_1}}{1}, \br{\mathsf{u_2}}{1}, \dots, \br{\mathsf{u_k}}{1}$ where $u = u_1 \cdot u_2 \cdots u_k$. From $\transstatetwo{\ltrans}$, there is  a sequence of transitions going to $\finstate{\lstate'}$ labeled by $\rec{\mathsf{\#}}{2}{\eqtestact}, \br{\mathsf{\#}}{1}$.
	% \end{itemize}

	\begin{figure}
	\begin{subfigure}[b]{0.99\textwidth}
	\centering
	\begin{tikzpicture}[node distance=2cm,auto, xscale = 2]
	\tikzset{every node/.style = {font = {\scriptsize}}}
	\node[state,initial, initial text = ] (0) at (0,0) {$q_0$};
	\node[state] at (0,1) (root) {\textsf{root}};
	\node[state] at (0,-1) (link) {\textsf{link}};
	\node[state] at (1,1) (root1) {}; 	
	\node[state] at (2,1) (root2) {};
	\node[state] at (3,1) (root3) {$\finstate{\lstate_0}$};
	\node[state] at (1.5,-1) (link1) {};
	\node[state] at (3,-1) (wait){$\waitstate$};
	

	\path[->] 	
		(0) edge (root)
		(0) edge (link)
		(root) edge node[above] {$\br{\mathsf{init}}{1}$} (root1)
		(root1) edge node[above] {$\br{\mathsf{\lstate_s}}{1}$} (root2)
		(root2) edge node[above] {$\br{\mathsf{\#}}{1}$} (root3)
		(link) edge node[below] {$\rec{\mathsf{init}}{2}{\enregact}$} (link1)
		(link1) edge node[below]{$\br{\mathsf{init}}{1}$} (wait)
	;
\end{tikzpicture}
	\caption{The initial part of $\prot$}\label{fig:lcs-choice}
	\end{subfigure}
	\begin{subfigure}[b]{0.99\textwidth}
	\centering
	\begin{tikzpicture}[node distance=2cm,auto]
	\tikzset{every node/.style = {font = {\scriptsize}}}
	\node[state] at (-0.5,1) (wait){$\waitstate$};
	\node[state, inner sep = 0pt] at (2,1) (l1) {$\startstate{\lstate_1}$};
	\node[state] at (4,2.5) (tr1) {$\transstateone{\ltrans}$};
	\node[state, inner sep = 0pt] at (6.5,2.5) (tr2) {$\transstatetwo{\ltrans}$};
	\node[state] at (6.5,4) (reba) {};
	\node[state] at (6.5,1) (rebb) {};
	\node[state] at (9,2.5) (endrec) {};
	\node[state, inner sep = 0pt] at (11,2.5) (endbr) {$\finstate{\lstate_2}$};
	\node[state, inner sep = 0pt] at (4, -0.5) (tr1prime) {$\transstateone{\ltrans'}$};
	\node[state] at (4,1) (rebaprime) {};
	\node[state] at (4, -2) (rebbprime) {};
	\node[state, inner sep = 0pt] at (6.5, -0.5) (tr2prime) {$\transstatetwo{\ltrans'}$};
	\node[state] at (9,-0.5) (endrecprime) {};
	\node[state, inner sep = 0pt] at (11,-0.5) (endbrprime) {$\finstate{\lstate_3}$};
	
	\path[->] 	
	(wait) edge node[sloped] {$\rec{\mathsf{\lstate_1}}{2}{\eqtestact}$} (l1)
	(l1) edge node[above, sloped] {$\br{\mathsf{\lstate_2}}{1}$} (tr1)
	(tr1) edge node[above] {$\rec{\mathsf{a}}{2}{\eqtestact}$} (tr2)
	(tr2) edge[bend left = 10] node[left] {$\rec{\mathsf{a}}{2}{\eqtestact}$} (reba)
	(reba) edge[bend left = 10] node[right] {$\br{\mathsf{a}}{1}$} (tr2)
	(tr2) edge[bend right = 10] node[left] {$\rec{\mathsf{b}}{2}{\eqtestact}$} (rebb)
	(rebb) edge[bend right = 10] node[right] {$\br{\mathsf{b}}{1}$} (tr2)
	(tr2) edge node[above] {$\rec{\mathsf{\#}}{2}{\eqtestact}$} (endrec)
	(endrec) edge node[above] {$\br{\#}{1}$} (endbr)
	(rebaprime) edge[bend left = 10] node[right] {$\br{\mathsf{a}}{1}$} (tr1prime)
	(tr1prime) edge[bend left = 10] node[left] {$\rec{\mathsf{a}}{2}{\eqtestact}$} (rebaprime)
	(tr1prime) edge[bend right = 10] node[left] {$\rec{\mathsf{b}}{2}{\eqtestact}$} (rebbprime)
	(rebbprime) edge[bend right = 10] node[right] {$\br{\mathsf{b}}{1}$} (tr1prime)
	(tr2prime) edge node[above] {$\rec{\mathsf{\#}}{2}{\eqtestact}$} (endrecprime)
	(endrecprime) edge node[above] {$\br{\#}{1}$} (endbrprime)
	(l1) edge[bend right = 40] node[left] {$\br{\mathsf{\lstate_3}}{1}$} (tr1prime)
	(tr1prime) edge node[below] {$\br{\mathsf{a}}{1}$} (tr2prime); 
\end{tikzpicture}
	\caption{Part of $\prot$ encoding transitions from $\lstate_1$. Here, $\los$ has symbols $\Sigma = \set{\mathsf{a},\mathsf{b}}$ and there are two transitions from $\lstate_1$: $\ltrans = (\lstate_1, \popact{a}, \lstate_2)$ and $\ltrans' = (\lstate_1, \pushact{b}, \lstate_3)$.}\label{fig:lcs-trans}
	\end{subfigure}
	\caption{Depiction of the protocol $\prot$ built in the "lossy channel system" reduction of Proposition~\ref{prop:reduction-LCS}.}
	\end{figure}
	We claim that $(\prot, q_f)$ is a positive instance of \COVER if and only if $(\los, \lstate_f)$ is a positive instance of the reachability problem for "lossy channel systems".
	First, suppose that there exists $w \in \Sigma^*$ such that $(\lstate_0, \epsilon) \lstep{*} (\lstate_f, w)$. Decompose the witness into $(\lstate_0, w_0) \lstep{} (\lstate_1, w_1) \lstep{} (\lstate_2, w_2) \cdots \lstep{} (\lstate_n,w_n)$ with $\lstate_n = \lstate_f$ and $w_n =w$. 
	We build an "initial run" of $\prot$ that "covers" $q_f$. It has set of agents $\agents := \set{0,\dots, n}$. Agent $0$ becomes the "root" and for all $i \geq 1$, agent $i$ becomes a "link" with predecessor agent $i-1$. By induction on $i$, we build an execution using agents $0$ to $i$ such that agent $i$ ends on state $\finstate{\lstate_i}$ and the sequence of "messages" sent by agent $i$ admits as subword $\mathsf{init} \cdot \mathsf{\lstate_i} \cdot w_i \cdot \mathsf{\#}$. For $i=0$, this condition is easuly met by making agent $0$ become "root". We make agent $i+1$ do the following. It receives from agent $i$ state $l_i$ and goes to $\startstate{l_i}$. It moves to $\transstateone{\ltrans}$ where $\ltrans \in \ltransitions$ is such that $(\lstate_i, w_i) \lstep{\ltrans} (\lstate_{i+1}, w_{i+1})$. It then follows the branch and gets to $\finstate{\lstate_{i+1}}$. 
	\begin{itemize}
		\item if $\ltrans =(\lstate_i,\pushact{x}, \lstate_{i+1})$ is a "push", then $w_{i+1} = w_{i}' \cdot y$ where $w_i' \subword w_i$ and $y \in \set{\epsilon, x}$. 
		% Agent $i+1$ receives $w_i'$ in full and rebroadcasts it while looping on $\transstateone{\ltrans}$. It then broadcasts $u$ to get to $\transstatetwo{\ltrans}$. It finally receives $\mathsf{\#}$ from agent $i$ and rebroadcasts it, going to state $\finstate{\lstate_{i+1}}$. 
		We can make agent $i+1$ broadcast $\mathsf{init} \cdot \mathsf{\lstate_i} \cdot w'_i \cdot x \cdot \mathsf{\#}$ which admits as subword $\mathsf{init} \cdot \mathsf{\lstate_{i+1}} \cdot w_{i+1} \cdot \mathsf{\#}$.
		\item if $\ltrans =(\lstate_i, \popact{x}, \lstate_{i+1})$ is a "pop" then $w_{i} = u \cdot w_{i+1}'$ where $w_{i+1} \subword w_{i+1}'$ and $x \subword u$. 
		% Overall, agent $i+1$ receives from agent $i$ a sequence of "message types" $\mathsf{init} \cdot \mathsf{\lstate_i} \cdot u \cdot w_{i+1}' \cdot \mathsf{\#}$ (some messages may get lost). 
		% From $\transstateone{\ltrans}$, agent $i+1$ receives $u$ and goes to $\transstatetwo{\ltrans}$. It then receives $w_{i+1}'$ in full and rebroadcasts it while looping on $\transstatetwo{\ltrans}$. It finally received $\mathsf{\#}$ from agent $i$ and rebroadcasts it, going to state $\finstate{\lstate_{i+1}}$.
		By lossiness, agent $i+1$ only receives $x$ from $u$ and goes to $\transstatetwo{\ltrans}$. We make it broadcast $\mathsf{init} \cdot \mathsf{\lstate_{i+1}} \cdot w_{i+1}'  \cdot \mathsf{\#}$ which admits as subword $\mathsf{init} \cdot \mathsf{\lstate_{i+1}} \cdot w_{i+1} \cdot \mathsf{\#}$.
	\end{itemize}
	This concludes the induction step.
	When applied to $i=n$, this builds an "initial run" where agent $n$ ends on $\finstate{\lstate_n}$, which is a witness that $(\prot, q_f)$ is positive.
	
	Suppose now that $(\prot, q_f)$ is positive. Let $\run: \config_0 \step{*} \config_f$ where $\config_f$ covers $q_f$. Agents in $\run$ are arranged in chains where a given chain starts with a "root" then has "link" agents each storing the value of their predecessor in the chain (some agents may be in no chain but these agents can be ignored). All chains are finite because $\agents$ is finite; morever, there can be no cycle because a "link" agent first store its predecessor's value then broadcast its own. Consider in $\run$ a chain of agents $a_0, \dots, a_n$ such that $a_0$ is the "root" and $a_n$ covers $q_f$. 
	
	From the "run" $\run$ projected on this chain, it is quite simple to build an execution of $\los$ that covers $q_f$. By structure of the protocol, because $a_n$ covers $q_f$, every agent $a_i$ ends on some $\finstate{\lstate_i}$ and broadcasts a word of the form $\mathsf{init} \cdot \lstate_i \cdot w_i \cdot \mathsf{\#}$; this can be proven with an immediate backwards induction. It then suffices to analyse the behavior of $a_{i+1}$ to prove that $(\lstate_i, w_i) \lstep{} (\lstate_{i+1}, w_{i+1})$. In particular, because $a_0$ is a "root", $\lstate_0 = \lstate_s$ and $w_0 = \epsilon$, which concludes the proof. 
\end{proof}
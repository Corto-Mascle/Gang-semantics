\section{Bounding the Size of the Minimal Coverability Witness}
\label{app:proofs_bounds}

In this section, we prove that we can obtain a computable bound, in complexity class $\Fomegaomega$, of $\tree$. All bounds provided in this section are independent from $\tree$. 

\begin{restatable}{lemma}{lemBoundLengthHeightH}
	\label{lem:bound-length-at-height-h}
	Let $\altmax$ the maximal "altitude" in $\tree$. There is a primitive recursive function $f_0$ such that, for every node $\node$ of $\tree$, $\size{\node} \leq f_0(\size{\prot} + \altmax - \altitude{\node})$.
\end{restatable}
\begin{proof}
Let $f_0: k \mapsto (\towerfun(k,k) (k^2 +2))^{k+1}$. Also, let $N := \size{\prot}$.
First, if $\node$ has altitude $\altmax$, then it has no "follower child", hence applying Lemma~\ref{lem:bound-successor-height} with $K=0$ bounds $\size{\node}$ by $\towerfun(\size{\prot},r) \leq \towerfun(N,N) \leq f_0(N)$. Let $\node$ be a node, and suppose that the statement is true for altitudes greater than $\altitude{\node}$. Let $d = \altmax - \altitude{\node}$, $d\geq 1$. Applying Lemma~\ref{lem:bound-successor-height} with $K := f_0(N + d - 1)$ gives: 
\begin{align*}
\size{\node} \leq \towerfun(N,N) (N^2 K + K + 1) \leq & \towerfun(N+d,N+d) f_0(N+d-1) ((N+d)^2 +2)   \\ \leq & f_0(N+d) 
\end{align*}
\end{proof}

% We actually prove the following, stronger statement:
% \begin{lemma}
% 	\label{lem:bound-length-at-height-h-extended}
% 	Let $\prot$ be a protocol, let $\node$ be a node of a "unfolding tree" of minimal "size@@tree" labeled by $\prot$ satisfying a "boss specification" $w$.
% 	Let $\altmax$ be the maximal "altitude" in $\tree$, let $N = \size{w} + \size{\prot} +1$ and let $f_0' : n \in \nats \mapsto ((n+2)^2 \towerfun(n,n))^{n+1}$, with $\towerfun$ the function defined in Lemma~\ref{lem:short-local-runs}.
	
% 	\begin{itemize}
% 		\item If $\node$ is a "boss node" then $\size{\bosslabel{\node}} \leq f_0'(\size{\prot} +\altmax - \altitude{\node})$ and $\size{\localrunlabel{\node}} \leq f_0'(\size{\prot} +\altmax - \altitude{\node}+1)$.
		
% 		\item If $\node$ is a "follower node" then $\size{\followlabelword{\node}} \leq \size{\localrunlabel{\node}} \leq f_0'(\size{\prot} + \altmax - \altitude{\node}+1)$.
% 	\end{itemize} 
% \end{lemma}
% \begin{proof}
% 	The proof is by strong induction on $\altmax-\altitude{\node}$.

% 	%initialization added for clarity although one could factorize it with the rest of the induction
% 	First, let $\node$ such that $\altitude{\node} = \altmax$. This implies that $\node$ has no follower node (hence we can set $K=0$ in Lemma~\ref{lem:bound-successor-height}).  If $\node$ is the root of the tree, then it is a "boss node"; we can impose that its "boss specification" is of length $1$ (coverability of $q_f$) and applying Lemma~\ref{lem:bound-successor-height} proves that $\size{\localrunlabel{\node}} \leq \towerfun(\size{\prot},r) \leq f_0'(\size{\prot})$ ($\towerfun$ defined in Lemma~\ref{lem:short-local-runs} is monotone for both arguments). If $\node$ is not the root of the tree, then it is necessarily a "follower node" (a "boss node" with a parent would not have maximum altitude), and the application of Lemma~\ref{lem:bound-successor-height} again yields that $\size{\followlabelword{\node}} \leq \size{\localrunlabel{\node}} \leq \towerfun(\size{\prot},r) \leq f_0'(\size{\prot})$.   

% 	Let $\node$ be a node, and suppose that the statement is true for all nodes of altitude greater than $\altitude{\node}$ and that $\altitude{\node} < \altmax$. 
% 	Let $\node_1, \ldots, \node_k$ be its "follower" children, and $K$ the maximum of all $\size{\followlabelword{\node_i}}$.  By induction hypothesis for all $i$ we have 
% 	$\size{\followlabelword{\node_i}} \leq f_0'(\size{\prot} + \altmax-\altitude{\node})$ because $\altitude{\node_i} = \altitude{\node}+1$, and thus $K \leq f_0'(\size{\prot} + \altmax-\altitude{\node})$. If $\node$ is not the root of $\tree$ then let $\node'$ be its parent.
	
% 	Suppose first that $\node$ is a "boss node". If it is at the root then we can assume that $\size{\bosslabel{\node}} \leq 1$.  If not, $\node'$ is above $\node$ therefore we have $\size{\bosslabel{\node}} \leq \size{\localrunlabel{\node'}} \leq f_0'(\size{\prot} + \altmax - \altitude{\node})$ by Lemma~\ref{lem:bound-successor-height} and induction hypothesis. In both cases we have $\size{\bosslabel{\node}} \leq f_0'(\size{\prot} +\altmax - \altitude{\node})$.	
% 	Again by Lemma~\ref{lem:bound-successor-height}, we obtain \[\size{\localrunlabel{\node}} \leq \towerfun(\size{\prot},r) \Big[f_0'(\size{\prot} + \altmax-\altitude{\node}) + \size{\messages}rK +1 \Big]. \]
	
% 	 As a result, 
% 	\begin{align*}
% 	\size{\localrunlabel{\node}}& 
% 	\leq \towerfun(\size{\prot},r) f_0'(\size{\prot} + \altmax-\altitude{\node})(\size{\messages} r +2) \\&  
% 	\leq (\size{\prot}+2)^2 \towerfun(\size{\prot},\size{\prot}) f_0'(\size{\prot} + \altmax - \altitude{\node}) \\& 
% 	\leq f_0'(\size{\prot} + \altmax - \altitude{\node}+1)
% 	\end{align*}
	
% 	Suppose now that $\node$ is a "follower node". By Lemma~\ref{lem:bound-successor-height}, we have 
% 	$\size{\followlabelword{\node}} \leq \size{\localrunlabel{\node}} \leq \towerfun(\size{\prot},r) \Big[1+\size{\messages}rK\Big]$.
% 	By the same argument as above, this proves that $\size{\localrunlabel{\node}} \leq f_0'(\size{\prot} + \altmax - \altitude{\node}+1)$.
% \end{proof}

\begin{restatable}{lemma}{lemBoundMaxHeight}
	\label{lem:bound-max-height}
	There exists a function $f_1 \in \Ffunction{\omega^{\size{\messages}}}$ such that $\altmax \leq f_1(\size{\prot})$.
\end{restatable}
\begin{proof}
	Let $\altmax$ be the maximal altitude of a node in $\tree$. Consider a branch of $\tree$ reaching altitude $\altmax$; for every $j \in \nset{1}{\altmax}$, let $\node_j$ the first node of this branch to reach altitude $j$. For every $j \geq 1$, $\node_j$ is a "follower nodes" as otherwise its predecessor in the branch would have altitude greater than $j$ hence the branch must have crossed altitude $j$ before.
	
	We will bound $\altmax$ using the "Length function theorem". For simplicity, we will encode the $\followmessagespec$ part using a fresh character added to our alphabet. This is why we obtain a function in $\Ffunction{w^{\size{\messages}}}$ and not $\Ffunction{w^{\size{\messages}-1}}$; in fact, one could obtain the latter bound using Theorem 5.3. of~\cite{SchmitzS2011upperHigman}, but the proof would be more involved.

	Let $\# \notin \messages$ be a fresh letter. For all $i \in \nset{1}{\altmax}$ let $\node'_i = \node_{\altmax - i+1}$ and $w_i = \followlabelword{\node'_{i}} \cdot \# \cdot \followlabelmessage{\node'_{i}} \in (\messages \cup \set{\#})^*$.
	
	We cannot have $w_{i}\subword w_{j}$ for $i< j$: indeed, this would imply $\followlabelword{\node'_{i}} \subword \followlabelmessage{\node'_{j}}$ and $\followlabelmessage{\node'_{i}} = \followlabelword{\node'_{j}}$, which is a contradiction according to Lemma~\ref{lem:shortening-branches}.
	As a result, the sequence $(w_i)_{1\leq i \leq \altmax}$ is a "bad sequence" over the alphabet $\messages\cup\set{\#}$.
	Furthermore, by Lemma~\ref{lem:bound-length-at-height-h}, for all $i$, $\size{\node_{i}} \leq f_0(\size{\prot} + i)$ therefore $\size{w_i} \leq f_0^{(i)}(\size{\prot})$.
	Because $f_0$ is primitive-recursive, we can apply Theorem~\ref{thm:lengthfcttheorem}: there exists a function $f_1 \in \Ffunction{\omega^{\size{\messages}}}$ such that the sequence $(w_i)_{i \in \nset{1}{\altmax}}$ is of length at most $f_1(\size{\prot})$, hence $\altmax \leq f_1(\size{\prot})$. 
\end{proof}
\begin{figure}[h]
	\begin{tikzpicture}
	
	\draw[dotted] (0,0) -- (4,0);
	\draw[dotted] (0,0.5) -- (4,0.5);
	\draw[dotted] (0,1) -- (4,1);
	\draw[dotted] (0,1.5) -- (4,1.5);
	\draw[dotted] (0,2) -- (4,2);
	\draw[dotted] (0,2.5) -- (4,2.5);
	\draw[dotted] (0,3) -- (4,3);
	
	\node at (-0.2,0) {$0$};
	\node at (-0.2,0.5) {$1$};
	\node at (-0.2,1) {$2$};
	\node at (-0.2,1.5) {$3$};
	\node at (-0.2,2) {$4$};
	\node at (-0.2,2.5) {$5$};
	\node at (-0.2,3) {$6$};
	
	\draw[very thick] (0.4,0) -- (0.6, 0.5) -- (0.8, 1) -- (1, 0.5) -- (1.2, 1) -- (1.3, 1.5) -- (1.7, 2) -- (1.8, 1.5) -- (2.1, 1) -- (2.3, 1.5) -- (2.4, 2) -- (2.8, 2.5);
	\draw (1, 0.5) -- (1.4, 0) -- (1.6, 0.5) -- (2.4, 1) -- (2.6, 1.5);
	\draw (1.6, 0.5) -- (2.2, 0) -- (2.5, 0.5);
	\draw (0.8, 1) -- (0.7, 1.5) -- (0.4, 1) -- (0.5, 1.5);
	\draw (0.7, 1.5) -- (0.9, 1) -- (1.1, 1.5) -- (0.8, 2);
	\draw (2.4, 2) -- (2.1, 1.5) -- (1.9, 2) -- (1.6, 2.5);
	\draw (2.4, 2) -- (2.1, 1.5) -- (1.9, 2) -- (1.6, 2.5);
	\draw (2.4, 2) -- (2.1, 1.5) -- (1.9, 2) -- (1.6, 2.5);
	\draw (1.9, 2) -- (2.2, 2.5);
	\draw (2.4, 2) -- (3, 1.5);
	\draw (1.9, 2) -- (2.2, 2.5);
	\draw (0.4, 0) -- (0.3, 0.5);
	
	\draw[->, >=triangle 45] (4.2, 0) -- (4.2, 3);
	
	
	\node (H) at (4.5, 3) {$\mathbf{alt}$};
	
	\draw[->] (0.4, -0.1) -- (0.8, -0.5);
	
	\node (R) at (2,-0.5) {root ($\mathbf{alt} =0$)};
	\node (AM) at (3,2.7) {$\altmax$};
	\draw[red, fill=red] (0.4, 0) circle (0.05);
	\draw[red, fill=red] (0.6, 0.5) circle (0.05);
	\draw[red, fill=red] (0.8, 1) circle (0.05);
	\draw[red, fill=red] (1.3, 1.5) circle (0.05);
	\draw[red, fill=red] (1.7, 2) circle (0.05);
	\draw[red, fill=red] (2.8, 2.5) circle (0.05);
	
	
	\node[red] at (0.7, 0.1) {\small $w_0$};
	\node[red] at (0.3, 0.6) {\small $w_1$};
	\node[red] at (0.5, 0.8) {\small $w_2$};
	\node[red] at (1.5, 1.3) {\small $w_3$};
	\node[red] at (1.3, 1.9) {\small $w_4$};
	\node[red] at (2.9, 2.3) {\small $w_5$};
	
	\node (txt) at (9.3, 2.7) {We consider a branch reaching $\altmax$. For all $i$,};
	\node (txt) at (9, 2.2) {let $w_i = \followlabelword{\node_i}\#\followlabelmessage{\node_i}$ where $\node_i$ is the first};
	\node (txt) at (7.6, 1.7) {"follower node" at altitude $i$.};
	
	\node (txt) at (9.8, 1) {Lemma~\ref{lem:bound-length-at-height-h} bounds the size of $w_i$ by $f_0(\altmax-i+1)$, };
	\node (txt) at (9.75, 0.5) {which allows us to bound the number of red dots using};
	\node (txt) at (8.95, 0) {Lemma~\ref{lem:shortening-branches} and the "Length function theorem".};
\end{tikzpicture}
	\caption{Illustration of the proof of Lemma~\ref{lem:bound-max-height}}.
	\label{fig:max-height-bound}
\end{figure}

The bound on the maximal altitude, along with Lemma~\ref{lem:bound-length-at-height-h}, bounds the size of the root of $\tree$ by $f_0(\size{\prot} + f_1(\size{\prot}))$. Let $\altmin := \min_{\node \in \tree} \altitude{\node} \leq 0$ the minimal altitude in $\tree$. By a similar argument but with "boss nodes", we can bound $|\altmin|$:


\begin{restatable}{lemma}{lemBoundMinHeight}
	\label{lem:bound-min-height}
	There exists a function $f_2 \in \Ffunction{\omega^{\size{\messages}}}$ such that $|\altmin| \leq f_2(\size{\prot})$.
\end{restatable}

\begin{proof}
	Consider a branch of $\tree$ reaching $\altmin$, let $\node_1, \ldots, \node_{|\altmin|}$ be nodes of $\tree$ such that $\node_i$ is the first node of the branch to reach altitude $-i$. All these nodes are "boss nodes" because a "follower node" has higher altitude than its parent. 
	For all $i \in \nset{0}{\altmin}$ let $\node'_i = \node_{|\altmin|-i}$ and $w_i = \bosslabel{\node'_{i}} \in \messages^*$.
	By Lemma~\ref{lem:shortening-branches}, $(w_i)_{i \in \nset{1}{\altmin}}$ is a "bad sequence".
	By Lemmas~\ref{lem:bound-length-at-height-h} and \ref{lem:bound-max-height}, for all $i$, we have $\size{w_i} \leq f_0(\size{\prot}+f_1(\size{\prot}) +i)$. Let $N := \size{\prot}+f_1(\size{\prot})+1$; we have $\size{w_i} \leq f_0^{(i)}(N)$.  
	
	Because $f_0$ is primitive recursive, we can apply Theorem~\ref{thm:lengthfcttheorem}: we obtain $h \in \Ffunction{\omega^{\size{\messages}-1}}$ such that $\altmin \leq h(N)$. Let $f_2: n \mapsto h(n+ f_1(n) + 1)$; we have $\altmin \leq f_2(\size{\prot})$. Moreover, $f_2$ is a function of $\Ffunction{\omega^{\size{\messages}}}$ because $f_1, h \in \Ffunction{\omega^{\size{\messages}}}$ and $\Ffunction{\omega^{\size{\messages}}}$ is closed by composition.
\end{proof}

With the bounds on the maximal and minimal altitudes, Lemma~\ref{lem:bound-length-at-height-h} yields a bound on the size of all nodes in the tree.


\begin{lemma}
	\label{lem:bound-node-size}
	There exists $f_3 \in \Ffunction{\omega^{\size{\messages}}}$ such that, for all $\node$ in $\tree$, $\size{\node} \leq f_3(\size{\prot})$. 
\end{lemma}
\begin{proof}
	Using Lemmas~\ref{lem:bound-length-at-height-h}, \ref{lem:bound-max-height} and \ref{lem:bound-min-height}, we obtain that, for every node $\node$, the value $\altmax - \altitude{\node}$ is at most $f_1(\size{\prot}) + f_2(\size{\prot})$. Applying Lemma~\ref{lem:bound-length-at-height-h} proves that, for all $\node$ in $\tree$, $\size{\node} \leq f_0(f_1(\size{\prot}) + f_2(\size{\prot})) \leq f_3(\size{\prot})$ where $f_3: n \mapsto f_0(f_1(n) + f_2(n))$; $f_3$ is in $\Ffunction{\omega^{\size{\messages}}}$ as $\Ffunction{\omega^{\size{\messages}}}$ is closed by composition. 
\end{proof}

Finally, as we now have a bound on all nodes, we can easily bound the length of branches and the number of children of all nodes to get abound on the total "size@@tree" of the tree.

\begin{restatable}{proposition}{PropBoundTreeSize}
	\label{prop:bound-tree-size}
	There exists a function $f$ of class $\Ffunction{\omega^{\size{\messages}}}$ s.t. $\size{\tree} \leq f(\size{\prot})$.
\end{restatable}
	% There exists a function $f$ of the class $\Ffunction{\omega^{\size{\messages}+1}}$ such that for all "protocol" $\prot$, for all "unfolding tree" $\tree$ of minimal size labeled by $\prot$ satisfying a "boss specification" $w$, the size of $\tree$ is bounded by $f(\size{\prot} + \size{w})$.

\begin{proof}
	Recall that the "size@@tree" of $\tree$ is defined as the sum of the sizes of its nodes. 

	Let $N = \size{\prot}$. 
	By Lemma~\ref{lem:bound-node-size}, for each $\node$ in $\tree$ we have $\size{\node} \leq f_3(N)$, hence the "local run" of $\node$ contains at most $f_3(N)$ values. By minimality of $\tree$, each value requires at most $1$ "boss child" and $\size{\messages}$ "follower children" hence $\node$ has at most $(\size{\messages}+1) f_3(N) \leq N f_3(N)$ children. Moreover, there are less than $f_4(N) := N^{f_3(N)}$ possible specifications in $\tree$ hence each branch of $\tree$ is of length less than $f_4(N)$ (by minimality, a branch does not have twice the same specification). This bounds the total number of nodes in $\tree$ by $f_5(N) := (N f_3(N))^{f_4(N) +1}$ hence we obtain $\size{\tree} \leq f(N)$ where $f: n \mapsto f_3(n) f_5(n)$. Because $\Ffunction{\omega^{\size{\messages}}}$ is stable by composition, exponentiation and product, we have $f \in \Ffunction{\omega^{\size{\messages}}}$.
\end{proof}


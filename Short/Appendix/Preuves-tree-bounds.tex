\section{Proof of Lemma~\ref{lem:bound-length-at-height-h}}
\label{app:bound-node-size-with-altitude}

\begin{restatable}{lemma}{lemBoundLengthHeightH}
	\label{lem:bound-length-at-height-h}
	Let $\tree$ be a minimal "coverability witness" for $(\prot,q_f)$, $\altmax$ the maximal "altitude" in $\tree$. There exists a primitive recursive function $f_0$ such that any node $\node$ of $\tree$ has size bounded by $f_0(\size{\prot} + \altmax - \altitude{\node})$.
\end{restatable}

We actually prove the following, stronger statement:
\begin{restatable}{lemma}{lemBoundLengthHeightH}
	\label{lem:bound-length-at-height-h-extended}
	Let $\prot$ be a protocol, let $\node$ be a node of a "unfolding tree" of minimal "size@@tree" labeled by $\prot$ satisfying a "boss specification" $w$.
	Let $\altmax$ be the maximal "altitude" in $\tree$, let $N = \size{w} + \size{\prot} +1$ and let $f_0' : n \in \nats \mapsto ((n+2)^2 \towerfun(n,n))^{n+1}$, with $\towerfun$ the function defined in Lemma~\ref{lem:short-local-runs}.
	
	\begin{itemize}
		\item If $\node$ is a "boss node" then $\size{\bosslabel{\node}} \leq f_0'(\size{\prot} +\altmax - \altitude{\node})$ and $\size{\localrunlabel{\node}} \leq f_0'(\size{\prot} +\altmax - \altitude{\node}+1)$.
		
		\item If $\node$ is a "follower node" then $\size{\followlabelword{\node}} \leq \size{\localrunlabel{\node}} \leq f_0'(\size{\prot} + \altmax - \altitude{\node}+1)$.
	\end{itemize} 
\end{restatable}
\begin{proof}
	The proof is by strong induction on $\altmax-\altitude{\node}$.

	%initialization added for clarity although one could factorize it with the rest of the induction
	First, let $\node$ such that $\altitude{\node} = \altmax$. This implies that $\node$ has no follower node (hence we can set $K=0$ in Lemma~\ref{lem:bound-successor-height}).  If $\node$ is the root of the tree, then it is a "boss node"; we can impose that its "boss specification" is of length $1$ (coverability of $q_f$) and applying Lemma~\ref{lem:bound-successor-height} proves that $\size{\localrunlabel{\node}} \leq \towerfun(\size{\prot},r) \leq f_0'(\size{\prot})$ ($\towerfun$ defined in Lemma~\ref{lem:short-local-runs} is of course monotone for both arguments). If $\node$ is not the root of the tree, then it is necessarily a "follower node" (a "boss node" with a parent would not have maximum altitude), and the application of Lemma~\ref{lem:bound-successor-height} again yields that $\size{\followlabelword{\node}} \leq \size{\localrunlabel{\node}} \leq \towerfun(\size{\prot},r) \leq f_0'(\size{\prot})$.   

	Let $\node$ be a node, and suppose that the statement is true for all nodes of altitude greater than $\altitude{\node}$ and that $\altitude{\node} < \altmax$. 
	Let $\node_1, \ldots, \node_k$ be its "follower" children, and $K$ the maximum of all $\size{\followlabelword{\node_i}}$.  By induction hypothesis for all $i$ we have 
	$\size{\followlabelword{\node_i}} \leq f_0'(\size{\prot} + \altmax-\altitude{\node})$ because $\altitude{\node_i} = \altitude{\node}+1$, and thus $K \leq f_0'(\size{\prot} + \altmax-\altitude{\node})$. If $\node$ is not the root of $\tree$ then let $\node'$ be its parent.
	
	Suppose first that $\node$ is a "boss node". If it is at the root then we can assume that $\size{\bosslabel{\node}} \leq 1$.  If not, $\node'$ is above $\node$ therefore we have $\size{\bosslabel{\node}} \leq \size{\localrunlabel{\node'}} \leq f_0'(\size{\prot} + \altmax - \altitude{\node})$ by Lemma~\ref{lem:bound-successor-height} and induction hypothesis. In both cases we have $\size{\bosslabel{\node}} \leq f_0'(\size{\prot} +\altmax - \altitude{\node})$.	
	Again by Lemma~\ref{lem:bound-successor-height}, we obtain \[\size{\localrunlabel{\node}} \leq \towerfun(\size{\prot},r) \Big[f_0'(\size{\prot} + \altmax-\altitude{\node}) + \size{\messages}rK +1 \Big]. \]
	
	 As a result, 
	\begin{align*}
	\size{\localrunlabel{\node}}& 
	\leq \towerfun(\size{\prot},r) f_0'(\size{\prot} + \altmax-\altitude{\node})(\size{\messages} r +2) \\&  
	\leq (\size{\prot}+2)^2 \towerfun(\size{\prot},\size{\prot}) f_0'(\size{\prot} + \altmax - \altitude{\node}) \\& 
	\leq f_0'(\size{\prot} + \altmax - \altitude{\node}+1)
	\end{align*}
	
	Suppose now that $\node$ is a "follower node". By Lemma~\ref{lem:bound-successor-height}, we have 
	$\size{\followlabelword{\node}} \leq \size{\localrunlabel{\node}} \leq \towerfun(\size{\prot},r) \Big[1+\size{\messages}rK\Big]$.
	By the same argument as above, this proves that $\size{\localrunlabel{\node}} \leq f_0'(\size{\prot} + \altmax - \altitude{\node}+1)$.
\end{proof}



\section{Proof of Proposition~\ref{prop:bound-tree-size}}
\label{app:bound-tree-size}

\begin{figure}[h]
	\begin{tikzpicture}
	
	\draw[dotted] (0,0) -- (4,0);
	\draw[dotted] (0,0.5) -- (4,0.5);
	\draw[dotted] (0,1) -- (4,1);
	\draw[dotted] (0,1.5) -- (4,1.5);
	\draw[dotted] (0,2) -- (4,2);
	\draw[dotted] (0,2.5) -- (4,2.5);
	\draw[dotted] (0,3) -- (4,3);
	
	\node at (-0.2,0) {$0$};
	\node at (-0.2,0.5) {$1$};
	\node at (-0.2,1) {$2$};
	\node at (-0.2,1.5) {$3$};
	\node at (-0.2,2) {$4$};
	\node at (-0.2,2.5) {$5$};
	\node at (-0.2,3) {$6$};
	
	\draw[very thick] (0.4,0) -- (0.6, 0.5) -- (0.8, 1) -- (1, 0.5) -- (1.2, 1) -- (1.3, 1.5) -- (1.7, 2) -- (1.8, 1.5) -- (2.1, 1) -- (2.3, 1.5) -- (2.4, 2) -- (2.8, 2.5);
	\draw (1, 0.5) -- (1.4, 0) -- (1.6, 0.5) -- (2.4, 1) -- (2.6, 1.5);
	\draw (1.6, 0.5) -- (2.2, 0) -- (2.5, 0.5);
	\draw (0.8, 1) -- (0.7, 1.5) -- (0.4, 1) -- (0.5, 1.5);
	\draw (0.7, 1.5) -- (0.9, 1) -- (1.1, 1.5) -- (0.8, 2);
	\draw (2.4, 2) -- (2.1, 1.5) -- (1.9, 2) -- (1.6, 2.5);
	\draw (2.4, 2) -- (2.1, 1.5) -- (1.9, 2) -- (1.6, 2.5);
	\draw (2.4, 2) -- (2.1, 1.5) -- (1.9, 2) -- (1.6, 2.5);
	\draw (1.9, 2) -- (2.2, 2.5);
	\draw (2.4, 2) -- (3, 1.5);
	\draw (1.9, 2) -- (2.2, 2.5);
	\draw (0.4, 0) -- (0.3, 0.5);
	
	\draw[->, >=triangle 45] (4.2, 0) -- (4.2, 3);
	
	
	\node (H) at (4.5, 3) {$\mathbf{alt}$};
	
	\draw[->] (0.4, -0.1) -- (0.8, -0.5);
	
	\node (R) at (2,-0.5) {root ($\mathbf{alt} =0$)};
	\node (AM) at (3,2.7) {$\altmax$};
	\draw[red, fill=red] (0.4, 0) circle (0.05);
	\draw[red, fill=red] (0.6, 0.5) circle (0.05);
	\draw[red, fill=red] (0.8, 1) circle (0.05);
	\draw[red, fill=red] (1.3, 1.5) circle (0.05);
	\draw[red, fill=red] (1.7, 2) circle (0.05);
	\draw[red, fill=red] (2.8, 2.5) circle (0.05);
	
	
	\node[red] at (0.7, 0.1) {\small $w_0$};
	\node[red] at (0.3, 0.6) {\small $w_1$};
	\node[red] at (0.5, 0.8) {\small $w_2$};
	\node[red] at (1.5, 1.3) {\small $w_3$};
	\node[red] at (1.3, 1.9) {\small $w_4$};
	\node[red] at (2.9, 2.3) {\small $w_5$};
	
	\node (txt) at (9.3, 2.7) {We consider a branch reaching $\altmax$. For all $i$,};
	\node (txt) at (9, 2.2) {let $w_i = \followlabelword{\node_i}\#\followlabelmessage{\node_i}$ where $\node_i$ is the first};
	\node (txt) at (7.6, 1.7) {"follower node" at altitude $i$.};
	
	\node (txt) at (9.8, 1) {Lemma~\ref{lem:bound-length-at-height-h} bounds the size of $w_i$ by $f_0(\altmax-i+1)$, };
	\node (txt) at (9.75, 0.5) {which allows us to bound the number of red dots using};
	\node (txt) at (8.95, 0) {Lemma~\ref{lem:shortening-branches} and the "Length function theorem".};
\end{tikzpicture}
	\caption{Illustration of the proof of Lemma~\ref{lem:bound-max-height}}.
	\label{fig:max-height-bound}
\end{figure}
\begin{restatable}{lemma}{lemBoundMaxHeight}
	\label{lem:bound-max-height}
	There exists a function $f_1$ of the class $\Ffunction{\omega^{\size{\messages}}}$ such that for all "protocol" $\prot$, for all "unfolding tree" $\tree$ of minimal "size@@tree" labeled by $\prot$ satisfying a "boss specification" $w$, the maximal altitude of a node of $\tree$ is bounded by $f_1(\size{\prot} + \size{w})$.
\end{restatable}
\begin{proof}
	Let $\tree$ be a "unfolding tree" of minimal "size@@tree" labeled by $\prot$ satisfying a "boss specification" $w$. Let $\altmax$ be the maximal altitude of a node in $\tree$. Let $\node_0$ the root of the tree, and  $\node_0, \ldots, \node_m$ a branch of the tree going from the root to a node $\node_m$ of altitude $\altmax$ (for all $i$, $\node_i$ is a child of $\node_{i-1}$).
	For all $a \in \nset{1}{\altmax}$, let $j_a$ be the minimal index such that $\altitude{\node_{j_a}} = a$. This index exists as $\altitude{\node_0} = 0$, $\altitude{\node_m} = \altmax$ and for all $i$, $\altitude{\node_{i}} \leq \altitude{\node_{i-1}}+1$.
	Furthermore, for every $a$, for every $j < j_a$, $\altitude{\node_j} < a$. 
	As a consequence, we have $j_1 < j_2 < \ldots < j_{\altmax}$. Moreover, $\node_{j_a}$ is necessarily a "follower node" as otherwise we would have $\altitude{\node_{j_a-1}} = a+1$ which contradicts minimality of $j_a$.
	
	We will bound $\altmax$ using Theorem~\ref{thm:lengthfcttheorem}. For simplicity, we will encode the $\followmessagespec$ part using a fresh character added to our alphabet (this is why we obtain a function in $\Ffunction{w^{\size{\messages}}}$ and not $\Ffunction{w^{\size{\messages}-1}}$; in fact, one could obtain the latter bound using Theorem 5.3. of~\cite{SchmitzS2011upperHigman}, but the proof would be more involved).

	Let $\# \notin \messages$ be a fresh letter, and let $N = \size{w} + \size{\prot}+1$. For all $i \in \nset{1}{\altmax}$ let $\node'_i = \node_{j_{\altmax - i+1}}$ and $w_i = \followlabelword{\node'_{i}} \# \followlabelmessage{\node'_{i}} \in (\messages \cup \set{\#})^*$.
	
	We claim that we cannot have $w_{i_1}\subword w_{i_2}$ for any  $i_1< i_2$.
	Suppose by contradiction that there exist such $i_1$ and $i_2$ with $i_1 < i_2$. Then we would have $\followlabelword{\node'_{i_1}} \subword \followlabelmessage{\node'_{i_2}}$ and $\followlabelmessage{\node'_{i_1}} = \followlabelword{\node'_{i_2}}$.
	Therefore, as $\node_{i_1}$ is a descendant of $\node_{i_2}$, by second case of Lemma~\ref{lem:shortening-branches} there would exist a "unfolding tree" of smaller "size@@tree" than $\tree$ satisfying $w$, contradicting the minimality of $\tree$.
	
	As a result, the sequence $(w_i)_{1\leq i \leq \altmax}$ is a "bad sequence" over the alphabet $\messages\cup\set{\#}$.
	Furthermore, by Lemma~\ref{lem:bound-length-at-height-h}, for all $i$, as $\altitude{\node'_{i}} = \altmax-i+1$, we have $\size{\followlabelword{\node_{j_a}}} \leq f_0(\size{\prot} + i)$ therefore $\size{w_i} \leq f_0^{(i)}(\size{\prot})$ (with $f_0^{(i)}$ denoting $f_0$ applied $i$ times).
	
	Because $f_0$ is primitive-recursive, we can apply Theorem~\ref{thm:lengthfcttheorem}: there exists a function $f_1' \in \Ffunction{\omega^{\size{\messages}}}$ such that the sequence $(w_i)_{i \in \nset{1}{\altmax}}$ is of length at most $f_1'(N)$. This prove that $\altmax \leq f_1'(N) = f_1'(\size{\prot} + \size{w} + 1)$. Letting $f_1 :n \mapsto f_1'(n+1)$ concludes the proof.
\end{proof}


The bound on the maximal altitude, along with Lemma~\ref{lem:bound-length-at-height-h}, gives us a bound on the size of the root. 
By using a similar argument as for the maximal altitude, we consider a branch reaching minimal altitude to obtain a "bad" sequence of "boss" nodes and apply the "Length function theorem" again to bound the minimal altitude.



\begin{restatable}{lemma}{lemBoundMinHeight}
	\label{lem:bound-min-height}
	There exists a function $f_2$ of the class $\Ffunction{\omega^{\size{\messages}+1}}$ such that for all "protocol" $\prot$, for all "unfolding tree" $\tree$ of minimal "size@@tree" labeled by $\prot$ satisfying a "boss specification" $w$, the absolute value of the minimal "altitude" of a node of $\tree$ is bounded by $f_2(\size{\prot} + \size{w})$.
\end{restatable}

\begin{proof}
	We proceed similarly as in the proof of Lemma~\ref{lem:bound-max-height}.
	
	Let $\tree$ be a "unfolding tree" of minimal "size@@tree" labeled by $\prot$ satisfying a "boss specification" $w$. Let $\altmin$ be the minimal altitude of a node in $\tree$. Let $\node_0, \ldots, \node_m$ be nodes of $\tree$ such that $\node_0$ is its root, $\altitude{\node_m} = \altmin$, and for all $i\in \nset{1}{m}$, $\node_i$ is a child of $\node_{i-1}$.
	For all $a \in \nset{\altmin}{0}$, let $j_a$ be the minimal index such that $\altitude{\node_{j_a}} = a$. This index exists as $\altitude{\node_0} = 0$, $\altitude{\node_m} = \altmin$ and for all $i$, $\altitude{\node_{i}} \geq \altitude{\node_{i-1}}-1$.
	Furthermore as $j_a$ is the minimal such index, we cannot have $\altitude{\node_j} \leq a$ for any $j < j_a$ as otherwise there would be a $j'$ such that $j' \leq j < j_a$ with $\altitude{\node_{j'}} = a$.
	
	As a consequence, we have $j_0 < j_1 < j_2 < \ldots < j_{|\altmin|}$. Moreover, $\node_{j_a}$ is necessarily a "boss node" for all $a<0$ as otherwise we would have $\altitude{\node_{j_a-1}} = a-1 \leq a$, and $\node_{j_0}$ is the root and therefore also a "boss node" as $\tree$ satisfies $w$.
	
	Let $N := \size{w} + \size{\prot}+1$. For all $i \in \nset{0}{|\altmin|}$ let $\node'_i = \node_{j_{-i}}$ and $w_i = \bosslabel{\node'_{i}} \in \messages^*$.
	
	We claim that we cannot have $w_{i_1}\subword w_{i_2}$ for any  $i_1< i_2$.
	Suppose by contradiction that there exist such $i_1$ and $i_2$ with $i_1 < i_2$. Then we would have $\bosslabel{\node'_{i_1}} \subword \bosslabel{\node'_{i_2}}$, hence, as $\node_{i_2}$ is a descendant of $\node_{i_1}$, by first case of Lemma~\ref{lem:shortening-branches} there would exist a "unfolding tree" of smaller "size@@tree" than $\tree$ satisfying $w$, contradicting the minimality of $\tree$.
	
	As a result, the sequence $(w_i)_{1\leq i \leq \altmax}$ is a "bad sequence" for the "subword order" over the alphabet $\messages\cup\set{\#}$.
	By Lemma~\ref{lem:bound-length-at-height-h}, for all $\node'_i$ we have $\size{\bosslabel{\node'_i}} \leq f_0(\size{\prot} + \altmax - \altitude{\node'_i}+1) = f_0(\size{\prot} + \altmax +i+1)$.
	By Lemma~\ref{lem:bound-max-height}, we have $\altmax \leq f_1(N)$ and thus $\size{\bosslabel{\node'_i}} \leq  f_0(N+f_1(N)+i)$. Let $g: k \geq 1 \mapsto f_0(k)$; clearly, for all $i$, $\size{w_i} \leq g^{(i)}(M)$ where $M := f_0(N + f_1(N)+1)$ (obviously, $f_0(k) > k+1$) and $g$ is primitive-recursive.
	
	We apply Theorem~\ref{thm:lengthfcttheorem} to obtain $h \in \Ffunction{\omega^{\size{\messages}}}$ such that $(w_i){1\leq i \leq \altmax}$ cannot have more than $h(M)$ elements, thus $|\altmin| \leq h(M)+1 = f_4(f_0(N+f_1(N)+1))$. By defining $f_2: n \mapsto f_4(f_0(n+f_1(n+1)+2))$, this means that $|\altmin| \leq f_2(\size{\prot} + \size{w})$. Moreover, $f_2$ is a function of $\Ffunction{\omega^{\size{\messages}+1}}$ as $f_0$, $f_1$ and $h$ are in $\Ffunction{\omega^{\size{\messages}}} \subseteq \Ffunction{\omega^{\size{\messages}+1}}$ and $\Ffunction{\omega^{\size{\messages}+1}}$ is closed by composition with functions of $\Ffunction{\omega^{\size{\messages}}}$.
\end{proof}

With the bounds on the maximal and minimal altitudes, Lemma~\ref{lem:bound-length-at-height-h} yields a bound on the size of all nodes in the tree.


\begin{lemma}
	\label{lem:bound-node-size}
	There exists a function $f_3$ of the class $\Ffunction{\omega^{\size{\messages}+1}}$ such that for all "protocol" $\prot$, for all "unfolding tree" $\tree$ of minimal "size@@tree" labeled by $\prot$ satisfying a "boss specification" $w$, for all node $\node$ of $\tree$,
	
		\begin{itemize}
		\item $\size{\localrunlabel{\node}} \leq f_3(\size{\prot}+ \size{w})$
			
		\item If $\node$ is a "boss node" then $\size{\bosslabel{\node}} \leq f_3(\size{\prot}+ \size{w})$.
		
		\item If $\node$ is a "follower node" then $\size{\followlabelword{\node}} \leq f_3(\size{\prot}+ \size{w})$.
	\end{itemize} 
\end{lemma}
\begin{proof}
	Using Lemmas~\ref{lem:bound-length-at-height-h}, \ref{lem:bound-max-height} and \ref{lem:bound-min-height}, we obtain that, for every node $\node$, the value $\altmax - \altitude{\node}$ is at most $f_1(\size{\prot} + \size{w}) + f_2(\size{\prot} + \size{w})$. We then apply Lemma~\ref{lem:bound-length-at-height-h} to bounds on $\localrunlabel{\node}$ and $\speclabel{\node}$ by $f_0(f_1(\size{\prot} + \size{w}) + f_2(\size{\prot} + \size{w})+1) \leq f_3(\size{\prot} + \size{w})$ where $f_3: n \mapsto f_0(f_1(n) + f_2(n)+1)$; $f_3$ is in $\Ffunction{\omega^{\size{\messages}+1}}$ as $\Ffunction{\omega^{\size{\messages}+1}}$ is closed by composition with elementary functions. 
\end{proof}

Finally, as we now have a bound on all nodes, we can easily bound the length of branches and the number of children of all nodes to get abound on the total "size@@tree" of the tree.

\begin{restatable}{proposition}{PropBoundTreeSize}
	\label{prop:bound-tree-size}
	There exists a function $f$ of class $\Ffunction{\omega^{\size{\messages}+1}}$ s.t. $\size{\tree} \leq f(\size{\prot})$.
\end{restatable}
	% There exists a function $f$ of the class $\Ffunction{\omega^{\size{\messages}+1}}$ such that for all "protocol" $\prot$, for all "unfolding tree" $\tree$ of minimal size labeled by $\prot$ satisfying a "boss specification" $w$, the size of $\tree$ is bounded by $f(\size{\prot} + \size{w})$.

\begin{proof}
	Recall that the "size@@tree" of $\tree$ is defined as the sum of the sizes of its nodes. 

	Thanks to Proposition~\ref{prop:trees-sound-complete}, $(\prot,q_f)$ is positive if and only if there exists a "coverability witness" for $(\prot,q_f)$.
	Let $\tree$ be a "coverability witness" of $(\prot,q_f)$, \emph{i.e.}, its root is a "boss node" whose "local run" covers $q_f$. Let $N = \size{\prot}+1$. Observe that one can encode the condition that this "local run" covers $q_f$ in the "boss specification" of the protocol (adding a special broadcast message only broadcast from $q_f$); in the following we will forget about the condition that the "local run" at the root covers $q_f$.
	
	We first bound the height of $\tree$ (\emph{i.e.}, the difference between its maximum and minimum "altitude"), then its arity (the maximal number of children of a node), which allows us to obtain a bound on the number of nodes. The sizes of the nodes will finally be bounded using the height. 
	
	By Lemma~\ref{lem:bound-node-size}, for all "boss node" $\node$ in $\tree$ we have $\size{\bosslabel{\node}} \leq f_3(N)$.
	If a branch contains strictly more than $\sum_{i=0}^{f_3(N)}\size{\messages}^{i}$ "follower nodes", then there are two nodes in it with the same "boss" label, hence $\tree$ can be reduced and still satisfy $w$, contradicting the minimality of $\tree$.

	Similarly, by Lemma~\ref{lem:bound-node-size}, for all "follower node" $\node$ in $\tree$ we have $\size{\followlabelword{\node}} \leq f_3(N)$.
	If a branch contains strictly more than $\size{\messages} \sum_{i=0}^{f_3(N)}\size{\messages}^{i}$ "follower nodes", then there are two nodes in it with the same "follower" label, hence $\tree$ can be reduced and still satisfy $w$, contradicting the minimality of $\tree$.
	
	As a result, each branch of $\tree$ contains at most $(\size{\messages}+1) \sum_{i=0}^{f_3(N)}\size{\messages}^{i}$ nodes. We simplify this expression: let $H: n \mapsto n (f_3(n)+1) n^{f_3(n)+1}$, each branch of $\tree$ has less than $H(N)$ nodes.
	
	Furthermore, by Lemma~\ref{lem:bound-successor-height}, each node has at most $\size{\messages}r$ "follower" children. Moreover, if a node $\node$ has more than $f_3(N)$ "boss" children, then one of them can be removed as at most $f_3(N)$ non-initial values appear in $\localrunlabel{\node}$; this would contradict the minimality of $\tree$.	
	As a consequence, each node in $\tree$ has at most $f_3(N) + \size{\messages}(r+1)$ children.

	% Let $\node$ be a node of $\tree$, let $V$ be the set of non-initial values appearing in $\localrunlabel{\node}$ apart from $\valuelabel{\node}$.
	% By condition~\ref{item:condition1_non_initial_value}, for each $v' \in V$, $\node$ has a "boss" child $\node_{v'}$ such that $\vinput{v'}{\localrunlabel{\node}} \subword \bosslabel{\node_{v'}}$.
	% We simply observe that $\size{V} \leq \size{\localrunlabel{\node}}$ as every non-initial value needs to be received at some point in $\localrunlabel{\node}$. We also note that all other "boss" children of $\node$ can be removed, and the resulting tree still satisfies $w$ and is still a "unfolding tree". By minimality of $\tree$, this means that $\mu$ has at most $\size{\localrunlabel{\node}}$ "boss" children, hence at most $f_3(N)$ by Lemma~\ref{lem:bound-node-size}.
	
	By combining the bounds on the length of branches and the arity of nodes, we obtain that $\tree$ has at most $\sum_{h=0}^{H} (f_3(N) + \size{\messages}r)^h$ nodes. Let $G: n \mapsto (H(n)+1)(f_3(n) + n^2)^{H(n)}$. The total number of nodes in $\tree$ is at most $G(N)$.

	Finally, we obtain a bound on the "size@@tree" of $\tree$ by taking the product of the bounds on the number of nodes and on the size of each node label: 
	\[ \size{\tree} \leq  G(N) (2 f_3(N)+1)\]
	
	Let $f: n \mapsto  G(n+1) (2 f_3(n+1) +1)$; using the fact that $n \leq f_3(n)$, we can bound $f_4$ by the composition of an elementary fonction with $f_3$, hence as $f_3 \in \Ffunction{\omega^{\messages}}$ and $\Ffunction{\omega^{\messages}}$ is closed by composition with elementary functions, we have that $f_4 \in \Ffunction{\omega^{\messages}}$.
	Overall, we have shown that the "size@@tree" of a minimal "coverability witness" for $(\prot,q_f)$ is bounded by $f(\size{\prot})$ with $f \in \Ffunction{\omega^{\size{\messages}}}$.
\end{proof}


In this section, we establish the \NP-completeness of the restriction of \COVER to "BNRA" with one register per agent, called 1-BNRA. Due to space constraints, formal proofs are not included in this section; they can be found in Appendix~\ref{app:cover-one-reg}. Here we simply sketch the key observations that allow us to abstract runs into short witnesses, leading to an \NP algorithm for the problem.
	
	In 1-BNRA, thanks to the "copycat principle", any  message can be broadcast with a fresh value, therefore one can always circumvent $\quotemarks{\diseqtestact}$ tests. In the end, our main challenge for 1-BNRA is $\quotemarks{\eqtestact}$ tests upon reception.
	For this reason, we look at clusters of agents that share the value in their registers. 

	Consider a "run" in which some agent $a$ reaches some state $q$,; we can duplicate $a$ many times to have an unlimited supply of agents in state $q$.
	Now assume that, at some point in the "run", agent $a$ stored a received value. Consider the last storing action performed by $a$: $a$ was in a state $q_1$ and performed transition $(q_1, \rec{m}{1}{\enregact}, q_2)$ upon reception of a message $(m,v)$. 
	Because we can assume that we have an unlimited supply of agents in $q_1$ thanks to the copycat principle, we can make as many agents as we want take transition $(q_1, \rec{m}{1}{\enregact}, q_2)$ at the same time as $a$ by receiving the same message $(m,v)$. These new agents end up in $q_2$ with value $\aval$, and then follow $a$ along every transition until they all reach $q$, still with value $v$. In summary, because $a$ has stored a value in the run, we can have an unlimited supply of agents in state $q$ with the same value as $a$. 
	
	Following those observations, we define an abstract semantics with abstract configurations of the form $(S, b, K)$ with $S, K \subseteq Q$ and $b \in Q \cup \set{\bot}$. The first component $S$ is a set of states that we know we can cover (hence we can assume that there are arbitrarily many agents in all these states).
	We start with $S = \set{q_0}$ and try to increase it. To do so, we use the two other components (the \emph{gang}) to keep track of the set of agents sharing a value $v$: $b$ (the \emph{boss}) is the state of the agent which had that value at the start, $K$ (the \emph{clique}) is the set of states covered by other agents with that value. As mentioned above, we may assume that every state of $K$ is filled with as many agents with value $v$ as we need. 
	We will thus define abstract steps which allow to simulate steps of the agents with the value we are following. When they cover states outside of $S$, we may add those to $S$ and reset $b$ to $q_0$ and $K$  to $\emptyset$, to then start following another value.
	We can bound the length of relevant abstract runs, and thus use them as witnesses for our \NP upper bound.
	
%	Say there is a run covering some state outside of $S$, and let $v$ be the value of one of the first agents to get out of $S$. We can then keep track of that value in the abstract semantics. 
	
%	Agent $a$ is unique and called \emph{boss}, agents like $a'$ are clonable and called \emph{followers}. For value $v$, the only relevant information is the state of the boss, \emph{i.e.}, the state of the agent $a$ that had $v$ initially (this agent cannot be cloned), and the \emph{clique}, \emph{i.e.}, the set of states reached by other agents with that value $v$. 
%
%	We thus define \emph{gangs}, an abstraction of configurations 
%	 of the 1-BNRA with respect to a value $v$. A gang is composed of a boss state and a clique. The clique may only increase throughout the abstract runs we will consider, because we assume that we always have enough copies of each agent so that we can leave many agents with value $v$ in every state they visited.
%	\AP More formally, let $(Q,\messages, \transitions, q_0)$ be a protocol.
%	A \emph{gang} is a pair $\gang = (\boss, \clique) \in (Q \cup \set{\noboss}) \times \powerset{Q}$. The element $\boss$ is the \emph{boss} and the set $\clique$ is the \emph{clique} of the gang. 	
%	Let $\run = \config_0 \step{} \config_1 \step{} \cdots \step{} \config_k$ be a "run" and $\aval \in \valsof{\run}$. The gang of value $\aval$ in $\run$ is the gang $(\boss, \clique)$ such that:
%	\begin{itemize}
%		\item if there exists $a_0$ an agent which has value $v$ in $\config_0$ and keeps it in all $\config_i$ then $\boss$ is its state in $\config_k$, otherwise $\boss := \noboss$, 
%		\item  $\clique$ is the set of states containing an agent with value $v$ in some $\config_i$.
%	\end{itemize}



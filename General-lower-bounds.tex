






""Lossy channel systems""  are systems where finite-state processes communicate via send messages from a finite alphabet through unbounded FIFO channels through which messages may be lost. The reachability problem asks whether a given control state of the system may be covered. Unlike in the non-lossy case \cite{BZ83}, reachability is decidable for "lossy channel systems" \cite{AK95,AbdullaJ1996undec}, but is proven to have non-primitive recursive complexity \cite{Schnoebelen2002verifying} and in fact lays at level $\mathcal{F}_{\omega^{\omega}}$ in the Fast-Growing Function Hierarchy \cite{ChambartS2008ordinal}.

We now prove that the coverability problem in BNRA is at least as hard as the reachability problem for "lossy channel system", and that this already hold for $2$-BNRAs.  
\begin{proposition}
	\label{prop:reduction-LCS}
	There is a polynomial-time reduction from the coverability problem for lossy channel systems to the coverability problem for BNRA with two registers.
\end{proposition}

\ifproofs
\begin{proof}
It is in fact sufficient to prove that it is as hard as reachability for lossy channel systems with a single channel, which corresponds to a single finite-state machine that buffering symbols in a lossy FIFO channel \cite{Schnoebelen2002verifying}.
Let $\los := (\lstates,\Sigma,\Delta)$ be a "lossy channel system", where $\lstates$ is a finite set of states, $\Sigma$ is a finite set of symbols and $\Delta \subseteq \lstates \times \Sigma^* \times \set{!, ?} \times \lstates$; $\quotemarks{!}$ corresponds to writing to the channel and $\quotemarks{?}$ to reading from the channel. A configuration of $\los$ is a pair of $\lstates \times \Sigma^*$ denoting the state of the system and the content of the channel. There exists a step from $(\alstate,w)$ to $(\alstate',w')$ using transition $\delta \in \Delta$, denoted $(\alstate,w) \lstep{\delta} (\alstate',w')$, when, by denoting $\subword$ for the "subword relation":
\begin{itemize}
\item $\delta = (\alstate,u,!,\alstate')$ for some $u \in \Sigma^*$ and $w' \subword w \cdot u$ (a ""push"", $u$ is written at the end of the channel), or
\item $\delta = (\alstate,u,?,\alstate')$ for some $u \in \Sigma^*$ and $u \cdot w' \subword w$ (a ""pop"", $u$ is read at the beginning of the channel).
\end{itemize}
The lossiness of the channel is encoded with the subword relations (instead of equalities for non-lossy channels); intuitively, it expresses the fact that letters in the channel may get lost. 

The existence of such a transition for some $\delta \in \transitions$ is denoted $(\alstate,w) \lstep{} (\alstate',w')$, and its transitive closure is denoted $\lstep{*}$. The ""reachability problem@@lcs"" asks, given $\los$ and two states $\alstate_i, \alstate_f \in \lstates$, whether there exists an exeuction $(\alstate_i,\epsilon) \step{*} (\alstate_f, \epsilon)$. 

We aim at constructing a $2$-BNRA $\prot$ with a distinguished state $q_f$ such that $q_f$ may be covered if and only if $(\los, q_i, q_f)$ is a positive instance of the "reachability problem@@lcs".
The intuition of $\prot$ is the following. Agents will form chains, each agent of the chain is meant to encode a configuration of $\lstates \times \Sigma$ along an execution of the "lossy channel system". 
An agent of the chain will only listen to message of their predecessor in the chain, from which they will obtain a state of the system and the content of the channel. It will then broadcast the new state of the system and the new, modified content of the register to the next agent of the chain. 

Note that the content of the channel might become big in a "lossy channel system", therefore agents will not store it but rather rebroadcast it letter by letter. An agent only applies a small modification at the beginning of the channel word if it decides to apply a "pop" transition and at the end of the channel if it decides to apply a "push" transition. Messages might get lost, which is why we are able to encode "lossy channel systems" but not non-lossy ones.

In some initial phase, agents decide whether they are ""root"" (at the beginning of their chain) or ""link"". A "root" agent receives no message; it simply broadcasts its identifiers and that it is on state $\alstate_0$ with an empty channel. To encode this option, in $\prot$, from the initial state $q_0$ one has the possibility to move to a part where the sequence of transitions performed is always $\br{\mathsf{init}}{1}, \br{\mathsf{q_0}{1}}, \br{\mathsf{\#}{1}}$. 
 
A "link" agent first receives a broadcast with an identifier which it decides to be their predecessor for the rest of the execution, then broadcast their own identifier. This construction guarantees that "link" have exactly one predecessor; it does not guarantee, however, than an execution forms a unique chain or that a given agent is the predecessor of exactly one other agent. 

We now define the protocol $\prot$. 


\end{proof}
\fi

\begin{remark}
	The reduction presented in the proof of Proposition~\ref{prop:reduction-LCS} can be adapted to show that the repeat coverability problem is undecidable for BNRA, as it is for LCS.
\end{remark}

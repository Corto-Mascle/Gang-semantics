\subsection{Soundness}
\label{sec:soundness}

%\begin{lemma}[Strong copycat]\label{lem:strong-copycat}
%	Let $q \in Q$ and  $\run : \config_0 \step{*} \config_f$ an "initial run"  that covers $q$. Write $\agents := \agentsof{\run}$. There exists $\agents' \subseteq \nats$ finite, a bijection $\psi : \agents \to \agents'$ and a "run" $\run': \config_0' \step{*} \config_f'$ over $\agents \cup \agents'$ such that:
%	\begin{itemize}
%		\item for all $a \in \agents$, $\config_f'(a) = \config_f(a)$ (agents of $\agents$ behave the same in $\run$ and $\run'$),
%		\item for all $a' \in \agents'$, $\st{\config_f'}(a') = \st{\config_f}(\psi^{-1}(a'))$,
%		\item for all $a \in \agents$, either $\data{\config_f}(a) = \data{\config_0}(a)$ and $\data{\config_f'}(a') = \data{\config_0'}(a')$ or $\data{\config_f'}(\psi(a)) = \data{\config_f'}(a)$.
%	\end{itemize}
%\end{lemma}
%
%\begin{proof}
%	Let $\agents' \subseteq \nats$ such that $|\agents'|= |\agents|$ and $\agents \cap \agents' = \emptyset$. 
%	There exists a bijection $\psi: \agents \mapsto \agents'$. Let $M = 1+ \max\set{\data{\config_0}(a) \mid a \in \agents}$.
%	Let $\config_0'$ an initial configuration over $\agents \cup \agents'$ which is equal to $\config_0$ on every agent of $\agents$, and such that $\config'_0(\psi(a)) = \config_0(a) +M$ for all $a \in \agents$. 
%	The constructed "run" $\run'$ over $\agents \cup \agents'$ starts on configuration $\config_0'$. We define it by induction on $\run$.
%	
%	Let $\run = \config_0 \to \config_1 \to \cdots \to \config_n$, let $u = \config_1 \to \cdots \to \config_{n-1}$, suppose we have a run $u' = \config_0' \to \cdots \to \config_{k}'$ over $\agents \cup \agents'$ such that $u$ and $u'$ satisfy the conditions of the lemma.
%	
%	Let $a$ be the broadcasting agent in the last step of $\run$, $m$ the corresponding message.
%	We have $\config_{n-1}(a) = \config_{k}'(a)$.
%	
%	First, we extend $u'$ by making $\psi(a)$ broadcast $m$, and no agent receives it.
%	We obtain a configuration $\config'_{k+1}$, which is equal to $\config'_k$ except that $\psi(a)$ is now in state $\st{\config_{n}}(a)$.
%	
%	Second we make $a$ broadcast $m$ and for all $a_r \in \agents$ that receives the message in $\config_{n-1} \to \config_n$ with an operation $\alpha$, we make both $a_r$ and $\psi(a_r)$ receive it with the same operation.
%	We have to show that this is a correct transition for all $a_r$.
%	
%	\paragraph{First case: $\config_{n-1}(a_r) \neq \config'_{k}(\psi(a_r))$}  Then $a_r$ and $\psi(a_r)$ both still have their initial value, hence the action $\alpha$ cannot be an equality test
%	
%	
%	First we make $a$ broadcast the same message, and 
%	
%	$\psi(a)$ moves in $\run'$ instead, and if the broadcasting agent has register value $v$ in $\run$ then it has value $v+M$ in $\run'$. In this second part, agents of $\agents$ are left idle. Write $\config_f'$ the final configuration of $\run'$. Since agents of $\agents$ behave the same in $\run$ and $\run'$, for all $a \in \agents$, $\config_f(a) = \config_f'(a')$.
%	A straightforward induction shows that this is indeed a correct run, and that in the end of $\run'$ the states and values of agents of $\agents$ are the same as at the end of $\run$, and that for all $a \in \agents$ we have $\config_f'(\psi(a)) = (\st{\config_f}(a), \data{\config_f}(a) +M)$.
%	
%	From this we immediately infer the conditions of the lemma.
%\end{proof}


\begin{lemma}
	\label{lem:correctness-construction}
	Let $\sigma_0 \in \aconfiginitset$, and $\sigma_0 \to \sigma_1 \to \cdots \to \sigma_n$ an abstract run. For all $i$ let $(S_i, b_i, K_i) := \sigma_i$. Let $M = \size{\Delta}+1$.
	
	For all $i$, there exist a set of agents $\agents_i$, a configuration $\config_i$, a run $\run_i : \config_0 \step{*} \config_i$ over $\agents_i$, agents $a_0, \cdots, a_n \in \agents_i$ and values $v_0, \ldots, v_n \in \nats$ such that:
	\begin{itemize}
		\item for all $s \in S_i$, there are at least $M^{n-i}$ agents (different from $a_i$) in state $s$ 
		
		\item for all $s \in K_i$, there are at least $M^{n-i}$ agents (different from $a_i$) in state $s$ with value $v_i$
		
		\item if $b_i \neq \noboss$, then $a_i$ is in state $b_i$ with value $v_i$.
	\end{itemize}
\end{lemma}

\ifproofs
\begin{proof}

We proceed by induction on $i$.
We set $\agents_0 = \set{1, \ldots, M^n}$, and we set $\config_0(a) = (q_0, a)$ for all $a$. Clearly $\config_0$ satisfies the requirements with respect to $\sigma_0$, with $a_0 = v_0 \in \agents$.

Now assume we constructed $\config_0 \step{*} \cdots \step{*} \config_{i}$ over $\agents_i$ satisfying the conditions of the lemma, we construct $\config_{i+1}$ using a case distinction on the form of the transition $\sigma_i \to \sigma_{i+1}$.
For each $s \in S\setminus K$ we define $\agents_{i,s}$ as the set of agents in state $s$ in $\config_{i}$. We have $\size{\agents_{i,s}} \geq M^{n-i}$ thus we can extract $M = \size{\Delta}+1$ disjoint sets of agents $(\agents_{i,s}^d)_{d \in \Delta\cup\set{\epsilon}}$ from it.
Similarly, for each $s \in K$ we define $\agents_{i,s}$ as the set of agents in state $s$ \textbf{with value $\mathbf{v_i}$} in $\config_{i}$. We have $\size{\agents_{i,s}} \geq \size{\Delta}^{n-i}$ thus we can extract $\size{\Delta}+1$ disjoint sets of agents $(\agents_{i,s}^d)_{d \in \Delta\cup\set{\epsilon}}$ from it.
\\

\textbf{Case 1: } If $\sigma_i \to \sigma_{i+1}$ is a \emph{broadcast from the clique} $d = (q, \brone{m}, q')$ with $q \in K_i$, then we make all agents $a \in \agents_{i,q}^{d}$ (which all have value $v_i$) execute that transition one by one.
None of those broadcasts are received by any other agent, except for the last one:
If $b \neq b'$ then there is a transition $(b, \rec{m}{\alpha}, b')$ and we make $a_i$ execute it upon receiving the broadcast. We then set $a_{i+1} = a_i$.
For all $k' \in K_{i+1} \setminus K_i$ there exists a transition $d'=(k, \rec{m}{\alpha}, k')$ such that either $\alpha$ is $\eqtestact$ or $*$ and $k \in K_i$ or $\alpha$ is $\enregact$ and $k\in S$.
In both cases we make all agents of $\agents_{i,k}^{d'}$ take that transition.

For all $s' \in S_{i+1} \setminus (S_i \cup K_{i+1})$ there exists a transition $d'=(s,\rec{m}{*},s')$ (the operation cannot be $\enregact$ or $\eqtestact$ as otherwise $s$ would be in $K_{i+1}$). We then make all agents of $\agents_{i,s}^{d'}$ follow that transition. 

We set $v_{i+1} = v_i$.
\\

\textbf{Case 2: }If $\sigma_i \to \sigma_{i+1}$ is a \emph{broadcast from the boss} $d = (b_i, \brone{m}, b_{i+1})$, then we make $a_i$ (which has value $v_i$) execute that transition, and we set $a_{i+1} = a_i$.
The agents receiving that message are as follows:

For all $k' \in K_{i+1} \setminus K_i $ there exists a transition $d'=(k, \rec{m}{\alpha}, k')$ such that $\alpha$ is either $\eqtestact$ or $*$ and $k \in K_i$ or $\alpha$ is $\enregact$ and $k\in S$.
In both cases we make all agents of $\agents_k^{d'}$ take that transition.

For all $s' \in S_{i+1} \setminus (S_i \cup K_{i+1} \cup \set{b_{i+1}})$ there exists a transition $d'=(s,\rec{m}{*},s')$ (the operation cannot be $\enregact$ or $\eqtestact$ as otherwise $s$ would be in $K_{i+1}$). We then make all agents of $\agents_s^{d'}$ follow that transition. 

By definition of an "abstract run", we must have $b_i \in S_i$.
Hence we can make all agents of $\agents_{i,s}^{d}$ execute $d$, with no agent receiving the corresponding broadcasts.

We set $v_{i+1} = v_i$.
\\

\textbf{Case 3: } If $\sigma_i \to \sigma_{i+1}$ is an \emph{external broadcast} $d = (q, \brone{m}, q')$ , then we make all agents $a \in \agents_q^{d}$ execute that transition one by one. None of those broadcasts are received by any other agent, except for the last one:
If $b_i \neq b_{i+1}$ then there is a transition $(b, \rec{m}{\alpha}, b'')$ and either $b_{i+1} = b'' \neq \noboss$ and $\alpha = *$ or $b_{i+1} = \noboss$ and $\alpha=\enregact$. In both cases we make $a_i$ execute that transition, and we set $a_{i+1} = a_i$.

For all $k' \in K_{i+1} \setminus K_i$ there exists a transition $d'=(k, \rec{m}{*}, k')$ with $k \in K_i$. We make all agents of $\agents_k^{d'}$ take that transition.

For all $s' \in S_{i+1} \setminus (S_i \cup K_{i+1})$ there exists a transition $d'=(s,\rec{m}{\alpha},s')$ with $\alpha \in \set{*, \enregact}$. We then make all agents of $\agents_s^{d'}$ follow that transition. 

We set $v_{i+1} = v_i$.
\\

\textbf{Case 4: }  If $\sigma_i \to \sigma_{i+1}$ is an \emph{gang reset} then no agent moves and we select some $a_{i+1}$ in $\agents_{q_0}$ and set $v_{i+1}$ to be its value.
\\
%\paragraph{In all cases:} After applying the given transitions, we use the copycat property: we add to $\agents_i$ $M^{n-i-1}$ disjoint copies of itself to obtain $\agents_{i+1}$, and repeat the run constructed thus far over each copy separately. This is to ensure that there are $M^{n-i-1}$ agents in $b_{i+1}$ (if it is not $\noboss$) after this step.

Throughout the case distinction we have ensured that:
\begin{itemize}
	\item If $b_{i+1} \neq \noboss$ then $a_{i+1}$ is an agent of value $v_{i+1}$.
	
	\item For all $k \in K_{i}$, the agents of $\agents_{i,k}^\epsilon$ do not move between configurations $\config_{i}$ and $\config_{i+1}$, hence they have state $k$ and value $v_{i+1}$ in $\config_{i+1}$.
	
	\item If the step is not a gang reset, then $v_{i+1} = v_i$ and for all $k' \in K_{i+1} \setminus K_i$, there exists $d \in \Delta$ from some $k$ to $k'$ such that all agents of $\agents_{i,k}^d$ take that transition. Furthermore, if $d$ is of the form $(k,\rec{m}{\enregact},k')$ then the broadcasting process has value $v_i$, thus all those agents keep value $v_i = v_{i+1}$. 
%	As $\size{\agents_{i,k}^d}\geq M^{n-i-1}$, there are at least that many agents with value $v_{i+1} = v_i$ in $k'$.
	
	\item For all $s \in S_{i}$, the agents of $\agents_{i,s}^\epsilon$ do not move between configurations $\config_{i}$ and $\config_{i+1}$, hence they have state $s$ in $\config_{i+1}$.
	
	\item If the step is not a gang reset, for all $s' \in S_{i+1} \setminus (S_i \cup \set{b_{i+1}})$, there exists $d \in \Delta$ from some $s \in S_i$ to $s'$ such that all agents of $\agents_{i,s}^d$ take that transition.
	
	\item If the step is a gang reset, the conditions of the lemma hold trivially.
\end{itemize}

As a result, we have ensured that the conditions of the lemma were respected.
This concludes our induction.
\end{proof}
\fi

\begin{corollary}
	\label{cor:soundness}
	For all $\sigma_0 \in \aconfiginitset$ and $\sigma = (S, b, K) \in \Sigma$ such that $\sigma_0 \step{*} \sigma$ there exists a reachable configuration $\gamma$ satisfying $K \cup \set{b}$ if $b \neq \noboss$ and $K$ otherwise.
\end{corollary}

\ifproofs
\begin{proof}
	We simply apply Lemma~\ref{lem:correctness-construction} to an "abstract run" $\sigma_0 \to \cdots \sigma_n = \sigma$ from $\sigma_0$ to $\sigma$.
	We obtain (by setting $i = n$) that there exists a value $v_n$ in the final configuration of the constructed run, for all $s \in K$, there is an agent with value $v_n$ in state $s$. Furthermore, if $b \neq \noboss$, then there is an agent in state $b$ with value $v_n$.  
\end{proof}
\fi
